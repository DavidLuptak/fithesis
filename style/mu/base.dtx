% \iffalse
%<*base>
% \fi\file{style/mu/fithesis3-base.sty}
% This is the base style file for theses written at the Masaryk
% University in Brno.
% If inheritance is enabled for style files, then this file is
% always the second style file to be loaded right after
% \texttt{style/base.sty}, regardless of the
% value of the |\thesis@style| macro.
%    \begin{macrocode}
\ProvidesPackage{fithesis3/style/mu/fithesis3-base}[2015/04/12]
\NeedsTeXFormat{LaTeX2e}
%    \end{macrocode}
% The file recognizes the following options: \begin{itemize}
%   \item\texttt{10pt}, \texttt{11pt}, \texttt{12pt} -- Sets the
%     type size to 10, 11 or 12 points respectively, along with the
%     page geometry. The default type size is 12 points.
%   \item\texttt{oneside}, \texttt{twoside} -- The document is
%     going to be single- or double-sided. In a double-sided
%     document headers, page numbering, margin notes and several
%     other elements are rearranged based on
%     the parity of the page. Blank pages are optionally inserted
%     prior the beginning of the next chapter to ensure that it
%     starts on an left-hand (even-numbered) page.
%
%     The \DescribeMacro{\ifthesis@twoside}|\ifthesis@twoside|
%     conditional is set to \textit{false} or \textit{true},
%     respectively. This value can be tested in the subsequently
%     loaded style files.
%   \item\texttt{onecolumn}, \texttt{twocolumn} -- The document
%     is going to be set in a single column or in two columns,
%     respectively.
%   \item\texttt{draft}, \texttt{final} -- Overful lines are or
%     aren't marked within the document, respectively.
%   \item\texttt{monochrome}, \texttt{color} -- The
%     \DescribeMacro{\ifthesis@color}|\ifthesis@color| conditional
%     is set to \textit{false} or \textit{true}, respectively.
%     This value can be tested in the subsequently loaded style
%     files.
%   \item\texttt{lot}, \texttt{nolot} -- |\listoftables| is or
%     isn't going to be included in the
%     \DescribeMacro{\thesis@blocks@tables}|\thesis@blocks@tables|
%     block, respectively.
%   \item\texttt{lof}, \texttt{nolof} -- |\listoffigures| is or
%     isn't going to be included in the
%     \DescribeMacro{\thesis@blocks@tables}|\thesis@blocks@tables|
%     block, respectively.
%   \item\texttt{cover}, \texttt{nocover} -- The
%     \DescribeMacro{\thesis@blocks@cover}|\thesis@blocks@cover|
%     is or isn't going to expand to the thesis cover.
% \end{itemize}
% The defaults are \texttt{12pt}, \texttt{oneside}, \texttt{final},
% \texttt{monochrome}, \texttt{nolof}, \texttt{nolot} and
% \texttt{nocover}.
%    \begin{macrocode}
\DeclareOption{10pt}{\def\thesis@ptsize{0}}
\DeclareOption{11pt}{\def\thesis@ptsize{1}}
\DeclareOption{12pt}{\def\thesis@ptsize{2}}
\newif\ifthesis@twoside
\DeclareOption{oneside}{%
  \thesis@twosidefalse\@twosidefalse\@mparswitchfalse}
\DeclareOption{twoside}{%
  \thesis@twosidetrue \@twosidetrue \@mparswitchtrue}
\DeclareOption{onecolumn}{\@twocolumnfalse}
\DeclareOption{twocolumn}{\@twocolumntrue}
\DeclareOption{draft}{\setlength\overfullrule{5pt}}
\DeclareOption{final}{\setlength\overfullrule{0pt}}
\newif\ifthesis@color\thesis@colorfalse
\DeclareOption{monochrome}{\thesis@colorfalse}
\DeclareOption{color}{\thesis@colortrue}
\DeclareOption{nolot}{\def\thesis@blocks@lot{}}
\DeclareOption{lot}{\let\thesis@blocks@lot\listoftables}
\DeclareOption{nolof}{\def\thesis@blocks@lof{}}
\DeclareOption{lof}{\let\thesis@blocks@lof\listoffigures}
\newif\ifthesis@cover
\DeclareOption{nocover}{\thesis@coverfalse}
\DeclareOption{cover}{\thesis@covertrue}

% Options executed by default
\ExecuteOptions{12pt,oneside,final,monochrome,lot,lof,nocover}
\ProcessOptions
%    \end{macrocode}
% The file loads the following packages: \begin{itemize}
%   \item\textsf{fontenc} -- The font encoding is set to Cork.
%   \item\textsf{mathpazo} -- The virtual \texttt{mathpazo} fonts
%     will be used for math.
%   \item\textsf{tgpagella} -- Changes the default roman font family
%     to \TeX\ Gyre Pagella.
%   \item\textsf{cmap} -- Places an an explicit \texttt{ToUnicode}
%     map in the resulting PDF file, allowing for extraction of
%     the text of the document.
%   \item\textsf{graphix} -- Adds support for the inclusion of
%     graphics files.
%   \item\textsf{pdfpages} -- Adds support for the injection of PDF
%     documents into the resulting document, namely the thesis
%     assignment.
%   \item\textsf{hyperref} -- Adds support for injecting metadata
%     into the resulting PDF document.
% \end{itemize}
% In addition to that, the \textsf{hyperref} package is configured
% to support both roman and arabic page numbering in one document.
%    \begin{macrocode}
\RequirePackage[resetfonts]{cmap}
\RequirePackage[T1]{fontenc}
\thesis@require{mathpazo}
\thesis@require{tgpagella}
\thesis@require{graphicx}
\thesis@require{pdfpages}
\thesis@require{hyperref}
\hypersetup{
  plainpages=false,          % Multiple page numbering support
  pdfpagelabels              % Generate pdf page labels
}
%    \end{macrocode}
% The file defines several blocks to be used in the redefinitions
% of the |\thesis@preamble| and |\thesis@postable| private macros
% by the subsequently loaded style files.
%
% \begin{macro}{\thesis@blocks@frontMatter}
% The |\thesis@blocks@frontMatter| private macro sets up the style
% of the front matter of the thesis.
% \begin{macrocode}
\def\thesis@blocks@frontMatter{%
  \thesis@blocks@clear
  \pagestyle{plain}
  \parindent 1.5em
  \setcounter{page}{1}
  \pagenumbering{roman}}
%    \end{macrocode}
% \end{macro}\begin{macro}{\thesis@blocks@clear}
% The |\thesis@blocks@clear| private macro clears the current page
% along with the next left-handed (even-numbered) page, when
% double-sided typesetting is enabled.
% \begin{macrocode}
\def\thesis@blocks@clear{%
  \ifthesis@twoside%
    \clearpage%
    \thispagestyle{empty}%
    \cleardoublepage%
  \else%
    \newpage%
  \fi}
%    \end{macrocode}
% \end{macro}\begin{macro}{\thesis@blocks@cover}
% The |\thesis@blocks@cover| private macro typesets the thesis
% cover. It is composed of three private macros:
% \begin{itemize}
%   \item\DescribeMacro{\thesis@blocks@cover@header}^^A
%        |\thesis@blocks@cover@header| -- The header of the cover
%        page
%   \item\DescribeMacro{\thesis@blocks@cover@content}^^A
%        |\thesis@blocks@cover@content| -- The content of the cover
%        page
%   \item\DescribeMacro{\thesis@blocks@cover@footer}^^A
%        |\thesis@blocks@cover@footer| -- The footer of the cover
%        page
% \end{itemize}
% This allows the subsequently loaded style files to only redefine
% certain parts of the cover page.
% \begin{macrocode}
\def\thesis@blocks@cover{%
  \ifthesis@cover%
    \thesis@blocks@clear%
    \ifthesis@twoside\@twosidefalse\fi % Temporarily disable twoside
    \begin{alwayssingle}%
      \thispagestyle{empty}%
      \begin{center}%
        \thesis@blocks@cover@header%
        \includegraphics[width=40mm]{%
          \thesis@logopath\thesis@facultyLogo}\\[0.4in]%
        \let\footnotesize\small%
        \let\footnoterule\relax{}%
        \thesis@blocks@cover@content%
        \par\vfill%
        \thesis@blocks@cover@footer%
      \end{center}%
    \end{alwayssingle}%
    \ifthesis@twoside\@twosidetrue\fi % Re-enable twoside
  \fi}
%    \end{macrocode}
% The output of the |\thesis@blocks@cover@header| private macro is
% controlled by the following conditional expressions:
% \begin{enumerate}
%   \item|\ifthesis@blocks@cover@university@| -- This
%        conditional expression determines, whether the university
%        name is going to be included in the header of the cover.
%        The default value of this conditional expression is true.
%   \item|\ifthesis@blocks@cover@faculty@| -- This
%        conditional expression determines, whether the faculty
%        name is going to be included in the header of the cover.
%        The default value of this conditional expression is true.
%   \item|\ifthesis@blocks@cover@department@| -- This
%        conditional expression determines, whether the department
%        name is going to be included in the header of the cover.
%        The default value of this conditional expression is false.
%   \item|\ifthesis@blocks@cover@field@| -- This
%        conditional expression determines, whether the field of
%        study is going to be included in the header of the cover.
%        The default value of this conditional expression is false.
% \end{enumerate}
% The sebsequently loaded style files can alter the value of these
% expressions to alter the output of the
% |\thesis@blocks@cover@header| private macro without altering its
% definition.
% \begin{macrocode}
\newif\ifthesis@blocks@cover@university@
\thesis@blocks@cover@university@true
\newif\ifthesis@blocks@cover@faculty@
\thesis@blocks@cover@faculty@true
\newif\ifthesis@blocks@cover@department@
\thesis@blocks@cover@department@false
\newif\ifthesis@blocks@cover@field@
\thesis@blocks@cover@field@false

\def\thesis@blocks@cover@header{%
  {\sc\ifthesis@blocks@cover@university@%
        \thesis@titlePage@LARGE\thesis@@{universityName}\\%
   \fi\ifthesis@blocks@cover@faculty@%
        \thesis@titlePage@Large\thesis@@{facultyName}\\%
   \fi\ifthesis@blocks@cover@department@%
        \thesis@titlePage@large\thesis@department\\%
      \fi}
  \ifthesis@blocks@cover@field@%
    {\thesis@titlePage@large\vskip 2em%
      {\bf\thesis@@{fieldTitle}:} \thesis@field}%
  \fi\vskip 2em}
\def\thesis@blocks@cover@content{%
  {\thesis@titlePage@Huge\bf\thesis@TeXtitle\par\vfil}%
  \vskip 0.8in%
  {\sc \thesis@@{typeName}}\\[0.3in]%
  {\thesis@titlePage@Large\bf\thesis@author}}
\def\thesis@blocks@cover@footer{%
  {\thesis@titlePage@large\thesis@place, \thesis@@{semester}}}
%    \end{macrocode}
% \end{macro}
% \begin{macro}{\thesis@blocks@titlePage}
% The |\thesis@blocks@titlePage| private macro typesets the thesis
% title page. It is composed of three private macros:
% \begin{itemize}
%   \item|\thesis@blocks@titlePage@header| -- The header of the
%        cover page
%   \item|\thesis@blocks@titlePage@content| -- The content of the
%        cover page
%   \item|\thesis@blocks@titlePage@footer| -- The footer of the
%        cover page
% \end{itemize}
% This allows the subsequently loaded style files to only redefine
% certain parts of the title page.
%    \begin{macrocode}
\def\thesis@blocks@titlePage{%
    \thesis@blocks@clear%
    \begin{alwayssingle}%
      \thispagestyle{empty}%
      \begin{center}%
        \thesis@blocks@titlePage@header%
        {\edef\thesis@logopath@color{%
          \thesis@logopath\ifthesis@color color/\fi}
        \includegraphics[width=40mm]{%
          \thesis@logopath@color\thesis@facultyLogo}\\[0.4in]}%
        \let\footnotesize\small%
        \let\footnoterule\relax{}%
        \thesis@blocks@titlePage@content%
        \par\vfill%
        \thesis@blocks@titlePage@footer%
      \end{center}%
    \end{alwayssingle}}
%    \end{macrocode}
% The output of the |\thesis@blocks@titlePage@header| private macro is
% controlled by the following conditional expressions:
% \begin{enumerate}
%   \item|\ifthesis@blocks@titlePage@university@| -- This
%        conditional expression determines, whether the university
%        name is going to be included in the header of the title
%        page. The default value of this conditional expression is
%        true.
%   \item|\ifthesis@blocks@titlePage@faculty@| -- This
%        conditional expression determines, whether the faculty of
%        study is going to be included in the header of the title
%        page.
%        The default value of this conditional expression is true.
%   \item|\ifthesis@blocks@titlePage@department@| -- This
%        conditional expression determines, whether the department
%        name is going to be included in the header of the title
%        page. The default value of this conditional expression is
%        false.
%   \item|\ifthesis@blocks@titlePage@field@| -- This
%        conditional expression determines, whether the field of
%        study is going to be included in the header of the title
%        page.
%        The default value of this conditional expression is false.
% \end{enumerate}
% The sebsequently loaded style files can alter the value of these
% expressions to alter the output of the
% |\thesis@blocks@titlePage@header| private macro without altering
% its definition.
% \begin{macrocode}
\newif\ifthesis@blocks@titlePage@university@
\thesis@blocks@titlePage@university@true
\newif\ifthesis@blocks@titlePage@faculty@
\thesis@blocks@titlePage@faculty@true
\newif\ifthesis@blocks@titlePage@department@
\thesis@blocks@titlePage@department@false
\newif\ifthesis@blocks@titlePage@field@
\thesis@blocks@titlePage@field@false

\def\thesis@blocks@titlePage@header{%
  {\sc\ifthesis@blocks@titlePage@university@%
        \thesis@titlePage@LARGE\thesis@@{universityName}\\%
   \fi\ifthesis@blocks@titlePage@faculty@%
        \thesis@titlePage@Large\thesis@@{facultyName}\\%
   \fi\ifthesis@blocks@titlePage@department@%
        \thesis@titlePage@large\thesis@department\\%
      \fi}
  \ifthesis@blocks@titlePage@field@%
    {\thesis@titlePage@large\vskip 2em%
      {\bf\thesis@@{fieldTitle}:} \thesis@field}%
  \fi\vskip 2em}
\let\thesis@blocks@titlePage@content=\thesis@blocks@cover@content
\let\thesis@blocks@titlePage@footer=\thesis@blocks@cover@footer
%    \end{macrocode}
% \end{macro}\begin{macro}{\thesis@blocks@tables}
% The |\thesis@blocks@tables| private macro typesets the table of
% contents and optionally the |\listoftables| and the
% |\listoffigures|.
% \begin{macrocode}
\def\thesis@blocks@tables{%
  \thesis@blocks@clear%
  \tableofcontents%
  \thesis@blocks@lot%
  \thesis@blocks@lof%
  \thesis@blocks@clear}
%    \end{macrocode}
% \end{macro}\begin{macro}{\thesis@blocks@declaration}
% The |\thesis@blocks@declaration| private macro typesets the
% declaration text.
% \begin{macrocode}
\def\thesis@blocks@declaration{%
  \thesis@blocks@clear%
  \begin{alwayssingle}%
    \chapter*{\thesis@@{declarationTitle}}%
    \thesis@@{declaration}%
    \vskip 2cm%
    \hfill\thesis@author%
  \end{alwayssingle}
  \thesis@blocks@clear}
%    \end{macrocode}
% \end{macro}\begin{macro}{\thesis@blocks@thanks}
% The |\thesis@blocks@thanks| private macro typesets the
% acknowledgement, if the |\thesis@thanks| private macro is
% defined.
% \begin{macrocode}
\def\thesis@blocks@thanks{%
  \ifx\thesis@thanks\undefined\else%
    \thesis@blocks@clear%
    \begin{alwayssingle}%
      \chapter*{\vspace*{\fill}\thesis@@{thanksTitle}}%
      \thesis@thanks%
    \end{alwayssingle}%
  \fi}
%    \end{macrocode}
% \end{macro}\begin{macro}{\thesis@blocks@abstract}
% The |\thesis@blocks@abstract| private macro typesets the
% abstract.
% \begin{macrocode}
\def\thesis@blocks@abstract{%
  \begin{alwayssingle}%
    \chapter*{\thesis@@{abstractTitle}}%
    \thesis@abstract%
    \par\vfil\null%
  \end{alwayssingle}}
%    \end{macrocode}
% \end{macro}\begin{macro}{\thesis@blocks@abstractEn}
% The |\thesis@blocks@abstractEn| private macro typesets the
% abstract in English. If the current locale is English, the
% macro produces no output. A style file that uses this block
% needs to require the English locale.
% \begin{macrocode}
\def\thesis@blocks@abstractEn{%
  \ifthesis@english\else%
    \newpage%
    \begin{alwayssingle}%
      % Start the new chapter without clearing the right page
      {\def\cleardoublepage{}%
      \chapter*{\thesis@{english@abstractTitle}}%
      \thesis@abstractEn}%
      \par\vfil\null%
    \end{alwayssingle}%
  \fi}
%    \end{macrocode}
% \end{macro}\begin{macro}{\thesis@blocks@keywords}
% The |\thesis@blocks@keywords| private macro typesets the
% keywords.
% \begin{macrocode}
\def\thesis@blocks@keywords{%
  \begin{alwayssingle}%
      % Start the new chapter without clearing the right page
      {\def\cleardoublepage{}%
      \chapter*{\thesis@@{keywordsTitle}}%
      \thesis@keywords}%
    \par\vfill%
  \end{alwayssingle}}
%    \end{macrocode}
% \end{macro}\begin{macro}{\thesis@blocks@keywordsEn}
% The |\thesis@blocks@keywordsEn| private macro typesets the
% keywords in English. If the current locale is English, the
% macro produces no output. A style file that uses this block
% needs to require the English locale.
% \begin{macrocode}
\def\thesis@blocks@keywordsEn{%
  \ifthesis@english\else%
    \begin{alwayssingle}%
      \newpage
      % Start the new chapter without clearing the right page
      {\def\cleardoublepage{}%
      \chapter*{\thesis@{english@keywordsTitle}}%
      \thesis@keywordsEn}%
      \par\vfill%
    \end{alwayssingle}%
  \fi}
%    \end{macrocode}
% \end{macro}\begin{macro}{\thesis@blocks@assignment}
% The |\thesis@blocks@assignment| private macro either typesets a
% blank page to be replaced with the official thesis assignment or
% injects the file located at the |\thesis@assignmentPDF| path, if
% defined. In case of a rigorous thesis, the macro expands to an
% empty token string.
% \begin{macrocode}
\def\thesis@blocks@assignment{%
  \ifx\thesis@type\thesis@rigorous\else%
    \thesis@blocks@clear%
    \ifx\thesis@assignmentFiles\undefined%
      \begin{alwayssingle}%
        \thispagestyle{empty}
        \addtocounter{page}{-\ifthesis@twoside2\else1\fi}
        \noindent\textit{\thesis@@{assignment}}%
      \end{alwayssingle}%
    \else%
      {\edef\@pdfList{\thesis@assignmentFiles}%
      \expandafter\includepdfmerge\expandafter{\@pdfList}}%
    \fi%
    \thesis@blocks@clear
  \fi}
%    \end{macrocode}
% \end{macro}\begin{macro}{\thesis@blocks@mainMatter}
% The |\thesis@blocks@mainMatter| private macro sets up the style
% of the main matter of the thesis.
% \begin{macrocode}
\def\thesis@blocks@mainMatter{%
  \thesis@blocks@clear
  \setcounter{page}{1}
  \pagenumbering{arabic}
  \pagestyle{thesisheadings}
  \parindent 1.5em\relax}
%    \end{macrocode}
% \end{macro}
% The rest of the file comprises redefinitions of \LaTeX\ commands
% and private \texttt{rapport3} class macros altering the layout of
% the resulting document. Depending on the type size of 10, 11 or
% 12 points, either the \texttt{fithesis3-10.clo},
% \texttt{fithesis3-11.clo} or \texttt{fithesis3-12.clo} file is
% loaded from the |\thesis@stylepath mu| directory, respectively.
%    \begin{macrocode}
% Table of contents will contain sectioning commands up to
% \subsubsection
\setcounter{tocdepth}{4}

% Load the `fithesis3-1*.clo` size option
\input\thesis@stylepath mu/fithesis3-1\thesis@ptsize.clo\relax

\def\ps@thesisheadings{%
\def\chaptermark##1{%
\markright{%
\ifnum\c@secnumdepth >\m@ne
\thechapter.\ %
\fi ##1}}
\let\@oddfoot\@empty
\let\@oddhead\@empty
\def\@oddhead{\vbox{\hbox to \textwidth{%
\hfil{\sc\rightmark}}\vskip 4pt\hrule}}
\if@twoside
 \def\@evenhead{\vbox{\hbox to \textwidth{%
 {\sc\rightmark}\hfil}\vskip 4pt\hrule}}
\else
 \let\@evenhead\@oddhead
\fi
\def\@oddfoot{\hfil\PageFont\thepage}
\if@twoside
 \def\@evenfoot{\PageFont\thepage\hfil}%
\else
 \let\@evenfoot\@oddfoot
\fi
\let\@mkboth\markboth
}

% Redefines the style of the chapter headings
\renewcommand*\chapter{%
\if@twoside
 \clearpage
 \thispagestyle{empty}
 \cleardoublepage
\else
 \clearpage
\fi
\thispagestyle{plain}%
\global\@topnum\z@
\@afterindentfalse
\secdef\@chapter\@schapter}

% Redefines the style of part headings
\renewcommand*\part{%
\clearpage
\thispagestyle{empty}
\cleardoublepage
\thispagestyle{empty}%
\if@twocolumn%
 \onecolumn
 \@tempswatrue
\else
 \@tempswafalse
\fi
\hbox{}\vfil
\secdef\@part\@spart}

\newif\if@restonecol
\def\alwayssingle{%
  \@restonecolfalse\if@twocolumn\@restonecoltrue\onecolumn\fi}
\def\endalwayssingle{\if@restonecol\twocolumn\fi}

\renewcommand*\l@part[2]{%
  \ifnum \c@tocdepth >-2\relax
    \addpenalty{-\@highpenalty}%
    \addvspace{0.5em \@plus\p@}%
    \begingroup
      \setlength\@tempdima{3em}%
      \parindent \z@ \rightskip \@pnumwidth
      \parfillskip -\@pnumwidth
      {\leavevmode
       \normalfont \bfseries #1\hfil \hb@xt@\@pnumwidth{\hss #2}}\par
       \nobreak
         \global\@nobreaktrue
         \everypar{\global\@nobreakfalse\everypar{}}%
    \endgroup
    \addvspace{0.2em \@plus\p@}%
  \fi}

\renewcommand*\l@chapter[2]{%
  \ifnum \c@tocdepth >\m@ne
    \addpenalty{-\@highpenalty}%
    \vskip 1.0em \@plus\p@
    \setlength\@tempdima{1.5em}%
    \begingroup
      \parindent \z@ \rightskip \@pnumwidth
      \parfillskip -\@pnumwidth
      \leavevmode \bfseries
      \advance\leftskip\@tempdima
      \hskip -\leftskip
      #1\nobreak\hfil \nobreak\hb@xt@\@pnumwidth{\hss #2}\par
      \penalty\@highpenalty
    \endgroup
  \fi}

\renewcommand*\l@chapter{\@dottedtocline{1}{0em}{1.5em}}
\renewcommand*\l@section{\@dottedtocline{2}{1.5em}{2.3em}}
\renewcommand*\l@subsection{\@dottedtocline{2}{3.8em}{3.2em}}
\renewcommand*\l@subsubsection{\@dottedtocline{2}{7.0em}{3.8em}}
%    \end{macrocode}\iffalse
%</base>
% \fi\file{style/mu/fit10.clo}
% This file is conditionally loaded by the
% \texttt{style/mu/base.sty} file to redefine the page geometry to
% match the type size of 10 points.
%    \begin{macrocode}
%<*opt>
%<*10pt>
\ProvidesFile{fit10.clo}[2015/04/08]

\renewcommand{\normalsize}{\fontsize\@xpt{12}\selectfont%
\abovedisplayskip 10\p@ plus2\p@ minus5\p@
\belowdisplayskip \abovedisplayskip
\abovedisplayshortskip  \z@ plus3\p@
\belowdisplayshortskip  6\p@ plus3\p@ minus3\p@
\let\@listi\@listI}

\renewcommand{\small}{\fontsize\@ixpt{11}\selectfont%
\abovedisplayskip 8.5\p@ plus3\p@ minus4\p@
\belowdisplayskip \abovedisplayskip
\abovedisplayshortskip \z@ plus2\p@
\belowdisplayshortskip 4\p@ plus2\p@ minus2\p@
\def\@listi{\leftmargin\leftmargini
\topsep 4\p@ plus2\p@ minus2\p@\parsep 2\p@ plus\p@ minus\p@
\itemsep \parsep}}

\renewcommand{\footnotesize}{\fontsize\@viiipt{9.5}\selectfont%
\abovedisplayskip 6\p@ plus2\p@ minus4\p@
\belowdisplayskip \abovedisplayskip
\abovedisplayshortskip \z@ plus\p@
\belowdisplayshortskip 3\p@ plus\p@ minus2\p@
\def\@listi{\leftmargin\leftmargini %% Added 22 Dec 87
\topsep 3\p@ plus\p@ minus\p@\parsep 2\p@ plus\p@ minus\p@
\itemsep \parsep}}

\renewcommand{\scriptsize}{\fontsize\@viipt{8pt}\selectfont}
\renewcommand{\tiny}{\fontsize\@vpt{6pt}\selectfont}
\renewcommand{\large}{\fontsize\@xiipt{14pt}\selectfont}
\renewcommand{\Large}{\fontsize\@xivpt{18pt}\selectfont}
\renewcommand{\LARGE}{\fontsize\@xviipt{22pt}\selectfont}
\renewcommand{\huge}{\fontsize\@xxpt{25pt}\selectfont}
\renewcommand{\Huge}{\fontsize\@xxvpt{30pt}\selectfont}

%</10pt>
%    \end{macrocode}
% \file{style/mu/fit11.clo}
% This file is conditionally loaded by the
% \texttt{style/mu/base.sty} file to redefine the page geometry to
% match the type size of 11 points.
%    \begin{macrocode}
%<*11pt>
\ProvidesFile{fit11.clo}[2015/04/08]

\renewcommand{\normalsize}{\fontsize\@xipt{14}\selectfont%
\abovedisplayskip 11\p@ plus3\p@ minus6\p@
\belowdisplayskip \abovedisplayskip
\belowdisplayshortskip  6.5\p@ plus3.5\p@ minus3\p@
%\abovedisplayshortskip  \z@ plus3\@p
\let\@listi\@listI}

\renewcommand{\small}{\fontsize\@xpt{12}\selectfont%
\abovedisplayskip 10\p@ plus2\p@ minus5\p@ 
\belowdisplayskip \abovedisplayskip
\abovedisplayshortskip  \z@ plus3\p@
\belowdisplayshortskip  6\p@ plus3\p@ minus3\p@
\def\@listi{\leftmargin\leftmargini
\topsep 6\p@ plus2\p@ minus2\p@\parsep 3\p@ plus2\p@ minus\p@
\itemsep \parsep}}

\renewcommand{\footnotesize}{\fontsize\@ixpt{11}\selectfont%
\abovedisplayskip 8\p@ plus2\p@ minus4\p@
\belowdisplayskip \abovedisplayskip
\abovedisplayshortskip \z@ plus\p@ 
\belowdisplayshortskip 4\p@ plus2\p@ minus2\p@
\def\@listi{\leftmargin\leftmargini
\topsep 4\p@ plus2\p@ minus2\p@\parsep 2\p@ plus\p@ minus\p@
\itemsep \parsep}}

\renewcommand{\scriptsize}{\fontsize\@viiipt{9.5pt}\selectfont}
\renewcommand{\tiny}{\fontsize\@vipt{7pt}\selectfont}
\renewcommand{\large}{\fontsize\@xiipt{14pt}\selectfont}
\renewcommand{\Large}{\fontsize\@xivpt{18pt}\selectfont}
\renewcommand{\LARGE}{\fontsize\@xviipt{22pt}\selectfont}
\renewcommand{\huge}{\fontsize\@xxpt{25pt}\selectfont}
\renewcommand{\Huge}{\fontsize\@xxvpt{30pt}\selectfont}

%</11pt>
%    \end{macrocode}
% \file{style/mu/fit12.clo}
% This file is conditionally loaded by the
% \texttt{style/mu/base.sty} file to redefine the page geometry to
% match the type size of 12 points. The type dimensions defined by
% the file are stored in the following private macros as well:
% \begin{itemize}
%  \item\DescribeMacro{\thesis@titlePage@normalsize}%
%    |\thesis@titlePage@normalsize| -- Equivalent to |\normalsize|
%  \item\DescribeMacro{\thesis@titlePage@small}%
%    |\thesis@titlePage@small| -- Equivalent to |\small|
%  \item\DescribeMacro{\thesis@titlePage@footnotesize}%
%    |\thesis@titlePage@footnotesize| -- Equivalent to
%    |\footnotesize|
%  \item\DescribeMacro{\thesis@titlePage@scriptsize}%
%    |\thesis@titlePage@scriptsize| -- Equivalent to |\scriptsize|
%  \item\DescribeMacro{\thesis@titlePage@tiny}%
%    |\thesis@titlePage@tiny| -- Equivalent to |\tiny|
%  \item\DescribeMacro{\thesis@titlePage@large}%
%    |\thesis@titlePage@large| -- Equivalent to |\large|
%  \item\DescribeMacro{\thesis@titlePage@Large}%
%    |\thesis@titlePage@Large| -- Equivalent to |\Large|
%  \item\DescribeMacro{\thesis@titlePage@LARGE}%
%    |\thesis@titlePage@LARGE| -- Equivalent to |\LARGE|
%  \item\DescribeMacro{\thesis@titlePage@huge}%
%    |\thesis@titlePage@huge| -- Equivalent to |\huge|
%  \item\DescribeMacro{\thesis@titlePage@Huge}%
%    |\thesis@titlePage@Huge| -- Equivalent to |\Huge|
% \end{itemize}
% These private macros can be used to typeset elements, whose size
% should remain constant regardless of the font size setting.
%    \begin{macrocode}
%<*12pt>
\ProvidesFile{fit12.clo}[2015/04/08]
%</12pt>

\def\thesis@titlePage@normalsize{\fontsize\@xiipt{14.5}%
\selectfont\abovedisplayskip 12\p@ plus3\p@ minus7\p@
\belowdisplayskip \abovedisplayskip
\abovedisplayshortskip  \z@ plus3\p@
\belowdisplayshortskip  6.5\p@ plus3.5\p@ minus3\p@
\let\@listi\@listI}

\def\thesis@titlePage@small{\fontsize\@xipt{13.6}\selectfont%
\abovedisplayskip 11\p@ plus3\p@ minus6\p@
\belowdisplayskip \abovedisplayskip
\abovedisplayshortskip  \z@ plus3\p@
\belowdisplayshortskip  6.5\p@ plus3.5\p@ minus3\p@
\def\@listi{\leftmargin\leftmargini %% Added 22 Dec 87
\parsep 4.5\p@ plus2\p@ minus\p@
            \itemsep \parsep
            \topsep 9\p@ plus3\p@ minus5\p@}}

\def\thesis@titlePage@footnotesize{\fontsize\@xpt{12}\selectfont%
\abovedisplayskip 10\p@ plus2\p@ minus5\p@
\belowdisplayskip \abovedisplayskip
\abovedisplayshortskip  \z@ plus3\p@
\belowdisplayshortskip  6\p@ plus3\p@ minus3\p@
\def\@listi{\leftmargin\leftmargini %% Added 22 Dec 87
\topsep 6\p@ plus2\p@ minus2\p@\parsep 3\p@ plus2\p@ minus\p@
\itemsep \parsep}}
            
\def\thesis@titlePage@scriptsize{\fontsize\@viiipt{9.5pt}\selectfont}
\def\thesis@titlePage@tiny{\fontsize\@vipt{7pt}\selectfont}
\def\thesis@titlePage@large{\fontsize\@xivpt{18pt}\selectfont}
\def\thesis@titlePage@Large{\fontsize\@xviipt{22pt}\selectfont}
\def\thesis@titlePage@LARGE{\fontsize\@xxpt{25pt}\selectfont}
\def\thesis@titlePage@huge{\fontsize\@xxvpt{30pt}\selectfont}
\def\thesis@titlePage@Huge{\fontsize\@xxvpt{30pt}\selectfont}

%<*12pt>
\renewcommand{\normalsize}{\thesis@titlePage@normalsize}
\renewcommand{\small}{\thesis@titlePage@small}
\renewcommand{\footnotesize}{\thesis@titlePage@footnotesize}
\renewcommand{\scriptsize}{\thesis@titlePage@scriptsize}
\renewcommand{\tiny}{\thesis@titlePage@tiny}
\renewcommand{\large}{\thesis@titlePage@large}
\renewcommand{\Large}{\thesis@titlePage@Large}
\renewcommand{\LARGE}{\thesis@titlePage@LARGE}
\renewcommand{\huge}{\thesis@titlePage@huge}
\renewcommand{\Huge}{\thesis@titlePage@Huge}
%</12pt>
\let\@normalsize\normalsize
\normalsize

\if@twoside               
   \oddsidemargin 0.75in  
   \evensidemargin 0.4in  
   \marginparwidth 0pt    
\else                     
   \oddsidemargin 0.75in  
   \evensidemargin 0.75in
   \marginparwidth 0pt
\fi
\marginparsep 10pt        

\topmargin 0.4in          
                          
\headheight 20pt          
\headsep 10pt             
\topskip 10pt    
\footskip 30pt 

%<*10pt>
\textheight = 43\baselineskip
\advance\textheight by \topskip
\textwidth 5.0truein
\columnsep 10pt       
\columnseprule 0pt

\footnotesep 6.65pt
\skip\footins 9pt plus 4pt minus 2pt
\floatsep 12pt plus 2pt minus 2pt
\textfloatsep 20pt plus 2pt minus 4pt
\intextsep 12pt plus 2pt minus 2pt
\dblfloatsep 12pt plus 2pt minus 2pt
\dbltextfloatsep 20pt plus 2pt minus 4pt

\@fptop 0pt plus 1fil
\@fpsep 8pt plus 2fil
\@fpbot 0pt plus 1fil
\@dblfptop 0pt plus 1fil
\@dblfpsep 8pt plus 2fil
\@dblfpbot 0pt plus 1fil
\marginparpush 5pt

\parskip 0pt plus 1pt
\partopsep 2pt plus 1pt minus 1pt

%</10pt>
%
%<*11pt>
\textheight = 39\baselineskip
\advance\textheight by \topskip
\textwidth 5.0truein
\columnsep 10pt
\columnseprule 0pt

\footnotesep 7.7pt
\skip\footins 10pt plus 4pt minus 2pt
\floatsep 12pt plus 2pt minus 2pt
\textfloatsep 20pt plus 2pt minus 4pt
\intextsep 12pt plus 2pt minus 2pt
\dblfloatsep 12pt plus 2pt minus 2pt
\dbltextfloatsep 20pt plus 2pt minus 4pt

\@fptop 0pt plus 1fil
\@fpsep 8pt plus 2fil
\@fpbot 0pt plus 1fil
\@dblfptop 0pt plus 1fil
\@dblfpsep 8pt plus 2fil
\@dblfpbot 0pt plus 1fil
\marginparpush 5pt 

\parskip 0pt plus 0pt
\partopsep 3pt plus 1pt minus 2pt

%</11pt>
%
%<*12pt>
\textheight = 37\baselineskip
\advance\textheight by \topskip
\textwidth 5.0truein
\columnsep 10pt
\columnseprule 0pt

\footnotesep 8.4pt
\skip\footins 10.8pt plus 4pt minus 2pt
\floatsep 14pt plus 2pt minus 4pt 
\textfloatsep 20pt plus 2pt minus 4pt
\intextsep 14pt plus 4pt minus 4pt
\dblfloatsep 14pt plus 2pt minus 4pt
\dbltextfloatsep 20pt plus 2pt minus 4pt

\@fptop 0pt plus 1fil
\@fpsep 10pt plus 2fil
\@fpbot 0pt plus 1fil
\@dblfptop 0pt plus 1fil
\@dblfpsep 10pt plus 2fil
\@dblfpbot 0pt plus 1fil
\marginparpush 7pt

\parskip 0pt plus 0pt
\partopsep 3pt plus 2pt minus 2pt

%</12pt>
\@lowpenalty   51
\@medpenalty  151
\@highpenalty 301
\@beginparpenalty -\@lowpenalty
\@endparpenalty   -\@lowpenalty
\@itempenalty     -\@lowpenalty

\def\@makechapterhead#1{%
  {%
    \setlength\parindent{\z@}%
    \setlength\parskip  {\z@}%
    \ifnum
      \c@secnumdepth >\m@ne
      \par\nobreak
      \vskip 10\p@
    \fi
    \Large \ChapFont \thechapter{} \space #1\par
    \nobreak
    \vskip 20\p@
  }%
}

\def\@makeschapterhead#1{%
  {%
    \setlength\parindent{\z@}%
    \setlength\parskip  {\z@}%
    \Large \ChapFont #1\par
    \nobreak
    \vskip 20\p@
  }%
}

\def\chapter{%
   \clearpage
   \thispagestyle{plain}
   \global\@topnum\z@ 
   \@afterindentfalse  
   \secdef\@chapter\@schapter
 }

\def\@chapter[#1]#2{%
  \ifnum \c@secnumdepth
    >\m@ne
    \refstepcounter{chapter}%
    \typeout{\@chapapp\space\thechapter.}% 
    \addcontentsline{toc}{chapter}{\protect
    \numberline{\thechapter}\bfseries #1}
  \else%
    \addcontentsline{toc}{chapter}{\bfseries #1}
  \fi
  \chaptermark{#1}%
  \addtocontents{lof}%
  {\protect\addvspace{4\p@}} 
  \addtocontents{lot}%
  {\protect\addvspace{4\p@}} 
  \if@twocolumn                   
    \@topnewpage[\@makechapterhead{#2}]%
  \else
    \@makechapterhead{#2}%
    \@afterheading          
  \fi
}

%\def\@schapter#1{\if@twocolumn \@topnewpage[\@makeschapterhead{#1}]%
%        \else \@makeschapterhead{#1}%
%              \markright{#1}
%              \@afterheading\fi}

\def\section{\@startsection {section}{1}{\z@}{-3.5ex plus-1ex minus
    -.2ex}{2.3ex plus.2ex}{\reset@font\large\bfseries}}
\def\subsection{\@startsection{subsection}{2}{\z@}{-3.25ex plus-1ex
    minus-.2ex}{1.5ex plus.2ex}{\reset@font\normalsize\bfseries}}
\def\subsubsection{\@startsection{subsubsection}{3}{\z@}{-3.25ex plus   
    -1ex minus-.2ex}{1.5ex plus.2ex}{\reset@font\normalsize}}
\def\paragraph{\@startsection
    {paragraph}{4}{\z@}{3.25ex plus1ex minus.2ex}{-1em}{\reset@font
    \normalsize\bfseries}}
\def\subparagraph{\@startsection
     {subparagraph}{4}{\parindent}{3.25ex plus1ex minus
     .2ex}{-1em}{\reset@font\normalsize\bfseries}}

\setcounter{secnumdepth}{2}

\def\appendix{\par
  \setcounter{chapter}{0}%
  \setcounter{section}{0}%
  \def\@chapapp{\appendixname}%
  \def\thechapter{\Alph{chapter}}}

\leftmargini 2.5em
\leftmarginii 2.2em     % > \labelsep + width of '(m)'
\leftmarginiii 1.87em   % > \labelsep + width of 'vii.'
\leftmarginiv 1.7em     % > \labelsep + width of 'M.'
\leftmarginv 1em
\leftmarginvi 1em

\leftmargin\leftmargini
\labelsep .5em
\labelwidth\leftmargini\advance\labelwidth-\labelsep

%<*10pt>
\def\@listI{\leftmargin\leftmargini \parsep 4\p@ plus2\p@ minus\p@%
\topsep 8\p@ plus2\p@ minus4\p@
\itemsep 4\p@ plus2\p@ minus\p@}

\let\@listi\@listI
\@listi

\def\@listii{\leftmargin\leftmarginii
   \labelwidth\leftmarginii\advance\labelwidth-\labelsep
   \topsep 4\p@ plus2\p@ minus\p@
   \parsep 2\p@ plus\p@ minus\p@
   \itemsep \parsep}

\def\@listiii{\leftmargin\leftmarginiii
    \labelwidth\leftmarginiii\advance\labelwidth-\labelsep
    \topsep 2\p@ plus\p@ minus\p@
    \parsep \z@ \partopsep\p@ plus\z@ minus\p@
    \itemsep \topsep}

\def\@listiv{\leftmargin\leftmarginiv
     \labelwidth\leftmarginiv\advance\labelwidth-\labelsep}
   
\def\@listv{\leftmargin\leftmarginv
     \labelwidth\leftmarginv\advance\labelwidth-\labelsep}
   
\def\@listvi{\leftmargin\leftmarginvi
     \labelwidth\leftmarginvi\advance\labelwidth-\labelsep}
%</10pt>
%
%<*11pt>
\def\@listI{\leftmargin\leftmargini \parsep 4.5\p@ plus2\p@ minus\p@
\topsep 9\p@ plus3\p@ minus5\p@
\itemsep 4.5\p@ plus2\p@ minus\p@}

\let\@listi\@listI
\@listi

\def\@listii{\leftmargin\leftmarginii
   \labelwidth\leftmarginii\advance\labelwidth-\labelsep
   \topsep 4.5\p@ plus2\p@ minus\p@
   \parsep 2\p@ plus\p@ minus\p@
   \itemsep \parsep}

\def\@listiii{\leftmargin\leftmarginiii
    \labelwidth\leftmarginiii\advance\labelwidth-\labelsep
    \topsep 2\p@ plus\p@ minus\p@
    \parsep \z@ \partopsep \p@ plus\z@ minus\p@
    \itemsep \topsep}

\def\@listiv{\leftmargin\leftmarginiv
     \labelwidth\leftmarginiv\advance\labelwidth-\labelsep}
   
\def\@listv{\leftmargin\leftmarginv
     \labelwidth\leftmarginv\advance\labelwidth-\labelsep}
    
\def\@listvi{\leftmargin\leftmarginvi
     \labelwidth\leftmarginvi\advance\labelwidth-\labelsep}
%</11pt>
%
%<*12pt>
\def\@listI{\leftmargin\leftmargini \parsep 5\p@ plus2.5\p@ minus\p@
\topsep 10\p@ plus4\p@ minus6\p@
\itemsep 5\p@ plus2.5\p@ minus\p@}

\let\@listi\@listI
\@listi

\def\@listii{\leftmargin\leftmarginii
   \labelwidth\leftmarginii\advance\labelwidth-\labelsep
   \topsep 5\p@ plus2.5\p@ minus\p@
   \parsep 2.5\p@ plus\p@ minus\p@
   \itemsep \parsep}

\def\@listiii{\leftmargin\leftmarginiii
    \labelwidth\leftmarginiii\advance\labelwidth-\labelsep
    \topsep 2.5\p@ plus\p@ minus\p@
    \parsep \z@ \partopsep \p@ plus\z@ minus\p@
    \itemsep \topsep}

\def\@listiv{\leftmargin\leftmarginiv
     \labelwidth\leftmarginiv\advance\labelwidth-\labelsep}
   
\def\@listv{\leftmargin\leftmarginv
     \labelwidth\leftmarginv\advance\labelwidth-\labelsep}
    
\def\@listvi{\leftmargin\leftmarginvi
     \labelwidth\leftmarginvi\advance\labelwidth-\labelsep}
%</12pt>
%</opt>
%    \end{macrocode}
