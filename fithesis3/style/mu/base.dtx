% \iffalse
%<*base>
% \fi\file{style/mu/fithesis3-base.sty}
% This is the base style file for theses written at the Masaryk
% University in Brno.
% If inheritance is enabled for style files, then this file is
% always the second style file to be loaded right after
% \texttt{style/base.sty}, regardless of the
% value of the |\thesis@style| macro.
%    \begin{macrocode}
\ProvidesPackage{fithesis3/style/mu/fithesis3-base}[2015/05/30]
\NeedsTeXFormat{LaTeX2e}
%    \end{macrocode}
% The file recognizes the following options: \begin{itemize}
%   \item\texttt{10pt}, \texttt{11pt}, \texttt{12pt} -- Sets the
%     type size to 10, 11 or 12 points respectively, along with the
%     page geometry. The default type size is 12 points.
%   \item\texttt{oneside}, \texttt{twoside} -- The document is
%     going to be single- or double-sided. In a double-sided
%     document headers, page numbering, margin notes and several
%     other elements are rearranged based on
%     the parity of the page. Blank pages are optionally inserted
%     prior the beginning of the next chapter to ensure that it
%     starts on an left-hand (even-numbered) page.
%
%     The \DescribeMacro{\ifthesis@twoside@}|\ifthesis@twoside@|
%     conditional is set to either \texttt{false} or \texttt{true},
%     respectively. This value can be tested in the subsequently
%     loaded style files.
%   \item\texttt{onecolumn}, \texttt{twocolumn} -- The document
%     is going to be set in a single column or in two columns,
%     respectively.
%   \item\texttt{draft}, \texttt{final} -- Overful lines are or
%     aren't marked within the document, respectively.
%   \item\texttt{palatino}, \texttt{nopalatino} -- The default
%     roman and math font family is going to be either set to
%     Palatino or left untouched, respectively. The
%     \DescribeMacro{\ifthesis@palatino@}|\ifthesis@palatino@|
%     conditional is set to either \texttt{true} or \texttt{false},
%     respectively.  This value can be tested in the subsequently
%     loaded style files.
%
%     The Palatino font is a part of the visual identity of the
%     Faculty of Informatics, at which the document class was
%     created.
%   \item\texttt{color}, \texttt{monochrome} -- Certain
%     typographical elements either are or aren't going to be
%     typeset in color, respectively. The
%     \DescribeMacro{\ifthesis@color@}|\ifthesis@color@|
%     conditional is set to either \texttt{true} or \texttt{false},
%     respectively.  This value can be tested in the subsequently
%     loaded style files.
%   \item\texttt{table}, \texttt{oldtable} -- If the
%     |\ifthesis@color@| conditional is \texttt{true}, then the
%     definition of the \texttt{tabular} and \texttt{tabularx}
%     commands either are or aren't going to be altered to better
%     match the style, respectively. The \DescribeMacro%
%     {\ifthesis@newtable@}|\ifthesis@newtable@| conditional is set
%     to either \texttt{true} or \texttt{false}, respectively.
%
%     The choice of the name is deliberate -- the redefinition of
%     the table environments depends on the \textsf{xcolor}
%     package, which needs to be loaded with the \texttt{table}
%     option. Since so many other packages depend on the
%     \textsf{xcolor} package and this style file is loaded at the
%     very end of the preamble, there would either be a great
%     chance of an option clash, or the option would have to be
%     passed to the \textsf{xcolor} package before the preamble
%     from the body of the \textsf{fithesis3} class thus breaking
%     the encapsulation. Naming the option \texttt{table} forces
%     the option to be processed by the \textsf{xcolor} package as
%     well and it is therefore an elegant solution to the problem
%     at hand.
%   \item\texttt{lot}, \texttt{nolot} -- |\listoftables| is or
%     isn't going to be included in the
%     \DescribeMacro{\thesis@blocks@tables}|\thesis@blocks@tables|
%     block, respectively.
%   \item\texttt{lof}, \texttt{nolof} -- |\listoffigures| is or
%     isn't going to be included in the
%     \DescribeMacro{\thesis@blocks@tables}|\thesis@blocks@tables|
%     block, respectively.
%   \item\texttt{cover}, \texttt{nocover} -- The
%     \DescribeMacro{\thesis@blocks@cover}|\thesis@blocks@cover|
%     is or isn't going to expand to the thesis cover.
%
%     The \DescribeMacro{\ifthesis@cover@}|\ifthesis@cover@|
%     conditional is set to \texttt{false} or \texttt{true},
%     respectively. This value can be tested in the subsequently
%     loaded style files.
% \end{itemize}
%    \begin{macrocode}
\DeclareOption{10pt}{\def\thesis@ptsize{0}}
\DeclareOption{11pt}{\def\thesis@ptsize{1}}
\DeclareOption{12pt}{\def\thesis@ptsize{2}}
\newif\ifthesis@twoside@
\DeclareOption{oneside}{%
  \thesis@twoside@false\@twosidefalse\@mparswitchfalse}
\DeclareOption{twoside}{%
  \thesis@twoside@true \@twosidetrue \@mparswitchtrue}
\DeclareOption{onecolumn}{\@twocolumnfalse}
\DeclareOption{twocolumn}{\@twocolumntrue}
\DeclareOption{draft}{\setlength\overfullrule{5pt}}
\DeclareOption{final}{\setlength\overfullrule{0pt}}
\newif\ifthesis@color@
\DeclareOption{monochrome}{\thesis@color@false}
\DeclareOption{color}{\thesis@color@true}
\newif\ifthesis@palatino@
\DeclareOption{palatino}{\thesis@palatino@true}
\DeclareOption{nopalatino}{\thesis@palatino@false}
\newif\ifthesis@newtable@
\DeclareOption{table}{\thesis@newtable@true}
\DeclareOption{oldtable}{\thesis@newtable@false}
\DeclareOption{nolot}{\def\thesis@blocks@lot{}}
\DeclareOption{lot}{\let\thesis@blocks@lot\listoftables}
\DeclareOption{nolof}{\def\thesis@blocks@lof{}}
\DeclareOption{lof}{\let\thesis@blocks@lof\listoffigures}
\newif\ifthesis@cover@
\DeclareOption{nocover}{\thesis@cover@false}
\DeclareOption{cover}{\thesis@cover@true}
%    \end{macrocode}
% These are the default options:
%    \begin{macrocode}
\ExecuteOptions{12pt,twoside,final,monochrome,palatino,
  oldtable,lot,lof,nocover}
\ProcessOptions
%    \end{macrocode}
% The file loads the following packages: \begin{itemize}
%   \item\textsf{xcolor} -- Adds support for color manipulation.
%   \item\textsf{ifxetex} -- Used to detect the \Hologo{XeTeX}
%     engine.
%   \item\textsf{ifluatex} -- Used to detect the \Hologo{LuaTeX}
%     engine.
%   \item\textsf{graphix} -- Adds support for the inclusion of
%     graphics files.
%   \item\textsf{pdfpages} -- Adds support for the injection of PDF
%     documents into the resulting document, namely the thesis
%     assignment.
%   \item\textsf{hyperref} -- Adds support for injecting metadata
%     into the resulting PDF document.
%   \item\textsf{keyval} -- Adds support for parsing
%     comma-delimited lists of key-value pairs.
% \end{itemize}
%    \begin{macrocode}
\thesis@require{xcolor}
\thesis@require{graphicx}
\thesis@require{pdfpages}
\thesis@require{keyval}
\thesis@require{ifxetex}
\thesis@require{ifluatex}
%    \end{macrocode}
% Using the |\ifxetex| and |\ifluatex| conditionals, a compound
% \DescribeMacro{\ifthesis@xeluatex}|\ifthesis@xeluatex|
% conditional was constructed. This conditional can be used by
% subsequently loaded style files to test,whether either the
% \Hologo{XeTeX} or the \Hologo{LuaTeX} engine is being used.
%    \begin{macrocode}
{\let\x\expandafter
\x\global\x\let\x\ifthesis@xeluatex\csname if%
  \ifxetex true\else
    \ifluatex\x\x\x t\x\x\x r\x\x\x u\x\x\x e%
    \else   f\x\x\x a\x\x\x l\x\x\x s\x\x\x e%
    \fi
  \fi\endcsname}
%    \end{macrocode}
% The following packages get only loaded, when the document is
% being typeset using the \Hologo{XeTeX} or \Hologo{LuaTeX}
% engine: \begin{itemize}
%   \item\textsf{fontspec} -- Allows the selection of
%     system-installed fonts.
%   \item\textsf{unicode-math} -- Allows the selection of
%     system-installed mathematical fonts.
% \end{itemize}
% Under \Hologo{XeTeX}, the \textsf{TeX Gyre Pagella} and
% \textsf{TeX Gyre Pagella Math} are also selected as the main text
% and math fonts.
%    \begin{macrocode}
\ifthesis@xeluatex
  \ifthesis@palatino@
    \thesis@require{fontspec}
    \thesis@require{unicode-math}
    \setmainfont[Ligatures=TeX]{TeX Gyre Pagella}
    \setmathfont[math-style=ISO,bold-style=ISO,vargreek-shape=TeX]%
      {TG Pagella Math}
  \fi
%    \end{macrocode}
% The following packages get only loaded, when the document is
% not being typeset using the \Hologo{XeTeX} engine and the
% |\ifthesis@palatino@| conditional is \texttt{true}:
% \begin{itemize}
%   \item\textsf{cmap} -- Places an an explicit \texttt{ToUnicode}
%     map in the resulting PDF file, allowing for extraction of
%     the text of the document.
%   \item\textsf{mathpazo} -- The virtual \texttt{mathpazo} fonts
%     will be used for math.
%   \item\textsf{tgpagella} -- Changes the default roman font
%     family to \TeX\ Gyre Pagella.
%   \item\textsf{lmodern} -- Changes the default sans-serif and
%     monotype font faces to Latin Modern instead of the default
%     Computer Modern font family.
%   \item\textsf{fontenc} -- The font encoding is set to Cork.
% \end{itemize}
%    \begin{macrocode}
\else
  \ifthesis@palatino@
    \RequirePackage[resetfonts]{cmap}
    \thesis@require{lmodern}
    \thesis@require{mathpazo}
    \thesis@require{tgpagella}
    \RequirePackage[T1]{fontenc}
  \fi
\fi
%    \end{macrocode}
% If the |\thesis@newtable@| and |\thesis@color@| conditionals are
% \texttt{true}, then the following package gets loaded:
% \begin{itemize}
%   \item\textsf{tabularx} -- Provides the \texttt{tabularx}
%     environment, which enables the typesetting of tables with
%     variable-width columns.
% \end{itemize}
% Subsequently, the \texttt{tabular} and \texttt{tabularx}
% environments are redefined to better match the style of the given
% faculty.
% 
% The |\thesis@newtable@old| and |\endthesis@newtable@old| macros
% containing the original definition of the |tabular| environment
% are always defined and are available for subsequently loaded
% styles in case the typesetting of unaltered tables is required.
%    \begin{macrocode}
\let\thesis@newtable@old\tabular
\let\endthesis@newtable@old\endtabular
\ifthesis@newtable@\ifthesis@color@
  \thesis@require{tabularx}
  \thesis@require{booktabs}
  % The redefinition of `tabular`
  \renewenvironment{tabular}%
    {\rowcolors{1}{thesis@color@tableOdd}%
                  {thesis@color@tableEven}%
     \thesis@newtable@old}%
    {\endthesis@newtable@old}
  % The redefinition of `tabularx`
  \let\thesis@newtable@oldx\tabularx
  \let\endthesis@newtable@oldx\endtabularx
  \renewenvironment{tabularx}%
    {\rowcolors{1}{thesis@color@tableOdd}%
                  {thesis@color@tableEven}%
     \thesis@newtable@oldx}%
    {\endthesis@newtable@oldx}
  % Adjust the measurements
  \setlength{\aboverulesep}{0pt}
  \setlength{\belowrulesep}{0pt}
  \setlength{\extrarowheight}{.75ex}
\fi\fi
%    \end{macrocode}
% The \textsf{hyperref} package is configured to support both roman
% and arabic page numbering in one document and to decorate
% hyperlinks with an underline instead of a rectangular box.
%    \begin{macrocode}
\thesis@require{hyperref}
\hypersetup{
  pdfborderstyle={/S/U/W 1}, % Less obtrusive borders
  plainpages=false,          % Multiple page numbering support
  pdfpagelabels              % Generate pdf page labels
}
%    \end{macrocode}
% \begin{macro}{\thesis@color@setup}
% The file defines the |\thesis@color@setup|\marg{colors} command,
% where \textit{colors} is a comma-delimited list of key-value
% pairs as defined by the \textsf{keyval} package. The command can
% be invoked either by the subsequently loaded style files or by
% the user to define which colors will be used, when the
% \texttt{color} option is specified.
%    \begin{macrocode}
\def\thesis@color@setup#1{%
  \setkeys{thesis@color}{#1}}
%    \end{macrocode}
% \end{macro}
% The following key-value pairs are supported:
% \begin{enumerate}
%   \item\marg{\texttt{links}=color} -- Sets the color of hyperref
%     links to \textit{color} and stores it under the name
%     \texttt{thesis@color@links}. The default color of links is
%     specified by the \textsf{hyperref} package.
%   \item\marg{\texttt{tableOdd}=color} -- Stores the color of the
%     odd rows of the redefined \texttt{tabular} and
%     \texttt{tabularx} environments under the name
%     \texttt{thesis@color@tableOdd}.
%   \item\marg{\texttt{tableEven}=color} -- Stores the color of the
%     even rows of the redefined \texttt{tabular} and
%     \texttt{tabularx} environments under the name
%     \texttt{thesis@color@tableEven}.
%   \item\marg{\texttt{tableEmph}=color} -- Stores the color of an
%     emphasized cell in the table for usage by the user under the
%     name \texttt{thesis@color@tableEmph}.
% \end{enumerate}
%    \begin{macrocode}
\define@key{thesis@color}{links}{%
  \definecolor{thesis@color@links}#1
  \hypersetup{linkbordercolor=thesis@color@links}}
\define@key{thesis@color}{tableOdd}{%
  \definecolor{thesis@color@tableOdd}#1}
\define@key{thesis@color}{tableEven}{%
  \definecolor{thesis@color@tableEven}#1}
\define@key{thesis@color}{tableEmph}{%
  \definecolor{thesis@color@tableEmph}#1}
%    \end{macrocode}
% The file defines several blocks to be used in the redefinitions
% of the |\thesis@preamble| and |\thesis@postable| private macros
% by the subsequently loaded style files.
%
% \begin{macro}{\thesis@blocks@frontMatter}
% The |\thesis@blocks@frontMatter| private macro sets up the style
% of the front matter of the thesis.
% \begin{macrocode}
\def\thesis@blocks@frontMatter{%
  \thesis@blocks@clear
  \pagestyle{plain}
  \parindent 1.5em
  \setcounter{page}{1}
  \pagenumbering{roman}}
%    \end{macrocode}
% \end{macro}\begin{macro}{\thesis@blocks@clear}
% The |\thesis@blocks@clear| private macro clears the current page
% along with the next left-handed (even-numbered) page, when
% double-sided typesetting is enabled.
% \begin{macrocode}
\def\thesis@blocks@clear{%
  \ifthesis@twoside@%
    \clearpage%
    \thispagestyle{empty}%
    \cleardoublepage%
  \else%
    \newpage%
  \fi}
%    \end{macrocode}
% \end{macro}\begin{macro}{\thesis@blocks@clearRight}
% The |\thesis@blocks@clearRight| private macro clears the current
% page along with the next right-handed (odd-numbered) page, when
% double-sided typesetting is enabled.
% \begin{macrocode}
\def\thesis@blocks@clearRight{%
  \ifthesis@twoside@%
    \clearpage%
    \ifodd\value{page}%
      \thispagestyle{empty}%
      \hbox{}%
      \newpage%
    \fi%
  \else%
    \newpage%
  \fi}
%    \end{macrocode}
% \end{macro}\begin{macro}{\thesis@blocks@facultyLogo@monochrome}
% The |\thesis@blocks@facultyLogo@monochrome|\oarg{options} private
% macro typesets the |\thesis@logopath\thesis@facultyLogo| logo
% with the given \textit{options} passed to |\includegraphics|.
% \begin{macrocode}
\newcommand{\thesis@blocks@facultyLogo@monochrome}[1]%
  [width=40mm]{{%
    \edef\@path{\thesis@logopath\thesis@facultyLogo}%
    \includegraphics[#1]{\@path}}}
%    \end{macrocode}
% \end{macro}\begin{macro}{\thesis@blocks@facultyLogo@color}
% The |\thesis@blocks@facultyLogo@color|\oarg{options} private
% macro typesets either the |\thesis@logopath\thesis@facultyLogo|
% logo, if the |\ifthesis|\discretionary{}{}{}|@color@| conditional
% is \texttt{false} or the
% |\thesis@logopath color/\thesis@facultyLogo| logo
% otherwise with the given \textit{options} passed to
% |\includegraphics|.
% \begin{macrocode}
\newcommand{\thesis@blocks@facultyLogo@color}[1]%
  [width=40mm]{{%
    \edef\@path{\thesis@logopath\ifthesis@color@%
      color/\fi\thesis@facultyLogo}%
    \includegraphics[#1]{\@path}}}
%    \end{macrocode}
% \end{macro}\begin{macro}{\thesis@blocks@universityLogo@monochrome}
% The |\thesis@blocks@universityLogo@monochrome|\oarg{options}
% private macro typesets the
% |\thesis@logopath\thesis@universityLogo| logo
% with the given \textit{options} passed to |\includegraphics|.
% \begin{macrocode}
\newcommand{\thesis@blocks@universityLogo@monochrome}[1]%
  [width=40mm]{{%
    \edef\@path{\thesis@logopath\thesis@universityLogo}%
    \includegraphics[#1]{\@path}}}
%    \end{macrocode}
% \end{macro}\begin{macro}{\thesis@blocks@universityLogo@color}
% The |\thesis@blocks@universityLogo@color|\oarg{options} private
% macro typesets either the |\thesis@logopath\thesis@universityLogo|
% logo, if the |\ifthesis|\discretionary{}{}{}|@color@| conditional
% is \texttt{false} or the
% |\thesis@logopath color/\thesis@universityLogo| logo otherwise
% with the given \textit{options} passed to |\includegraphics|.
% \begin{macrocode}
\newcommand{\thesis@blocks@universityLogo@color}[1]%
  [width=40mm]{{%
    \edef\@path{\thesis@logopath\ifthesis@color@%
      color/\fi\thesis@universityLogo}%
    \includegraphics[#1]{\@path}}}
%    \end{macrocode}
% \end{macro}\begin{macro}{\thesis@blocks@cover}
% The |\thesis@blocks@cover| private macro typesets the thesis
% cover. It is composed of three private macros:
% \begin{itemize}
%   \item\DescribeMacro{\thesis@blocks@cover@header}^^A
%        |\thesis@blocks@cover@header| -- The header of the cover
%        page
%   \item\DescribeMacro{\thesis@blocks@cover@content}^^A
%        |\thesis@blocks@cover@content| -- The content of the cover
%        page
%   \item\DescribeMacro{\thesis@blocks@cover@footer}^^A
%        |\thesis@blocks@cover@footer| -- The footer of the cover
%        page
% \end{itemize}
% This allows the subsequently loaded style files to only redefine
% certain parts of the cover page.
% \begin{macrocode}
\def\thesis@blocks@cover{%
  \ifthesis@cover@%
    \thesis@blocks@clear%
    \ifthesis@twoside@\@twosidefalse\fi % Temporarily disable twoside
    \begin{alwayssingle}%
      \thispagestyle{empty}%
      \begin{center}%
        \thesis@blocks@cover@header%
        \thesis@blocks@facultyLogo@monochrome\\[0.4in]%
        \let\footnotesize\small%
        \let\footnoterule\relax{}%
        \thesis@blocks@cover@content%
        \par\vfill%
        \thesis@blocks@cover@footer%
      \end{center}%
    \end{alwayssingle}%
    \ifthesis@twoside@\@twosidetrue\fi % Re-enable twoside
  \fi}
%    \end{macrocode}
% The output of the |\thesis@blocks@cover@header| private macro is
% controlled by the following conditionals:
% \begin{enumerate}
%   \item|\ifthesis@blocks@cover@university@| -- This
%        conditional expression determines, whether the university
%        name is going to be included in the header of the cover.
%        The default value of this conditional is \texttt{true}.
%   \item|\ifthesis@blocks@cover@faculty@| -- This
%        conditional expression determines, whether the faculty
%        name is going to be included in the header of the cover.
%        The default value of this conditional is \texttt{true}.
%   \item|\ifthesis@blocks@cover@department@| -- This
%        conditional expression determines, whether the department
%        name is going to be included in the header of the cover.
%        The default value of this conditional is \texttt{false}.
%   \item|\ifthesis@blocks@cover@field@| -- This
%        conditional expression determines, whether the field of
%        study is going to be included in the header of the cover.
%        The default value of this conditional is \texttt{false}.
% \end{enumerate}
% The sebsequently loaded style files can alter the value of these
% conditionals to alter the output of the
% |\thesis@blocks@cover@header| private macro without altering its
% definition.
% \begin{macrocode}
\newif\ifthesis@blocks@cover@university@
\thesis@blocks@cover@university@true
\newif\ifthesis@blocks@cover@faculty@
\thesis@blocks@cover@faculty@true
\newif\ifthesis@blocks@cover@department@
\thesis@blocks@cover@department@false
\newif\ifthesis@blocks@cover@field@
\thesis@blocks@cover@field@false

\def\thesis@blocks@cover@header{%
  {\sc\ifthesis@blocks@cover@university@%
        \thesis@titlePage@LARGE\thesis@@{universityName}\\%
   \fi\ifthesis@blocks@cover@faculty@%
        \thesis@titlePage@Large\thesis@@{facultyName}\\%
   \fi\ifthesis@blocks@cover@department@%
        \thesis@titlePage@large\thesis@department\\%
      \fi}
  \ifthesis@blocks@cover@field@%
    {\thesis@titlePage@large\vskip 1em%
      {\bf\thesis@@{fieldTitle}:} \thesis@field}%
  \fi\vskip 2em}
\def\thesis@blocks@cover@content{%
  {\thesis@titlePage@Huge\bf\thesis@TeXtitle\par\vfil}%
  \vskip 0.8in%
  {\thesis@titlePage@large\sc\thesis@@{typeName}}\\[0.3in]%
  {\thesis@titlePage@Large\bf\thesis@author}}
\def\thesis@blocks@cover@footer{%
  {\thesis@titlePage@large\thesis@place, \thesis@@{semester}}}
%    \end{macrocode}
% \end{macro}
% \begin{macro}{\thesis@blocks@titlePage}
% The |\thesis@blocks@titlePage| private macro typesets the thesis
% title page. It is composed of three private macros:
% \begin{itemize}
%   \item|\thesis@blocks@titlePage@header| -- The header of the
%        cover page
%   \item|\thesis@blocks@titlePage@content| -- The content of the
%        cover page
%   \item|\thesis@blocks@titlePage@footer| -- The footer of the
%        cover page
% \end{itemize}
% This allows the subsequently loaded style files to only redefine
% certain parts of the title page. Depending on the value of the
% |\ifthesis@color@| conditional, the faculty logo is loaded from
% either |\thesis@logopath|, if \texttt{false}, or from
% |\thesis@logopath color/|, if \texttt{true}.
%    \begin{macrocode}
\def\thesis@blocks@titlePage{%
    \thesis@blocks@clear%
    \begin{alwayssingle}%
      \thispagestyle{empty}%
      \begin{center}%
        \thesis@blocks@titlePage@header%
        \thesis@blocks@facultyLogo@color\\[0.4in]%
        \let\footnotesize\small%
        \let\footnoterule\relax{}%
        \thesis@blocks@titlePage@content%
        \par\vfill%
        \thesis@blocks@titlePage@footer%
      \end{center}%
    \end{alwayssingle}}
%    \end{macrocode}
% The output of the |\thesis@blocks@titlePage@header| private macro is
% controlled by the following conditionals:
% \begin{enumerate}
%   \item|\ifthesis@blocks@titlePage@university@| -- This
%        conditional expression determines, whether the university
%        name is going to be included in the header of the title
%        page. The default value of this conditional is
%        \texttt{true}.
%   \item|\ifthesis@blocks@titlePage@faculty@| -- This
%        conditional expression determines, whether the faculty of
%        study is going to be included in the header of the title
%        page. The default value of this conditional is
%        \texttt{true}.
%   \item|\ifthesis@blocks@titlePage@department@| -- This
%        conditional expression determines, whether the department
%        name is going to be included in the header of the title
%        page. The default value of this conditional is
%        \texttt{false}.
%   \item|\ifthesis@blocks@titlePage@field@| -- This
%        conditional expression determines, whether the field of
%        study is going to be included in the header of the title
%        page. The default value of this conditional is
%        \texttt{false}.
% \end{enumerate}
% The sebsequently loaded style files can alter the value of these
% conditionals to alter the output of the
% |\thesis@blocks@titlePage@header| private macro without altering
% its definition.
% \begin{macrocode}
\newif\ifthesis@blocks@titlePage@university@
\thesis@blocks@titlePage@university@true
\newif\ifthesis@blocks@titlePage@faculty@
\thesis@blocks@titlePage@faculty@true
\newif\ifthesis@blocks@titlePage@department@
\thesis@blocks@titlePage@department@false
\newif\ifthesis@blocks@titlePage@field@
\thesis@blocks@titlePage@field@false

\def\thesis@blocks@titlePage@header{%
  {\sc\ifthesis@blocks@titlePage@university@%
        \thesis@titlePage@LARGE\thesis@@{universityName}\\%
   \fi\ifthesis@blocks@titlePage@faculty@%
        \thesis@titlePage@Large\thesis@@{facultyName}\\%
   \fi\ifthesis@blocks@titlePage@department@%
        \thesis@titlePage@large\thesis@department\\%
      \fi}
  \ifthesis@blocks@titlePage@field@%
    {\thesis@titlePage@large\vskip 1em%
      {\bf\thesis@@{fieldTitle}:} \thesis@field}%
  \fi\vskip 2em}
\let\thesis@blocks@titlePage@content=\thesis@blocks@cover@content
\let\thesis@blocks@titlePage@footer=\thesis@blocks@cover@footer
%    \end{macrocode}
% \end{macro}\begin{macro}{\thesis@blocks@tables}
% The |\thesis@blocks@tables| private macro typesets the table of
% contents and optionally the |\listoftables| and the
% |\listoffigures|.
% \begin{macrocode}
\def\thesis@blocks@tables{%
  \thesis@blocks@clear%
  \tableofcontents%
  \thesis@blocks@lot%
  \thesis@blocks@lof}
%    \end{macrocode}
% \end{macro}\begin{macro}{\thesis@blocks@declaration}
% The |\thesis@blocks@declaration| private macro typesets the
% declaration text.
% \begin{macrocode}
\def\thesis@blocks@declaration{%
  \thesis@blocks@clear%
  \begin{alwayssingle}%
    \chapter*{\thesis@@{declarationTitle}}%
    \thesis@declaration%
    \vskip 2cm%
    \hfill\thesis@author%
  \end{alwayssingle}}
%    \end{macrocode}
% \end{macro}\begin{macro}{\thesis@blocks@thanks}
% The |\thesis@blocks@thanks| private macro typesets the
% acknowledgement, if the |\thesis@thanks| private macro is
% defined.
% \begin{macrocode}
\def\thesis@blocks@thanks{%
  \ifx\thesis@thanks\undefined\else%
    \thesis@blocks@clear%
    \begin{alwayssingle}%
      \chapter*{\vspace*{\fill}\thesis@@{thanksTitle}}%
      \thesis@thanks%
    \end{alwayssingle}%
  \fi}
%    \end{macrocode}
% \end{macro}\begin{macro}{\thesis@blocks@abstract}
% The |\thesis@blocks@abstract| private macro typesets the
% abstract.
% \begin{macrocode}
\def\thesis@blocks@abstract{%
  \begin{alwayssingle}%
    % Start the new chapter without clearing the right page
    {\def\cleardoublepage{}
    \chapter*{\thesis@@{abstractTitle}}}%
    \noindent\thesis@abstract%
    \par\vfil\null%
  \end{alwayssingle}}
%    \end{macrocode}
% \end{macro}\begin{macro}{\thesis@blocks@abstractEn}
% The |\thesis@blocks@abstractEn| private macro typesets the
% abstract in English. If the current locale is English, the
% macro produces no output. A style file that uses this block
% needs to require the English locale.
% \begin{macrocode}
\def\thesis@blocks@abstractEn{%
  \ifthesis@english\else%
    \begin{alwayssingle}%
      % Start the new chapter without clearing the right page
      {\def\cleardoublepage{}%
      \chapter*{\thesis@{english@abstractTitle}}%
      \thesis@abstractEn}%
      \par\vfil\null%
    \end{alwayssingle}%
  \fi}
%    \end{macrocode}
% \end{macro}\begin{macro}{\thesis@blocks@keywords}
% The |\thesis@blocks@keywords| private macro typesets the
% keywords.
% \begin{macrocode}
\def\thesis@blocks@keywords{%
  \begin{alwayssingle}%
    % Start the new chapter without clearing the right page
    {\def\cleardoublepage{}%
    \chapter*{\thesis@@{keywordsTitle}}%
    \noindent\thesis@TeXkeywords}%
    \par\vfil\null%
  \end{alwayssingle}}
%    \end{macrocode}
% \end{macro}\begin{macro}{\thesis@blocks@keywordsEn}
% The |\thesis@blocks@keywordsEn| private macro typesets the
% keywords in English. If the current locale is English, the
% macro produces no output. A style file that uses this block
% needs to require the English locale.
% \begin{macrocode}
\def\thesis@blocks@keywordsEn{%
  \ifthesis@english\else%
    \begin{alwayssingle}%
      % Start the new chapter without clearing the right page
      {\def\cleardoublepage{}%
      \chapter*{\thesis@{english@keywordsTitle}}%
      \thesis@TeXkeywordsEn}%
      \par\vfil\null%
    \end{alwayssingle}%
  \fi}
%    \end{macrocode}
% \end{macro}\begin{macro}{\thesis@blocks@assignment}
% The |\thesis@blocks@assignment| private macro either:
% \begin{itemize}
%   \item Expands to an empty token list, when the first
%     expansion of |\thesis@assignmentPDF| yields an empty token
%     list or when the |\ifthesis@blocks@assignment@| conditional
%     is \texttt{false}.
%   \item Typesets a placeholder page to be replaced with the
%     official thesis assignment, when |\thesis@assignmentPDF|
%     is undefined and the |\ifthesis@blocks|\discretionary{}{}^^A
%     {}|@assignment@| conditional is \texttt{true}.
%   \item Injects the PDF file(s) specified within the
%     |\thesis@assignmentPDF| macro, when |\thesis@assignmentPDF|
%     is defined, its first expansion yields a non-empty token list
%     and the |\ifthesis@blocks@assignment@| conditional is
%     \texttt{true}.
% \end{itemize}
% The default value of the \DescribeMacro{\ifthesis@blocks@assig^^A
% nment@}|\ifthesis@blocks@assignment@| conditional is
% \texttt{true}.
% \begin{macrocode}
\newif\ifthesis@blocks@assignment@
\thesis@blocks@assignment@true
\def\thesis@blocks@assignment{%
  \ifthesis@blocks@assignment@%
    \ifx\thesis@assignmentFiles\undefined%
      % Typeset a placeholder page.
      \thesis@blocks@clear%
      \begin{alwayssingle}%
        \thispagestyle{empty}
        \addtocounter{page}{-\ifthesis@twoside@2\else1\fi}
        \noindent\textit{\thesis@@{assignment}}%
      \end{alwayssingle}%
    \else%
      {\edef\@pdfList{\thesis@assignmentFiles}%
      \if\@pdfList\empty\else%
        % Inject the specified PDF documents.
        \thesis@blocks@clear%
        \expandafter\includepdfmerge\expandafter{\@pdfList}%
      \fi}%
    \fi%
  \fi}
%    \end{macrocode}
% \end{macro}\begin{macro}{\thesis@blocks@mainMatter}
% The |\thesis@blocks@mainMatter| private macro sets up the style
% of the main matter of the thesis.
% \begin{macrocode}
\def\thesis@blocks@mainMatter{%
  \thesis@blocks@clear
  \setcounter{page}{1}
  \pagenumbering{arabic}
  \pagestyle{thesisheadings}
  \parindent 1.5em\relax}
%    \end{macrocode}
% \end{macro}
% The rest of the file comprises redefinitions of \LaTeX\ commands
% and private \texttt{rapport3} class macros altering the layout of
% the resulting document. Depending on the type size of 10, 11 or
% 12 points, either the \texttt{fithesis3-10.clo},
% \texttt{fithesis3-11.clo} or \texttt{fithesis3-12.clo} file is
% loaded from the |\thesis@stylepath| |mu| directory, respectively.
%    \begin{macrocode}
% Table of contents will contain sectioning commands up to
% \subsubsection
\setcounter{tocdepth}{4}

% Load the `fithesis3-1*.clo` size option
\input\thesis@stylepath mu/fithesis3-1\thesis@ptsize.clo\relax

\def\ps@thesisheadings{%
\def\chaptermark##1{%
\markright{%
\ifnum\c@secnumdepth >\m@ne
\thechapter.\ %
\fi ##1}}
\let\@oddfoot\@empty
\let\@oddhead\@empty
\def\@oddhead{\vbox{\hbox to \textwidth{%
\hfil{\sc\rightmark}}\vskip 4pt\hrule}}
\if@twoside
 \def\@evenhead{\vbox{\hbox to \textwidth{%
 {\sc\rightmark}\hfil}\vskip 4pt\hrule}}
\else
 \let\@evenhead\@oddhead
\fi
\def\@oddfoot{\hfil\PageFont\thepage}
\if@twoside
 \def\@evenfoot{\PageFont\thepage\hfil}%
\else
 \let\@evenfoot\@oddfoot
\fi
\let\@mkboth\markboth
}

% Redefines the style of the chapter headings
\renewcommand*\chapter{%
\if@twoside
 \clearpage
 \thispagestyle{empty}
 \cleardoublepage
\else
 \clearpage
\fi
\thispagestyle{plain}%
\global\@topnum\z@
\@afterindentfalse
\secdef\@chapter\@schapter}

% Redefines the style of part headings
\renewcommand*\part{%
\clearpage
\thispagestyle{empty}
\cleardoublepage
\thispagestyle{empty}%
\if@twocolumn%
 \onecolumn
 \@tempswatrue
\else
 \@tempswafalse
\fi
\hbox{}\vfil
\secdef\@part\@spart}

\newif\if@restonecol
\def\alwayssingle{%
  \@restonecolfalse\if@twocolumn\@restonecoltrue\onecolumn\fi}
\def\endalwayssingle{\if@restonecol\twocolumn\fi}

\renewcommand*\l@part[2]{%
  \ifnum \c@tocdepth >-2\relax
    \addpenalty{-\@highpenalty}%
    \addvspace{0.5em \@plus\p@}%
    \begingroup
      \setlength\@tempdima{3em}%
      \parindent \z@ \rightskip \@pnumwidth
      \parfillskip -\@pnumwidth
      {\leavevmode
       \normalfont \bfseries #1\hfil \hb@xt@\@pnumwidth{\hss #2}}\par
       \nobreak
         \global\@nobreaktrue
         \everypar{\global\@nobreakfalse\everypar{}}%
    \endgroup
    \addvspace{0.2em \@plus\p@}%
  \fi}

\renewcommand*\l@chapter[2]{%
  \ifnum \c@tocdepth >\m@ne
    \addpenalty{-\@highpenalty}%
    \vskip 1.0em \@plus\p@
    \setlength\@tempdima{1.5em}%
    \begingroup
      \parindent \z@ \rightskip \@pnumwidth
      \parfillskip -\@pnumwidth
      \leavevmode \bfseries
      \advance\leftskip\@tempdima
      \hskip -\leftskip
      #1\nobreak\hfil \nobreak\hb@xt@\@pnumwidth{\hss #2}\par
      \penalty\@highpenalty
    \endgroup
  \fi}

\renewcommand*\l@chapter{\@dottedtocline{1}{0em}{1.5em}}
\renewcommand*\l@section{\@dottedtocline{2}{1.5em}{2.3em}}
\renewcommand*\l@subsection{\@dottedtocline{3}{3.8em}{3.2em}}
\renewcommand*\l@subsubsection{\@dottedtocline{4}{7.0em}{3.8em}}
%    \end{macrocode}\iffalse
%</base>
% \fi\file{style/mu/fit10.clo}
% This file is conditionally loaded by the
% \texttt{style/mu/base.sty} file to redefine the page geometry to
% match the type size of 10 points.
%    \begin{macrocode}
%<*opt>
%<*10pt>
\ProvidesFile{fit10.clo}[2015/04/08]

\renewcommand{\normalsize}{\fontsize\@xpt{12}\selectfont%
\abovedisplayskip 10\p@ plus2\p@ minus5\p@
\belowdisplayskip \abovedisplayskip
\abovedisplayshortskip  \z@ plus3\p@
\belowdisplayshortskip  6\p@ plus3\p@ minus3\p@
\let\@listi\@listI}

\renewcommand{\small}{\fontsize\@ixpt{11}\selectfont%
\abovedisplayskip 8.5\p@ plus3\p@ minus4\p@
\belowdisplayskip \abovedisplayskip
\abovedisplayshortskip \z@ plus2\p@
\belowdisplayshortskip 4\p@ plus2\p@ minus2\p@
\def\@listi{\leftmargin\leftmargini
\topsep 4\p@ plus2\p@ minus2\p@\parsep 2\p@ plus\p@ minus\p@
\itemsep \parsep}}

\renewcommand{\footnotesize}{\fontsize\@viiipt{9.5}\selectfont%
\abovedisplayskip 6\p@ plus2\p@ minus4\p@
\belowdisplayskip \abovedisplayskip
\abovedisplayshortskip \z@ plus\p@
\belowdisplayshortskip 3\p@ plus\p@ minus2\p@
\def\@listi{\leftmargin\leftmargini %% Added 22 Dec 87
\topsep 3\p@ plus\p@ minus\p@\parsep 2\p@ plus\p@ minus\p@
\itemsep \parsep}}

\renewcommand{\scriptsize}{\fontsize\@viipt{8pt}\selectfont}
\renewcommand{\tiny}{\fontsize\@vpt{6pt}\selectfont}
\renewcommand{\large}{\fontsize\@xiipt{14pt}\selectfont}
\renewcommand{\Large}{\fontsize\@xivpt{18pt}\selectfont}
\renewcommand{\LARGE}{\fontsize\@xviipt{22pt}\selectfont}
\renewcommand{\huge}{\fontsize\@xxpt{25pt}\selectfont}
\renewcommand{\Huge}{\fontsize\@xxvpt{30pt}\selectfont}

%</10pt>
%    \end{macrocode}
% \file{style/mu/fit11.clo}
% This file is conditionally loaded by the
% \texttt{style/mu/base.sty} file to redefine the page geometry to
% match the type size of 11 points.
%    \begin{macrocode}
%<*11pt>
\ProvidesFile{fit11.clo}[2015/04/08]

\renewcommand{\normalsize}{\fontsize\@xipt{14}\selectfont%
\abovedisplayskip 11\p@ plus3\p@ minus6\p@
\belowdisplayskip \abovedisplayskip
\belowdisplayshortskip  6.5\p@ plus3.5\p@ minus3\p@
%\abovedisplayshortskip  \z@ plus3\@p
\let\@listi\@listI}

\renewcommand{\small}{\fontsize\@xpt{12}\selectfont%
\abovedisplayskip 10\p@ plus2\p@ minus5\p@ 
\belowdisplayskip \abovedisplayskip
\abovedisplayshortskip  \z@ plus3\p@
\belowdisplayshortskip  6\p@ plus3\p@ minus3\p@
\def\@listi{\leftmargin\leftmargini
\topsep 6\p@ plus2\p@ minus2\p@\parsep 3\p@ plus2\p@ minus\p@
\itemsep \parsep}}

\renewcommand{\footnotesize}{\fontsize\@ixpt{11}\selectfont%
\abovedisplayskip 8\p@ plus2\p@ minus4\p@
\belowdisplayskip \abovedisplayskip
\abovedisplayshortskip \z@ plus\p@ 
\belowdisplayshortskip 4\p@ plus2\p@ minus2\p@
\def\@listi{\leftmargin\leftmargini
\topsep 4\p@ plus2\p@ minus2\p@\parsep 2\p@ plus\p@ minus\p@
\itemsep \parsep}}

\renewcommand{\scriptsize}{\fontsize\@viiipt{9.5pt}\selectfont}
\renewcommand{\tiny}{\fontsize\@vipt{7pt}\selectfont}
\renewcommand{\large}{\fontsize\@xiipt{14pt}\selectfont}
\renewcommand{\Large}{\fontsize\@xivpt{18pt}\selectfont}
\renewcommand{\LARGE}{\fontsize\@xviipt{22pt}\selectfont}
\renewcommand{\huge}{\fontsize\@xxpt{25pt}\selectfont}
\renewcommand{\Huge}{\fontsize\@xxvpt{30pt}\selectfont}

%</11pt>
%    \end{macrocode}
% \file{style/mu/fit12.clo}
% This file is conditionally loaded by the
% \texttt{style/mu/base.sty} file to redefine the page geometry to
% match the type size of 12 points. The type dimensions defined by
% the file are stored in the following private macros as well:
% \begin{itemize}
%  \item\DescribeMacro{\thesis@titlePage@normalsize}%
%    |\thesis@titlePage@normalsize| -- Equivalent to |\normalsize|
%  \item\DescribeMacro{\thesis@titlePage@small}%
%    |\thesis@titlePage@small| -- Equivalent to |\small|
%  \item\DescribeMacro{\thesis@titlePage@footnotesize}%
%    |\thesis@titlePage@footnotesize| -- Equivalent to
%    |\footnotesize|
%  \item\DescribeMacro{\thesis@titlePage@scriptsize}%
%    |\thesis@titlePage@scriptsize| -- Equivalent to |\scriptsize|
%  \item\DescribeMacro{\thesis@titlePage@tiny}%
%    |\thesis@titlePage@tiny| -- Equivalent to |\tiny|
%  \item\DescribeMacro{\thesis@titlePage@large}%
%    |\thesis@titlePage@large| -- Equivalent to |\large|
%  \item\DescribeMacro{\thesis@titlePage@Large}%
%    |\thesis@titlePage@Large| -- Equivalent to |\Large|
%  \item\DescribeMacro{\thesis@titlePage@LARGE}%
%    |\thesis@titlePage@LARGE| -- Equivalent to |\LARGE|
%  \item\DescribeMacro{\thesis@titlePage@huge}%
%    |\thesis@titlePage@huge| -- Equivalent to |\huge|
%  \item\DescribeMacro{\thesis@titlePage@Huge}%
%    |\thesis@titlePage@Huge| -- Equivalent to |\Huge|
% \end{itemize}
% These private macros can be used to typeset elements, whose size
% should remain constant regardless of the font size setting.
%    \begin{macrocode}
%<*12pt>
\ProvidesFile{fit12.clo}[2015/04/08]
%</12pt>

\def\thesis@titlePage@normalsize{\fontsize\@xiipt{14.5}%
\selectfont\abovedisplayskip 12\p@ plus3\p@ minus7\p@
\belowdisplayskip \abovedisplayskip
\abovedisplayshortskip  \z@ plus3\p@
\belowdisplayshortskip  6.5\p@ plus3.5\p@ minus3\p@
\let\@listi\@listI}

\def\thesis@titlePage@small{\fontsize\@xipt{13.6}\selectfont%
\abovedisplayskip 11\p@ plus3\p@ minus6\p@
\belowdisplayskip \abovedisplayskip
\abovedisplayshortskip  \z@ plus3\p@
\belowdisplayshortskip  6.5\p@ plus3.5\p@ minus3\p@
\def\@listi{\leftmargin\leftmargini %% Added 22 Dec 87
\parsep 4.5\p@ plus2\p@ minus\p@
            \itemsep \parsep
            \topsep 9\p@ plus3\p@ minus5\p@}}

\def\thesis@titlePage@footnotesize{\fontsize\@xpt{12}\selectfont%
\abovedisplayskip 10\p@ plus2\p@ minus5\p@
\belowdisplayskip \abovedisplayskip
\abovedisplayshortskip  \z@ plus3\p@
\belowdisplayshortskip  6\p@ plus3\p@ minus3\p@
\def\@listi{\leftmargin\leftmargini %% Added 22 Dec 87
\topsep 6\p@ plus2\p@ minus2\p@\parsep 3\p@ plus2\p@ minus\p@
\itemsep \parsep}}
            
\def\thesis@titlePage@scriptsize{\fontsize\@viiipt{9.5pt}\selectfont}
\def\thesis@titlePage@tiny{\fontsize\@vipt{7pt}\selectfont}
\def\thesis@titlePage@large{\fontsize\@xivpt{18pt}\selectfont}
\def\thesis@titlePage@Large{\fontsize\@xviipt{22pt}\selectfont}
\def\thesis@titlePage@LARGE{\fontsize\@xxpt{25pt}\selectfont}
\def\thesis@titlePage@huge{\fontsize\@xxvpt{30pt}\selectfont}
\def\thesis@titlePage@Huge{\fontsize\@xxvpt{30pt}\selectfont}

%<*12pt>
\renewcommand{\normalsize}{\thesis@titlePage@normalsize}
\renewcommand{\small}{\thesis@titlePage@small}
\renewcommand{\footnotesize}{\thesis@titlePage@footnotesize}
\renewcommand{\scriptsize}{\thesis@titlePage@scriptsize}
\renewcommand{\tiny}{\thesis@titlePage@tiny}
\renewcommand{\large}{\thesis@titlePage@large}
\renewcommand{\Large}{\thesis@titlePage@Large}
\renewcommand{\LARGE}{\thesis@titlePage@LARGE}
\renewcommand{\huge}{\thesis@titlePage@huge}
\renewcommand{\Huge}{\thesis@titlePage@Huge}
%</12pt>
\let\@normalsize\normalsize
\normalsize

\if@twoside               
   \oddsidemargin 0.75in  
   \evensidemargin 0.4in  
   \marginparwidth 0pt    
\else                     
   \oddsidemargin 0.75in  
   \evensidemargin 0.75in
   \marginparwidth 0pt
\fi
\marginparsep 10pt        

\topmargin 0.4in          
                          
\headheight 20pt          
\headsep 10pt             
\topskip 10pt    
\footskip 30pt 

%<*10pt>
\textheight = 43\baselineskip
\advance\textheight by \topskip
\textwidth 5.0truein
\columnsep 10pt       
\columnseprule 0pt

\footnotesep 6.65pt
\skip\footins 9pt plus 4pt minus 2pt
\floatsep 12pt plus 2pt minus 2pt
\textfloatsep 20pt plus 2pt minus 4pt
\intextsep 12pt plus 2pt minus 2pt
\dblfloatsep 12pt plus 2pt minus 2pt
\dbltextfloatsep 20pt plus 2pt minus 4pt

\@fptop 0pt plus 1fil
\@fpsep 8pt plus 2fil
\@fpbot 0pt plus 1fil
\@dblfptop 0pt plus 1fil
\@dblfpsep 8pt plus 2fil
\@dblfpbot 0pt plus 1fil
\marginparpush 5pt

\parskip 0pt plus 1pt
\partopsep 2pt plus 1pt minus 1pt

%</10pt>
%
%<*11pt>
\textheight = 39\baselineskip
\advance\textheight by \topskip
\textwidth 5.0truein
\columnsep 10pt
\columnseprule 0pt

\footnotesep 7.7pt
\skip\footins 10pt plus 4pt minus 2pt
\floatsep 12pt plus 2pt minus 2pt
\textfloatsep 20pt plus 2pt minus 4pt
\intextsep 12pt plus 2pt minus 2pt
\dblfloatsep 12pt plus 2pt minus 2pt
\dbltextfloatsep 20pt plus 2pt minus 4pt

\@fptop 0pt plus 1fil
\@fpsep 8pt plus 2fil
\@fpbot 0pt plus 1fil
\@dblfptop 0pt plus 1fil
\@dblfpsep 8pt plus 2fil
\@dblfpbot 0pt plus 1fil
\marginparpush 5pt 

\parskip 0pt plus 0pt
\partopsep 3pt plus 1pt minus 2pt

%</11pt>
%
%<*12pt>
\textheight = 37\baselineskip
\advance\textheight by \topskip
\textwidth 5.0truein
\columnsep 10pt
\columnseprule 0pt

\footnotesep 8.4pt
\skip\footins 10.8pt plus 4pt minus 2pt
\floatsep 14pt plus 2pt minus 4pt 
\textfloatsep 20pt plus 2pt minus 4pt
\intextsep 14pt plus 4pt minus 4pt
\dblfloatsep 14pt plus 2pt minus 4pt
\dbltextfloatsep 20pt plus 2pt minus 4pt

\@fptop 0pt plus 1fil
\@fpsep 10pt plus 2fil
\@fpbot 0pt plus 1fil
\@dblfptop 0pt plus 1fil
\@dblfpsep 10pt plus 2fil
\@dblfpbot 0pt plus 1fil
\marginparpush 7pt

\parskip 0pt plus 0pt
\partopsep 3pt plus 2pt minus 2pt

%</12pt>
\@lowpenalty   51
\@medpenalty  151
\@highpenalty 301
\@beginparpenalty -\@lowpenalty
\@endparpenalty   -\@lowpenalty
\@itempenalty     -\@lowpenalty

\def\@makechapterhead#1{%
  {%
    \setlength\parindent{\z@}%
    \setlength\parskip  {\z@}%
    \ifnum
      \c@secnumdepth >\m@ne
      \par\nobreak
      \vskip 10\p@
    \fi
    \Large \ChapFont \thechapter{} \space #1\par
    \nobreak
    \vskip 20\p@
  }%
}

\def\@makeschapterhead#1{%
  {%
    \setlength\parindent{\z@}%
    \setlength\parskip  {\z@}%
    \Large \ChapFont #1\par
    \nobreak
    \vskip 20\p@
  }%
}

\def\chapter{%
   \clearpage
   \thispagestyle{plain}
   \global\@topnum\z@ 
   \@afterindentfalse  
   \secdef\@chapter\@schapter
 }

\def\@chapter[#1]#2{%
  \ifnum \c@secnumdepth
    >\m@ne
    \refstepcounter{chapter}%
    \typeout{\@chapapp\space\thechapter.}% 
    \addcontentsline{toc}{chapter}{\protect
    \numberline{\thechapter}\bfseries #1}
  \else%
    \addcontentsline{toc}{chapter}{\bfseries #1}
  \fi
  \chaptermark{#1}%
  \addtocontents{lof}%
  {\protect\addvspace{4\p@}} 
  \addtocontents{lot}%
  {\protect\addvspace{4\p@}} 
  \if@twocolumn                   
    \@topnewpage[\@makechapterhead{#2}]%
  \else
    \@makechapterhead{#2}%
    \@afterheading          
  \fi
}

%\def\@schapter#1{\if@twocolumn \@topnewpage[\@makeschapterhead{#1}]%
%        \else \@makeschapterhead{#1}%
%              \markright{#1}
%              \@afterheading\fi}

\def\section{\@startsection {section}{1}{\z@}{-3.5ex plus-1ex minus
    -.2ex}{2.3ex plus.2ex}{\reset@font\large\bfseries}}
\def\subsection{\@startsection{subsection}{2}{\z@}{-3.25ex plus-1ex
    minus-.2ex}{1.5ex plus.2ex}{\reset@font\normalsize\bfseries}}
\def\subsubsection{\@startsection{subsubsection}{3}{\z@}{-3.25ex plus   
    -1ex minus-.2ex}{1.5ex plus.2ex}{\reset@font\normalsize}}
\def\paragraph{\@startsection
    {paragraph}{4}{\z@}{3.25ex plus1ex minus.2ex}{-1em}{\reset@font
    \normalsize\bfseries}}
\def\subparagraph{\@startsection
     {subparagraph}{4}{\parindent}{3.25ex plus1ex minus
     .2ex}{-1em}{\reset@font\normalsize\bfseries}}

\setcounter{secnumdepth}{2}

\def\appendix{\par
  \setcounter{chapter}{0}%
  \setcounter{section}{0}%
  \def\@chapapp{\appendixname}%
  \def\thechapter{\Alph{chapter}}}

\leftmargini 2.5em
\leftmarginii 2.2em     % > \labelsep + width of '(m)'
\leftmarginiii 1.87em   % > \labelsep + width of 'vii.'
\leftmarginiv 1.7em     % > \labelsep + width of 'M.'
\leftmarginv 1em
\leftmarginvi 1em

\leftmargin\leftmargini
\labelsep .5em
\labelwidth\leftmargini\advance\labelwidth-\labelsep

%<*10pt>
\def\@listI{\leftmargin\leftmargini \parsep 4\p@ plus2\p@ minus\p@%
\topsep 8\p@ plus2\p@ minus4\p@
\itemsep 4\p@ plus2\p@ minus\p@}

\let\@listi\@listI
\@listi

\def\@listii{\leftmargin\leftmarginii
   \labelwidth\leftmarginii\advance\labelwidth-\labelsep
   \topsep 4\p@ plus2\p@ minus\p@
   \parsep 2\p@ plus\p@ minus\p@
   \itemsep \parsep}

\def\@listiii{\leftmargin\leftmarginiii
    \labelwidth\leftmarginiii\advance\labelwidth-\labelsep
    \topsep 2\p@ plus\p@ minus\p@
    \parsep \z@ \partopsep\p@ plus\z@ minus\p@
    \itemsep \topsep}

\def\@listiv{\leftmargin\leftmarginiv
     \labelwidth\leftmarginiv\advance\labelwidth-\labelsep}
   
\def\@listv{\leftmargin\leftmarginv
     \labelwidth\leftmarginv\advance\labelwidth-\labelsep}
   
\def\@listvi{\leftmargin\leftmarginvi
     \labelwidth\leftmarginvi\advance\labelwidth-\labelsep}
%</10pt>
%
%<*11pt>
\def\@listI{\leftmargin\leftmargini \parsep 4.5\p@ plus2\p@ minus\p@
\topsep 9\p@ plus3\p@ minus5\p@
\itemsep 4.5\p@ plus2\p@ minus\p@}

\let\@listi\@listI
\@listi

\def\@listii{\leftmargin\leftmarginii
   \labelwidth\leftmarginii\advance\labelwidth-\labelsep
   \topsep 4.5\p@ plus2\p@ minus\p@
   \parsep 2\p@ plus\p@ minus\p@
   \itemsep \parsep}

\def\@listiii{\leftmargin\leftmarginiii
    \labelwidth\leftmarginiii\advance\labelwidth-\labelsep
    \topsep 2\p@ plus\p@ minus\p@
    \parsep \z@ \partopsep \p@ plus\z@ minus\p@
    \itemsep \topsep}

\def\@listiv{\leftmargin\leftmarginiv
     \labelwidth\leftmarginiv\advance\labelwidth-\labelsep}
   
\def\@listv{\leftmargin\leftmarginv
     \labelwidth\leftmarginv\advance\labelwidth-\labelsep}
    
\def\@listvi{\leftmargin\leftmarginvi
     \labelwidth\leftmarginvi\advance\labelwidth-\labelsep}
%</11pt>
%
%<*12pt>
\def\@listI{\leftmargin\leftmargini \parsep 5\p@ plus2.5\p@ minus\p@
\topsep 10\p@ plus4\p@ minus6\p@
\itemsep 5\p@ plus2.5\p@ minus\p@}

\let\@listi\@listI
\@listi

\def\@listii{\leftmargin\leftmarginii
   \labelwidth\leftmarginii\advance\labelwidth-\labelsep
   \topsep 5\p@ plus2.5\p@ minus\p@
   \parsep 2.5\p@ plus\p@ minus\p@
   \itemsep \parsep}

\def\@listiii{\leftmargin\leftmarginiii
    \labelwidth\leftmarginiii\advance\labelwidth-\labelsep
    \topsep 2.5\p@ plus\p@ minus\p@
    \parsep \z@ \partopsep \p@ plus\z@ minus\p@
    \itemsep \topsep}

\def\@listiv{\leftmargin\leftmarginiv
     \labelwidth\leftmarginiv\advance\labelwidth-\labelsep}
   
\def\@listv{\leftmargin\leftmarginv
     \labelwidth\leftmarginv\advance\labelwidth-\labelsep}
    
\def\@listvi{\leftmargin\leftmarginvi
     \labelwidth\leftmarginvi\advance\labelwidth-\labelsep}
%</12pt>
%</opt>
%    \end{macrocode}
