\documentclass[color,table,cover,twoside,lot,lof]{../../fithesis3}
\usepackage[english]{babel}
\usepackage{hologo}
\usepackage{fancyvrb}
\usepackage{minted}
\usepackage{ltxcmds}
\usepackage[
  backend=biber,
  style=numeric,
  citestyle=numeric-comp,
  sorting=none
]{biblatex}
\addbibresource{guide.bib}
\makeatletter
\thesissetup{
  title=A fithesis3 user guide for the \thesis@english@facultyName,
  TeXtitle=A \textit{fithesis3} user guide\medskip\\\Large for the
    \thesis@english@facultyName,
  type=bc,
  department=Department of Computer Graphics and Design,
  programme=Applied Informatics,
  field=Typesetting,
  titleEn=\thesis@title,
  departmentEn=\thesis@department,
  programmeEn=\thesis@programme,
  fieldEn=\thesis@field,
  author=Vít Novotný,
  advisor={doc. RNDr. Petr Sojka, Ph.D.},
  assignment={},
  keywords={thesis, typesetting, LaTeX},
  abstract=This user guide describes the installation process of
    the \textsf{fithesis3} document class and documents selected
    parts of the public API of the \textsf{fithesis3} document
    class that bear relevance to the style of the
    \thesis@english@facultyName.,
  keywordsEn=\thesis@keywords,
  abstractEn=\thesis@abstract,
  faculty=?,
  basepath=../..,
  autoLayout=false}
\makeatother
\begin{document}
  \makeatletter\thesis@preamble\makeatother
  \chapter{Introduction}
  \textsf{fithesis3} is a \LaTeX{} document class, which
  aims to streamline the typesetting of mandatory parts of theses
  and dissertations so that the author can focus on the content
  alone. Unlike its predecessors, \textsf{fithesis3} aims to
  provide modular enough design to enable its usage across the
  faculties of the \makeatletter\thesis@english@universityName%
  \makeatother. To this end, the class comprises:
  \begin{itemize}
    \item\emph{style files}, which are unique for each faculty and
      which encapsulate the look and the arrangement of the final
      documents
    \item\emph{locale files}, which define the strings for the
      given locale
    \item\emph{base class}, which serves as a bridge between style
      files, locale files and the input document
  \end{itemize}
  The overarching desing and the interactions between the style
  files, locale files and the base class are documented in the
  technical documentation of the class \cite{novotny15}
  distributed along with the package. This
  guide, on the other hand, only aims to document the selected
  parts of the public API of the \textsf{fithesis3} class
  that bear relevance to the style file of the
  \makeatletter\thesis@english@facultyName\makeatother. Note that
  this guide is typeset using the said style file.

  \section{Required packages and fonts}\label{sec:req-packages}
  In order to be able to use the \textsf{fithesis3} class, your
  \TeX{} installation needs to include the \textsf{rapport3}
  class\footnote{See
  \url{https://www.ctan.org/pkg/ntgclass}} and the following
  packages:\begin{itemize}
    \item\textsf{keyval}, \textsf{etoolbox}, \textsf{ifxetex}
    \item\textsf{inputenc} -- only when typesetting with \TeX{} or
      \hologo{pdfTeX}
  \end{itemize} In order to be able to use the \textsf{fithesis3}
  class with the style files of the
  \makeatletter\thesis@english@facultyName% \makeatother, your
  \TeX{} installation needs to also include the following
  packages:\begin{itemize}
    \item\textsf{xcolor}, \textsf{graphix}, \textsf{pdfpages},
      \textsf{hyperref}, \textsf{keyval}%
      {\makeatletter % Typeset additional loaded packages
      \def\thguide@prnpkg#1{\ltx@ifpackageloaded%
        {#1}{, \textsf{#1}}{}}%
      \thguide@prnpkg{tikz}%
      \thguide@prnpkg{changepage}%
      \thguide@prnpkg{setspace}%
      \thguide@prnpkg{geometry}}
    \item\textsf{fontspec}, \textsf{unicode-math} -- only when
      typesetting with \Hologo{XeTeX}
    \item\textsf{mathpazo}, \textsf{tgpagella}, \textsf{lmodern},
       \textsf{cmap}, \textsf{fontenc} -- only when typesetting
       with \TeX{} or \hologo{pdfTeX}
    \item\textsf{tabularx}, \textsf{booktabs} -- only when the
      \texttt{table} option is specified (see Section
      \ref{sec:options})
  \end{itemize}
  The \TeX{} Gyre Pagella\footnote{See
  \url{https://www.ctan.org/pkg/tex-gyre-pagella}} and
  \TeX{} Gyre Pagella Math\footnote{See
  \url{https://www.ctan.org/pkg/tex-gyre-math-pagella}} OpenType
  fonts are also required, when typesetting with \Hologo{XeTeX}.
  All these are likely to be a part of any reasonably modern \TeX{}
  distribution.

  \section{Installation}
  This section covers the installation of the \textsf{fithesis3}
  class. Please note that the installation of the class is fully
  optional. You can typeset your thesis by either directly editing
  either the \path{pdflatex.tex} or the \path{xelatex.tex} example
  files distributed along with the package or by pointing your
  source document to the class as follows:
  \begin{minted}{latex}
    \documentclass{path/fithesis3}
    \thesissetup{basepath=path}
    % The rest of the document
  \end{minted}
  where \texttt{path} corresponds to the path of the directory
  containing the \path{fithesis3.cls} file.

  When installing, first make sure that the \textsf{fithesis3} class
  is not a part of
  your \TeX{} distribution already. This can be easily verified by
  creating the minimal document described in the next section and
  typesetting it using either the \Hologo{XeTeX} or
  the \hologo{pdfTeX} engine, respectively. In case
  \textsf{fithesis3} is not a part of your \TeX{}
  distribution, the typesetting will prematurely terminate with the
  following error: \begin{Verbatim}[frame=single]
! LaTeX Error: File `fithesis3/fithesis3.cls' not found.
  \end{Verbatim}
  If the \textsf{fithesis3} class is not a part of
  your distribution, you can proceed to the installation.
  This can be achieved by extracting the \path{fithesis3.tds.zip}
  archive distributed along with the package into into one of
  the \TeX{} directory structure trees within your \TeX{}
  distribution. If you are using \TeX{}Live\footnote{See
  \url{https://www.tug.org/texlive/doc.html}, Chapter 2.3},
  this can be achieved by creating a \texttt{texmf}
  directory within your user home directory and by extracting the
  \path{fithesis3.tds.zip} archive into it. For \Hologo{MiKTeX},
  see the online documentation\footnote{See
  \url{http://docs.miktex.org/manual/localadditions.html}}.

  \section{A minimal document}
  Before using the \textsf{fithesis3} class, you should be familiar
  with the \LaTeX{} typesetting system. A good way to get started
  is to read one of the introductory texts in English
  \cite{veryshortlatex,shortlatex,longlatex,latex} or in Czech
  \cite{rybicka03,satrapa11}.
  We will start by creating a plain text document named
  \path{helloworld.tex} in the UTF-8 encoding with the following
  content:
  \begin{minted}{latex}
    \documentclass{fithesis3}
    \thesissetup{faculty=?}
    \begin{document}
      Hello world!
    \end{document}
  \end{minted}
  Now typeset the document using either the \hologo{pdfLaTeX} or
  the \Hologo{XeLaTeX} engine. If everything is set up correctly,
  you should end up with a document containing all the mandatory
  parts of a thesis and one page at the end containing a
  \texttt{Hello world!} line. You should notice that the document
  is implicitly typeset in English and that it contains lots of
  placeholder stringss for missing metadata (see Figure
  \ref{fig:example01}). In the next chapter, we are going to
  address that.
  \begin{figure}[!bt]
    \centering\makeatletter
    \fbox{\includegraphics[clip,trim=0cm 14.8cm 0cm 4.2cm,%
      width=0.975\textwidth]{examples/\thesis@faculty-01.pdf}}
    \makeatother
    \caption{The placeholder strings in the minimal document}
    \label{fig:example01}
  \end{figure}

  \chapter{Configuring the class}
  In this chapter, we will configure the class to use the correct
  locale, to insert the correct metadata into the output document
  and to be laid out in a meaningful way.

  \section{Setting the locale}
  First, we are going to set the locale of the document class. This
  locale affects the locale of the mandatory parts of thesis. To
  see what locales are available, list the contents of the
  \path{fithesis3/locale/} directory. It should contain several
  \textit{locale}\texttt{.dtx} files. Each of these
  \textit{locale}s can be used by the class. To load a
  \textit{locale}, insert \texttt{\string\thesis\-setup\{locale=}%
  \textit{locale}\texttt{\}} into the preamble of the document.

  If you use the \textsf{babel} or the \textsf{polyglossia} package
  to load the hyphenation patterns for your locale, you don't need
  to set the locale at all, \textsf{fithesis3} will use the main
  language of \textsf{babel} or \textsf{polyglossia}.
  \begin{minted}{latex}
    \documentclass{fithesis3}
    \thesissetup{faculty=?}
    % Using the babel package:
    \usepackage[czech]{babel}
    \begin{document}
      The mandatory parts of the thesis
      are going to be typeset in Czech.
    \end{document}

    \documentclass{fithesis3}
    \thesissetup{faculty=?}
    % Using the polyglossia package:
    \usepackage{polyglossia}
    \setmainlanguage{czech}
    \begin{document}
      The mandatory parts of the thesis
      are going to be typeset in Czech.
    \end{document}
  \end{minted}
  The \textsf{babel} package can be used with Latin scripts, while
  the \textsf{polyglossia} package supports non-Latin scripts as
  well and is intended as a replacement of \textsf{babel} for
  \Hologo{XeLaTeX}. You are advised to use one of them, depending
  on your requirements.

  \section{Inserting metadata}
  Next, we are going to insert some metadata into the document. The
  metadata can be inserted into the thesis using the
  \texttt{\string\thesis\-setup} command. This command accepts
  a comma-delimited \textit{key}\texttt{=}\textit{value} list.
  The placeholder strings in our minimal document map directly
  into the
  \textit{key}s, so to change the \emph{<<author>>} placeholder
  into \emph{Jane Doe}, simply insert the
  \texttt{\string\thesis\-setup\{author=Jane Doe\}} command into the
  preamble of your document.

  Note, however, that some keys can not be deduced directly from
  the output
  document. For example the \texttt{thanks} key is not visible,
  since the acknowledgement is not a mandatory
  part of the thesis and therefore it only gets inserted into
  the document, when the \texttt{thanks} key is defined. Some other
  \textit{key}s, like the \texttt{abstract} key, can also span
  multiple paragraphs, in which case they need to be set using the
  \texttt{\string\thesis\-long}\textit{key}%
  \texttt{\}\-\{}\textit{value}\texttt{\}}
  command as follows:
  \begin{minted}{latex}
    \documentclass{fithesis3}
    \thesissetup{
      faculty=?,
      author=Jane Doe}
    \thesislong{abstract}{
      In this document, I am going to
      explore the craft of creating
      abstracts \ldots
        
      \ldots spanning multiple paragraphs.}
    \begin{document}
      Hello world!
    \end{document}
  \end{minted}
  If the \textit{value} of a \textit{key} contains a comma, the
  \texttt{\string\thesis\-setup\{}\textit{key}\texttt{=}%
  \textit{value}\texttt{\}} command would erroneously interpret it
  as a delimiter. To prevent this from happening, enclose the
  \textit{value} in curly braces as follows:
  \begin{minted}{latex}
    \documentclass{fithesis3}
    \thesissetup{
      faculty=?,
      author=Jane Doe,
      advisor={RNDr. John Doe, Ph.D.},
      keywords={keyword1, keyword2}}
    \begin{document}
      Hello world!
    \end{document}
  \end{minted}
  The complete list of \textit{key}s and their effects can
  be found in the technical documentation of the
  class \cite[chapter \emph{Public API}]{novotny15} distributed
  along with the package.

  \section{Style options}\label{sec:options}
  The look of the resulting document can be affected by
  \textit{options} passed to the style file using the
  \texttt{\string\documentclass[}\textit{options}\texttt{]\{fithes%
  is3\}} syntax. The complete list of options for the style files
  of the \makeatletter\thesis@english@universityName\makeatother\
  can be found in the technical documentation of the class
  \cite[chapter \emph{Style files}]{novotny15} distributed along
  with the package. Some of the more important options are listed
  in Table \ref{tab:options} for your convenience.
  
  \begin{table}[bt]
    \begin{tabularx}{\textwidth}{lX}
      \toprule
        \emph{Name} & \emph{Description} \\
      \midrule
        \texttt{oneside} & Enables one-sided typesetting. This
        is generally discouraged. Use only if you don't
        have access to a double-sided printer, or if one-sided
        typesetting is a formal requirement. \\
        \texttt{twoside} & Enables double-sided typesetting.
        Double-sided typesetting consumes less paper, is generally
        regarded as more visually pleasing and is enabled by default.
        Use at least 120 grams per square meter paper to prevent
        show-through. \\
        \texttt{color} & Enables colors. A colorful version of the
        document is more visually pleasing, but shouldn't be used
        in a printed version, if you don't have access to a color
        printer. Unless you have a compelling reason not to, always use
        this option in the e-version. \\
        \texttt{monochrome} & Disables colors. Disabling colors is
        generally discouraged, unless you don't have access to a
        color printer. \\
        \texttt{cover} & Typesets the cover of the thesis. Should
        be generally used in the e-version. \\
        \texttt{nocover} & Forbids the typesetting of the thesis
        cover. Use, if you're typesetting the printed version and
        have no desire to have a cover made for your thesis. This
        option is the default. \\
        \texttt{table} & Redefines the \texttt{tabular} and
        \texttt{tabularx} environments to use alternating colors
        for odd and even rows like this table does. \\
        \texttt{oldtable} & Instructs the style not to redefine the
        \texttt{tabular} and \texttt{tabularx} environments. This
        option is the default. \\
      \bottomrule
    \end{tabularx}
    \caption{A non-exhaustive list of basic options accepted by the
      styles of the Masaryk University}
    \label{tab:options}
  \end{table}

  \chapter{Advanced usage}
  This chapter is dedicated to more advanced \LaTeX\ users, who
  wish to customize the class to better suit their needs.

  \section{Throubleshooting option clashes}
  When a package is required twice, each time with different
  options, an option clash error occurs: \begin{Verbatim}%
  [frame=single]
! LaTeX Error: Option clash for package hyperref
  \end{Verbatim}
  If you need to pass \textit{options} to a \textit{package}
  required by the \textsf{fithesis3} class, as specified in
  Section \ref{sec:req-packages}, prepend the
  \texttt{\string\Pass\-Options\-To\-Package\discretionary{}{}{}%
  \{}\textit{options}\texttt{\}\discretionary{}{}{}%
  \{}\textit{package}\texttt{\}} command before the
  \texttt{\string\documentclass\{\ldots\}} statement.

  If you need to pass \textit{options} to a \textit{package}
  required by the style files of the \makeatletter%
  \thesis@english@universityName\makeatother, you should, in most
  cases, be able to just
  \texttt{\string\use\-package\discretionary{}{}{}[}%
  \textit{options}\texttt{]\discretionary{}{}{}\{}%
  \textit{package}\texttt{\}} within the preamble. That is because
  even though style files are loaded at the very end of the
  preamble \cite[chapter \emph{Main routine}]{novotny15}, they load
  most of the packages \emph{lazily}.  This means that the package
  is only loaded, if it hasn't already been loaded by the user.
  This has the advantage of avoiding option clashes.
  
  Some packages, however, need to be loaded with a specific set of
  options in which case the lazy loading is not applicable and an
  option clash may occur. In this case, you may not load the
  package within the preamble and you can only alter the package
  configuration after it has been loaded, if that's in any way
  supported by the given package. You can do this by either
  appending the package-specific code after the preamble, or by
  loading the style files prematurely using the
  \texttt{\string\thesis@load} command, if the package-specific
  code needs to be included into the preamble:\begin{minted}{latex}
    \documentclass{fithesis3}
    % Preamble
    \begin{document}
      % The package-specific code goes here
    \end{document}

    \documentclass{fithesis3}
    % Preamble
    \makeatletter\thesis@load\makeatother
    % The package-specific code goes here
    \begin{document}
    \end{document}
  \end{minted}
  
  \section{Changing the layout}
  If you are unsatisfied with the arrangement of the mandatory
  parts of the thesis, you can disable it using the
  \texttt{autoLayout} metadata key:
  \begin{minted}{latex}
    \documentclass{fithesis3}
    \thesissetup{faculty=?,autoLayout=false}
    \begin{document}
      A document which, except for this line,
      is completely empty.
    \end{document}
  \end{minted}
  This results in a document, which is completely devoid of any
  mandatory parts of the thesis (see Figure \ref{fig:example02}).
  You can now manually insert the preamble and the postamble of the
  document as follows:
  \begin{minted}{latex}
    \documentclass{fithesis3}
    \thesissetup{
      faculty=?,
      autoLayout=false}
    \begin{document}
      \makeatletter\thesis@preamble\makeatother
      A document which once again contains all
      the mandatory parts of a thesis.
      \makeatletter\thesis@postamble\makeatother
    \end{document}
  \end{minted}
  \begin{figure}[!bt]
    \centering\makeatletter
    \fbox{\includegraphics[clip,trim=2cm 24.5cm 2cm 4.5cm,%
      width=0.975\textwidth]{examples/02.pdf}}
    \makeatother
    \caption{A document with disabled \texttt{autoLayout}}
    \label{fig:example02}
  \end{figure}
  This alone would be a useless excercise, as we're now back to the
  original document. However, instead of inserting the
  \texttt{\string\thesis\-@preamble} and the
  \texttt{\string\thesis\-@postamble} commands into the document, we
  can insert only certain blocks to which these commands expand.
  \texttt{\string\thesis\-@preamble} expands to the
  following commands:
  {% The following macro typesets the meaning of another macro
  \makeatletter
  \def\thguide@macromeaning#1{%
      \let\ea\expandafter%
      \ea\let\ea\thguide@macro\csname#1\endcsname%
      \ea\def\ea\thguide@meaning\ea{\meaning\thguide@macro}%
      \def\thguide@parse##1 ##2{%
        ##1\ifx##2\relax\ea\@gobbletwo\else\\\fi\thguide@parse##2}%
      \begin{verse}%
        \ea\ifx\@empty\thguide@macro\@empty%
          $\langle$\emph{empty}$\rangle$
        \else%
          \tt\ea\ea\ea\ea\ea\ea\ea\ea\ea\ea\ea\ea\ea\ea\ea\ea\ea\ea%
          \ea\ea\ea\ea\ea\ea\ea\ea\ea\ea\ea\ea\ea\thguide@parse\ea%
          \ea\ea\ea\ea\ea\ea\ea\ea\ea\ea\ea\ea\ea\ea\@gobbletwo\ea%
          \ea\ea\ea\ea\ea\ea\@gobbletwo\ea\ea\ea\@gobbletwo\ea%
          \@gobbletwo\thguide@meaning \relax%
        \fi%
      \end{verse}}
  \thguide@macromeaning{thesis@preamble} and
  \texttt{\string\thesis\-@postamble} 
  expands to the following commands:
  \thguide@macromeaning{thesis@postamble}
  \makeatother}
  To create a document, which only contains the thesis cover prior
  to the text, we would use the following:
  \begin{minted}{latex}
    \documentclass{fithesis3}
    \thesissetup{
      faculty=?,
      autoLayout=false}
    \begin{document}
      \makeatletter
        \thesis@blocks@cover
        \thesis@blocks@mainMatter
      \makeatother
      A document which contains only the cover.
    \end{document}
  \end{minted}
  The available blocks are documented in the technical
  documentation of the class \cite[chapter
  \emph{Style files}]{novotny15} distributed along with the
  package.

  \makeatletter\thesis@postamble\makeatother
  \printbibliography[heading=bibintoc]
\end{document}
