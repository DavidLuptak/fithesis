% \iffalse meta-comment
% fithesis.dtx
% Copyright 1998--2015 Daniel Marek (DM), Jan Pavlovič (JP),
% Vít Novotný (VN), Petr Sojka (PS)
% https://www.fi.muni.cz/tech/unix/tex/fithesis.xhtml
% Faculty of Informatics, Masaryk University
%
% This work may be distributed and/or modified under the
% conditions of the LaTeX Project Public License, either version 1.3
% of this license or (at your option) any later version.
% The latest version of this license is in
%   http://www.latex-project.org/lppl.txt
% and version 1.3 or later is part of all distributions of LaTeX
% version 2005/12/01 or later.
%
% This work has the LPPL maintenance status `maintained'.
% 
% The Current Maintainer of this work is Vít Novotný.
% Send bug reports, requests for additions and questions
% to the fithesis discussion forum at
% <https://is.muni.cz/auth/df/fithesis-sazba/>.
%
% This work consists of the files fithesis.dtx and fithesis.ins
% and the derived files fithesis3.cls, fithesis2.cls, fithesis.cls,
% fit10.clo, fit11.clo, fit12.clo.
%
%    \begin{macrocode}
%<*driver>

\documentclass{ltxdoc}
\usepackage[utf8]{inputenc} % this file uses UTF-8
\usepackage[english]{babel}
\usepackage{tgpagella}
\usepackage{tabularx}
\usepackage{hologo}
\usepackage{booktabs}
\usepackage[scaled=0.86]{berasans}
\usepackage[scaled=1.03]{inconsolata}
\usepackage[resetfonts]{cmap}
\usepackage[T1]{fontenc} % use 8bit fonts
\emergencystretch 2dd
\usepackage{hypdoc}

% Making paragraphs numbered
\makeatletter
\renewcommand\paragraph{\@startsection{paragraph}{4}{\z@}%
            {-2.5ex\@plus -1ex \@minus -.25ex}%
            {1.25ex \@plus .25ex}%
            {\normalfont\normalsize\bfseries}}
\makeatother
\setcounter{secnumdepth}{4} % how many sectioning levels to assign
\setcounter{tocdepth}{4}    % how many sectioning levels to show

% ltxdoc class options
\CodelineIndex
\MakeShortVerb{|}
\EnableCrossrefs
\DoNotIndex{}
\makeatletter
\c@IndexColumns=2
\makeatother

\begin{document}
  \RecordChanges
  \DocInput{fithesis.dtx}
  \PrintIndex
  \PrintChanges
\end{document}

%</driver>
%    \end{macrocode}
%<*class>
\NeedsTeXFormat{LaTeX2e}
% \fi
\def\thesis@version{2015/04/26 v0.3.09 fithesis3 MU thesis class}
%
%%%%%%%%%%%%%%%%%%%%%%%%%%%%%%%%%%%%%%%%%%%%%%%%%%%%%%%%%%%%%%%%%%%%%%%%%%%%%%%
%
% \changes{v0.3.10}  {2015/05/09}{Fixed a typo in the technical
%   documentation. Updated the \emph{Advanced usage} chapter of the
%   user guide. The required packaged listed in Section 2.2 of the
%   user guide are now always correct. Adjusted the footer spacing
%   in the styles of econ and fi. Added \emph{Advanced usage}
%   chapter to the user guide. Added the description of basic
%   options into the user guide. Added the \texttt{table} and
%   \texttt{oldtable} options. Added the \texttt{type} field to the
%   guide for completeness. [VN]}
% \changes{v0.3.09}  {2015/04/26}{A complete refactoring of the class. The class
%   was decomposed into a base class, locale files and style files. [VN]}
% \changes{v0.3.08}  {2015/03/04}{Fixed a non-terminated \cs{if} condition.
%   [VN] (backport of v0.2.18)\\Fixed mostly documentation errors reported
%   at the new fithesis discussion forum (-ti, eco$\rightarrow$econ, implicit
%   twocolumn, example extended (font setup), etc.). [PS] (backport of v0.2.17)}
% \changes{v0.3.07}  {2015/02/03}{Replaced the \cs{thesiswoman} command with
%   \cs{thesisgender}. [VN]}
% \changes{v0.3.06}  {2015/01/26}{Added the colorx package and the base colors
%   for each faculty. If the color option is specified, the tabular environment
%   gets redefined and uses the faculty colors to color alternating table rows
%   to improve readability. The hyperref links in the e-version are now likewise
%   colored according to the chosen faculty, in this case regardless of the
%   presence of the color option. Dropped the support for typesetting theses
%   outside MU. [VN]}
% \changes{v0.3.05}  {2015/01/21}{Added support for change typesetting.
%   Restructured the code to make it more amenable to literal programming.
%   Added support for \cs{CodelineIndex} typesetting. Added information about
%   the usage of \textsf{fithesis1} and \textsf{fithesis2} on the FI unix
%   machines. (backport of v0.2.16) [VN]\\Minor changes throughout the text,
%   added a link to the the fithesis forums [PS] (backport of v0.2.15@r14:15)}
% \changes{v0.3.04}  {2015/01/14}{Import the url package to allow for the use of
%   \cs{url} within the documentation. (backport of v0.2.15@r13) [VN]}
% \changes{v0.3.03}  {2015/01/14}{Small fixes (added \cs{relax} at
%   \cs{MainMatter}), generating both fithesis.cls (obsolete, loading
%   \texttt{fithesis2.cls}) and \texttt{fithesis2.cls}, minor doc edits,
%   version numbering of \texttt{.clo} fixed, switch to utf8 and ensuring that
%   \texttt{.dtx} compiles. Documentation adjusted to the status quo, added
%   link to discussion forum (backport of v0.2.14) [PS]}
% \changes{v0.3.02}  {2015/01/13}{pdf metadata stamping added for
%   \cs{thesistitle} and \cs{thesisstudent} [VN]}
% \changes{v0.3.01}  {2015/01/09}{documentation now uses babel and cmap
%   packages. the entire file was transcoded into utf8, \cs{thesiscolor} was
%   replaced by color class option, added pdf metadata stamping support [VN]}
% \changes{v0.3.00}  {2015/01/01}{fi logo is no longer special-cased (added eps
%   and pdf), \cs{thesislogopath} added to set the logo directory path,
%   \cs{thesiscolor} added to enable colorful typo elements [VN]}
% \changes{v0.2.12a}{2008--2011}{fork fithesis2 by Mr. Filipčík and Janoušek;
%   cf. \protect\url{https://github.com/liskin/fithesis}}
% \changes{v0.2.12} {2008/07/27}{Licence change to the LPPL [JP]}
% \changes{v0.2.11} {2008/01/07}{fix missing \texttt{fi-logo.mf} [JP,PS]}
% \changes{v0.2.10} {2006/05/12}{fix EN name of Acknowledgement [JP]}
% \changes{v0.2.09}  {2006/05/08}{add EN version of University name [JP]}
% \changes{v0.2.08}  {2006/01/20}{add change of University name [JP]}
% \changes{v0.2.07}  {2005/05/10}{escape all Czech letters [JP]
%   babel is used instead of stupid package czech [JP]
%   \cs{MainMatter} should be placed after \cs{tablesofcontents} [PS]}
% \changes{v0.2.06}  {2004/12/22}{fix : behind Advisor [JP]}
% \changes{v0.2.05}  {2004/05/13}{add English abstract [JP]}
% \changes{v0.2.04}  {2004/05/13}{fix SK declaration [Peter Cerensky, JP]}
% \changes{v0.2.03}  {2004/05/13}{fix title spacing [PS, JP]}
% \changes{v0.2.02}  {2004/05/12}{fix encoding bug [JP]}
% \changes{v0.2.01}  {2004/05/11}{add subsubsection to toc [JP]}
% \changes{v0.2.00}  {2004/05/03}{add sk lang [JP, Peter Cerensky]
%   set default cls class to \textsf{rapport3} [JP]}
% \changes{v0.1g}   {2004/04/01}{change of default size (12pt$\rightarrow$11pt) [JP]}
% \changes{v0.1f}   {2004/01/24}{add documentation for hyperref [JP]}
% \changes{v0.1e}   {2004/01/07}{add Brno to MU title [JP]}
% \changes{v0.1d}   {2003/03/24}{removed def schapter from fit1*.clo [JP]}
% \changes{v0.1c}   {2003/02/21}{default values of \cs{facultyname} and
%   \\\cs{@thesissubtitle} set for backward compatibility) [PS]}
% \changes{v0.1b}   {2003/02/14}{change of default size (11pt$\rightarrow$12pt) [JP]}
% \changes{v0.1a}   {2003/02/12}{minor documentation changes (CZ only,
%   sorry) [PS]}
% \changes{v0.1}    {2003/02/11}{new release, documentation editing (CZ only,
%   sorry) [PS]}
% \changes{v0.0a}   {2002}{changes by Jan Pavlovič to allow fithesis being
%   backend of docbook based system for thesis writing}
% \changes{v0.0}    {1998}{bachelor project of Daniel Marek under
%   supervision of Petr Sojka}
%
%%%%%%%%%%%%%%%%%%%%%%%%%%%%%%%%%%%%%%%%%%%%%%%%%%%%%%%%%%%%%%%%%%%%%%%%%%%%%%%
%
% \title{The \textsf{fithesis3} class for the typesetting of theses written
%   at the Masaryk Univerzity in Brno}
% \author{Daniel Marek, Jan Pavlovič, Vít Novotný, Petr Sojka}
% \date{\today}
% \maketitle
%
% \begin{abstract}
% \noindent This document describes the design and implementation
% of the \textsf{fithesis3} document class. It contains technical
% information for anyone who wishes to extend the class with their
% locale files or style files. Users who only wish to use the class
% are advised to consult the guides within the |guide/| directory,
% which only document the parts of the public API that are relevant
% to the given style files.
% \end{abstract}
%
% \tableofcontents
%
% \section{Required classes and packages}
% The class loads the \texttt{rapport3} base class and the
% following packages: \begin{itemize}
%   \item\textsf{keyval} -- Adds support for parsing
%     comma-delimited lists of key-value pairs.
%   \item\textsf{etoolbox} -- Adds support for expanding
%     code after the preamble.
%   \item\textsf{ifxetex} -- Used to detect the \Hologo{XeTeX}
%     engine.
%   \item\textsf{inputenc} -- Used to enable the input utf-8
%     encoding. This package does not get loaded under
%     \Hologo{XeTeX}.
% \end{itemize}
% for string parsing and comparisons.
% The \texttt{hyperref} package is also conditionally loaded during
% the expansion of the |\thesis@load| macro (see section
% \ref{sec:thesis@load}). Other packages may be required by the
% style files (see section \ref{sec:style-files}) you are using.
%    \begin{macrocode}
\ProvidesClass{fithesis3}[\thesis@version]
\LoadClass[a4paper]{rapport3}
\RequirePackage{keyval}
\RequirePackage{etoolbox}
\RequirePackage{ifxetex}
\ifxetex\else
  \RequirePackage[utf8]{inputenc}
\fi
%    \end{macrocode}
% \section{Public API}
% \label{sec:public-api}
% \subsection{Options}
% Any \oarg{options} passed to the class will be handed down to the
% loaded style files. The supported options are therefore documented
% in the subsection of section \ref{sec:style-files} dedicated to
% the respective style files.
%
% \subsection{The \cs{thesissetup} macro}
% \begin{macro}{\thesissetup}
% The main public macro is the |\thesissetup|\marg{keyvals}
% command, where \textit{keyvals} is a comma-delimited list of
% key-value pairs as defined by the \textsf{keyval} package. This
% macro needs to be included prior to the beginning of a \LaTeX\ 
% document. When used, the \textit{keyvals} are processed.
%
% Note that the values specified to the |\thesissetup| public macro
% may only contain one paragraph of text. If you wish to set
% multiple paragraphs of text as the value, you need to use
% the |\thesislong| public macro.
%    \begin{macrocode}
\def\thesissetup#1{%
  \setkeys{thesis}{#1}}
%    \end{macrocode}
% \subsubsection{The \texttt{basepath} key}
% \begin{macro}{\thesis@basepath}
% The \marg{\texttt{basepath}=path} pair sets the \textit{path}
% containing the class files. The \textit{path} is prepended to
% each other path (|\thesis@logopath|, |\thesis@stylepath| and
% |\thesis@localepath|) used by the class. If non-empty, the
% \textit{path} gets normalized to \textit{path/}. The normalized
% \textit{path} is stored within the private |\thesis@basepath| macro,
% whose implicit value is |fithesis3/|.
%    \begin{macrocode}
\def\thesis@basepath{fithesis3/}
\define@key{thesis}{basepath}{%
  \ifx\@empty#1\@empty%
    \def\thesis@basepath{}%
  \else%
    \def\thesis@basepath{#1/}%
  \fi}
%    \end{macrocode}
% \end{macro}
% \begin{macro}{\thesis@logopath}
% \subsubsection{The \texttt{logopath} key}
% The \marg{\texttt{logopath}=path} pair sets the \textit{path}
% containing the logo files, which is used by the style files
% loading the logo. If the \textit{path} doesn't begin with a
% The \textit{path} is normalized via the private |\thesis@subdir|
% macro and stored within the private |\thesis@logopath| macro,
% whose implicit value is
% |\thesis@basepath| followed by |logo/\thesis@university/|.
% By default, this expands to \texttt{fithesis3/logo/mu/}.
%    \begin{macrocode}
\def\thesis@logopath{\thesis@basepath logo/\thesis@university/}
\define@key{thesis}{logopath}{%
  \def\thesis@logopath{\thesis@subdir#1\relax}}
%    \end{macrocode}
% \end{macro}
% \begin{macro}{\thesis@subdir}
% The |\thesis@subdir| private macro expands to the concatenation
% of the |\thesis@basepath| private macro with |#1|, if the first
% token of |#1| isn't a slash character (|/|), or to |#1|
% otherwise.
%    \begin{macrocode}
\def\thesis@subdir#1#2\relax{%
  \ifx\@empty#1\@empty%
    \thesis@basepath%
  \else%
    \if#1/%
      #1#2/%
    \else%
      \thesis@basepath#1#2/%
    \fi%
  \fi}
%    \end{macrocode}
% \end{macro}
% \begin{macro}{\thesis@stylepath}
% \subsubsection{The \texttt{stylepath} key}
% The \marg{\texttt{stylepath}=path} pair sets the \textit{path}
% containing the style files. If the \textit{path} doesn't begin
% with a slash (\texttt{/}), it is normalized to
% \cmd{/thesis@basepath} followed by \textit{path}. The
% normalized \textit{path} is stored within the private
% |\thesis@stylepath| macro,
% whose implicit value is |\thesis@basepath style/|. By default,
% this expands to \texttt{fithesis3/style/}.
%    \begin{macrocode}
\def\thesis@stylepath{\thesis@basepath style/}
\define@key{thesis}{stylepath}{%
  \def\thesis@stylepath{\thesis@subdir#1\relax}}
%    \end{macrocode}
% \end{macro}
% \begin{macro}{\thesis@localepath}
% \subsubsection{The \texttt{localepath} key}
% The \marg{\texttt{localepath}=path} pair sets the \textit{path}
% containing the locale files. If the \textit{path} doesn't begin
% with a slash (\texttt{/}), it is normalized to
% \cmd{/thesis@basepath} followed by \textit{path}. The
% normalized \textit{path} is stored within the private
% |\thesis@localepath| macro,
% whose implicit value is |\thesis@basepath| followed by |locale/|.
% By default, this expands to \texttt{fithesis3/locale/}.
%    \begin{macrocode}
\def\thesis@localepath{\thesis@basepath locale/}
\define@key{thesis}{localepath}{%
  \def\thesis@localepath{\thesis@subdir#1\relax}}
%    \end{macrocode}
% \end{macro}
% \begin{macro}{\thesis@def}
% The |\thesis@def|\oarg{key}\marg{name} private macro defines
% the private |\thesis@|\textit{name} macro to expand
% to either <<\textit{key}>>, if specified, or to
% <<\textit{name}>>. The macro serves to provide the placeholder
% string for user-defined macros with no default value.
%    \begin{macrocode}
\newcommand{\thesis@def}[2][]{%
  \expandafter\def\csname thesis@#2\endcsname{%
    <<\ifx\@empty#1\@empty#2\else#1\fi>>}}
%    \end{macrocode}
% \end{macro}
% \begin{macro}{\thesis@declaration}
% \subsubsection{The \texttt{declaration} key}
% The \marg{\texttt{declaration}=text} pair sets the
% declaration \textit{text} to be included into the document.
% \cmd{/thesis@basepath} followed by \textit{path}. The
% \textit{text} is stored within the private |\thesis@declaration|
% macro, whose implicit value is |\thesis@@{declaration}|.
%    \begin{macrocode}
\def\thesis@declaration{\thesis@@{declaration}}
\long\def\KV@thesis@declaration#1{%
  \long\def\thesis@declaration{#1}}
%    \end{macrocode}
% \end{macro}
% \begin{macro}{\ifthesis@woman}
% \subsubsection{The \texttt{gender} key}
% The \marg{\texttt{gender}=char} pair sets the author's gender to
% either a male, if \textit{char} is the character \texttt{m}, or
% to a female. The gender can be tested using the
% |\ifthesis@woman| \ldots |\else| \ldots |\fi| conditional. The
% implicit gender is male.
%    \begin{macrocode}
\newif\ifthesis@woman\thesis@womanfalse
\define@key{thesis}{gender}{%
  \def\thesis@male{m}%
  \def\thesis@arg{#1}%
  \ifx\thesis@male\thesis@arg%
    \thesis@womanfalse%
  \else%
    \thesis@womantrue%
  \fi}
%    \end{macrocode}
% \end{macro}
% \begin{macro}{\thesis@author}
% \subsubsection{The \texttt{author} key}
% The \marg{\texttt{author}=name} pair sets the author's full
% name to \textit{name}. The \textit{name} is parsed using the
% \DescribeMacro{\thesis@parseAuthor} private macro and stored
% within the following private macros:
% \begin{itemize}
%   \item\DescribeMacro{\thesis@author}|\thesis@author|
%     -- The full name of the author.
%   \item\DescribeMacro{\thesis@author@head}|\thesis@author@head|
%     -- The first space-delimited part of the name. This
%     corresponds to the author's first name.
%   \item\DescribeMacro{\thesis@author@tail}|\thesis@author@tail|
%     -- The full name without the first space-delimited part of
%     the name. This corresponds to the author's surname.
% \end{itemize}
%    \begin{macrocode}
\def\thesis@parseAuthor#1{%
  \def\thesis@author{#1}%
  \def\thesis@author@head{\expandafter\expandafter\expandafter%
    \@gobble\thesis@head#1 \relax}%
  \def\thesis@author@tail{\thesis@tail#1 \relax}}
\thesis@def{author}%
\thesis@def[author]{author@head}%
\thesis@def[author]{author@tail}%
\define@key{thesis}{author}{%
  \thesis@parseAuthor{#1}}
%    \end{macrocode}
% \end{macro}
% \begin{macro}{\thesis@id}
% \subsubsection{The \texttt{id} key}
% The \marg{\texttt{id}=identifier} pair sets the identifier
% of the thesis author to \textit{identifier}. This usually
% corresponds to a unique identifier of the author within the
% information system of the given university.
%    \begin{macrocode}
\thesis@def{id}
\define@key{thesis}{id}{%
  \def\thesis@id{#1}}
%    \end{macrocode}
% \end{macro}
% \begin{macro}{\thesis@type}
% \subsubsection{The \texttt{type} key}
% The \marg{\texttt{type}=type} pair sets the type of the thesis
% to \textit{type}. The following types of theses are recognized:
% \begin{center}\begin{tabular}{lc}\toprule
%   The thesis type & The value of \textit{type} \\\midrule
%   Bachelor's thesis & \texttt{bc} \\
%   Master's thesis & \texttt{mgr} \\
%   Doctoral thesis & \texttt{d} \\
%   Rigorous thesis & \texttt{r} \\\bottomrule
% \end{tabular}\end{center}
% The \textit{type} is stored within the private |\thesis@type|
% macro, whose implicit value is |bc|. For the ease of testing of
% the thesis type via |\ifx| conditions within style and locale
% files, the \DescribeMacro{\thesis@bachelors}|\thesis@bachelors|,
% \DescribeMacro{\thesis@masters}|\thesis@masters|,
% \DescribeMacro{\thesis@doctoral}|\thesis@doctoral| and
% \DescribeMacro{\thesis@rigorous}|\thesis@rigorous| macros
% containing the corresponding \textit{type} values are available
% as a part of the private API.
%    \begin{macrocode}
\def\thesis@bachelors{bc}
\def\thesis@masters{mgr}
\def\thesis@doctoral{d}
\def\thesis@rigorous{r}
\let\thesis@type\thesis@bachelors
\define@key{thesis}{type}{%
  \def\thesis@type{#1}}
%    \end{macrocode}
% \end{macro}
% \begin{macro}{\thesis@university}
% \subsubsection{The \texttt{university} key}
% The \marg{\texttt{university}=id} pair sets the identifier of
% the university, at which the thesis is being written,
% to \textit{id}. The \textit{id} is stored within the private
% |\thesis@university| macro, whose implicit value is \texttt{mu}.
% The |\thesis@university|
% macro is used by the |\thesis@logopath| macro and when loading
% the style and locale files using the |\thesis@load| macro. It
% allows for the usage of the class at universities other than
% the Masaryk University in Brno without the need to alter the
% code.
%    \begin{macrocode}
\def\thesis@university{mu}
\define@key{thesis}{university}{%
  \def\thesis@university{#1}}
%    \end{macrocode}
% \end{macro}
% \begin{macro}{\thesis@faculty}
% \subsubsection{The \texttt{faculty} key}
% The \marg{\texttt{faculty}=domain} pair sets the faculty, at
% which the thesis is being written, to \textit{domain}. The
% following \textit{domain} names are recognized:
% \begin{center}\begin{tabularx}{\textwidth}{Xc}\toprule
%   The Faculty & The \textit{domain} name \\\midrule
%   The Faculty of Informatics & \texttt{fi} \\
%   The Faculty of Science & \texttt{sci} \\
%   The Faculty of Law & \texttt{law} \\
%   The Faculty of Economics and Administration & \texttt{econ} \\
%   The Faculty of Social Studies & \texttt{fss} \\
%   The Faculty of Medicine & \texttt{med} \\
%   The Faculty of Education & \texttt{ped} \\
%   The Faculty of Arts & \texttt{phil} \\
%   The Faculty of Sports Studies & \texttt{fsps} \\\bottomrule
% \end{tabularx}\end{center}
% The \textit{domain} name is stored within the private
% |\thesis@faculty| macro, whose implicit value is \texttt{fi}.
%    \begin{macrocode}
\def\thesis@faculty{fi}
\define@key{thesis}{faculty}{%
  \def\thesis@faculty{#1}}
%    \end{macrocode}
% \end{macro}
% \begin{macro}{\thesis@department}
% \subsubsection{The \texttt{department} key}
% The \marg{\texttt{department}=name} pair sets the name of the
% department, at which the thesis is being written, to
% \textit{name}. The \textit{name} is stored within the private
% |\thesis@department| macro.
%    \begin{macrocode}
\thesis@def{department}
\define@key{thesis}{department}{%
  \def\thesis@department{#1}}
%    \end{macrocode}
% \end{macro}
% \begin{macro}{\thesis@departmentEn}
% \subsubsection{The \texttt{departmentEn} key}
% The \marg{\texttt{departmentEn}=name} pair sets the English
% name of the department, at which the thesis is being written, to
% \textit{name}. The \textit{name} is stored within the private
% |\thesis@departmentEn| macro.
%    \begin{macrocode}
\thesis@def{departmentEn}
\define@key{thesis}{departmentEn}{%
  \def\thesis@departmentEn{#1}}
%    \end{macrocode}
% \end{macro}
% \begin{macro}{\thesis@programme}
% \subsubsection{The \texttt{programme} key}
% The \marg{\texttt{programme}=name} pair sets the name of the
% author's study programme to \textit{name}. The \textit{name}
% is stored within the private |\thesis@programme| macro.
%    \begin{macrocode}
\thesis@def{programme}
\define@key{thesis}{programme}{%
  \def\thesis@programme{#1}}
%    \end{macrocode}
% \end{macro}
% \begin{macro}{\thesis@programmeEn}
% \subsubsection{The \texttt{programmeEn} key}
% The \marg{\texttt{programmeEn}=name} pair sets the English name
% of the author's study programme to \textit{name}. The
% \textit{name} is stored within the private |\thesis@programmeEn|
% macro.
%    \begin{macrocode}
\thesis@def{programmeEn}
\define@key{thesis}{programmeEn}{%
  \def\thesis@programmeEn{#1}}
%    \end{macrocode}
% \end{macro}
% \begin{macro}{\thesis@field}
% \subsubsection{The \texttt{field} key}
% The \marg{\texttt{field}=name} pair sets the name of the
% author's field of stufy to \textit{name}. The \textit{name}
% is stored within the private |\thesis@field| macro.
%    \begin{macrocode}
\thesis@def{field}
\define@key{thesis}{field}{%
  \def\thesis@field{#1}}
%    \end{macrocode}
% \end{macro}
% \begin{macro}{\thesis@fieldEn}
% \subsubsection{The \texttt{fieldEn} key}
% The \marg{\texttt{fieldEn}=name} pair sets the English name of
% the author's field of stufy to \textit{name}. The \textit{name}
% is stored within the private |\thesis@fieldEn| macro.
%    \begin{macrocode}
\thesis@def{fieldEn}
\define@key{thesis}{fieldEn}{%
  \def\thesis@fieldEn{#1}}
%    \end{macrocode}
% \end{macro}
% \begin{macro}{\thesis@universityLogo}
% \subsubsection{The \texttt{universityLogo} key}
% The \marg{\texttt{universityLogo}=filename} pair sets the
% filename of the logo file to be used to \textit{filename}. The
% \textit{filename} is stored within the private
% |\thesis@universityLogo| macro, whose implicit value is
% \texttt{base}. The logo file is loaded from the
% |\thesis@logopath\thesis@logo| path.
%    \begin{macrocode}
\def\thesis@universityLogo{base}
\define@key{thesis}{universityLogo}{%
  \def\thesis@universityLogo{#1}}
%    \end{macrocode}
% \end{macro}
% \begin{macro}{\thesis@facultyLogo}
% \subsubsection{The \texttt{facultyLogo} key}
% The \marg{\texttt{facultyLogo}=filename} pair sets the filename
% of the logo file to be used to \textit{filename}. The
% \textit{filename} is stored within the private
% |\thesis@facultyLogo| macro, whose implicit value is
% |\thesis@faculty|. The logo file is loaded from the
% |\thesis@logopath\thesis@logo| path.
%    \begin{macrocode}
\def\thesis@facultyLogo{\thesis@faculty}
\define@key{thesis}{facultyLogo}{%
  \def\thesis@facultyLogo{#1}}
%    \end{macrocode}
% \end{macro}
% \begin{macro}{\thesis@style}
% \subsubsection{The \texttt{style} key}
% The \marg{\texttt{style}=filename} pair sets the filename of the
% style file to be used to \textit{filename}. The \textit{filename}
% is stored within the private |\thesis@style| macro, whose
% implicit value is |\thesis@university/fithesis3-\thesis@faculty|.
% The style file is loaded from the
% |\thesis@stylepath\thesis@style| path.
%    \begin{macrocode}
\def\thesis@style{\thesis@university/fithesis3-\thesis@faculty}
\define@key{thesis}{style}{%
  \def\thesis@style{#1}}
%    \end{macrocode}
% \end{macro}
% \begin{macro}{\thesis@style@inheritance}
% \subsubsection{The \texttt{styleInheritance} key}
% The \marg{\texttt{styleInheritance}=bool} pair either enables,
% if \textit{bool} is \texttt{true} or unspecified, or disables the
% inheritance for style files. The effects of the inheritance
% are documented within the subsection documenting the
% |\thesis@load| macro. The setting can be tested using the
% |\ifthesis@style@inheritance| \ldots
% |\else| \ldots |\fi| conditional. Inheritance is enabled for
% style files by default.
%    \begin{macrocode}
\newif\ifthesis@style@inheritance\thesis@style@inheritancetrue
\define@key{thesis}{styleInheritance}[true]{%
  \def\@true{true}%
  \def\@arg{#1}%
  \ifx\@true\@arg%
    \thesis@style@inheritancetrue%
  \else%
    \thesis@style@inheritancefalse%
  \fi}
%    \end{macrocode}
% \end{macro}
% \begin{macro}{\thesis@locale}
% \subsubsection{The \texttt{locale} key}
% The \marg{\texttt{locale}=filename} pair sets the filename of the
% locale file(s) to be used to \textit{filename}. The
% \textit{filename} is stored within the private |\thesis@locale|
% macro, whose implicit value is the main language of either the
% \textsf{babel} or the \textsf{polyglossia} package, or
% \texttt{english}, when undefined. If the inheritance is disabled
% for locale files, the locale file is loaded from the
% |\thesis@localepath\thesis@locale| path.
%    \begin{macrocode}
\def\thesis@locale{%
  % Babel detection
  \ifx\languagename\undefined%
  english\else\languagename\fi}
\define@key{thesis}{locale}{%
  \def\thesis@locale{#1}}
%    \end{macrocode}
% \end{macro}
% \begin{macro}{\ifthesis@english}
% The English locale is special. Several parts of the document will
% typically be typeset in both the current locale and English.
% However, if the current locale is English, this would result in
% duplicity. To avoid this, the |\ifthesis@english| \ldots |\else|
% \ldots |\fi| conditional is made available for testing, whether
% or not the current locale is English.
%    \begin{macrocode}
\def\ifthesis@english{
  \expandafter\def\expandafter\@english\expandafter{\string%
  \english}%
  \expandafter\expandafter\expandafter\def\expandafter%
  \expandafter\expandafter\@locale\expandafter\expandafter%
  \expandafter{\expandafter\string\csname\thesis@locale\endcsname}%
  \expandafter\csname\expandafter i\expandafter f\ifx\@locale%
  \@english%
    true%
  \else%
    false%
  \fi\endcsname}
%    \end{macrocode}
% \end{macro}
% \begin{macro}{\thesis@locale@inheritance}
% \subsubsection{The \texttt{localeInheritance} key}
% The \marg{\texttt{localeInheritance}=bool} pair either enables,
% if \textit{bool} is \texttt{true} or unspecified, or disables the
% inheritance. The effects of the inheritance are
% documented within the subsection documenting the |\thesis@load| 
% macro. The setting can be tested using the
% |\ifthesis@locale@inheritance| \ldots
% |\else| \ldots |\fi| conditional. Inheritance is enabled for locale
% files by default.
%    \begin{macrocode}
\newif\ifthesis@locale@inheritance\thesis@locale@inheritancetrue
\define@key{thesis}{localeInheritance}[true]{%
  \def\@true{true}%
  \def\@arg{#1}%
  \ifx\@true\@arg%
    \thesis@locale@inheritancetrue%
  \else%
    \thesis@locale@inheritancefalse%
  \fi}
%    \end{macrocode}
% \end{macro}
% \subsubsection{The \texttt{date} key}
% The \marg{\texttt{date}=date} pair sets the date of the thesis
% defence to \textit{date}, where \textit{date} is a string
% in the \texttt{YYYY/MM/DD} format, where \texttt{YYYY} stands
% for full year, \texttt{MM} stands for month and \texttt{DD}
% stands for day. The \textit{date} is parsed and stored using
% the \DescribeMacro{\thesis@parseDate}|\thesis@parseDate| private
% macro within the following private macros:
% \begin{itemize}
%   \item\DescribeMacro{\thesis@date}|\thesis@date| -- The whole
%     date
%   \item\DescribeMacro{\thesis@year}|\thesis@year| -- The year
%   \item\DescribeMacro{\thesis@month}|\thesis@month| -- The month
%   \item\DescribeMacro{\thesis@day}|\thesis@day| -- The day of
%     month
%   \item\DescribeMacro{\thesis@season}|\thesis@season| -- Expands
%     to either:
%     \begin{itemize}
%       \item\texttt{winter} if \texttt{MM} $<7$.
%       \item\texttt{summer} if \texttt{MM} $\geq7$.
%     \end{itemize}
%   \item\DescribeMacro{\thesis@academicYear}|\thesis@academicYear|
%     -- The academic year of the given semester:
%     \begin{itemize}
%       \item\texttt{YYYY/YYYY}$+1$ in case of a summer semester
%       \item\texttt{YYYY}$-1$\texttt{/YYYY} in case of a winter
%            semester
%     \end{itemize}
% \end{itemize}
% To set up the default values, the |\thesis@parseDate| macro is
% called with the fully expanded |\the\year/\the\month/\the\day|
% string.
%    \begin{macrocode}
\def\thesis@parseDate#1/#2/#3|{{
  % Basic info
  \gdef\thesis@date{#1/#2/#3}%
  \gdef\thesis@year{#1}%
  \gdef\thesis@month{#2}%
  \gdef\thesis@day{#3}%
  
  % Season and academic year
  \newcount\@year \expandafter\@year \thesis@year \relax%
  \newcount\@month\expandafter\@month\thesis@month\relax%
  \ifnum\@month<7%
    \gdef\thesis@season{winter}%
    \advance\@year-1\edef\@yearA{\the\@year}%
    \advance\@year 1\edef\@yearB{\the\@year}%
  \else%
    \gdef\thesis@season{summer}%
                    \edef\@yearA{\the\@year}%
    \advance\@year 1\edef\@yearB{\the\@year}%
  \fi%
  \global\edef\thesis@academicYear{\@yearA/\@yearB}}}

\edef\thesis@date{\the\year/\the\month/\the\day}%
\expandafter\thesis@parseDate\thesis@date|%

\define@key{thesis}{date}{{%
  \edef\@date{#1}%
  \expandafter\thesis@parseDate\@date|}}
%    \end{macrocode}
% \begin{macro}{\thesis@place}
% \subsubsection{The \texttt{place} key}
% The \marg{\texttt{place}=place} pair sets the location of the
% faculty, at which the thesis is being prepared, to \textit{place}.
% The \textit{place} is stored within the private |\thesis@place|
% macro, whose implicit value is \texttt{Brno}.
%    \begin{macrocode}
\def\thesis@place{Brno}
\define@key{thesis}{place}{%
  \def\thesis@place{#1}}
%    \end{macrocode}
% \end{macro}
% \begin{macro}{\thesis@title}
% \subsubsection{The \texttt{title} key}
% The \marg{\texttt{title}=title} pair sets the title of the
% thesis to \textit{title}. The \textit{title} is stored within the
% private |\thesis@title| macro.
%    \begin{macrocode}
\thesis@def{title}
\define@key{thesis}{title}{%
  \def\thesis@title{#1}}
%    \end{macrocode}
% \end{macro}
% \begin{macro}{\thesis@TeXtitle}
% \subsubsection{The \texttt{TeXtitle} key}
% The \marg{\texttt{TeXtitle}=title} pair sets the \TeX\ title of
% the thesis to \textit{title}. The \textit{title} is used, when
% typesetting the title, whereas |\thesis@title| is a plain text,
% which gets included in the PDF header of the
% resulting document as well as in the \BibTeX\ file containing
% the bibliographical entry for the thesis. The \textit{title}
% is stored within the private |\thesis@TeXtitle| macro, whose
% implicit value is |\thesis@title|.
%    \begin{macrocode}
\def\thesis@TeXtitle{\thesis@title}
\define@key{thesis}{TeXtitle}{%
  \def\thesis@TeXtitle{#1}}
%    \end{macrocode}
% \end{macro}
% \begin{macro}{\thesis@titleEn}
% \subsubsection{The \texttt{titleEn} key}
% The \marg{\texttt{titleEn}=title} pair sets the English title of
% the thesis to \textit{title}. The \textit{title} is stored within
% the private |\thesis@titleEn| macro.
%    \begin{macrocode}
\thesis@def{titleEn}
\define@key{thesis}{titleEn}{%
  \def\thesis@titleEn{#1}}
%    \end{macrocode}
% \end{macro}
% \begin{macro}{\thesis@TeXtitleEn}
% \subsubsection{The \texttt{TeXtitleEn} key}
% The \marg{\texttt{TeXtitleEn}=title} pair sets the English \TeX\ 
% title of the thesis to \textit{title}. The \textit{title} is
% used, when typesetting the title, whereas |\thesis@titleEn| is a
% plain text. The \textit{title} is stored within the private
% |\thesis@TeXtitleEn| macro, whose implicit value is
% |\thesis@titleEn|.
%    \begin{macrocode}
\def\thesis@TeXtitleEn{\thesis@titleEn}
\define@key{thesis}{TeXtitleEn}{%
  \def\thesis@TeXtitleEn{#1}}
%    \end{macrocode}
% \end{macro}
% \begin{macro}{\thesis@keywords}
% \subsubsection{The \texttt{keywords} key}
% The \marg{\texttt{keywords}=list} pair sets the keywords of the
% thesis to the comma-delimited \textit{list}. The \textit{list}
% is stored within the private |\thesis@keywords| macro.
%    \begin{macrocode}
\thesis@def{keywords}
\define@key{thesis}{keywords}{%
  \def\thesis@keywords{#1}}
%    \end{macrocode}
% \end{macro}
% \begin{macro}{\thesis@keywordsEn}
% \subsubsection{The \texttt{keywordsEn} key}
% The \marg{\texttt{keywordsEn}=list} pair sets the English
% keywords of the thesis to the comma-delimited \textit{list}. The
% \textit{list} is stored within the private |\thesis@keywordsEn|
% macro.
%    \begin{macrocode}
\thesis@def{keywordsEn}
\define@key{thesis}{keywordsEn}{%
  \def\thesis@keywordsEn{#1}}
%    \end{macrocode}
% \end{macro}
% \begin{macro}{\thesis@abstract}
% \subsubsection{The \texttt{abstract} key}
% The \marg{\texttt{abstract}=text} pair sets the abstract of the
% thesis to \textit{text}. The \textit{text} is stored within the
% private |\thesis@abstract| macro.
%    \begin{macrocode}
\thesis@def{abstract}
\long\def\KV@thesis@abstract#1{%
  \long\def\thesis@abstract{#1}}
%    \end{macrocode}
% \end{macro}
% \begin{macro}{\thesis@abstractEn}
% \subsubsection{The \texttt{abstractEn} key}
% The \marg{\texttt{abstractEn}=text} pair sets the English
% abstract of the thesis to \textit{text}. The \textit{text}
% is stored within the private |\thesis@abstractEn| macro.
%    \begin{macrocode}
\thesis@def{abstractEn}
\long\def\KV@thesis@abstractEn#1{%
  \long\def\thesis@abstractEn{#1}}
%    \end{macrocode}
% \end{macro}
% \begin{macro}{\thesis@advisor}
% \subsubsection{The \texttt{advisor} key}
% The \marg{\texttt{advisor}=name} pair sets the thesis advisor's
% full name to \textit{name}. The \textit{name} is stored within
% the private |\thesis@advisor| macro.
%    \begin{macrocode}
\thesis@def{advisor}
\define@key{thesis}{advisor}{\def\thesis@advisor{#1}}
%    \end{macrocode}
% \end{macro}
% \begin{macro}{\thesis@thanks}
% \subsubsection{The \texttt{thanks} key}
% The \marg{\texttt{thanks}=text} pair sets the acknowledgement
% text to \textit{text}. The \textit{text} is stored within
% the private |\thesis@thanks| macro.
%    \begin{macrocode}
\long\def\KV@thesis@thanks#1{%
  \long\def\thesis@thanks{#1}}
%    \end{macrocode}
% \end{macro}
% \begin{macro}{\thesis@assignmentFiles}
% \subsubsection{The \texttt{assignment} key}
% The \marg{\texttt{assignment}=list} pair sets the comma-separated
% list of paths (and optional page spec specifiers, see the
% \textsf{pdfpages} package |\includepdfmerge| command
% documentation) to the pdf files containing the thesis assignment
% to \textit{list}. The \textit{list} is stored within the
% |\thesis@assignmentFiles| private macro. When defined, the PDF
% files are injected into the resulting document instead of the
% placeholder |\thesis@@{assignment}| string.
%    \begin{macrocode}
\define@key{thesis}{assignment}{%
  \def\thesis@assignmentFiles{#1}}
%    \end{macrocode}
% \end{macro}
% \begin{macro}{\ifthesis@auto}
% \subsubsection{The \texttt{autoLayout} key}
% The \marg{\texttt{autoLayout}=bool} pair either enables,
% if \textit{bool} is \texttt{true} or unspecified, or disables
% autolayout. Autolayout injects the
% |\thesis@preamble| and |\thesis@postamble| private macros
% at the beginning and the end of the document, respectively. The
% setting can be tested using the |\ifthesis@auto| \ldots |\else|
% \ldots |\fi| conditional. The autolayout is enabled by default.
%    \begin{macrocode}
\newif\ifthesis@auto\thesis@autotrue
\define@key{thesis}{autoLayout}[true]{%
  \def\@true{true}%
  \def\@arg{#1}%
  \ifx\@true\@arg%
    \thesis@autotrue%
  \else%
    \thesis@autofalse%
  \fi}
%    \end{macrocode}
% \end{macro} ^^A The nested \ifthesis@auto macro definition
% \end{macro} ^^A The \thesissetup macro definition
% The \DescribeMacro{\thesis@preamble}|\thesis@postamble|
% and \DescribeMacro{\thesis@postamble}|\thesis@preamble|
% private macros are defined as empty strings by default and are
% subject to redefinition by the style files.
%    \begin{macrocode}
\def\thesis@preamble{}
\def\thesis@postamble{}
%    \end{macrocode}
% \subsection{The \cs{thesislong} macro}
% \begin{macro}{\thesislong}
% The public macro |\thesislong|\marg{key}\marg{value},
% where \textit{value} may contain multiple paragraphs of text, can
% be used for the following \textit{key}s as an alternative to the
% |\thesissetup| public macro, which only permits a single
% paragraph as the \textit{value}:
% \begin{itemize}
%   \item\texttt{abstract}
%   \item\texttt{abstractEn}
%   \item\texttt{thanks}
%   \item\texttt{declaration}
% \end{itemize}
%    \begin{macrocode}
\long\def\thesislong#1#2{%
  \csname KV@thesis@#1\endcsname{#2}}
%    \end{macrocode}
% \end{macro}
% \section{Private API}
% \subsection{Main routine}\label{sec:thesis@load}
% \begin{macro}{\thesis@load}
% The |\thesis@load| macro is responsible for preparing the
% environment for, and consequently loading, the necessary locale
% and style files. By default, the |\thesis@load| macro gets
% expanded at the end of the preamble,
% but it can be inserted manually prior to that, if necessary to
% prevent package clashes. The \DescribeMacro{\ifthesis@loaded}
% |\ifthesis@loaded| semaphore ensures that the expansion is only
% performed once.
%    \begin{macrocode}
\newif\ifthesis@loaded\thesis@loadedfalse
\AtEndPreamble{\thesis@load}
\def\thesis@load{%
  \ifthesis@loaded\else%
    \thesis@loadedtrue
    \makeatletter%
%    \end{macrocode}
% First, the main locale file is loaded using the
% |\thesis@requireLocale| macro.
%    \begin{macrocode}
      % Load the main locale file
      \thesis@requireLocale{\thesis@locale}
%    \end{macrocode}
% Consequently, the style files are loaded with the class options
% passed onto them. If inheritance is enabled for style files, then
% each of the following files is loaded in sequence, if they exist:
% \begin{enumerate}
%   \item|\thesis@stylepath fithesis3-base.sty|
%   \item|\thesis@stylepath\thesis@university/fithesis3-base.sty|
%   \item|\thesis@stylepath\thesis@style.sty|
% \end{enumerate}If inheritance is disabled for style files,
% then only the |\thesis@stylepath\thesis@|^^A
% \discretionary{}{}{}|style.sty| file is loaded. The
% \texttt{fithesis3-} prefix serves to prevent package clashes
% with other similarly named package files within the \TeX\
% directory structure.
%    \begin{macrocode}
      \ifthesis@style@inheritance
        \thesis@requireStyle{\thesis@stylepath fithesis3-base}
        \thesis@requireStyle{\thesis@stylepath\thesis@university%
          /fithesis3-base}
      \fi
      \thesis@requireStyle{\thesis@stylepath\thesis@style}
%    \end{macrocode}
% With the placeholder strings loaded from the locale files, we
% can now inject metadata into the resulting PDF file. To this
% end, the \textsf{hyperref} package is conditionally included with
% the \texttt{unicode} option. Consequently, the following values
% are assigned to the PDF headers:\begin{itemize}
%   \item\texttt{Title} is set to |\thesis@title|.
%   \item\texttt{Author} is set to |\thesis@author|.
%   \item\texttt{Keywords} is set to |\thesis@keywords|.
%   \item\texttt{Creator} is set to \texttt{\thesis@version}.
% \end{itemize}
%    \begin{macrocode}
       \thesis@require{hyperref}%
      {\hypersetup{unicode,
         pdftitle={\thesis@title},%
         pdfauthor={\thesis@author},%
         pdfkeywords={\thesis@keywords},%
         pdfcreator={\thesis@version},%
     }}%
%    \end{macrocode}
% If autolayout is enabled, the |\thesis@preamble| and
% |\thesis@postamble| macros are scheduled for expansion at the
% beginning and at the end of the document, respectively.
%    \begin{macrocode}
      \ifthesis@auto%
        \AtBeginDocument{\thesis@preamble}%
        \AtEndDocument{\thesis@postamble}%
      \fi%
%    \end{macrocode}
% Lastly, a \BibTeX\ file named |\jobname.bib| containing the
% bibliographical entry for the thesis is scheduled to be
% generated at the end of the document in the working directory
% using the |\thesis@bibgen| macro and the
% \DescribeMacro{\thesis@pages}|\thesis@pages| private macro
% definition containing the length of the document is scheduled to
% be included in the auxiliary file.
%    \begin{macrocode}
      \AtEndDocument{%
        % Define \thesis@pages for the next run
        \write\@auxout{\noexpand\gdef\noexpand%
          \thesis@pages{\thepage}}}
    \makeatother%
  \fi}
%    \end{macrocode}
% \end{macro}
% \subsection{File manipulation macros}
% \begin{macro}{\thesis@exists}
% The |\thesis@exists|\marg{file}\marg{tokens} private macro is
% used to test for the existence of a given \textit{file}. If the
% \textit{file} exists, the macro expands to \textit{tokens}.
% Otherwise, a class warning is written to the output.
%    \begin{macrocode}
\def\thesis@input#1{%
  \thesis@exists{#1}{\input{#1}}}
%    \end{macrocode}
% \end{macro}\begin{macro}{\thesis@input}
% The |\thesis@input|\marg{file} private macro inputs the given
% \textit{file}, if it exists.
%    \begin{macrocode}
\def\thesis@exists#1#2{%
  \IfFileExists{#1}{#2}{%
  \ClassWarning{fithesis3}{File #1 doesn't exist}}}
%    \end{macrocode}
% \end{macro}\begin{macro}{\thesis@require}
% The |\thesis@require| \marg{package} expands to
% |\RequirePackage|\marg{package}, if the specified
% \textit{package} has not yet been loaded. This generally serves
% to prevents options clashes, when the options with which the
% package had been loaded are of no consequence.
%    \begin{macrocode}
\def\thesis@require#1{%
  \@ifpackageloaded{#1}{}{\RequirePackage{#1}}}
%    \end{macrocode}
% \end{macro}\begin{macro}{\thesis@requireStyle}
% The |\thesis@requireStyle|\marg{package} expands to
% |\RequirePackageWithOptions|\marg{package}, if the specified
% \textit{package} exists and has not yet been loaded. This
% generally serves to load style files.
%    \begin{macrocode}
\def\thesis@requireStyle#1{\thesis@exists{#1.sty}{%
  \@ifpackageloaded{#1}{}{\RequirePackageWithOptions{#1}}}}
%    \end{macrocode}
% \end{macro}\begin{macro}{\thesis@requireLocale}
% The |\thesis@requireLocale|\marg{locale} private macro loads
% locale files of the specified \textit{locale}, if they haven't
% been loaded before. If inheritance is enabled for locale files,
% then the following directories are used:
% \begin{enumerate}
%   \item|\thesis@localepath|
%   \item|\thesis@localepath\thesis@university/|
%   \item|\thesis@localepath\thesis@university/\thesis@faculty/|
% \end{enumerate}If inheritance is disabled for locale files,
% then only the |\thesis@localepath| directory is used. The macro
% can be used within both locale and style files, although the
% usage within locale files is strongly discouraged to prevent
% circular dependencies.
%    \begin{macrocode}
\def\thesis@requireLocale#1{%
  % Prevent redundant entries
  \expandafter\ifx\csname thesis@#1@required\endcsname\relax%
    \expandafter\def\csname thesis@#1@required\endcsname{}%
      \thesis@input{\thesis@localepath#1.def}
      \ifthesis@locale@inheritance%
        \thesis@input{\thesis@localepath\thesis@university/#1.def}% 
        \thesis@input{\thesis@localepath\thesis@university/%
          \thesis@faculty/#1.def}% 
      \fi%
  \fi}
%    \end{macrocode}\end{macro}
% \subsection{String manipulation macros}
% \begin{macro}{\thesis@}
% The |\thesis@|\marg{name} macro expands to |\thesis@|
% \textit{name}, where \textit{name} gets fully expanded and can
% therefore contain active characters and command sequences.
%    \begin{macrocode}
\def\thesis@#1{\csname thesis@#1\endcsname}
%    \end{macrocode}
% \end{macro}\begin{macro}{\thesis@@}
% The |\thesis@@|\marg{name} macro expands to |\thesis@|
% \textit{locale}|@|\textit{name}, where \textit{locale}
% corresponds to the name of the current locale. 
% \textit{name} gets fully expanded and can
% therefore contain active characters and command sequences.
%    \begin{macrocode}
\def\thesis@@#1{\thesis@{\thesis@locale @#1}}
%    \end{macrocode}
% \end{macro}
% The \DescribeMacro{\thesis@lower}|\thesis@lower|
% and \DescribeMacro{\thesis@upper}|\thesis@upper|
% private macros are used for upper- and lowercasing within
% locale files. To cast the |\thesis@|\textit{name} macro
% to the lower- or uppercase, |\thesis@lower{|\textit{name}|}| or
% |\thesis@upper{|\textit{name}|}| would be used, respectively.
% \textit{name} gets fully expanded and can
% therefore contain active characters and command sequences.
%    \begin{macrocode}
\def\thesis@lower#1{{%
  \let\ea\expandafter%
  \ea\ea\ea\ea\ea\ea\ea\ea\ea\ea\ea\ea\ea\ea\ea\lowercase\ea\ea\ea
  \ea\ea\ea\ea\ea\ea\ea\ea\ea\ea\ea\ea{\ea\ea\ea\ea\ea\ea\ea\ea\ea
  \ea\ea\ea\ea\ea\ea\@gobble\ea\ea\ea\string\ea\csname\csname the%
  sis@#1\endcsname\endcsname}}}
\def\thesis@upper#1{{%
  \let\ea\expandafter%
  \ea\ea\ea\ea\ea\ea\ea\ea\ea\ea\ea\ea\ea\ea\ea\uppercase\ea\ea\ea
  \ea\ea\ea\ea\ea\ea\ea\ea\ea\ea\ea\ea{\ea\ea\ea\ea\ea\ea\ea\ea\ea
  \ea\ea\ea\ea\ea\ea\@gobble\ea\ea\ea\string\ea\csname\csname the%
  sis@#1\endcsname\endcsname}}}
%    \end{macrocode}
% The \DescribeMacro{\thesis@@lower}|\thesis@@lower|
% and \DescribeMacro{\thesis@@upper}|\thesis@@upper|
% private macros are used for upper- and lowercasing current
% \textit{locale} strings within style files. To cast the
% |\thesis@|\textit{locale}|@|\textit{name} macro to the
% lower- or uppercase, |\thesis@@lower{|\textit{name}|}| or
% |\thesis@@upper{|\textit{name}|}| would be used,
% respectively. \textit{name} gets fully expanded and can
% therefore contain active characters and command sequences.
%    \begin{macrocode}
\def\thesis@@lower#1{\thesis@lower{\thesis@locale @#1}}
\def\thesis@@upper#1{\thesis@upper{\thesis@locale @#1}}
%    \end{macrocode}
% The \DescribeMacro{\thesis@head}|\thesis@head|
% and \DescribeMacro{\thesis@tail}|\thesis@tail|
% private macros are used for retrieving a head or a tail of
% space-separated token sequences, which end with |\relax|.
%    \begin{macrocode}
\def\thesis@head#1 #2{%
  \ifx\relax#2%
    \expandafter\@gobbletwo%
  \else%
    \ #1%
  \fi%
  \thesis@head#2}%
\def\thesis@tail#1 #2{%
  \ifx\relax#2%
    #1%
    \expandafter\@gobbletwo%
  \fi%
  \thesis@tail#2}%
%    \end{macrocode}
% \subsection{General purpose macros}
% The \DescribeMacro{\thesis@pages}|\thesis@pages| macro is defined
% at the beginning of the second \LaTeX\ run as a part of the main
% routine (see section \ref{sec:thesis@load}). During the first
% run, the macro expands to \texttt{??}.
%    \begin{macrocode}
\ifx\thesis@pages\undefined\def\thesis@pages{??}\fi
%    \end{macrocode}
% \iffalse
%</class>
% ^^A Old fithesis classes
%<*oldclass1>

\NeedsTeXFormat{LaTeX2e}
\ProvidesClass{oldfithesis1}[2015/03/04 old fithesis will load fithesis3 MU thesis class]

\ClassWarning{oldfithesis1}{%
  You are using the fithesis class, which has been deprecated.
  The fithesis3 class will be used instead.
  For more information, see <https://www.fi.muni.cz/tech/unix/tex/fithesis.xhtml>%
}\LoadClass{fithesis3}

%</oldclass1>
%
%<*oldclass2>

\NeedsTeXFormat{LaTeX2e}
\ProvidesClass{oldfithesis2}[2015/03/04 old fithesis2 will load fithesis3 MU thesis class]

\ClassWarning{oldfithesis2}{%
  You are using the fithesis2 class, which has been deprecated.
  The fithesis3 class will be used instead.
  For more information, see <https://www.fi.muni.cz/tech/unix/tex/fithesis.xhtml>%
}\LoadClass{fithesis3}

%</oldclass2>
% \fi
%
% \subsection{Locale files}
% \label{sec:locale-files}
% Locale files contain macro definitions for various locales. They
% live in the \texttt{locale/} subtree and they are loaded during
% the main routine (see section \ref{sec:thesis@load}).
%
% When creating a new locale file, it is advisable to create one
% self-contained \texttt{dtx} file, which is then partitioned into
% locale files via the \textsf{docstrip} tool based on the
% respective \texttt{ins} file. A \DescribeMacro{\file} macro
% |\file|\marg{filename} is available for the sectioning the
% documentation of various files within the \texttt{dtx} file.
% \textit{filename}. For more information about \texttt{dtx} files
% and the \textsf{docstrip} tool, consult the \textsf{dtxtut,
% docstrip, doc} and \textsf{ltxdoc} manuals.
%
% \subsubsection{Interface}
% The union of locale files named \textit{locale}\texttt{.def},
% where \textit{locale} is the result of the expansion of
% |\thesis@locale|, loaded via main routine's inheritance scheme
% (see section \ref{sec:thesis@load}) needs to define the following
% private macros:
% \begin{itemize}
%   \item|\thesis@|\textit{locale}|@universityName| -- The name of
%     the university
%   \item|\thesis@|\textit{locale}|@facultyName| -- The name of the
%     faculty
%   \item|\thesis@|\textit{locale}|@assignment| -- Instructions to
%     replace the current page with the official thesis assignment
%   \item|\thesis@|\textit{locale}|@declaration| -- The declaration
%     text
%   \item|\thesis@|\textit{locale}|@fieldTitle| -- The title of
%     the field of study entry
%   \item|\thesis@|\textit{locale}|@advisorTitle| -- The title of
%     the advisor
%   \item|\thesis@|\textit{locale}|@authorTitle| -- The title of
%     the author
%   \item|\thesis@|\textit{locale}|@abstractTitle| -- The title of
%     the abstract section
%   \item|\thesis@|\textit{locale}|@keywordsTitle| -- The title of
%     the keywords section
%   \item|\thesis@|\textit{locale}|@thanksTitle| -- The title of
%     the acknowledgement section
%   \item|\thesis@|\textit{locale}|@declarationTitle| -- The title
%     of the declaration section
%   \item|\thesis@|\textit{locale}|@idTitle| -- The title of the
%     thesis author's identifier field
%   \item|\thesis@|\textit{locale}|@winter| -- The name of the
%     winter semester
%   \item|\thesis@|\textit{locale}|@summer| -- The name of the
%     summer semester
%   \item|\thesis@|\textit{locale}|@semester| -- The full name of
%     the current semester
%   \item|\thesis@|\textit{locale}|@typeName| -- The name of the
%     thesis type
% \end{itemize}
%
% \def\file#1{\paragraph{The \texttt{#1} file}}
% \subsubsection{English locale files}
% % \file{locale/fithesis-english.def}
% This is the base file of the English locale.\iffalse
%<*base>
% \fi\begin{macrocode}
\ProvidesFile{fithesis/locale/fithesis-english.def}[2017/09/08]
%    \end{macrocode}
% The locale file defines all the private macros mandated by the
% locale file interface.
% \begin{macrocode}

% Placeholders
\gdef\thesis@english@universityName{University name}
\gdef\thesis@english@facultyName{Faculty name}
%    \end{macrocode}
% \changes{v0.3.47}{2017/07/09}{Moved the \cs{ifthesis@digital}
%   tests from \texttt{locale/*.def} to \texttt{locale/mu/*.def},
%   since \cs{ifthesis@digital} is undefined in
%   \texttt{fithesis3.cls}. [VN]}
%    \begin{macrocode}
\gdef\thesis@english@assignment{%
  This is where a copy of the official signed thesis assignment
  is located in the printed version of the document.}
\gdef\thesis@english@declaration{Declaration text ...}

% Csquotes style
\gdef\thesis@english@csquotesStyle{english}

% Time strings
\gdef\thesis@english@spring{Spring}
\gdef\thesis@english@fall{Fall}
\gdef\thesis@english@semester{%
  \thesis@{english@\thesis@season} \thesis@seasonYear}
\gdef\thesis@english@formattedDate{{%
  \thesis@day.
  \newcount\@month\expandafter\@month\thesis@month\relax
  \ifnum\@month=1%
    January
  \else\ifnum\@month=2%
    February
  \else\ifnum\@month=3%
    March
  \else\ifnum\@month=4%
    April
  \else\ifnum\@month=5%
    May
  \else\ifnum\@month=6%
    June
  \else\ifnum\@month=7%
    July
  \else\ifnum\@month=8%
    August
  \else\ifnum\@month=9%
    September
  \else\ifnum\@month=10%
    October
  \else\ifnum\@month=11%
    November
  \else\ifnum\@month=12%
    December
  \else
    <<unknown month (\the\@month)>>
  \fi\fi\fi\fi\fi\fi
  \fi\fi\fi\fi\fi\fi
  \thesis@year}}

% Miscellaneous
\gdef\thesis@english@authorSignature{Author's signature}
\gdef\thesis@english@fieldTitle{Field of study}
\gdef\thesis@english@advisorTitle{Advisor}
\gdef\thesis@english@authorTitle{Author}
\gdef\thesis@english@abstractTitle{Abstract}
\gdef\thesis@english@keywordsTitle{Keywords}
%    \end{macrocode}
% \changes{v0.3.48}{2017/09/08}{Changed
%   \cs{thesis@english@thanksTitle} to plural. [VN]}
%    \begin{macrocode}
\gdef\thesis@english@thanksTitle{Acknowledgements}
\gdef\thesis@english@declarationTitle{Declaration}
\gdef\thesis@english@idTitle{ID}
\gdef\thesis@english@typeName@sempaper{Seminar Paper}
\gdef\thesis@english@typeName@bachelors{Bachelor's Thesis}
\gdef\thesis@english@typeName@masters{Master's Thesis}
\gdef\thesis@english@typeName@proposal{Thesis Proposal}
\gdef\thesis@english@typeName@doctoral{Doctoral Thesis}
\gdef\thesis@english@typeName@rigorous{Rigorous Thesis}
\gdef\thesis@english@typeName{%
  \ifx\thesis@type\thesis@sempaper
    \thesis@english@typeName@sempaper
  \else\ifx\thesis@type\thesis@bachelors
    \thesis@english@typeName@bachelors
  \else\ifx\thesis@type\thesis@masters
    \thesis@english@typeName@masters
  \else\ifx\thesis@type\thesis@proposal
    \thesis@english@typeName@proposal
  \else\ifx\thesis@type\thesis@doctoral
    \thesis@english@typeName@doctoral
  \else\ifx\thesis@type\thesis@rigorous
    \thesis@english@typeName@rigorous
  \else
    <<Unknown thesis type (\thesis@type)>>%
  \fi\fi\fi\fi\fi\fi}
%    \end{macrocode}\iffalse
%</base>
% \fi\file{locale/mu/fithesis-english.def}
% This is the English locale file specific to the Masaryk
% University in Brno. It replaces the \texttt{universityName}
% placeholder with the correct value and defines the
% \texttt{declaration} and \texttt{idTitle} strings.
% \iffalse
%<*mu>
% \fi\begin{macrocode}
\ProvidesFile{fithesis/locale/mu/fithesis-english.def}[2017/07/09]
\gdef\thesis@english@universityName{Masaryk University}
\gdef\thesis@english@declaration{%
  Hereby I declare that this paper is my original authorial work,
  which I have worked out on my own. All sources, references, and
  literature used or excerpted during elaboration of this work are
  properly cited and listed in complete reference to the due source.}

% Placeholders
%    \end{macrocode}
% \changes{v0.3.47}{2017/07/09}{Moved the \cs{ifthesis@digital}
%   tests from \texttt{locale/*.def} to \texttt{locale/mu/*.def},
%   since \cs{ifthesis@digital} is undefined in
%   \texttt{fithesis3.cls}. [VN]}
%    \begin{macrocode}
\gdef\thesis@english@assignment{%
  \ifthesis@digital@
  \else
  \fi}
%    \end{macrocode}
% \changes{v0.3.47}{2017/07/09}{Moved the \cs{ifthesis@digital}
%   tests from \texttt{locale/*.def} to \texttt{locale/mu/*.def},
%   since \cs{ifthesis@digital} is undefined in
%   \texttt{fithesis3.cls}. [VN]}
% \changes{v0.3.47}{2017/07/09}{Added an
% \cs{ifthesis@blocks@assignment@hideIfDigital@} test to the
% definition of the \texttt{assignment} string for the Masaryk
% University in Brno. [VN]}
%    \begin{macrocode}
\gdef\thesis@english@assignment{%
  \ifthesis@blocks@assignment@hideIfDigital@
    \ifthesis@digital@
      This is where a copy of the official signed thesis assignment
      is located in the printed version of the document.
    \else
      Replace this page with a copy of the official signed thesis
      assignment.
    \fi
  \else
    Set the PDF document containing the official signed thesis
    assignment using the <<assignment>> key.
  \fi}

% Bibliographic entry
\gdef\thesis@english@bib@title{Bibliographic record}
\gdef\thesis@english@bib@pages{p}
%    \end{macrocode}
% \changes{v0.3.46}{2017/06/02}{Lifted the \texttt{bib@author},
%   \texttt{bib@thesisTitle}, and \texttt{bib@advisor} strings from
%   \texttt{locale/mu/sci/*.def} to \texttt{locale/mu/*.def},
%   so that they can be shared with \texttt{locale/mu/econ/*.def}.
%   [VN]}
%    \begin{macrocode}
\global\let\thesis@english@bib@author\thesis@english@authorTitle
\gdef\thesis@english@bib@thesisTitle{Title of Thesis}
\gdef\thesis@english@bib@advisor{Supervisor}

% Miscellaneous
\gdef\thesis@english@idTitle{UČO}
%    \end{macrocode}\iffalse
%</mu>
% \fi\file{locale/mu/law/fithesis-english.def}
% This is the English locale file specific to the Faculty of Law at
% the Masaryk University in Brno. It replaces the
% \texttt{facultyName} placeholder with the correct value and
% defines the \texttt{facultyLongName} required by the
% |\thesis@blocks@cover| and the |\thesis@blocks@titlePage| blocks.
% \iffalse
%<*mu/law>
% \fi\begin{macrocode}
\ProvidesFile{fithesis/locale/mu/law/fithesis-english.def}[2015/06/26]
\gdef\thesis@english@facultyName{Faculty of Law}
\gdef\thesis@english@facultyLongName{The Faculty of Law of the
  Masaryk University}
%    \end{macrocode}\iffalse
%</mu/law>
% \fi\file{locale/mu/fsps/fithesis-english.def}
% This is the English locale file specific to the Faculty of Sports
% Studies at the Masaryk University in Brno. It replaces the
% \texttt{facultyName} placeholder with the correct value and
% redefines the \texttt{fieldTitle} string in accordance with the
% common usage at the faculty.
% \iffalse
%<*mu/fsps>
% \fi\begin{macrocode}
\ProvidesFile{fithesis/locale/mu/fsps/fithesis-english.def}[2017/06/02]

% Placeholders
\gdef\thesis@english@facultyName{Faculty of Sports Studies}

% Miscellaneous
\gdef\thesis@english@fieldTitle{Specialization}
%    \end{macrocode}\iffalse
%</mu/fsps>
% \fi\file{locale/mu/fss/fithesis-english.def}
% This is the English locale file specific to the Faculty of Social
% Studies at the Masaryk University in Brno. It replaces the
% \texttt{facultyName} and \texttt{assignment} strings with the
% correct values.
% \iffalse
%<*mu/fss>
% \fi\begin{macrocode}
\ProvidesFile{fithesis/locale/mu/fss/fithesis-english.def}[2016/05/25]

% Placeholders
\gdef\thesis@english@facultyName{Faculty of Social Studies}
\gdef\thesis@english@assignment{%
  \ifthesis@digital@
    This is where a copy of the official signed thesis assignment
    or a copy of the Statement of an Author or both are located
    in the printed version of the document.
  \else
    Replace this page with a copy of the official signed thesis
    assignment or a copy of the Statement of an Author or both,
    depending on the requirements of the respective department.
  \fi}
%    \end{macrocode}\iffalse
%</mu/fss>
% \fi\file{locale/mu/econ/fithesis-english.def}
% This is the English locale file specific to the Faculty of
% Economics and Administration at the Masaryk University in Brno.
% It replaces the \texttt{facultyName} and \texttt{abstractTitle}
% placeholders with the correct value. The locale file also defines
% the private macros required by the
% |\thesis@blocks@|\discretionary{}{}{}|bibEntry| block defined
% within the \texttt{style/mu/fithesis-econ.sty} style file.
% \iffalse
%<*mu/econ>
% \fi\begin{macrocode}
\ProvidesFile{fithesis/locale/mu/econ/fithesis-english.def}[2017/06/02]

% Placeholders
\gdef\thesis@english@facultyName{Faculty of Economics
  and Administration}

% Bibliographic entry
%    \end{macrocode}
% \changes{v0.3.46}{2017/06/02}{Defined strings required by
%   \cs{thesis@blocks@bibEntry} from
%   \texttt{style/mu/fithesis-econ.sty} in
%   \texttt{locale/mu/econ/*.def}. [VN]}
%    \begin{macrocode}
\gdef\thesis@english@bib@department{Department}
\gdef\thesis@english@bib@year{Year of Defense}

% Miscellaneous
%    \end{macrocode}
% \changes{v0.3.46}{2017/06/02}{Updated the
%   \cs{abstractTitle} string in \texttt{locale/mu/econ/*.def} in
%   accordance with the 2/2017 dean's directive. The patch was
%   submitted by Jana Ratajská. [VN]}
%    \begin{macrocode}
\gdef\thesis@english@abstractTitle{Annotation}
%    \end{macrocode}\iffalse
%</mu/econ>
% \fi\file{locale/mu/med/fithesis-english.def}
% This is the English locale file specific to the Faculty of
% Medicine at the Masaryk University in Brno.
% It replaces the \texttt{facultyName} placeholder with the
% correct value and redefines the \texttt{abstractTitle} string
% with the common usage at the faculty. The file also defines
% the \texttt{bib@title} and \texttt{bib@pages} strings required
% by the |\thesis@blocks@bibEntry| block defined within the
% \texttt{style/mu/\discretionary{}{}{}fithesis-med.sty}
% style file.
% \iffalse
%<*mu/med>
% \fi\begin{macrocode}
\ProvidesFile{fithesis/locale/mu/med/fithesis-english.def}[2016/03/23]

% Placeholders
\gdef\thesis@english@facultyName{Faculty of Medicine}

% Miscellaneous
\gdef\thesis@english@abstractTitle{Annotation}
%    \end{macrocode}\iffalse
%</mu/med>
% \fi\file{locale/mu/fi/fithesis-english.def}
% This is the English locale file specific to the Faculty of
% Informatics at the Masaryk University in Brno.  It replaces the
% \texttt{facultyName} placeholder with the correct value and
% redefines the string in accordance with the requirements of the
% faculty.  The file also defines the \texttt{advisorSignature}
% string required by the |\thesis@blocks@titlePage| block defined
% within the
% \texttt{style/mu/\discretionary{}{}{}fithesis-fi.sty}
% style file.
% \iffalse
%<*mu/fi>
% \fi\begin{macrocode}
\ProvidesFile{fithesis/locale/mu/fi/fithesis-english.def}[2016/05/25]

% Placeholders
\gdef\thesis@english@facultyName{Faculty of Informatics}
\gdef\thesis@english@assignment{Replace this page with a copy
  of the official signed thesis assignment and a copy of the
  Statement of an Author.}
\gdef\thesis@english@assignment{%
  \ifthesis@digital@
    This is where a copy of the official signed thesis assignment
    and a copy of the Statement of an Author is located in the
    printed version of the document.
  \else
    Replace this page with a copy of the official signed thesis
    assignment and a copy of the Statement of an Author.
  \fi}

% Others
\gdef\thesis@english@advisorSignature{Signature of Thesis
  \thesis@english@advisorTitle}
\gdef\thesis@english@typeName@proposal{Ph.D. Thesis Proposal}
%    \end{macrocode}\iffalse
%</mu/fi>
% \fi\file{locale/mu/phil/fithesis-english.def}
% This is the English locale file specific to the Faculty of
% Arts at the Masaryk University in Brno.
% It replaces the \texttt{facultyName} placeholder with the
% correct value. It also defines the \texttt{departmentName}
% string, which is used by the \texttt{style/mu/fithesis-phil^^A
% .sty} style file, when typesetting the names of known
% departments.
% \iffalse
%<*mu/phil>
% \fi\begin{macrocode}
\ProvidesFile{fithesis/locale/mu/phil/fithesis-english.def}[2016/03/22]
\gdef\thesis@english@facultyName{Faculty of Arts}
\gdef\thesis@english@departmentName{%
  \ifx\thesis@department\thesis@departments@kisk
    Division of Information and Library Studies%
  \else
    <<Unknown department (\thesis@department)>>%
  \fi}
%    \end{macrocode}\iffalse
%</mu/phil>
% \fi\file{locale/mu/ped/fithesis-english.def}
% This is the Slovak locale file specific to the Faculty of
% Education at the Masaryk University in Brno.  It replaces the
% \texttt{facultyName} placeholder with the correct value. The file
% also defines the \texttt{bib@title} and \texttt{bib@pages}
% strings required by the |\thesis@blocks@bibEntry| block defined
% within the
% \texttt{style/mu/\discretionary{}{}{}fithesis-ped.sty}
% style file.
% \iffalse
%<*mu/ped>
% \fi\begin{macrocode}
\ProvidesFile{fithesis/locale/mu/ped/fithesis-english.def}[2016/03/22]

% Placeholders
\gdef\thesis@english@facultyName{Faculty of Education}
%    \end{macrocode}\iffalse
%</mu/ped>
% \fi\file{locale/mu/sci/fithesis-english.def}
% This is the English locale file specific to the Faculty of
% Science at the Masaryk University in Brno.  It defines the
% private macros required by the |\thesis@blocks@bibEntryEn| block
% defined within the
% \texttt{style/mu/\discretionary{}{}{}fithesis-sci.sty}
% style file. It also replaces the \texttt{facultyName}
% placeholder with the correct value and redefines the
% \texttt{advisorTitle} string in accordance with the formal
% requirements of the faculty.
% \iffalse
%<*mu/sci>
% \fi\begin{macrocode}
\ProvidesFile{fithesis/locale/mu/sci/fithesis-english.def}[2017/06/02]

% Placeholders
\gdef\thesis@english@facultyName{Faculty of Science}

% Miscellaneous
\global\let\thesis@english@advisorTitleEn=\thesis@english@bib@advisor

% Bibliographic entry
\gdef\thesis@english@bib@programme{Degree Programme}
\global\let\thesis@english@bib@field\thesis@english@fieldTitle
\gdef\thesis@english@bib@academicYear{Academic Year}
\gdef\thesis@english@bib@pages{Number of Pages}
\global\let\thesis@english@bib@keywords\thesis@english@keywordsTitle
%    \end{macrocode}\iffalse
%</mu/sci>
% \fi

% \subsubsection{Czech locale files}
% % \file{locale/fithesis-czech.def}
% This is the base file of the Czech locale.\iffalse
%<*base>
% \fi\begin{macrocode}
\ProvidesFile{fithesis/locale/fithesis-czech.def}[2017/07/09]
%    \end{macrocode}
% The locale file defines all the private macros mandated by the
% locale file interface.
% \begin{macro}{\thesis@czech@gender@koncovka}
% The locale file also defines the |\thesis@czech@gender@koncovka|
% macro, which expands to the correct verb ending based on the
% value of the |\thesis@ifwoman| macro and the
% \end{macro}\begin{macro}{\thesis@czech@typeName@akuzativ}
% |\thesis@czech@typeName@akuzativ| containing the accusative case
% of the thesis type name.
% \end{macro}\begin{macrocode}

% Pomocná makra
\gdef\thesis@czech@gender@koncovka{%
  \ifthesis@woman a\fi}

% Csquotes styl
\gdef\thesis@czech@csquotesStyle{german}

% Zástupné texty
\gdef\thesis@czech@universityName{Název univerzity}
\gdef\thesis@czech@facultyName{Název fakulty}
%    \end{macrocode}
% \changes{v0.3.47}{2017/07/09}{Moved the \cs{ifthesis@digital}
%   tests from \texttt{locale/*.def} to \texttt{locale/mu/*.def},
%   since \cs{ifthesis@digital} is undefined in
%   \texttt{fithesis3.cls}. [VN]}
%    \begin{macrocode}
\gdef\thesis@czech@assignment{%
  Na tomto místě se v~tištěné práci nachází oficiální podepsané
  zadání práce.}
\gdef\thesis@czech@declaration{Text prohlášení ...}

% Časové údaje
\gdef\thesis@czech@spring{jaro}
\gdef\thesis@czech@fall{podzim}
\gdef\thesis@czech@semester{%
  \thesis@{czech@\thesis@season} \thesis@seasonYear}
\gdef\thesis@czech@formattedDate{{%
  \thesis@day.
  \newcount\@month\expandafter\@month\thesis@month\relax
  \ifnum\@month=1%
    ledna
  \else\ifnum\@month=2%
    února
  \else\ifnum\@month=3%
    března
  \else\ifnum\@month=4%
    dubna
  \else\ifnum\@month=5%
    května
  \else\ifnum\@month=6%
    června
  \else\ifnum\@month=7%
    července
  \else\ifnum\@month=8%
    srpna
  \else\ifnum\@month=9%
    září
  \else\ifnum\@month=10%
    října
  \else\ifnum\@month=11%
    listopadu
  \else\ifnum\@month=12%
    prosince
  \else
    <<neznámý měsíc (\the\@month)>>
  \fi\fi\fi\fi\fi\fi
  \fi\fi\fi\fi\fi\fi
  \thesis@year}}

% Různé
\gdef\thesis@czech@authorSignature{Podpis autora}
\gdef\thesis@czech@fieldTitle{Obor}
\gdef\thesis@czech@advisorTitle{Vedoucí práce}
\gdef\thesis@czech@authorTitle{Autor}
\gdef\thesis@czech@abstractTitle{Shrnutí}
\gdef\thesis@czech@keywordsTitle{Klíčová slova}
\gdef\thesis@czech@thanksTitle{Poděkování}
\gdef\thesis@czech@declarationTitle{Prohlášení}
\gdef\thesis@czech@idTitle{ID}
\gdef\thesis@czech@typeName@sempaper{Seminární práce}
\gdef\thesis@czech@typeName@bachelors{Bakalářská práce}
\gdef\thesis@czech@typeName@masters{Diplomová práce}
\gdef\thesis@czech@typeName@proposal{Teze závěrečné práce}
\gdef\thesis@czech@typeName@doctoral{Disertační práce}
\gdef\thesis@czech@typeName@rigorous{Rigorózní práce}
\gdef\thesis@czech@typeName{%
  \ifx\thesis@type\thesis@sempaper
    \thesis@czech@typeName@sempaper
  \else\ifx\thesis@type\thesis@bachelors
    \thesis@czech@typeName@bachelors
  \else\ifx\thesis@type\thesis@masters
    \thesis@czech@typeName@masters
  \else\ifx\thesis@type\thesis@proposal
    \thesis@czech@typeName@proposal
  \else\ifx\thesis@type\thesis@doctoral
    \thesis@czech@typeName@doctoral
  \else\ifx\thesis@type\thesis@rigorous
    \thesis@czech@typeName@rigorous
  \else
    <<Neznámý typ práce (\thesis@type)>>%
  \fi\fi\fi\fi\fi\fi}
\gdef\thesis@czech@typeName@akuzativ@sempaper{Seminární práci}
\gdef\thesis@czech@typeName@akuzativ@bachelors{Bakalářskou práci}
\gdef\thesis@czech@typeName@akuzativ@masters{Diplomovou práci}
\gdef\thesis@czech@typeName@akuzativ@proposal{Tezi závěrečné práce}
\gdef\thesis@czech@typeName@akuzativ@doctoral{Disertační práci}
\gdef\thesis@czech@typeName@akuzativ@rigorous{Rigorózní práci}
\gdef\thesis@czech@typeName@akuzativ{%
  \ifx\thesis@type\thesis@sempaper
    \thesis@czech@typeName@akuzativ@sempaper
  \else\ifx\thesis@type\thesis@bachelors
    \thesis@czech@typeName@akuzativ@bachelors
  \else\ifx\thesis@type\thesis@masters
    \thesis@czech@typeName@akuzativ@masters
  \else\ifx\thesis@type\thesis@proposal
    \thesis@czech@typeName@akuzativ@proposal
  \else\ifx\thesis@type\thesis@doctoral
    \thesis@czech@typeName@akuzativ@doctoral
  \else\ifx\thesis@type\thesis@rigorous
    \thesis@czech@typeName@akuzativ@rigorous
  \else
    <<Neznámý typ práce (\thesis@type)>>%
  \fi\fi\fi\fi\fi\fi}
%    \end{macrocode}\iffalse
%</base>
% \fi\file{locale/mu/fithesis-czech.def}
% This is the Czech locale file specific to the Masaryk
% University in Brno. It replaces the \texttt{universityName}
% placeholder with the correct value and defines the
% \texttt{declaration} and \texttt{idTitle} strings.
% \iffalse
%<*mu>
% \fi\begin{macrocode}
\ProvidesFile{fithesis/locale/mu/fithesis-czech.def}[2017/07/09]

% Zástupné texty
\gdef\thesis@czech@universityName{Masarykova univerzita}
\gdef\thesis@czech@declaration{Prohlašuji, že jsem
  \thesis@lower{czech@typeName@akuzativ} zpracoval%
  \thesis@czech@gender@koncovka\ samostatně a~%
  použil\thesis@czech@gender@koncovka\ jen prameny
  uvedené v~seznamu literatury.}
%    \end{macrocode}
% \changes{v0.3.47}{2017/07/09}{Moved the \cs{ifthesis@digital}
%   tests from \texttt{locale/*.def} to \texttt{locale/mu/*.def},
%   since \cs{ifthesis@digital} is undefined in
%   \texttt{fithesis3.cls}. [VN]}
% \changes{v0.3.47}{2017/07/09}{Added an
% \cs{ifthesis@blocks@assignment@hideIfDigital@} test to the
% definition of the \texttt{assignment} string for the Masaryk
% University in Brno. [VN]}
%    \begin{macrocode}
\gdef\thesis@czech@assignment{%
  \ifthesis@blocks@assignment@hideIfDigital@
    \ifthesis@digital@
      Na tomto místě se v~tištěné práci nachází oficiální podepsané
      zadání práce.
    \else
      Místo tohoto listu vložte kopii oficiálního podepsaného zadání
      práce.
    \fi
  \else
    Nastavte pomocí klíče <<assignment>> název PDF souboru
    s~oficiálním podepsaným zadáním práce.
  \fi}

% Bibliografický záznam
\gdef\thesis@czech@bib@title{Bibliografický záznam}
\gdef\thesis@czech@bib@pages{str}
%    \end{macrocode}
% \changes{v0.3.46}{2017/06/02}{Lifted the \texttt{bib@author},
%   \texttt{bib@thesisTitle}, and \texttt{bib@advisor} strings from
%   \texttt{locale/mu/sci/*.def} to \texttt{locale/mu/*.def},
%   so that they can be shared with \texttt{locale/mu/econ/*.def}.
%   [VN]}
%    \begin{macrocode}
\global\let\thesis@czech@bib@author\thesis@czech@authorTitle
\gdef\thesis@czech@bib@thesisTitle{Název práce}
\global\let\thesis@czech@bib@advisor\thesis@czech@advisorTitle

% Různé
\gdef\thesis@czech@idTitle{UČO}
%    \end{macrocode}\iffalse
%</mu>
% \fi\file{locale/mu/law/fithesis-czech.def}
% This is the Czech locale file specific to the Faculty of Law at
% the Masaryk University in Brno. It replaces the
% \texttt{facultyName} placeholder with the correct value, defines
% the \texttt{facultyLongName} required by the
% |\thesis@blocks@cover| and the |\thesis@blocks@titlePage| blocks
% and replaces the \texttt{abstractTitle} string in accordance
% with the requirements of the faculty.
% \iffalse
%<*mu/law>
% \fi\begin{macrocode}
\ProvidesFile{fithesis/locale/mu/law/fithesis-czech.def}[2015/06/26]

% Různé
\gdef\thesis@czech@abstractTitle{Abstrakt}

% Zástupné texty
\gdef\thesis@czech@facultyName{Právnická fakulta}
\gdef\thesis@czech@facultyLongName{Právnická fakulta Masarykovy
  univerzity}
%    \end{macrocode}\iffalse
%</mu/law>
% \fi\file{locale/mu/fsps/fithesis-czech.def}
% This is the Czech locale file specific to the Faculty of Sports
% Studies at the Masaryk University in Brno. It replaces the
% \texttt{facultyName} placeholder with the correct value and
% redefines the \texttt{fieldTitle} string in accordance with the
% common usage at the faculty. The locale file also redefines the
% \texttt{declaration} string in accordance with the requirements
% of the faculty.
% \iffalse
%<*mu/fsps>
% \fi\begin{macrocode}
\ProvidesFile{fithesis/locale/mu/fsps/fithesis-czech.def}[2017/05/15]

% Zástupné texty
\gdef\thesis@czech@facultyName{Fakulta sportovních studií}
\gdef\thesis@czech@declaration{Prohlašuji, že jsem
  \thesis@lower{czech@typeName@akuzativ} vypracoval%
  \thesis@czech@gender@koncovka\ samostatně a~na základě
  literatury a~pramenů uvedených v~použitých zdrojích.}

% Různé
\gdef\thesis@czech@fieldTitle{Specializace}
%    \end{macrocode}\iffalse
%</mu/fsps>
% \fi\file{locale/mu/fss/fithesis-czech.def}
% This is the Czech locale file specific to the Faculty of Social
% Studies at the Masaryk University in Brno. It replaces the
% \texttt{facultyName} and \texttt{assignment} placeholders with
% the correct values.
% \iffalse
%<*mu/fss>
% \fi\begin{macrocode}
\ProvidesFile{fithesis/locale/mu/fss/fithesis-czech.def}[2016/05/25]

% Zástupné texty
\gdef\thesis@czech@facultyName{Fakulta sociálních studií}
\gdef\thesis@czech@assignment{%
  \ifthesis@digital@
    Na tomto místě se v~tištěné práci nachází oficiální podepsané
    zadání práce, prohlášení autora školního díla nebo obojí.
  \else
    Místo tohoto listu vložte kopie oficiálního podepsaného zadání
    práce nebo prohlášení autora školního díla nebo obojí
    v~závislosti na požadavcích příslušné katedry.
  \fi}

%    \end{macrocode}\iffalse
%</mu/fss>
% \fi\file{locale/mu/econ/fithesis-czech.def}
% This is the Czech locale file specific to the Faculty of
% Economics and Administration at the Masaryk University in Brno.
% It replaces the \texttt{facultyName} and \texttt{abstractTitle}
% placeholders with the correct values. The locale file also
% redefines the \texttt{declaration} string in accordance with
% the requirements of the faculty and defines the private macros
% required by the |\thesis@blocks@|\discretionary{}{}{}|bibEntry|
% block defined within the \texttt{style/mu/fithesis-econ.sty}
% style file.
% \iffalse
%<*mu/econ>
% \fi\begin{macrocode}
\ProvidesFile{fithesis/locale/mu/econ/fithesis-czech.def}[2017/07/09]

% Zástupné texty
\gdef\thesis@czech@facultyName{Ekonomicko-správní fakulta}

% Bibliografický záznam
%    \end{macrocode}
% \changes{v0.3.46}{2017/06/02}{Defined strings required by
%   \cs{thesis@blocks@bibEntry} from
%   \texttt{style/mu/fithesis-econ.sty} in
%   \texttt{locale/mu/econ/*.def}. [VN]}
%    \begin{macrocode}
\gdef\thesis@czech@bib@thesisTitleEn{Název práce v angličtině}
\gdef\thesis@czech@bib@department{Katedra}
\gdef\thesis@czech@bib@year{Rok obhajoby}

% Různé
%    \end{macrocode}
% \changes{v0.3.46}{2017/06/02}{Updated the
%   \cs{abstractTitle} string in \texttt{locale/mu/econ/*.def} in
%   accordance with the 2/2017 dean's directive. The patch was
%   submitted by Jana Ratajská. [VN]}
%    \begin{macrocode}
\gdef\thesis@czech@abstractTitle{Anotace}
%    \end{macrocode}
% \changes{v0.3.46}{2017/06/02}{Updated the \texttt{declaration} string
%   in \texttt{locale/mu/econ/*.def} in accordance with the 2/2017
%   dean's directive. [VN]}
% \changes{v0.3.47}{2017/07/09}{Updated the \texttt{declaration} string
%   in \texttt{locale/mu/econ/*.def} in accordance with the 2/2017
%   dean's directive. [VN]}
% The following extra data field is defined for
% \texttt{declaration} string: \begin{itemize}
%   \item|advisorCsGenitiv| -- the advisor's name in
%     genitive following Czech morphology.
% \end{itemize}
%    \begin{macrocode}
\thesis@def@extra{advisorCsGenitiv}
\gdef\thesis@czech@declaration{Prohlašuji, že jsem
  \thesis@lower{czech@typeName@akuzativ} \thesis@title{} zpracoval%
  \thesis@czech@gender@koncovka\ samostatně pod vedením
  \thesis@extra@advisorCsGenitiv\
  a~uvedl\thesis@czech@gender@koncovka\ v~ní všechny
  odborné zdroje v~souladu s~právními předpisy, vnitřními
  předpisy Masarykovy univerzity a~vnitřními akty řízení
  Masarykovy univerzity a~Ekonomicko-správní fakulty MU.}
%    \end{macrocode}\iffalse
%</mu/econ>
% \fi\file{locale/mu/med/fithesis-czech.def}
% This is the Czech locale file specific to the Faculty of
% Medicine at the Masaryk University in Brno.
% It replaces the \texttt{facultyName} placeholder with the
% correct value and redefines the \texttt{abstractTitle} string in
% accordance with the common usage at the faculty. The file also
% defines the \texttt{bib@title} and \texttt{bib@pages} strings
% required by the |\thesis@blocks@bibEntry| block defined within
% the \texttt{style/mu/fithesis-med.sty} style file.
% \iffalse
%<*mu/med>
% \fi\begin{macrocode}
\ProvidesFile{fithesis/locale/mu/med/fithesis-czech.def}[2016/03/23]

% Zástupné texty
\gdef\thesis@czech@facultyName{Lékařská fakulta}

% Různé
\gdef\thesis@czech@abstractTitle{Anotace}
%    \end{macrocode}\iffalse
%</mu/med>
% \fi\file{locale/mu/fi/fithesis-czech.def}
% This is the Czech locale file specific to the Faculty of
% Informatics at the Masaryk University in Brno.
% It replaces the \texttt{facultyName} placeholder with the
% correct value and redefines the \texttt{declaration} string in
% accordance with the requirements of the faculty. The file also
% defines the \texttt{advisorSignature} string required by the
% |\thesis@blocks@titlePage| block defined within the
% \texttt{style/mu/\discretionary{}{}{}fithesis-fi.sty}
% style file.
% \iffalse
%<*mu/fi>
% \fi\begin{macrocode}
\ProvidesFile{fithesis/locale/mu/fi/fithesis-czech.def}[2016/05/25]

% Zástupné texty
\gdef\thesis@czech@facultyName{Fakulta informatiky}
\gdef\thesis@czech@assignment{%
  \ifthesis@digital@
    Na tomto místě se v~tištěné práci nachází oficiální podepsané
    zadání práce a prohlášení autora školního díla.
  \else
    Místo tohoto listu vložte kopie oficiálního podepsaného zadání
    práce a prohlášení autora školního díla.
  \fi}
\gdef\thesis@czech@declaration{%
  Prohlašuji, že tato \thesis@lower{czech@typeName} je mým
  původním autorským dílem, které jsem vypracoval%
  \thesis@czech@gender@koncovka\ samostatně. Všechny zdroje,
  prameny a~literaturu, které jsem při vypracování
  používal\thesis@czech@gender@koncovka\ nebo z~nich
  čerpal\thesis@czech@gender@koncovka, v~práci řádně cituji
  s~uvedením úplného odkazu na příslušný zdroj.}

% Ostatní
\gdef\thesis@czech@advisorSignature{Podpis vedoucího}
\gdef\thesis@czech@typeName@proposal{Teze disertační práce}
\gdef\thesis@czech@typeName@akuzativ@proposal{Tezi disertační práce}
%    \end{macrocode}\iffalse
%</mu/fi>
% \fi\file{locale/mu/phil/fithesis-czech.def}
% This is the Czech locale file specific to the Faculty of
% Arts at the Masaryk University in Brno.
% It replaces the \texttt{facultyName} placeholder with the
% correct value. It also redefines the \texttt{declaration},
% \texttt{typeName} and \texttt{typeName@akuzativ} strings in
% accordance with the requirements of the faculty.
%
% The locale file also defines the \texttt{departmentName}
% string, which is used by the \texttt{style/mu/fithesis-phil^^A
% .sty} style file, when typesetting the names of known
% departments.
% \iffalse
%<*mu/phil>
% \fi\begin{macrocode}
\ProvidesFile{fithesis/locale/mu/phil/fithesis-czech.def}[2016/03/22]

% Zástupné texty
\gdef\thesis@czech@facultyName{Filozofická fakulta}
\gdef\thesis@czech@departmentName{%
  \ifx\thesis@department\thesis@departments@kisk
    Kabinet informačních studií a knihovnictví%
  \else
    <<Neznámé oddělení (\thesis@department)>>%
  \fi}
\gdef\thesis@czech@declaration{%
  \ifx\thesis@department\thesis@departments@kisk
    Prohlašuji, že jsem předkládanou práci zpracoval%
    \thesis@czech@gender@koncovka\ samostatně a~použil%
    \thesis@czech@gender@koncovka\ jen uvedené prameny a~%
    literaturu. Současně dávám svolení k~tomu, aby elektronická
    verze této práce byla zpřístupněna přes informační systém
    Masarykovy univerzity.%
  \else
    Prohlašuji, že jsem \thesis@lower{czech@typeName@akuzativ}
    vypracoval\thesis@czech@gender@koncovka\ samostatně s~využitím
    uvedené literatury.%
  \fi}

% Ostatní
\global\let\thesis@czech@typeName@super
  \thesis@czech@typeName
\gdef\thesis@czech@typeName{%
  \ifx\thesis@type\thesis@bachelors
    Bakalářská diplomová práce%
  \else\ifx\thesis@type\thesis@masters
    Magisterská diplomová práce%
  \else
    \thesis@czech@typeName@super
  \fi\fi}

\global\let\thesis@czech@typeName@akuzativ@super
  \thesis@czech@typeName@akuzativ
\gdef\thesis@czech@typeName@akuzativ{%
  \ifx\thesis@type\thesis@bachelors
    Diplomovou práci%
  \else\ifx\thesis@type\thesis@masters
    Diplomovou práci%
  \else
    \thesis@czech@typeName@akuzativ@super
  \fi\fi}
%    \end{macrocode}\iffalse
%</mu/phil>
% \fi\file{locale/mu/ped/fithesis-czech.def}
% This is the Czech locale file specific to the Faculty of
% Education at the Masaryk University in Brno.
% It replaces the \texttt{facultyName} placeholder with the
% correct value. The file also defines the
% \texttt{bib@title} and \texttt{bib@pages} strings required by the
% |\thesis@blocks@bibEntry| block defined within the
% \texttt{style/mu/\discretionary{}{}{}fithesis-ped.sty}
% style file.
% \iffalse
%<*mu/ped>
% \fi\begin{macrocode}
\ProvidesFile{fithesis/locale/mu/ped/fithesis-czech.def}[2016/03/22]

% Zástupné texty
\gdef\thesis@czech@facultyName{Pedagogická fakulta}
%    \end{macrocode}\iffalse
%</mu/ped>
% \fi\file{locale/mu/sci/fithesis-czech.def}
% This is the Czech locale file specific to the Faculty of Science
% at the Masaryk University in Brno.  It defines the private macros
% required by the |\thesis@blocks@|\discretionary{}{}{}|bibEntryCs|
% block defined within the
% \texttt{style/mu/fithesis-sci.sty} style file.  It also
% replaces the \texttt{facultyName} placeholder with the correct
% value and redefines the \texttt{abstractTitle} and
% \texttt{declaration} strings in accordance with the formal
% requirements of the faculty.
% \iffalse
%<*mu/sci>
% \fi\begin{macrocode}
\ProvidesFile{fithesis/locale/mu/sci/fithesis-czech.def}[2017/10/28]

% Zástupné texty
\gdef\thesis@czech@facultyName{Přírodovědecká fakulta}

% Ostatní
\gdef\thesis@czech@abstractTitle{Abstrakt}
\gdef\thesis@czech@declaration{%
  Prohlašuji, že jsem \thesis@lower{czech@typeName@%
  akuzativ} vypracoval\thesis@czech@gender@koncovka\ samostatně,
  s~využitím pouze citovaných pramenů, dalších informací a zdrojů
  v~souladu s~Disciplinárním řádem pro studenty Pedagogické fakulty
  Masarykovy univerzity a se zákonem č.\ 121/2000 Sb., o~právu
  autorském, o~právech souvisejících s~právem autorským a o~změně
  některých zákonů (autorský zákon), ve znění pozdějších předpisů.}

% Bibliografický záznam
\gdef\thesis@czech@bib@programme{Studijní program}
\global\let\thesis@czech@bib@field\thesis@czech@fieldTitle
\gdef\thesis@czech@bib@academicYear{Akademický rok}
\gdef\thesis@czech@bib@pages{Počet stran}
\global\let\thesis@czech@bib@keywords\thesis@czech@keywordsTitle
%    \end{macrocode}\iffalse
%</mu/sci>
% \fi

% \subsubsection{Slovak locale files}
% % \file{locale/slovak.def}
% This is the base file of the Slovak locale. It defines all the
% private macros mandated by the locale file interface. 
% \begin{macro}{\thesis@slovak@gender@koncovka}
% The locale file defines the |\thesis@slovak@gender@koncovka|
% macro, which expands to the correct verb ending based on the
% value of the |\thesis@ifwoman| macro.
% \end{macro}\iffalse
%<*base>
% \fi\begin{macrocode}
\ProvidesFile{fithesis3/locale/slovak.def}[2015/04/18]

% Pomocná makrá
\def\thesis@slovak@gender@koncovka{%
  \ifthesis@woman a\fi}

% Zástupné texty
\def\thesis@slovak@placeholders@title{Názov práce}
\def\thesis@slovak@placeholders@keywords{kľúčové slovo 1,
  kľúčové slovo 2, ...}
\def\thesis@slovak@placeholders@abstract{Text zhrnutie}
\def\thesis@slovak@placeholders@author{Meno autora}
\def\thesis@slovak@placeholders@author@firstName{Meno}
\def\thesis@slovak@placeholders@author@surname{Priezvisko}
\def\thesis@slovak@universityName{Názov univerzity}
\def\thesis@slovak@facultyName{Názov fakulty}
\def\thesis@slovak@placeholders@advisor{Meno vedúceho}
\def\thesis@slovak@placeholders@department{Názov katedry}
\def\thesis@slovak@placeholders@programme{Názov študijného
  programu}
\def\thesis@slovak@placeholders@field{Název študijného odboru}
\def\thesis@slovak@assignment{Namiesto tejto stránky vložte kópiu
  oficiálneho podpísaného zadania práce.}
\def\thesis@slovak@declaration{Text prehlásenie ...}

% Rôzne
\def\thesis@slovak@fieldTitle{Odbor}
\def\thesis@slovak@advisorTitle{Vedúci práce}
\def\thesis@slovak@authorTitle{Autor}
\def\thesis@slovak@abstractTitle{Zhrnutie}
\def\thesis@slovak@keywordsTitle{Kľúčové slová}
\def\thesis@slovak@thanksTitle{Poďakovanie}
\def\thesis@slovak@declarationTitle{Prehlásenie}
\def\thesis@slovak@winter{Jar}
\def\thesis@slovak@summer{Jeseň}
\def\thesis@slovak@semester{%
  \thesis@{slovak@\thesis@season}\ \thesis@year}
\def\thesis@slovak@typeName{%
  \ifx\thesis@type\thesis@bachelors%
    Bakalárska práca%
  \else\ifx\thesis@type\thesis@masters%
    Diplomová práca%
  \else\ifx\thesis@type\thesis@doctoral%
    Dizertačná práca%
  \else\ifx\thesis@type\thesis@rigorous%
    Rigorózna práca%
  \else%
    <<Neznámy typ práce (\thesis@type)>>%
  \fi\fi\fi\fi}
\def\thesis@slovak@typeName@akuzativ{%
  \ifx\thesis@type\thesis@bachelors%
    Bakalársku prácu%
  \else\ifx\thesis@type\thesis@masters%
    Diplomovú prácu%
  \else\ifx\thesis@type\thesis@doctoral%
    Dizertačnú prácu%
  \else\ifx\thesis@type\thesis@rigorous%
    Rigoróznu prácu%
  \else%
    <<Neznámy typ práce (\thesis@type)>>%
  \fi\fi\fi\fi}
%    \end{macrocode}\iffalse
%</base>
% \fi\file{locale/mu/slovak.def}
% This is the Slovak locale file specific to the Masaryk
% University in Brno. It replaces the \texttt{universityName}
% placeholder with the correct value and defines a placeholder
% declaration text.
% \iffalse
%<*mu>
% \fi\begin{macrocode}
\ProvidesFile{fithesis3/locale/mu/slovak.def}[2015/04/26]

% Zástupné texty
\def\thesis@slovak@universityName{Masarykova Univerzita}
\def\thesis@slovak@declaration{%
  Prehlašujem, že som predloženú \thesis@lower{%
  slovak@typeName@akuzativ} vypracoval%
  \thesis@slovak@gender@koncovka\ samostatne len s~použitím
  uvedenej literatúry a prameňov.}
%    \end{macrocode}\iffalse
%</mu>
% \fi\file{locale/mu/law/slovak.def}
% This is the Slovak locale file specific to the Faculty of Law at
% the Masaryk University in Brno. It replaces the
% \texttt{facultyName} placeholder with the correct value and
% replaces the \texttt{abstractTitle} and
% \texttt{placeholders@abstract} strings in accordance with the
% requirements of the faculty.
% \iffalse
%<*mu/law>
% \fi\begin{macrocode}
\ProvidesFile{fithesis3/locale/mu/law/slovak.def}[2015/04/26]

% Rôzne
\def\thesis@slovak@abstractTitle{Abstrakt}

% Zástupné texty
\def\thesis@slovak@placeholders@abstract{Text abstraktu}
\def\thesis@slovak@facultyName{Právnická fakulta Masarykovej
  univerzity}
%    \end{macrocode}\iffalse
%</mu/law>
% \fi\file{locale/mu/fsps/slovak.def}
% This is the Slovak locale file specific to the Faculty of Sports
% Studies at the Masaryk University in Brno. It replaces the
% \texttt{facultyName} placeholder with the correct value and the
% \texttt{placeholders@field} and \texttt{fieldTitle} strings in
% accordance with the common usage at the faculty.
% \iffalse
%<*mu/fsps>
% \fi\begin{macrocode}
\ProvidesFile{fithesis3/locale/mu/fsps/slovak.def}[2015/04/18]

% Zástupné texty
\def\thesis@slovak@placeholders@field{Názov špecializácie}
\def\thesis@slovak@facultyName{Fakulta športových štúdií}

% Rôzne
\def\thesis@slovak@fieldTitle{Špecializácie}
%    \end{macrocode}\iffalse
%</mu/fsps>
% \fi\file{locale/mu/fss/slovak.def}
% This is the Slovak locale file specific to the Faculty of Social
% Studies at the Masaryk University in Brno. It replaces the
% \texttt{facultyName} and \texttt{placeholders@assignment}
% strings with the correct value.
% \iffalse
%<*mu/fss>
% \fi\begin{macrocode}
\ProvidesFile{fithesis3/locale/mu/fss/slovak.def}[2015/04/26]

% Zástupné texty
\def\thesis@slovak@facultyName{Fakulta sociálnych štúdií}
\def\thesis@slovak@assignment{Namiesto tejto stránky vložte kópiu
  oficiálneho podpísaného zadania práce alebo prehlásenie autora
  školského diela alebo obidve~v závislosti na požiadavkách
  príslušnej katedry.}
%    \end{macrocode}\iffalse
%</mu/fss>
% \fi\file{locale/mu/econ/slovak.def}
% This is the Slovak locale file specific to the Faculty of
% Economics and Administration at the Masaryk University in Brno.
% It replaces the \texttt{facultyName} placeholder with the
% correct value.
% \iffalse
%<*mu/econ>
% \fi\begin{macrocode}
\ProvidesFile{fithesis3/locale/mu/econ/slovak.def}[2015/04/18]
\def\thesis@slovak@facultyName{Ekonomicko-správna fakulta}
%    \end{macrocode}\iffalse
%</mu/econ>
% \fi\file{locale/mu/med/slovak.def}
% This is the Slovak locale file specific to the Faculty of
% Medicine at the Masaryk University in Brno.
% It replaces the \texttt{facultyName} placeholder with the
% correct value and redefines the \texttt{abstractTitle},
% and \texttt{placeholders@abstract} strings in accordance
% with the common usage at the faculty. The file also defines the
% \texttt{bib@title} and \texttt{bib@pages} strings required by the
% |\thesis@blocks@bibEntry| block defined within the
% \texttt{style/mu/fithesis3-med.sty} style file.

% \iffalse
%<*mu/med>
% \fi\begin{macrocode}
\ProvidesFile{fithesis3/locale/mu/med/slovak.def}[2015/04/26]

% Rôzne
\def\thesis@slovak@abstractTitle{Anotácie}

% Zástupné texty
\def\thesis@slovak@placeholders@abstract{Text abstraktu}
\def\thesis@slovak@facultyName{Lekárska fakulta}

% Bibliografický zoznam
\def\thesis@slovak@bib@title{Bibliografický záznam}
\def\thesis@slovak@bib@pages{str}
%    \end{macrocode}\iffalse
%</mu/med>
% \fi\file{locale/mu/fi/slovak.def}
% This is the Slovak locale file specific to the Faculty of
% Informatics at the Masaryk University in Brno.
% It replaces the \texttt{facultyName} placeholder with the
% correct value and updates the \texttt{placeholders@assignment}
% and \texttt{declaration} strings in accordance with the
% requirements of the faculty. The file also defines the
% \texttt{advisorSignature} string required by the
% |\thesis@blocks@titlePage| block defined within the
% \texttt{style/mu/fithesis3-fi.sty} style file.
% \iffalse
%<*mu/fi>
% \fi\begin{macrocode}
\ProvidesFile{fithesis3/locale/mu/fi/slovak.def}[2015/04/18]

% Zástupné texty
\def\thesis@slovak@facultyName{Fakulta informatiky}
\def\thesis@slovak@assignment{Namiesto tejto stránky vložte kópiu
  oficiálneho podpísaného zadania práce a prehlásenie autora
  školského diela.}
\def\thesis@slovak@declaration{%
  Prehlasujem, že táto \thesis@lower{slovak@typeName} je mojím
  pôvodným autorským dielom, ktoré som vypracoval%
  \thesis@slovak@gender@koncovka samostatne. Všetky zdroje,
  pramene a literatúru, ktoré som pri vypracovaní
  používal\thesis@slovak@gender@koncovka\ alebo z~nich
  čerpal\thesis@slovak@gender@koncovka, v~práci riadne citujem
  s~uvedením úplného odkazu na príslušný zdroj.}

% Rôzne
\def\thesis@slovak@advisorSignature{Podpis vedúceho}
%    \end{macrocode}\iffalse
%</mu/fi>
% \fi\file{locale/mu/phil/slovak.def}
% This is the Slovak locale file specific to the Faculty of
% Arts at the Masaryk University in Brno.
% It replaces the \texttt{facultyName} placeholder with the
% correct value. It also defines the \texttt{declaration} string
% and redefines the \texttt{typeName} and
% \texttt{typeName@akuzativ} strings in accordance with the
% requirements of the faculty.
% \iffalse
%<*mu/phil>
% \fi\begin{macrocode}
\ProvidesFile{fithesis3/locale/mu/phil/slovak.def}[2015/04/26]

% Zástupné texty
\def\thesis@slovak@facultyName{Filozofická fakulta}
\def\thesis@slovak@declaration{%
  Prehlašujem, že som predloženú \thesis@lower{%
  slovak@typeName@akuzativ} vypracoval%
  \thesis@slovak@gender@koncovka\ samostatne na
  základe vlastných zistení a len s~použitím
  uvedenej literatúry a prameňov.}

% Rôzne
\def\thesis@slovak@typeName{%
  \ifx\thesis@type\thesis@bachelors%
    Bakalárska diplomová práca%
  \else\ifx\thesis@type\thesis@masters%
    Magisterská diplomová práca%
  \else\ifx\thesis@type\thesis@doctoral%
    Dizertačná práca%
  \else%
    <<Neznámy typ práce (\thesis@type)>>%
  \fi\fi\fi}
\def\thesis@slovak@typeName@akuzativ{%
  \ifx\thesis@type\thesis@bachelors%
    Diplomovú prácu%
  \else\ifx\thesis@type\thesis@masters%
    Diplomovú prácu%
  \else\ifx\thesis@type\thesis@doctoral%
    Dizertačnú prácu%
  \else%
    <<Neznámý typ práce (\thesis@type)>>%
  \fi\fi\fi}
%    \end{macrocode}\iffalse
%</mu/phil>
% \fi\file{locale/mu/ped/slovak.def}
% This is the Slovak locale file specific to the Faculty of
% Education at the Masaryk University in Brno.
% It replaces the \texttt{facultyName} placeholder with the
% correct value. The file also defines the
% \texttt{bib@title} and \texttt{bib@pages} strings required by the
% |\thesis@blocks@bibEntry| block defined within the
% \texttt{style/mu/fithesis3-ped.sty} style file.
% \iffalse
%<*mu/ped>
% \fi\begin{macrocode}
\ProvidesFile{fithesis3/locale/mu/ped/slovak.def}[2015/04/18]

% Zástupné texty
\def\thesis@slovak@facultyName{Pedagogická fakulta}

% Bibliografický zoznam
\def\thesis@slovak@bib@title{Bibliografický záznam}
\def\thesis@slovak@bib@pages{str}
%    \end{macrocode}\iffalse
%</mu/ped>
% \fi\file{locale/mu/sci/slovak.def}
% This is the Slovak locale file specific to the Faculty of
% Science at the Masaryk University in Brno.
% It replaces the \texttt{facultyName} placeholder with the
% correct value.
% \iffalse
%<*mu/sci>
% \fi\begin{macrocode}
\ProvidesFile{fithesis3/locale/mu/sci/slovak.def}[2015/04/18]

% Zástupné texty
\def\thesis@slovak@facultyName{Prírodovedecká fakulta}
%    \end{macrocode}\iffalse
%</mu/sci>
% \fi

%
% \subsection{Style files}
% \label{sec:style-files}
% Style files define the structure and the look of the resulting
% document. They live in the \texttt{style/} subtree and they are
% loaded during the main routine (see section
% \ref{sec:thesis@load}).
%
% When creating a new style file, it is advisable to create one
% self-contained \texttt{dtx} file, which can contain several
% files to be extracted via the \textsf{docstrip} tool based on the
% respective \texttt{ins} file. A \DescribeMacro{\file} macro
% |\file|\marg{filename} is available for sectioning the
% documentation of various files within the \texttt{dtx} file.
% For more information about \texttt{dtx} files and the
% \textsf{docstrip} tool, consult the \textsf{dtxtut, docstrip,
% doc} and \textsf{ltxdoc} manuals.
%
% \subsubsection{Interface}
% The union of locale files named
% \texttt{fithesis3-}\textit{style}\texttt{.sty},
% where \textit{style} is the result of the expansion of
% |\thesis@locale|, loaded via main routine's inheritance scheme
% (see section \ref{sec:thesis@load}) should define at least one of
% the following private macros:
% \begin{itemize}
%   \item\DescribeMacro{\thesis@preamble}
%                      |\thesis@preamble| -- If autolayout is
%                      enabled, then this macro is expanded at the
%                      very beginning of the document.
%   \item\DescribeMacro{\thesis@postamble}
%                      |\thesis@postamble| -- If autolayout is
%                      enabled, then this macro is expanded at the
%                      very end of the document.
% \end{itemize}
%
% \subsubsection{Base style files}
% % \file{style/fithesis-base.sty}
% If inheritance is enabled for style files, then this file is
% always the first style file to be loaded, regardless of the
% value of the |\thesis@style| macro. This style file is
% currently a dummy file.
%    \begin{macrocode}
\NeedsTeXFormat{LaTeX2e}
\ProvidesPackage{fithesis/style/fithesis-base}[2018/02/11]
%    \end{macrocode}
% \changes{v0.3.49}{2018/02/11}{\cs{emph} uses the italic type
%   face rather than the slanted type face. [VN]}
%    \begin{macrocode}
\DeclareRobustCommand\em
        {\@nomath\em \ifdim \fontdimen\@ne\font >\z@
                       \eminnershape \else \itshape \fi}%
%    \end{macrocode}

% % \iffalse
%<*base>
% \fi\file{style/mu/fithesis-base.sty}\label{sec:fithesis-mu-base}
% This is the base style file for theses written at the Masaryk
% University in Brno. When inheritance is enabled for style files,
% this file is always the second style file to be loaded right
% after \texttt{style/fithesis-base.sty}, regardless of the value
% of the |\thesis@style| macro.
%    \begin{macrocode}
\ProvidesPackage{fithesis/style/mu/fithesis-base}[2017/05/21]
\NeedsTeXFormat{LaTeX2e}
%    \end{macrocode}
% The file recognizes the following options: \begin{itemize}
%   \item\texttt{10pt}, \texttt{11pt}, \texttt{12pt} -- Sets the
%     type size to 10, 1o or 12 points respectively.
%    \begin{macrocode}
\DeclareOption{10pt}{\def\thesis@ptsize{0}}
\DeclareOption{11pt}{\def\thesis@ptsize{1}}
\DeclareOption{12pt}{\def\thesis@ptsize{2}}
%    \end{macrocode}
%   \item\texttt{oneside}, \texttt{twoside} -- The document is
%     going to be either single- or double-sided, respectively. In
%     a double-sided document, headers, page numbering, margin
%     notes and several other elements will be arranged based on
%     the parity of the page. Blank pages will also be inserted
%     prior the beginning of each chapter to ensure that it starts
%     on a right-hand (odd-numbered) page. The
%     \DescribeMacro{\ifthesis@twoside@}|\ifthesis@twoside@|
%     conditional is set to either \texttt{false} or \texttt{true},
%     respectively.
%    \begin{macrocode}
\newif\ifthesis@twoside@
\DeclareOption{oneside}{%
  \thesis@twoside@false\@twosidefalse\@mparswitchfalse}
\DeclareOption{twoside}{%
  \thesis@twoside@true \@twosidetrue \@mparswitchtrue}
%    \end{macrocode}
%   \item\texttt{onecolumn}, \texttt{twocolumn} -- The document
%     is going to be set in either a single column or in two
%     columns, respectively.
%    \begin{macrocode}
\DeclareOption{onecolumn}{\@twocolumnfalse}
\DeclareOption{twocolumn}{\@twocolumntrue}
%    \end{macrocode}
%   \item\texttt{draft}, \texttt{final} -- Overful lines either are
%     or aren't marked within the document, respectively, and
%     graphics either aren't or are inserted into the document,
%     respectively.
%    \begin{macrocode}
\DeclareOption{draft}{\setlength\overfullrule{5pt}}
\DeclareOption{final}{\setlength\overfullrule{0pt}}
%    \end{macrocode}
%   \item\texttt{palatino}, \texttt{nopalatino} -- The roman
%     text font family and the math font family is going to be
%     either set to Palatino or left untouched, respectively. The
%     \DescribeMacro{\ifthesis@palatino@}|\ifthesis@|^^A
%     \discretionary{}{}{}|palatino@| conditional is set to either
%     \texttt{true} or \texttt{false}, respectively. The
%     Palatino font is a part of the visual identity of the Faculty
%     of Informatics at which the document class was created.
%    \begin{macrocode}
\newif\ifthesis@palatino@
\DeclareOption{palatino}{\thesis@palatino@true}
\DeclareOption{nopalatino}{\thesis@palatino@false}
%    \end{macrocode}
%   \item\texttt{color}, \texttt{monochrome} -- Certain
%     typographical elements either are or aren't going to be
%     typeset in color, respectively. The
%     \DescribeMacro{\ifthesis@color@}|\ifthesis@color@|
%     conditional is set to either \texttt{true} or \texttt{false},
%     respectively.
%    \begin{macrocode}
\newif\ifthesis@color@
\DeclareOption{monochrome}{\thesis@color@false}
\DeclareOption{color}{\thesis@color@true}
%    \end{macrocode}
%   \item\texttt{microtype}, \texttt{nomicrotype} -- The
%     microtypographic extension of modern \TeX\ engines -- such as
%     \hologo{pdfTeX}, \Hologo{XeTeX}, or \Hologo{LuaTeX} -- is or isn't
%     going to be enabled, respectively. The
%     \DescribeMacro{\ifthesis@microtype@}|\ifthesis@microtype@|
%     conditional is set to either \texttt{true} or \texttt{false},
%     respectively.
%    \begin{macrocode}
\newif\ifthesis@microtype@
\DeclareOption{microtype}{\thesis@microtype@true}
\DeclareOption{nomicrotype}{\thesis@microtype@false}
%    \end{macrocode}
%   \item\texttt{table}, \texttt{oldtable} -- If the
%     |\ifthesis@color@| conditional is \texttt{true}, then the
%     definitions of the \texttt{tabular}, \texttt{tabularx}, and
%     \texttt{tabu} commands either are or aren't going to be
%     altered to better match the style, respectively. The
%     \DescribeMacro{\ifthesis@newtable@}|\ifthesis@newtable@|
%     conditional is set to either \texttt{true} or \texttt{false},
%     respectively.
%
%    \begin{macrocode}
\newif\ifthesis@newtable@
\DeclareOption{table}{\thesis@newtable@true}
\DeclareOption{oldtable}{\thesis@newtable@false}
%    \end{macrocode}
%
%     The choice of the option name is deliberate -- the
%     redefinition of the table environments depends on the
%     \textsf{xcolor} package, which needs to be loaded with the
%     \texttt{table} option. Since so many other packages depend on
%     the \textsf{xcolor} package and this style file is loaded at
%     the very end of the preamble, there would either be a great
%     chance of an option clash, or the option would have to be
%     passed to the \textsf{xcolor} package before the preamble
%     from the body of the \textsf{fithesis3} class thus breaking
%     the encapsulation. Naming the option \texttt{table} forces
%     the option to be processed by the \textsf{xcolor} package as
%     well and it is therefore an elegant solution to the problem
%     at hand.
%   \item\texttt{lot}, \texttt{nolot} -- The \DescribeMacro{^^A
%     \thesis@blocks@lot}|\thesis@blocks@lot| macro will be defined
%     as either |\listoftables| or |\relax|, respectively. As a
%     side effect, the |\listoftables| either is or isn't going to
%     be included in the \DescribeMacro{\thesis@blocks@tables}^^A
%     |\thesis@|\discretionary{}{}{}|blocks@tables| block,
%     respectively.
%    \begin{macrocode}
\DeclareOption{nolot}{\let\thesis@blocks@lot\relax}
\DeclareOption{lot}{\let\thesis@blocks@lot\listoftables}
%    \end{macrocode}
%   \item\texttt{lot}, \texttt{nolot} -- The \DescribeMacro{^^A
%     \thesis@blocks@lof}|\thesis@blocks@lof| macro will be defined
%     as either |\listoffigures| or |\relax|, respectively. As a
%     side effect, the |\listoffigures| either is or isn't going to
%     be included in the \DescribeMacro{\thesis@blocks@tables}^^A
%     |\thesis@|\discretionary{}{}{}|blocks@tables| block,
%     respectively.
%    \begin{macrocode}
\DeclareOption{nolof}{\let\thesis@blocks@lof\relax}
\DeclareOption{lof}{\let\thesis@blocks@lof\listoffigures}
%    \end{macrocode}
%   \item\texttt{cover}, \texttt{nocover} -- The
%     \DescribeMacro{\thesis@blocks@cover}|\thesis@blocks@cover|
%     either is going to expand to either the thesis cover or
%     produces no output, respectively. The
%     \DescribeMacro{\ifthesis@cover@}|\ifthesis@cover@|
%     conditional is set to \texttt{false} or \texttt{true},
%     respectively.
%    \begin{macrocode}
\newif\ifthesis@cover@
\DeclareOption{nocover}{\thesis@cover@false}
\DeclareOption{cover}{\thesis@cover@true}
%    \end{macrocode}
%   \item\texttt{digital}, \texttt{printed} -- These macrooptions
%     set the options that are appropriate for either the printed or
%     for the digital version of the document, respectively. The
%     \DescribeMacro{\ifthesis@digital@}|\ifthesis@digital@|
%     conditional is set to \texttt{true} or \texttt{false},
%     respectively.
%    \begin{macrocode}
\newif\ifthesis@digital@
\DeclareOption{digital}{%
  \ExecuteOptions{color,cover}%
  \thesis@digital@true}
\DeclareOption{printed}{%
  \ExecuteOptions{monochrome,nocover}%
  \thesis@digital@false}
%    \end{macrocode}
% \end{itemize}
% These are the default options:
%    \begin{macrocode}
\ExecuteOptions{%
  printed,12pt,twoside,final,microtype,palatino,oldtable,lot,lof}
\ProcessOptions*
%    \end{macrocode}
% The file uses English locale strings within the macros.
%    \begin{macrocode}
\thesis@requireLocale{english}
%    \end{macrocode}
% The file loads the following packages: \begin{itemize}
%   \item\textsf{xcolor} -- Adds support for color manipulation.
%   \item\textsf{ifxetex} -- Used to detect the \Hologo{XeTeX}
%     engine.
%   \item\textsf{ifluatex} -- Used to detect the \Hologo{LuaTeX}
%     engine.
%   \item\textsf{graphix} -- Adds support for the inclusion of
%     graphics files.
%   \item\textsf{pdfpages} -- Adds support for the injection of PDF
%     documents into the resulting document, namely the thesis
%     assignment.
%   \item\textsf{hyperref} -- Adds support for the injection of
%     metadata into the resulting PDF document.
%   \item\textsf{keyval} -- Adds support for parsing
%     comma-delimited lists of key-value pairs.
% \end{itemize}
%    \begin{macrocode}
\thesis@require{xcolor}
\thesis@require{graphicx}
\thesis@require{pdfpages}
\thesis@require{keyval}
\thesis@require{ifxetex}
\thesis@require{ifluatex}
%    \end{macrocode}
% If the |\thesis@microtype@| is set to true, then the
% \textsf{microtype} package gets loaded.
%    \begin{macrocode}
\ifthesis@microtype@
  \thesis@require[final,babel]{microtype}
\fi
%    \end{macrocode}
% Using the |\ifxetex| and |\ifluatex| conditionals, a compound
% \DescribeMacro{\ifthesis@xeluatex}|\ifthesis@xeluatex|
% conditional was constructed. This conditional can be used by
% subsequently loaded style files to test, whether either the
% \Hologo{XeTeX} or the \Hologo{LuaTeX} engine is being used.
%    \begin{macrocode}
{\let\x\expandafter
\x\global\x\let\x\ifthesis@xeluatex\csname if%
  \ifxetex true\else
    \ifluatex\x\x\x t\x\x\x r\x\x\x u\x\x\x e%
    \else   f\x\x\x a\x\x\x l\x\x\x s\x\x\x e%
    \fi
  \fi\endcsname}
%    \end{macrocode}
% The following packages get only loaded, when the document is
% being typeset using the \Hologo{XeTeX} or \Hologo{LuaTeX}
% engine: \begin{itemize}
%   \item\textsf{fontspec} -- Allows the selection of
%     system-installed fonts.
%   \item\textsf{unicode-math} -- Allows the selection of
%     system-installed mathematical fonts.
% \end{itemize}
% Under \Hologo{XeTeX} or \Hologo{LuaTeX}, the \textsf{TeX Gyre
% Pagella} and \textsf{TeX Gyre Pagella Math} are also selected as
% the main text and math fonts.
%    \begin{macrocode}
\ifthesis@xeluatex
  \ifthesis@palatino@
    \thesis@require{fontspec}
    \thesis@require{unicode-math}
    \setmainfont[Ligatures=TeX]{TeX Gyre Pagella}
    \setmathfont[math-style=ISO,bold-style=ISO]{texgyrepagella-math.otf}
  \fi
%    \end{macrocode}
% The following packages get only loaded, when the document is not
% being typeset using the \Hologo{XeTeX} or \Hologo{LuaTeX} engine
% and the |\ifthesis@palatino@| conditional is \texttt{true}:
% \begin{itemize}
%   \item\textsf{cmap} -- Places an explicit \texttt{ToUnicode}
%     map in the resulting PDF file, allowing for the extraction of
%     the text from the document.
%   \item\textsf{mathpazo} -- Changes the default math font family
%     to \texttt{mathpazo}.
%   \item\textsf{tgpagella} -- Changes the default roman font
%     family to \TeX\ Gyre Pagella.
%   \item\textsf{lmodern} -- Changes the default sans-serif and
%     monotype font faces to Latin Modern instead of the default
%     Computer Modern font family.
%   \item\textsf{fontenc} -- The font encoding is set to Cork.
% \end{itemize}
%    \begin{macrocode}
\else
  \ifthesis@palatino@
    \RequirePackage[resetfonts]{cmap}
    \thesis@require{lmodern}
    \thesis@require{mathpazo}
    \thesis@require{tgpagella}
    \RequirePackage[T1]{fontenc}
  \fi
\fi
%    \end{macrocode}
% If the |\thesis@newtable@| conditional is \texttt{true}, then
% some of the dimensions associated with tables are modified in
% preparation for the coloring of the table cells. The following
% packages are also loaded:
% \begin{itemize}
%   \item\textsf{tabularx} -- Provides the \texttt{tabularx}
%     environment, which enables the typesetting of tables with
%     flexible-width columns.
%   \item\textsf{tabu} -- Provides the \texttt{tabu} environment,
%     which enables the typesetting of complex tables.
%   \item\textsf{booktabs} -- A package, which allows the creation
%     of publication-quality tables in \LaTeX.
% \end{itemize}
%    \begin{macrocode}
\let\thesis@newtable@old\tabular
\let\endthesis@newtable@old\endtabular
\ifthesis@newtable@
  % Load the packages.
  \thesis@require{tabularx}
  \thesis@require{tabu}
  \thesis@require{booktabs}
  % Adjust the measurements.
  \setlength{\aboverulesep}{0pt}
  \setlength{\belowrulesep}{0pt}
  \setlength{\extrarowheight}{.75ex}
%    \end{macrocode}
% When both the |\thesis@newtable@| and |\thesis@color@| conditionals are
% \texttt{true}, then the \texttt{tabular} and \texttt{tabularx}
% environments are redefined to better match the style of the given
% faculty.
% 
% The \DescribeMacro{\thesis@newtable@old}|\thesis@newtable@old|
% and \DescribeMacro{\endthesis@newtable@old}
% |\endthesis@newtable@old| macros containing the original
% definition of the |tabular| environment are always defined and
% are available for subsequently loaded styles in case the
% typesetting of unaltered tables is required. Similarly, the
% \DescribeMacro{\thesis@newtable@oldx}|\thesis@newtable@oldx| and
% \DescribeMacro{\endthesis@newtable@oldx}|\endthesis@newtable@oldx|
% macros are defined for the |tabularx| environment and the
% \DescribeMacro{\thesis@newtable@oldtabu}|\thesis@newtable@oldtabu|
% and \DescribeMacro{\endthesis@newtable@oldtabu}
% |\endthesis@newtable@oldtabu| for the |tabu| environment.
%    \begin{macrocode}
  \ifthesis@color@
    % The redefinition of `tabular`
    \renewenvironment{tabular}%
      {\rowcolors{1}{thesis@color@tableOdd}%
                    {thesis@color@tableEven}%
       \thesis@newtable@old}%
      {\endthesis@newtable@old}
    % The redefinition of `tabularx`
    \let\thesis@newtable@oldx\tabularx
    \let\endthesis@newtable@oldx\endtabularx
    \renewenvironment{tabularx}%
      {\rowcolors{1}{thesis@color@tableEven}%
                    {thesis@color@tableOdd}%
       \thesis@newtable@oldx}%
      {\endthesis@newtable@oldx}
    % The redefinition of `tabu`
    \let\thesis@newtable@oldtabu\tabu
    \let\endthesis@newtable@oldtabu\endtabu
    \renewenvironment{tabu}%
      {\rowcolors{1}{thesis@color@tableEven}%
                    {thesis@color@tableOdd}%
       \thesis@newtable@oldtabu}%
      {\endthesis@newtable@oldtabu}
  \fi
\fi
%    \end{macrocode}
% \begin{macro}{\ifthesis@bibliography@}
% A new conditional, |\ifthesis@bibliography@|, is defined. This
% conditional is true, when |\thesis@bibFiles| expands to a
% non-empty token list.
%    \begin{macrocode}
\newif\ifthesis@bibliography@
\thesis@bibliography@false
\ifx\thesis@bibFiles\undefined\else
  {\edef\@bibList{\thesis@bibFiles}%
  \ifx\@bibList\empty\else
    \global\thesis@bibliography@true
  \fi}
\fi
%    \end{macrocode}
% \end{macro}
% \begin{macro}{\thesis@bibliography@setup}
% The file defines the |\thesis@bibliography@setup|\marg{options}
% command, where \textit{options} is a comma-delimited list of
% key-value pairs as defined by the \textsf{keyval} package. The
% command can be invoked by the subsequently loaded style
% files to define the bibliography options.
%    \begin{macrocode}
\def\thesis@bibliography@setup#1{%
  \setkeys{thesis@bibliography}{#1}}
%    \end{macrocode}
% \end{macro}
% The following key-value pairs are supported:
% \begin{enumerate}
%   \item\marg{\texttt{style}=style} -- Stores \texttt{style} in
%     \DescribeMacro{\thesis@bibliography@style}^^A
%     |\thesis@bibliography@style|, unless it has already been
%     defined (presumably by the user).
%   \item\marg{\texttt{sorting}=mode} -- Stores \texttt{mode} in
%     \DescribeMacro{\thesis@bibliography@sorting}^^A
%     |\thesis@bibliography@sorting|, unless it has already been
%     defined (presumably by the user).
% \end{enumerate}
%    \begin{macrocode}
\define@key{thesis@bibliography}{style}{%
  \ifx\thesis@bibliography@style\undefined
    \def\thesis@bibliography@style{#1}%
  \fi}
\define@key{thesis@bibliography}{sorting}{%
  \ifx\thesis@bibliography@sorting\undefined
    \def\thesis@bibliography@sorting{#1}%
  \fi}
%    \end{macrocode}
% \begin{macro}{\thesis@bibliography@load}
% When |\ifthesis@bibliography@| is true and
% |\ifthesis@bibliography@loaded@| is false, the
% |\thesis@bibliography@load| macro loads the \textsf{csquotes} and
% Bib\LaTeX\ packages with the bibliography databases specified in
% |\thesis@bibFiles|. The macro also sets the
% \DescribeMacro{\thesis@bibliography@loaded}^^A
% |\ifthesis@bibliography@loaded@| conditional to true.
%    \begin{macrocode}
\newif\ifthesis@bibliography@loaded@
  \thesis@bibliography@loaded@false
\newif\ifthesis@bibliography@included@
  \thesis@bibliography@included@false
\def\thesis@bibliography@load{%
  \ifthesis@bibliography@
    \ifthesis@bibliography@loaded@\else
      \thesis@bibliography@loaded@true
      % Load csquotes and BibLaTeX.
      \thesis@require{csquotes}
%    \end{macrocode}
% If |\thesis@bibliography@style| is undefined, the bibliography
% and citation styles default to |iso-numeric|. If
% |\thesis@bibliography@sorting| is undefined, the sorting scheme
% defaults to |none|.
%    \begin{macrocode}
      \thesis@bibliography@setup{
        style = iso-numeric,
        sorting = none}
      \thesis@require[
        backend=biber,
        style=\thesis@bibliography@style,
        sorting=\thesis@bibliography@sorting,
        autolang=other,
        sortlocale=auto]{biblatex}
      % Load the bibliography databases.
      {\edef\@bibList{\thesis@bibFiles}%
      \def\@inject##1,{%
        \def\@args{##1}\def\@relax{\relax}%
        \ifx\@args\@relax\else
          % Trim leading spaces.
          \edef\@trimmed{\romannumeral-`\.##1}%
          \addbibresource\@trimmed
          \expandafter\@inject\fi}%
      \expandafter\@inject\@bibList,\relax,}%
%    \end{macrocode}
% \begin{macro}{\ifthesis@bibliography@included@}
% The |\ifthesis@bibliography@included@| conditional is true, when
% the user has manually included a bibliography into their
% document. The default value of the conditional is false and
% |\printbibliography| is patched to set the conditional to true on
% expansion. This enables the user to place the bibliography
% manually without it appearing in the autolayout. The original
% macro is stored in the
% \DescribeMacro{\thesis@printbibliography@old}^^A
% |\thesis@printbibliography@old| macro.
%    \begin{macrocode}
      \let\thesis@printbibliography@old\printbibliography
      \def\printbibliography{%
        \global\thesis@bibliography@included@true
        \thesis@printbibliography@old}
    \fi
  \fi}
%    \end{macrocode}
% \end{macro} ^^A The nested \ifthesis@bibliography@included@ def
% \end{macro} ^^A The \thesis@bibliography@load macro definition
% The \textsf{hyperref} package is configured to support both roman
% and arabic page numbering in one document and to decorate
% hyperlinks with an underline instead of a rectangular box.
%    \begin{macrocode}
\thesis@require{hyperref}
\hypersetup{pdfborderstyle={/S/U/W 1}} % Less obtrusive borders
%    \end{macrocode}
% Clubs and widows are set to be infinitely bad.
%    \begin{macrocode}
\widowpenalty 10000
\clubpenalty  10000
%    \end{macrocode}
% \begin{macro}{\thesis@color@setup}
% The file defines the |\thesis@color@setup|\marg{colors} command,
% where \textit{colors} is a comma-delimited list of key-value
% pairs as defined by the \textsf{keyval} package. The command can
% be invoked either by the subsequently loaded style files or by
% the user to define which colors will be used, when the
% \texttt{color} option is specified.
%    \begin{macrocode}
\def\thesis@color@setup#1{%
  \setkeys{thesis@color}{#1}}
%    \end{macrocode}
% \end{macro}
% The following key-value pairs are supported:
% \begin{enumerate}
%   \item\marg{\texttt{links}=color} -- Sets the color of hyperref
%     links to \textit{color} and stores it under the name
%     \texttt{thesis@color@links}. The default color of links is
%     specified by the \textsf{hyperref} package.
%   \item\marg{\texttt{tableOdd}=color} -- Stores the color of the
%     odd rows of the redefined \texttt{tabular} and
%     \texttt{tabularx} environments under the name
%     \texttt{thesis@color@tableOdd}.
%   \item\marg{\texttt{tableEven}=color} -- Stores the color of the
%     even rows of the redefined \texttt{tabular} and
%     \texttt{tabularx} environments under the name
%     \texttt{thesis@color@tableEven}.
%   \item\marg{\texttt{tableEmph}=color} -- Stores the color of an
%     emphasized cell in a table user under the name
%     \texttt{thesis@color@tableEmph}. This color is meant to be
%     used manually by the user.
% \end{enumerate}
%    \begin{macrocode}
\define@key{thesis@color}{links}{%
  \definecolor{thesis@color@links}#1
  \hypersetup{linkbordercolor=thesis@color@links}}
\define@key{thesis@color}{tableOdd}{%
  \definecolor{thesis@color@tableOdd}#1}
\define@key{thesis@color}{tableEven}{%
  \definecolor{thesis@color@tableEven}#1}
\define@key{thesis@color}{tableEmph}{%
  \definecolor{thesis@color@tableEmph}#1}
%    \end{macrocode}
% The file defines several blocks to be used in the redefinitions
% of the |\thesis@blocks@preamble| and |\thesis@blocks@postamble|
% macros by the subsequently loaded style files.
%
% \begin{macro}{\thesis@blocks@coverMatter}
% The |\thesis@blocks@coverMatter| macro sets up the style
% of the cover and the title page of the thesis. This amounts
% to disabling the page numbering, so that hyperref links do not
% point to the cover page and the title page instead of the
% initial pages of the main matter.
% \begin{macrocode}
\def\thesis@blocks@coverMatter{%
  \pagenumbering{gobble}}
%    \end{macrocode}
% \end{macro}\begin{macro}{\thesis@blocks@frontMatter}
% The |\thesis@blocks@frontMatter| macro sets up the style
% of the front matter of the thesis.
% \begin{macrocode}
\def\thesis@blocks@frontMatter{%
  \thesis@blocks@clear
  \pagestyle{plain}
  \parindent 1.5em
  \setcounter{page}{1}
  \pagenumbering{roman}}
%    \end{macrocode}
% \end{macro}\begin{macro}{\thesis@blocks@clear}
% The |\thesis@blocks@clear| macro clears the current page.
% It also clears the next left-hand (even-numbered) page, when
% double-sided typesetting is enabled.
% \begin{macrocode}
\def\thesis@blocks@clear{%
  \ifthesis@twoside@
    \clearpage
    \thispagestyle{empty}%
    \cleardoublepage
  \else
    \newpage
  \fi}
%    \end{macrocode}
% \end{macro}\begin{macro}{\thesis@blocks@clearRight}
% The |\thesis@blocks@clearRight| macro clears the current
% page. It also clears the next right-hand (odd-numbered) page,
% when double-sided typesetting is enabled.
% \begin{macrocode}
\def\thesis@blocks@clearRight{%
  \ifthesis@twoside@
    \clearpage
    \ifodd\value{page}%
      \thispagestyle{empty}%
      \hbox{}%
      \newpage
    \fi
  \else
    \newpage
  \fi}
%    \end{macrocode}
% \end{macro}\begin{macro}{\thesis@blocks@facultyLogo@monochrome}
% The |\thesis@blocks@facultyLogo@monochrome|\oarg{options} 
% macro typesets the |\thesis@logopath\thesis@facultyLogo| logo
% with the given \textit{options} passed to |\includegraphics|.
% \begin{macrocode}
\newcommand{\thesis@blocks@facultyLogo@monochrome}[1]%
  [width=40mm]{{%
    \edef\@path{\thesis@logopath\thesis@facultyLogo}%
    \includegraphics[#1]{\@path}}}
%    \end{macrocode}
% \end{macro}\begin{macro}{\thesis@blocks@facultyLogo@color}
% The |\thesis@blocks@facultyLogo@color|\oarg{options} 
% macro typesets either the |\thesis@logopath\thesis@facultyLogo|
% logo, if the |\ifthesis|\discretionary{}{}{}|@color@| conditional
% is \texttt{false}, or the
% |\thesis@logopath\thesis@facultyLogo-color| logo
% otherwise with the given \textit{options} passed to
% |\includegraphics|.
% \begin{macrocode}
\newcommand{\thesis@blocks@facultyLogo@color}[1]%
  [width=40mm]{{%
    \edef\@path{\thesis@logopath\thesis@facultyLogo
      \ifthesis@color@-color\fi}%
    \includegraphics[#1]{\@path}}}
%    \end{macrocode}
% \end{macro}\begin{macro}{\thesis@blocks@universityLogo@monochrome}
% The |\thesis@blocks@universityLogo@monochrome|\oarg{options}
% macro typesets the
% |\thesis@logopath\thesis@universityLogo| logo
% with the given \textit{options} passed to |\includegraphics|.
% \begin{macrocode}
\newcommand{\thesis@blocks@universityLogo@monochrome}[1]%
  [width=40mm]{{%
    \edef\@path{\thesis@logopath\thesis@universityLogo}%
    \includegraphics[#1]{\@path}}}
%    \end{macrocode}
% \end{macro}\begin{macro}{\thesis@blocks@universityLogo@color}
% The |\thesis@blocks@universityLogo@color|\oarg{options} 
% macro typesets either the |\thesis@logopath\thesis@universityLogo|
% logo, if the |\ifthesis|\discretionary{}{}{}|@color@| conditional
% is \texttt{false}, or the
% |\thesis@logopath\thesis@universityLogo|\discretionary{}{}{}|-color|
% logo otherwise with the given \textit{options} passed to
% |\includegraphics|.
% \begin{macrocode}
\newcommand{\thesis@blocks@universityLogo@color}[1]%
  [width=40mm]{{%
    \edef\@path{\thesis@logopath\thesis@universityLogo
      \ifthesis@color@-color\fi}%
    \includegraphics[#1]{\@path}}}
%    \end{macrocode}
% The |\thesis@department@name| and |\thesis@field@name| macros and
% their English counterparts provide a level of indirection that
% allows the subsequently loaded style files to parse the values of
% |\thesis@department| and |\thesis@field| (and their English
% counterparts) and map them to human-readable names, which will
% then be typeset.
% \begin{macrocode}
\let\thesis@department@name\thesis@department
\let\thesis@departmentEn@name\thesis@departmentEn
\let\thesis@field@name\thesis@field
\let\thesis@fieldEn@name\thesis@fieldEn
%    \end{macrocode}
% \end{macro}\begin{macro}{\thesis@blocks@cover}
% The |\thesis@blocks@cover| macro typesets the thesis
% cover. It is composed of three macros:
% \begin{itemize}
%   \item\DescribeMacro{\thesis@blocks@cover@header}^^A
%        |\thesis@blocks@cover@header| -- The header of the cover
%        page
%   \item\DescribeMacro{\thesis@blocks@cover@content}^^A
%        |\thesis@blocks@cover@content| -- The content of the cover
%        page
%   \item\DescribeMacro{\thesis@blocks@cover@footer}^^A
%        |\thesis@blocks@cover@footer| -- The footer of the cover
%        page
% \end{itemize}
% This allows the subsequently loaded style files to only redefine
% certain parts of the cover page.
% \begin{macrocode}
\def\thesis@blocks@cover{%
  \ifthesis@cover@
    \thesis@blocks@clear
    \begin{alwayssingle}%
      \thispagestyle{empty}%
      \begin{center}%
        \thesis@blocks@cover@header
        \thesis@blocks@facultyLogo@monochrome\\[0.4in]%
        \let\footnotesize\small
        \let\footnoterule\relax{}%
        \thesis@blocks@cover@content
        \par\vfill
        \thesis@blocks@cover@footer
      \end{center}%
    \end{alwayssingle}%
  \fi}
%    \end{macrocode}
% The output of the |\thesis@blocks@cover@header| macro is
% controlled by the following conditionals:
% \begin{enumerate}
%   \item|\ifthesis@blocks@cover@university@| -- This
%        conditional expression determines, whether the university
%        name is going to be included in the header of the cover.
%        The default value of this conditional is \texttt{true}.
%   \item|\ifthesis@blocks@cover@faculty@| -- This
%        conditional expression determines, whether the faculty
%        name is going to be included in the header of the cover.
%        The default value of this conditional is \texttt{true}.
%   \item|\ifthesis@blocks@cover@department@| -- This
%        conditional expression determines, whether the department
%        name is going to be included in the header of the cover.
%        The default value of this conditional is \texttt{false}.
%   \item|\ifthesis@blocks@cover@field@| -- This
%        conditional expression determines, whether the field of
%        study is going to be included in the header of the cover.
%        The default value of this conditional is \texttt{false}.
% \end{enumerate}
% The sebsequently loaded style files can modify the value of these
% conditionals to alter the output of the
% |\thesis@blocks@cover@header| macro without altering its
% definition.
% \begin{macrocode}
\newif\ifthesis@blocks@cover@university@
\thesis@blocks@cover@university@true
\newif\ifthesis@blocks@cover@faculty@
\thesis@blocks@cover@faculty@true
\newif\ifthesis@blocks@cover@department@
\thesis@blocks@cover@department@false
\newif\ifthesis@blocks@cover@field@
\thesis@blocks@cover@field@false

\def\thesis@blocks@cover@header{%
  {\sc\ifthesis@blocks@cover@university@
        \thesis@titlePage@LARGE\thesis@@{universityName}\\%
   \fi\ifthesis@blocks@cover@faculty@
        \thesis@titlePage@Large\thesis@@{facultyName}\\%
   \fi\ifthesis@blocks@cover@department@
        \thesis@titlePage@large\thesis@department@name\\%
      \fi}
  \ifthesis@blocks@cover@field@
    {\thesis@titlePage@large\vskip 1em%
      {\bf\thesis@@{fieldTitle}:} \thesis@field@name}%
  \fi\vskip 2em}
\def\thesis@blocks@cover@content{%
  {\thesis@titlePage@Huge\bf\thesis@TeXtitle\par\vfil}%
  \vskip 0.8in%
  {\thesis@titlePage@large\sc\thesis@@{typeName}}\\[0.3in]%
  {\thesis@titlePage@Large\bf\thesis@author}}
\def\thesis@blocks@cover@footer{%
  {\thesis@titlePage@large\thesis@place, \thesis@@{semester}}}
%    \end{macrocode}
% \end{macro}
% \begin{macro}{\thesis@blocks@titlePage}
% The |\thesis@blocks@titlePage| macro typesets the thesis
% title page. It is composed of three macros:
% \begin{itemize}
%   \item|\thesis@blocks@titlePage@header| -- The header of the
%        cover page
%   \item|\thesis@blocks@titlePage@content| -- The content of the
%        cover page
%   \item|\thesis@blocks@titlePage@footer| -- The footer of the
%        cover page
% \end{itemize}
% This allows the subsequently loaded style files to only redefine
% certain parts of the title page.
%    \begin{macrocode}
\def\thesis@blocks@titlePage{%
    \thesis@blocks@clear
    \begin{alwayssingle}%
      \thispagestyle{empty}%
      \begin{center}%
        \thesis@blocks@titlePage@header
        \thesis@blocks@facultyLogo@color\\[0.4in]%
        \let\footnotesize\small
        \let\footnoterule\relax{}%
        \thesis@blocks@titlePage@content
        \par\vfill
        \thesis@blocks@titlePage@footer
      \end{center}%
    \end{alwayssingle}}
%    \end{macrocode}
% The output of the |\thesis@blocks@titlePage@header| macro is
% controlled by the following conditionals:
% \begin{enumerate}
%   \item|\ifthesis@blocks@titlePage@university@| -- This
%        conditional expression determines, whether the university
%        name is going to be included in the header of the title
%        page. The default value of this conditional is
%        \texttt{true}.
%   \item|\ifthesis@blocks@titlePage@faculty@| -- This
%        conditional expression determines, whether the faculty of
%        study is going to be included in the header of the title
%        page. The default value of this conditional is
%        \texttt{true}.
%   \item|\ifthesis@blocks@titlePage@department@| -- This
%        conditional expression determines, whether the department
%        name is going to be included in the header of the title
%        page. The default value of this conditional is
%        \texttt{false}.
%   \item|\ifthesis@blocks@titlePage@field@| -- This
%        conditional expression determines, whether the field of
%        study is going to be included in the header of the title
%        page. The default value of this conditional is
%        \texttt{false}.
% \end{enumerate}
% The sebsequently loaded style files can modify the value of these
% conditionals to alter the output of the
% |\thesis@blocks@titlePage@header| macro without altering its
% definition.
% \begin{macrocode}
\newif\ifthesis@blocks@titlePage@university@
\thesis@blocks@titlePage@university@true
\newif\ifthesis@blocks@titlePage@faculty@
\thesis@blocks@titlePage@faculty@true
\newif\ifthesis@blocks@titlePage@department@
\thesis@blocks@titlePage@department@false
\newif\ifthesis@blocks@titlePage@field@
\thesis@blocks@titlePage@field@false

\def\thesis@blocks@titlePage@header{%
  {\sc\ifthesis@blocks@titlePage@university@
        \thesis@titlePage@LARGE\thesis@@{universityName}\\%
   \fi\ifthesis@blocks@titlePage@faculty@
        \thesis@titlePage@Large\thesis@@{facultyName}\\%
   \fi\ifthesis@blocks@titlePage@department@
        \thesis@titlePage@large\thesis@department@name\\%
      \fi}
  \ifthesis@blocks@titlePage@field@
    {\thesis@titlePage@large\vskip 1em%
      {\bf\thesis@@{fieldTitle}:} \thesis@field@name}%
  \fi\vskip 2em}
\let\thesis@blocks@titlePage@content=\thesis@blocks@cover@content
\let\thesis@blocks@titlePage@footer=\thesis@blocks@cover@footer
%    \end{macrocode}
% \end{macro}\begin{macro}{\thesis@blocks@toc}
% The |\thesis@blocks@toc| macro typesets the table of contents.
% \begin{macrocode}
\def\thesis@blocks@toc{%
  \thesis@blocks@clear
  \tableofcontents}
%    \end{macrocode}
% \end{macro}\begin{macro}{\thesis@blocks@tables}
% The |\thesis@blocks@tables| macro typesets the table of
% contents and optionally the list of tables and the
% list of figures.
% \begin{macrocode}
\def\thesis@blocks@tables{%
  \thesis@blocks@toc
  \thesis@blocks@lot
  \thesis@blocks@lof}
%    \end{macrocode}
% \end{macro}\begin{macro}{\thesis@blocks@declaration}
% The |\thesis@blocks@declaration| macro typesets the
% declaration text.
% \begin{macrocode}
\def\thesis@blocks@declaration{%
  \thesis@blocks@clear
  \begin{alwayssingle}%
    \chapter*{\thesis@@{declarationTitle}}%
    \thesis@declaration
    \vskip 2cm%
    \hfill\thesis@author
  \end{alwayssingle}}
%    \end{macrocode}
% \end{macro}\begin{macro}{\thesis@blocks@thanks}
% The |\thesis@blocks@thanks| macro typesets the
% acknowledgement, if the |\thesis@thanks| macro is
% defined. Otherwise, the macro produces no output.
% \begin{macrocode}
\def\thesis@blocks@thanks{%
  \ifx\thesis@thanks\undefined\else
    \thesis@blocks@clear
    \begin{alwayssingle}%
      \chapter*{\vspace*{\fill}\thesis@@{thanksTitle}}%
      \leavevmode\thesis@thanks
    \end{alwayssingle}%
  \fi}
%    \end{macrocode}
% \end{macro}\begin{macro}{\thesis@blocks@abstract}
% The |\thesis@blocks@abstract| macro typesets the
% abstract.
% \begin{macrocode}
\def\thesis@blocks@abstract{%
  \begin{alwayssingle}%
    % Start the new chapter without clearing the right page
    {\def\cleardoublepage{}
    \chapter*{\thesis@@{abstractTitle}}}%
    \noindent\thesis@abstract
    \par\vfil\null
  \end{alwayssingle}}
%    \end{macrocode}
% \end{macro}\begin{macro}{\thesis@blocks@abstractEn}
% The |\thesis@blocks@abstractEn| macro typesets the
% abstract in English. If the current locale is English, the
% macro produces no output.
% \begin{macrocode}
\def\thesis@blocks@abstractEn{%
  \ifthesis@english\else
    {\thesis@selectLocale{english}%
    \begin{alwayssingle}%
      % Start the new chapter without clearing the right page
      {\def\cleardoublepage{}%
      \chapter*{\thesis@english@abstractTitle}%
      \thesis@abstractEn}%
      \par\vfil\null
    \end{alwayssingle}}%
  \fi}
%    \end{macrocode}
% \end{macro}\begin{macro}{\thesis@blocks@keywords}
% The |\thesis@blocks@keywords| macro typesets the
% keywords.
% \begin{macrocode}
\def\thesis@blocks@keywords{%
  \begin{alwayssingle}%
    % Start the new chapter without clearing the right page
    {\def\cleardoublepage{}%
    \chapter*{\thesis@@{keywordsTitle}}%
    \noindent\thesis@TeXkeywords}%
    \par\vfil\null
  \end{alwayssingle}}
%    \end{macrocode}
% \end{macro}\begin{macro}{\thesis@blocks@keywordsEn}
% The |\thesis@blocks@keywordsEn| macro typesets the
% keywords in English. If the current locale is English, the
% macro produces no output.
% \begin{macrocode}
\def\thesis@blocks@keywordsEn{%
  \ifthesis@english\else
    {\thesis@selectLocale{english}%
    \begin{alwayssingle}%
      % Start the new chapter without clearing the right page
      {\def\cleardoublepage{}%
      \chapter*{\thesis@english@keywordsTitle}%
      \thesis@TeXkeywordsEn}%
      \par\vfil\null
    \end{alwayssingle}}%
  \fi}
%    \end{macrocode}
% \end{macro}\begin{macro}{\thesis@rewind}
% The |\thesis@rewind| macro rewinds the page numbers by either one
% or two pages, depending on whether one-side or two-side
% typesetting is enabled, respectively.
% \begin{macrocode}
\def\thesis@rewind{%
  \addtocounter{page}{-\ifthesis@twoside@2\else1\fi}}
%    \end{macrocode}
% \end{macro}\begin{macro}{\thesis@blocks@assignment}
% The |\thesis@blocks@assignment| macro produces a different output
% depending on the values of the |\thesis@assignmentFiles|.
% |\ifthesis@blocks@assignment@|,
% |\ifthesis@blocks@assignment@hideIfDigital@|, and
% |\ifthesis@digital| macros.
%
% The default value of the
% \DescribeMacro{\ifthesis@blocks@assignment@}^^A
% |\ifthesis@blocks@assignment@| and
% \DescribeMacro{\ifthesis@blocks@assignment@hideIfDigital}^^A
% |\ifthesis@blocks@assignment@hideIfDigital@| conditionals is
% \texttt{true}.
% \begin{macrocode}
\newif\ifthesis@blocks@assignment@
\thesis@blocks@assignment@true
\newif\ifthesis@blocks@assignment@hideIfDigital@
\thesis@blocks@assignment@hideIfDigital@true
\def\thesis@blocks@assignment{%
%    \end{macrocode}
% If the |\ifthesis@blocks@assignment@| conditional is
% \textsf{true} and the |\thesis@assignmentFiles| macro is
% undefined, then typeset a placeholder page.
% \begin{macrocode}
  \ifthesis@blocks@assignment@
    \ifx\thesis@assignmentFiles\undefined
      % Rewind the pages and typeset a placeholder page.
      \thesis@blocks@clear
      \begin{alwayssingle}%
        \thispagestyle{empty}\thesis@rewind
        \noindent\textit{\thesis@@{assignment}}%
      \end{alwayssingle}%
    \else
%    \end{macrocode}
% Locally define \DescribeMacro{\@inject}|\@inject| as our routine
% for injecting lists of PDF documents.
% \begin{macrocode}
      {\edef\@pdfList{\thesis@assignmentFiles}%
      \let\ea\expandafter
      % Injects the specified PDF documents.
      \def\@inject##1,{\thesis@blocks@clear
        \def\@args{##1}\def\@relax{\relax}%
        \ifx\@args\@relax\else
          % Trim leading spaces.
          \edef\@trimmed{\romannumeral-`\.##1}%
          % Rewind the pages and include the PDF.
          \thesis@rewind\includepdf[pages=-]\@trimmed
          \ea\@inject\fi}%
%    \end{macrocode}
% If the |\ifthesis@blocks@assignment@| conditional is
% \textsf{true} and the |\thesis@assignmentFiles| macro is neither
% undefined nor empty, then typeset a placeholder page, if the
% |\ifthesis@digital@| conditional is \textsf{true} and the
% |\ifthesis@blocks@assignmane@hideIfDigital@| conditional is
% \textsf{true}
% \begin{macrocode}
      \ifx\@pdfList\empty\else
        \ifthesis@digital@
          \ifthesis@blocks@assignment@hideIfDigital@
            \thesis@blocks@clear
            \begin{alwayssingle}%
              \thispagestyle{empty}\thesis@rewind
              \noindent\textit{\thesis@@{assignment}}%
            \end{alwayssingle}%
%    \end{macrocode}
% If the |\ifthesis@blocks@assignment@| conditional is
% \textsf{true} and the |\thesis@assignmentFiles| macro is neither
% undefined nor empty, then inject the PDF documents specified in
% the |\thesis@assignmentFiles|, if the |\ifthesis@digital@|
% conditional is \textsf{false} or the
% |\ifthesis@blocks@assignment@hideIfDigital@| conditional is
% \textsf{false}.
% \begin{macrocode}
          \else
            \ea\@inject\@pdfList,\relax,%
          \fi
        \else
          \ea\@inject\@pdfList,\relax,%
        \fi
      \fi}%
    \fi
  \fi}
%    \end{macrocode}
% \end{macro}\begin{macro}{\thesis@pages@frontMatter}
% The \cs{thesis@pages@frontMatter} macro contains the last page
% number within the front matter of the document. During the
% first \TeX{} compilation, the macro expands to ??.
% \changes{v0.3.45}{2017/05/21}{Defined the
%   \cs{thesis@pages@frontMatter} macro in
%   \texttt{style/mu/fithesis-base.sty}. The patch was submitted
%   by Juraj Pálenik. [VN]}
% \begin{macrocode}
\ifx\thesis@pages@frontMatter\undefined
  \def\thesis@pages@frontMatter{??}\fi
%    \end{macrocode}
% \end{macro}\begin{macro}{\thesis@blocks@mainMatter}
% The |\thesis@blocks@mainMatter| macro sets up the style
% of the main matter of the thesis, defines the
% \cs{thesis@pages@frontMatter} macro and also writes the
% definition to the auxiliary file.
% \begin{macrocode}
\def\thesis@blocks@mainMatter{%
  \gdef\thesis@pages@frontMatter{\thepage}
  \write\@auxout{\noexpand\gdef\noexpand
    \thesis@pages@frontMatter{\thepage}}
  \thesis@blocks@clear
  \setcounter{page}{1}
  \pagenumbering{arabic}
  \pagestyle{thesisheadings}
  \parindent 1.5em\relax}
%    \end{macrocode}
% \end{macro}\begin{macro}{\thesis@blocks@bibEntry}
% The |\thesis@blocks@bibEntry| macro typesets a bibliographical
% entry. Along with the macros required by the locale file
% interface, the \textit{locale} files need to define the following
% strings:
% \begin{itemize}
%   \item\texttt{bib@title} -- The title of the entire block
%   \item\texttt{bib@pages} -- The abbreviation of pages used in
%     the bibliographical entry
% \end{itemize}
%    \begin{macrocode}
\def\thesis@blocks@bibEntry{%
  \chapter*{\thesis@@{bib@title}}
  \noindent\thesis@upper{author@tail}, \thesis@author@head.
  \emph{\thesis@title}. \thesis@place: \thesis@@{universityName},
  \thesis@@{facultyName}, \thesis@department@name, \thesis@year.
  \thesis@pages\ \thesis@@{bib@pages}.
  \thesis@@{advisorTitle}: \thesis@advisor
  \thesis@blocks@clearRight}
%    \end{macrocode}
% \end{macro}\begin{macro}{\thesis@blocks@bibliography}
% When |\ifthesis@bibliography@loaded@| is true and
% |\ifthesis@bibliography@included@| false, then the
% |\thesis@blocks@bibliography| macro typesets a bibliography via
% the Bib\LaTeX\ package. Otherwise, this macro produces no output.
%    \begin{macrocode}
\def\thesis@blocks@bibliography{%
  \ifthesis@bibliography@loaded@
    \ifthesis@bibliography@included@\else
      \thesis@blocks@clear
      {\emergencystretch=3em%
      \printbibliography[heading=bibintoc]}%
    \fi
  \fi}
%    \end{macrocode}
% \end{macro}
% The rest of the file comprises redefinitions of \LaTeX\ commands
% and private \texttt{rapport3} class macros altering the layout of
% the resulting document. Depending on the type size of 10, 11 or
% 12 points, either the \texttt{fithesis-10.clo},
% \texttt{fithesis-11.clo} or \texttt{fithesis-12.clo} file is
% loaded from the |\thesis@stylepath| |mu| directory, respectively.
%    \begin{macrocode}
% Table of contents will contain sectioning commands up to
% \subsection.
\setcounter{tocdepth}{2}

% Sections up to \subsection will be numbered.
\setcounter{secnumdepth}{2}

% Load the `fithesis-1*.clo` size option.
\input\thesis@stylepath mu/fithesis-1\thesis@ptsize.clo\relax

% Define the running heading style.
\def\ps@thesisheadings{%
  \def\chaptermark##1{%
    \markright{%
      \ifnum\c@secnumdepth >\m@ne
        \thechapter.\ %
      \fi ##1}}
  \let\@oddfoot\@empty
  \let\@oddhead\@empty
  \def\@oddhead{%
    \vbox{%
      \hbox to \textwidth{%
      \hfil{\sc\rightmark}}%
      \vskip 4pt\hrule}}
  \if@twoside
    \def\@evenhead{%
      \vbox{%
        \hbox to \textwidth{%
          {\sc\rightmark}%
          \hfil}
        \vskip 4pt\hrule}}
  \else
    \let\@evenhead\@oddhead
  \fi
  \def\@oddfoot{\hfil\PageFont\thepage}
  \if@twoside
    \def\@evenfoot{\PageFont\thepage\hfil}%
  \else
    \let\@evenfoot\@oddfoot
  \fi
  \let\@mkboth\markboth}

% Redefine the style of the chapter headings.
\renewcommand*\chapter{%
  \thesis@blocks@clear
  \thispagestyle{plain}%
  \global\@topnum\z@
  \@afterindentfalse
  \secdef\@chapter\@schapter}

% Redefine the style of part headings.
\renewcommand*\part{%
  \thesis@blocks@clear
  \if@twocolumn
    \onecolumn
    \@tempswatrue
  \else
    \@tempswafalse
  \fi
  \hbox{}\vfil
  \secdef\@part\@spart}

% A macro for temporary multicol -> singlecol switching.
\newif\if@restonecol
\def\alwayssingle{%
  \@restonecolfalse\if@twocolumn\@restonecoltrue\onecolumn\fi}
\def\endalwayssingle{\if@restonecol\twocolumn\fi}

% Disable uppercasing in PDF strings.
\pdfstringdefDisableCommands{%
  \let\MakeUppercase\relax}

% Set up the ToC entries appearance.
\renewcommand*\toc@font[1]{%
  \ifcase #1\relax
    \bfseries % \chapter (0)
  \or
    \slshape  % \section (1)
  \else
    \rmfamily % \subsection (2), \subsubsection (3)
              % \paragraph, \subparagraph (4)
  \fi}

% Set up the table of contents entries for sectioning commands.
\renewcommand*\l@part[2]{%
  \ifnum \c@tocdepth >-2\relax
    \addpenalty{-\@highpenalty}%
    \ifnum \c@tocdepth >0\relax
      \addvspace{2.25em \@plus\p@}%
    \else
      \addvspace{1.0em \@plus\p@}%
    \fi
    \begingroup
      \setlength\@tempdima{3em}%
      \parindent \z@ \rightskip \@pnumwidth
      \parfillskip -\@pnumwidth
      {\leavevmode
       \normalfont \bfseries #1\hfil \hb@xt@\@pnumwidth{\hss #2}}\par
       \nobreak
         \global\@nobreaktrue
         \everypar{\global\@nobreakfalse\everypar{}}%
    \endgroup
  \fi}

\renewcommand*\l@chapter[2]{%
  \ifnum \c@tocdepth >0\relax
    \addpenalty{-\@highpenalty}%
    \addvspace{1.0em \@plus\p@}%
    \setlength\@tempdima{1.5em}%
    \begingroup
      \parindent \z@ \rightskip \@pnumwidth
      \parfillskip -\@pnumwidth
      \leavevmode \bfseries
      \advance\leftskip\@tempdima
      \hskip -\leftskip
      #1\nobreak\hfil \nobreak\hb@xt@\@pnumwidth{\hss #2}\par
      \penalty\@highpenalty
    \endgroup
  \else
    \@dottedtocline{0}{0em}{1.5em}{#1}{#2}
  \fi}

\renewcommand*\l@section{\@dottedtocline{1}{1.5em}{2.3em}}
\renewcommand*\l@subsection{\@dottedtocline{2}{3.8em}{3.2em}}
\renewcommand*\l@subsubsection{\@dottedtocline{3}{7.0em}{4.1em}}
\renewcommand*\l@paragraph{\@dottedtocline{4}{10.0em}{5.0em}}
\renewcommand*\l@subparagraph{\@dottedtocline{4}{12.0em}{6.0em}}
%    \end{macrocode}\iffalse
%</base>
% \fi\file{style/mu/fithesis-10.clo}
% This file is conditionally loaded by the
% \texttt{style/mu/base.sty} file to redefine the page geometry to
% match the type size of 10 points.
%    \begin{macrocode}
%<*opt>
%<*10pt>
\ProvidesFile{fithesis/style/mu/fithesis-10.clo}[2016/05/15]

\renewcommand{\normalsize}{\fontsize\@xpt{12}\selectfont
\abovedisplayskip 10\p@ plus2\p@ minus5\p@
\belowdisplayskip \abovedisplayskip
\abovedisplayshortskip  \z@ plus3\p@
\belowdisplayshortskip  6\p@ plus3\p@ minus3\p@
\let\@listi\@listI}

\renewcommand{\small}{\fontsize\@ixpt{11}\selectfont
\abovedisplayskip 8.5\p@ plus3\p@ minus4\p@
\belowdisplayskip \abovedisplayskip
\abovedisplayshortskip \z@ plus2\p@
\belowdisplayshortskip 4\p@ plus2\p@ minus2\p@
\def\@listi{\leftmargin\leftmargini
\topsep 4\p@ plus2\p@ minus2\p@\parsep 2\p@ plus\p@ minus\p@
\itemsep \parsep}}

\renewcommand{\footnotesize}{\fontsize\@viiipt{9.5}\selectfont
\abovedisplayskip 6\p@ plus2\p@ minus4\p@
\belowdisplayskip \abovedisplayskip
\abovedisplayshortskip \z@ plus\p@
\belowdisplayshortskip 3\p@ plus\p@ minus2\p@
\def\@listi{\leftmargin\leftmargini %% Added 22 Dec 87
\topsep 3\p@ plus\p@ minus\p@\parsep 2\p@ plus\p@ minus\p@
\itemsep \parsep}}

\renewcommand{\scriptsize}{\fontsize\@viipt{8pt}\selectfont}
\renewcommand{\tiny}{\fontsize\@vpt{6pt}\selectfont}
\renewcommand{\large}{\fontsize\@xiipt{14pt}\selectfont}
\renewcommand{\Large}{\fontsize\@xivpt{18pt}\selectfont}
\renewcommand{\LARGE}{\fontsize\@xviipt{22pt}\selectfont}
\renewcommand{\huge}{\fontsize\@xxpt{25pt}\selectfont}
\renewcommand{\Huge}{\fontsize\@xxvpt{30pt}\selectfont}

%</10pt>
%    \end{macrocode}
% \file{style/mu/fithesis-11.clo}
% This file is conditionally loaded by the
% \texttt{style/mu/base.sty} file to redefine the page geometry to
% match the type size of 11 points.
%    \begin{macrocode}
%<*11pt>
\ProvidesFile{fithesis/style/mu/fithesis-11.clo}[2016/05/15]

\renewcommand{\normalsize}{\fontsize\@xipt{14}\selectfont
\abovedisplayskip 11\p@ plus3\p@ minus6\p@
\belowdisplayskip \abovedisplayskip
\belowdisplayshortskip  6.5\p@ plus3.5\p@ minus3\p@
% \abovedisplayshortskip  \z@ plus3\@p
\let\@listi\@listI}

\renewcommand{\small}{\fontsize\@xpt{12}\selectfont
\abovedisplayskip 10\p@ plus2\p@ minus5\p@ 
\belowdisplayskip \abovedisplayskip
\abovedisplayshortskip  \z@ plus3\p@
\belowdisplayshortskip  6\p@ plus3\p@ minus3\p@
\def\@listi{\leftmargin\leftmargini
\topsep 6\p@ plus2\p@ minus2\p@\parsep 3\p@ plus2\p@ minus\p@
\itemsep \parsep}}

\renewcommand{\footnotesize}{\fontsize\@ixpt{11}\selectfont
\abovedisplayskip 8\p@ plus2\p@ minus4\p@
\belowdisplayskip \abovedisplayskip
\abovedisplayshortskip \z@ plus\p@ 
\belowdisplayshortskip 4\p@ plus2\p@ minus2\p@
\def\@listi{\leftmargin\leftmargini
\topsep 4\p@ plus2\p@ minus2\p@\parsep 2\p@ plus\p@ minus\p@
\itemsep \parsep}}

\renewcommand{\scriptsize}{\fontsize\@viiipt{9.5pt}\selectfont}
\renewcommand{\tiny}{\fontsize\@vipt{7pt}\selectfont}
\renewcommand{\large}{\fontsize\@xiipt{14pt}\selectfont}
\renewcommand{\Large}{\fontsize\@xivpt{18pt}\selectfont}
\renewcommand{\LARGE}{\fontsize\@xviipt{22pt}\selectfont}
\renewcommand{\huge}{\fontsize\@xxpt{25pt}\selectfont}
\renewcommand{\Huge}{\fontsize\@xxvpt{30pt}\selectfont}

%</11pt>
%    \end{macrocode}
% \file{style/mu/fithesis-12.clo}
% This file is conditionally loaded by the
% \texttt{style/mu/base.sty} file to redefine the page geometry to
% match the type size of 12 points. The type dimensions defined by
% the file are stored in the following macros as well:
% \begin{itemize}
%  \item\DescribeMacro{\thesis@titlePage@normalsize}%
%    |\thesis@titlePage@normalsize| -- Equivalent to |\normalsize|
%  \item\DescribeMacro{\thesis@titlePage@small}%
%    |\thesis@titlePage@small| -- Equivalent to |\small|
%  \item\DescribeMacro{\thesis@titlePage@footnotesize}%
%    |\thesis@titlePage@footnotesize| -- Equivalent to
%    |\footnotesize|
%  \item\DescribeMacro{\thesis@titlePage@scriptsize}%
%    |\thesis@titlePage@scriptsize| -- Equivalent to |\scriptsize|
%  \item\DescribeMacro{\thesis@titlePage@tiny}%
%    |\thesis@titlePage@tiny| -- Equivalent to |\tiny|
%  \item\DescribeMacro{\thesis@titlePage@large}%
%    |\thesis@titlePage@large| -- Equivalent to |\large|
%  \item\DescribeMacro{\thesis@titlePage@Large}%
%    |\thesis@titlePage@Large| -- Equivalent to |\Large|
%  \item\DescribeMacro{\thesis@titlePage@LARGE}%
%    |\thesis@titlePage@LARGE| -- Equivalent to |\LARGE|
%  \item\DescribeMacro{\thesis@titlePage@huge}%
%    |\thesis@titlePage@huge| -- Equivalent to |\huge|
%  \item\DescribeMacro{\thesis@titlePage@Huge}%
%    |\thesis@titlePage@Huge| -- Equivalent to |\Huge|
% \end{itemize}
% These macros can be used to typeset elements whose size
% should remain constant regardless of the font size setting.
%    \begin{macrocode}
%<*12pt>
\ProvidesFile{fithesis/style/mu/fithesis-12.clo}[2016/05/15]
%</12pt>

\def\thesis@titlePage@normalsize{\fontsize\@xiipt{14.5}%
\selectfont\abovedisplayskip 12\p@ plus3\p@ minus7\p@
\belowdisplayskip \abovedisplayskip
\abovedisplayshortskip  \z@ plus3\p@
\belowdisplayshortskip  6.5\p@ plus3.5\p@ minus3\p@
\let\@listi\@listI}

\def\thesis@titlePage@small{\fontsize\@xipt{13.6}\selectfont
\abovedisplayskip 11\p@ plus3\p@ minus6\p@
\belowdisplayskip \abovedisplayskip
\abovedisplayshortskip  \z@ plus3\p@
\belowdisplayshortskip  6.5\p@ plus3.5\p@ minus3\p@
\def\@listi{\leftmargin\leftmargini %% Added 22 Dec 87
\parsep 4.5\p@ plus2\p@ minus\p@
            \itemsep \parsep
            \topsep 9\p@ plus3\p@ minus5\p@}}

\def\thesis@titlePage@footnotesize{\fontsize\@xpt{12}\selectfont
\abovedisplayskip 10\p@ plus2\p@ minus5\p@
\belowdisplayskip \abovedisplayskip
\abovedisplayshortskip  \z@ plus3\p@
\belowdisplayshortskip  6\p@ plus3\p@ minus3\p@
\def\@listi{\leftmargin\leftmargini %% Added 22 Dec 87
\topsep 6\p@ plus2\p@ minus2\p@\parsep 3\p@ plus2\p@ minus\p@
\itemsep \parsep}}
            
\def\thesis@titlePage@scriptsize{\fontsize\@viiipt{9.5pt}\selectfont}
\def\thesis@titlePage@tiny{\fontsize\@vipt{7pt}\selectfont}
\def\thesis@titlePage@large{\fontsize\@xivpt{18pt}\selectfont}
\def\thesis@titlePage@Large{\fontsize\@xviipt{22pt}\selectfont}
\def\thesis@titlePage@LARGE{\fontsize\@xxpt{25pt}\selectfont}
\def\thesis@titlePage@huge{\fontsize\@xxvpt{30pt}\selectfont}
\def\thesis@titlePage@Huge{\fontsize\@xxvpt{30pt}\selectfont}

%<*12pt>
\renewcommand{\normalsize}{\thesis@titlePage@normalsize}
\renewcommand{\small}{\thesis@titlePage@small}
\renewcommand{\footnotesize}{\thesis@titlePage@footnotesize}
\renewcommand{\scriptsize}{\thesis@titlePage@scriptsize}
\renewcommand{\tiny}{\thesis@titlePage@tiny}
\renewcommand{\large}{\thesis@titlePage@large}
\renewcommand{\Large}{\thesis@titlePage@Large}
\renewcommand{\LARGE}{\thesis@titlePage@LARGE}
\renewcommand{\huge}{\thesis@titlePage@huge}
\renewcommand{\Huge}{\thesis@titlePage@Huge}
%</12pt>
\let\@normalsize\normalsize
\normalsize

\if@twoside               
   \oddsidemargin 0.75in  
   \evensidemargin 0.4in  
   \marginparwidth 0pt    
\else                     
   \oddsidemargin 0.75in  
   \evensidemargin 0.75in
   \marginparwidth 0pt
\fi
\marginparsep 10pt        

\topmargin 0.4in          
                          
\headheight 20pt          
\headsep 10pt             
\topskip 10pt    
\footskip 30pt 

%<*10pt>
\textheight = 43\baselineskip
\advance\textheight by \topskip
\textwidth 5.0truein
\columnsep 10pt       
\columnseprule 0pt

\footnotesep 6.65pt
\skip\footins 9pt plus 4pt minus 2pt
\floatsep 12pt plus 2pt minus 2pt
\textfloatsep 20pt plus 2pt minus 4pt
\intextsep 12pt plus 2pt minus 2pt
\dblfloatsep 12pt plus 2pt minus 2pt
\dbltextfloatsep 20pt plus 2pt minus 4pt

\@fptop 0pt plus 1fil
\@fpsep 8pt plus 2fil
\@fpbot 0pt plus 1fil
\@dblfptop 0pt plus 1fil
\@dblfpsep 8pt plus 2fil
\@dblfpbot 0pt plus 1fil
\marginparpush 5pt

\parskip 0pt plus 1pt
\partopsep 2pt plus 1pt minus 1pt

%</10pt>
%
%<*11pt>
\textheight = 39\baselineskip
\advance\textheight by \topskip
\textwidth 5.0truein
\columnsep 10pt
\columnseprule 0pt

\footnotesep 7.7pt
\skip\footins 10pt plus 4pt minus 2pt
\floatsep 12pt plus 2pt minus 2pt
\textfloatsep 20pt plus 2pt minus 4pt
\intextsep 12pt plus 2pt minus 2pt
\dblfloatsep 12pt plus 2pt minus 2pt
\dbltextfloatsep 20pt plus 2pt minus 4pt

\@fptop 0pt plus 1fil
\@fpsep 8pt plus 2fil
\@fpbot 0pt plus 1fil
\@dblfptop 0pt plus 1fil
\@dblfpsep 8pt plus 2fil
\@dblfpbot 0pt plus 1fil
\marginparpush 5pt 

\parskip 0pt plus 0pt
\partopsep 3pt plus 1pt minus 2pt

%</11pt>
%
%<*12pt>
\textheight = 37\baselineskip
\advance\textheight by \topskip
\textwidth 5.0truein
\columnsep 10pt
\columnseprule 0pt

\footnotesep 8.4pt
\skip\footins 10.8pt plus 4pt minus 2pt
\floatsep 14pt plus 2pt minus 4pt 
\textfloatsep 20pt plus 2pt minus 4pt
\intextsep 14pt plus 4pt minus 4pt
\dblfloatsep 14pt plus 2pt minus 4pt
\dbltextfloatsep 20pt plus 2pt minus 4pt

\@fptop 0pt plus 1fil
\@fpsep 10pt plus 2fil
\@fpbot 0pt plus 1fil
\@dblfptop 0pt plus 1fil
\@dblfpsep 10pt plus 2fil
\@dblfpbot 0pt plus 1fil
\marginparpush 7pt

\parskip 0pt plus 0pt
\partopsep 3pt plus 2pt minus 2pt

%</12pt>
\@lowpenalty   51
\@medpenalty  151
\@highpenalty 301
\@beginparpenalty -\@lowpenalty
\@endparpenalty   -\@lowpenalty
\@itempenalty     -\@lowpenalty

\def\@makechapterhead#1{{%
  \setlength\parindent{\z@}%
  \setlength\parskip  {\z@}%
  \Large \ChapFont 
  \ifnum
    \c@secnumdepth >\m@ne
    \par\nobreak
    \vskip 10\p@
    \thechapter{} \space
  \fi #1\par
  \nobreak
  \vskip 20\p@}}

\def\@makeschapterhead#1{{%
  \setlength\parindent{\z@}%
  \setlength\parskip  {\z@}%
  \Large \ChapFont #1\par
  \nobreak
  \vskip 20\p@}}

\def\chapter{%
 \clearpage
 \thispagestyle{plain}
 \global\@topnum\z@ 
 \@afterindentfalse  
 \secdef\@chapter\@schapter}

\def\@chapter[#1]#2{%
  \ifnum \c@secnumdepth
    >\m@ne
    \refstepcounter{chapter}%
    \typeout{\@chapapp\space\thechapter.}% 
    \addcontentsline{toc}{chapter}{\protect
    \numberline{\thechapter}\bfseries #1}
  \else
    \addcontentsline{toc}{chapter}{\bfseries #1}
  \fi
  \chaptermark{#1}%
  \addtocontents{lof}%
  {\protect\addvspace{4\p@}} 
  \addtocontents{lot}%
  {\protect\addvspace{4\p@}} 
  \if@twocolumn                   
    \@topnewpage[\@makechapterhead{#2}]%
  \else
    \@makechapterhead{#2}%
    \@afterheading          
  \fi
}

\def\section{\@startsection{section}{1}{\z@}{%
  -3.5ex plus-1ex minus-.2ex}{2.3ex plus.2ex}{%
  \reset@font\large\bfseries}}
\def\subsection{\@startsection{subsection}{2}{\z@}{%
  -3.25ex plus-1ex minus-.2ex}{1.5ex plus.2ex}{%
  \reset@font\normalsize\bfseries}}
\def\subsubsection{\@startsection{subsubsection}{3}{\z@}{%
  -3.25ex plus-1ex minus-.2ex}{1.5ex plus.2ex}{%
  \reset@font\normalsize}}
\def\paragraph{\@startsection{paragraph}{4}{\z@}{%
  3.25ex plus1ex minus.2ex}{-1em}{%
  \reset@font\normalsize\bfseries}}
\def\subparagraph{\@startsection{subparagraph}{4}{\parindent}{%
  3.25ex plus1ex minus.2ex}{-1em}{%
  \reset@font\normalsize\bfseries}}

\leftmargini 2.5em
\leftmarginii 2.2em     % > \labelsep + width of '(m)'
\leftmarginiii 1.87em   % > \labelsep + width of 'vii.'
\leftmarginiv 1.7em     % > \labelsep + width of 'M.'
\leftmarginv 1em
\leftmarginvi 1em

\leftmargin\leftmargini
\labelsep .5em
\labelwidth\leftmargini\advance\labelwidth-\labelsep

%<*10pt>
\def\@listI{\leftmargin\leftmargini \parsep 4\p@ plus2\p@ minus\p@
\topsep 8\p@ plus2\p@ minus4\p@
\itemsep 4\p@ plus2\p@ minus\p@}

\let\@listi\@listI
\@listi

\def\@listii{\leftmargin\leftmarginii
  \labelwidth\leftmarginii\advance\labelwidth-\labelsep
  \topsep 4\p@ plus2\p@ minus\p@
  \parsep 2\p@ plus\p@ minus\p@
  \itemsep \parsep}

\def\@listiii{\leftmargin\leftmarginiii
  \labelwidth\leftmarginiii\advance\labelwidth-\labelsep
  \topsep 2\p@ plus\p@ minus\p@
  \parsep \z@ \partopsep\p@ plus\z@ minus\p@
  \itemsep \topsep}

\def\@listiv{\leftmargin\leftmarginiv
  \labelwidth\leftmarginiv\advance\labelwidth-\labelsep}
   
\def\@listv{\leftmargin\leftmarginv
  \labelwidth\leftmarginv\advance\labelwidth-\labelsep}
   
\def\@listvi{\leftmargin\leftmarginvi
  \labelwidth\leftmarginvi\advance\labelwidth-\labelsep}
%</10pt>
%
%<*11pt>
\def\@listI{\leftmargin\leftmargini \parsep 4.5\p@ plus2\p@ minus\p@
\topsep 9\p@ plus3\p@ minus5\p@
\itemsep 4.5\p@ plus2\p@ minus\p@}

\let\@listi\@listI
\@listi

\def\@listii{\leftmargin\leftmarginii
  \labelwidth\leftmarginii\advance\labelwidth-\labelsep
  \topsep 4.5\p@ plus2\p@ minus\p@
  \parsep 2\p@ plus\p@ minus\p@
  \itemsep \parsep}

\def\@listiii{\leftmargin\leftmarginiii
  \labelwidth\leftmarginiii\advance\labelwidth-\labelsep
  \topsep 2\p@ plus\p@ minus\p@
  \parsep \z@ \partopsep \p@ plus\z@ minus\p@
  \itemsep \topsep}

\def\@listiv{\leftmargin\leftmarginiv
  \labelwidth\leftmarginiv\advance\labelwidth-\labelsep}
   
\def\@listv{\leftmargin\leftmarginv
  \labelwidth\leftmarginv\advance\labelwidth-\labelsep}
    
\def\@listvi{\leftmargin\leftmarginvi
  \labelwidth\leftmarginvi\advance\labelwidth-\labelsep}
%</11pt>
%
%<*12pt>
\def\@listI{\leftmargin\leftmargini \parsep 5\p@ plus2.5\p@ minus\p@
\topsep 10\p@ plus4\p@ minus6\p@
\itemsep 5\p@ plus2.5\p@ minus\p@}

\let\@listi\@listI
\@listi

\def\@listii{\leftmargin\leftmarginii
  \labelwidth\leftmarginii\advance\labelwidth-\labelsep
  \topsep 5\p@ plus2.5\p@ minus\p@
  \parsep 2.5\p@ plus\p@ minus\p@
  \itemsep \parsep}

\def\@listiii{\leftmargin\leftmarginiii
  \labelwidth\leftmarginiii\advance\labelwidth-\labelsep
  \topsep 2.5\p@ plus\p@ minus\p@
  \parsep \z@ \partopsep \p@ plus\z@ minus\p@
  \itemsep \topsep}

\def\@listiv{\leftmargin\leftmarginiv
  \labelwidth\leftmarginiv\advance\labelwidth-\labelsep}
   
\def\@listv{\leftmargin\leftmarginv
  \labelwidth\leftmarginv\advance\labelwidth-\labelsep}
    
\def\@listvi{\leftmargin\leftmarginvi
  \labelwidth\leftmarginvi\advance\labelwidth-\labelsep}
%</12pt>
%</opt>
%    \end{macrocode}

% \subsubsection{The style files of the Faculty of Informatics}
% % \file{style/mu/fi.sty}
% This is the style file for the theses written at the Faculty of
% Informatics at the Masaryk University in Brno. It has been
% prepared in accordance with the formal requirements published at
% the website of the faculty\footnote{See
% \url{https://www.fi.muni.cz/docs/BP_DP_na_FI.pdf}}.
%    \begin{macrocode}
\NeedsTeXFormat{LaTeX2e}
%    \end{macrocode}
% The style file defines the autolayout preamble as the cover and
% the title page followed by the assignment, declaration,
% acknowledgement, abstract, keywords, table of contents and list
% of tables and figures as a part of the front matter. All of the
% blocks are defined in the \texttt{style/mu/base.tex} file.
%    \begin{macrocode}
\def\thesis@preamble{%
  \thesis@blocks@cover%
  \thesis@blocks@titlePage%
  \thesis@blocks@frontMatter%
    \thesis@blocks@assignment%
    \thesis@blocks@declaration%
    \thesis@blocks@thanks%
    \thesis@blocks@abstract%
    \thesis@blocks@abstractEn%
    \thesis@blocks@keywords%
    \thesis@blocks@keywordsEn%
    \thesis@blocks@tables%
  \thesis@blocks@mainMatter%
}
%    \end{macrocode}

% \subsubsection{The style files of the Faculty of Science}
% % \file{style/mu/fithesis-sci.sty}
% This is the style file for the theses written at the Faculty of
% Science at the Masaryk University in Brno. It has been
% prepared in accordance with the formal requirements published at
% the website of the faculty\footnote{See
% \url{http://www.sci.muni.cz/NW/predpisy/od/OD-2014-05.pdf}}.
%    \begin{macrocode}
\NeedsTeXFormat{LaTeX2e}
\ProvidesPackage{fithesis/style/mu/fithesis-sci}[2017/05/28]
%    \end{macrocode}
% The file defines the color scheme of the respective faculty. Note
% the the color definitions are in RGB, which makes the resulting
% files generally unsuitable for printing.
%    \begin{macrocode}
\thesis@color@setup{
  links={HTML}{20E366},
  tableEmph={HTML}{8EDEAA},
  tableOdd={HTML}{EDF7F1},
  tableEven={HTML}{CCEDD8}}
%    \end{macrocode}
% The bibliography support is enabled. The |numeric| citations are
% used and the bibliography is sorted in citation order.
%    \begin{macrocode}
\thesis@bibliography@setup{
  style=iso-numeric,
  sorting=none}
\thesis@bibliography@load
%    \end{macrocode}
% The file uses Czech locale strings within the macros.
%    \begin{macrocode}
\thesis@requireLocale{czech}
%    \end{macrocode}
% \begin{macro}{\ifthesis@czech}
% The |\ifthesis@czech| \ldots |\else| \ldots |\fi| conditional is made
% available for testing, whether or not the current locale is Czech.
% \changes{v0.3.45}{2017/05/23}{Defined the
%   \cs{ifthesis@czech} macro in
%   \texttt{style/mu/fithesis-sci.sty}. The patch was submitted by
%   Juraj Pálenik. [VN]}
%    \begin{macrocode}
\def\ifthesis@czech{
  \expandafter\def\expandafter\@czech\expandafter{\string
  \czech}%
  \expandafter\expandafter\expandafter\def\expandafter
  \expandafter\expandafter\@locale\expandafter\expandafter
  \expandafter{\expandafter\string\csname\thesis@locale\endcsname}%
  \expandafter\csname\expandafter i\expandafter f\ifx\@locale
  \@czech
    true%
  \else
    false%
  \fi\endcsname}
\ifthesis@czech
  \expandafter\expandafter\expandafter\let\expandafter\expandafter
    \csname ifthesis@czech\endcsname\csname iftrue\endcsname
\else
  \expandafter\expandafter\expandafter\let\expandafter\expandafter
    \csname ifthesis@czech\endcsname\csname iffalse\endcsname
\fi
%    \end{macrocode}
% The file recognizes the following options: \begin{itemize}
%   \item\texttt{abstractonsinglepage} -- The abstracts are going
%   to be typeset on a single page as opposed to being spread
%   across several pages. This is the default for theses whose main
%   locale is neither Czech nor English.
% \end{itemize}
% \changes{v0.3.45}{2017/05/24}{Defined the
%   \texttt{abstractonsinglepage} option in
%   \texttt{style/mu/fithesis-sci.sty}. The patch was submitted by
%   Juraj Pálenik. [VN]}
%    \begin{macrocode}
\newif\ifthesis@abstractonsinglepage@
\DeclareOption{abstractonsinglepage}{\thesis@abstractonsinglepage@true}
\ifthesis@czech\else\ifthesis@english\else
  \ExecuteOptions{abstractonsinglepage}
\fi\fi
\ProcessOptions*
%    \end{macrocode}
% \end{macro}
% The file loads the following packages:
% \begin{itemize}
%   \item\textsf{tikz} -- Used for dimension arithmetic.
%   \item\textsf{changepage} -- Used for width adjustments.
% \end{itemize}
%    \begin{macrocode}
\thesis@require{tikz}
\thesis@require{changepage}
%    \end{macrocode}
% In case of rigorous and doctoral theses, the style file hides the
% thesis assignment in accordance with the formal requirements of
% the faculty.
% \begin{macrocode}
\ifx\thesis@type\thesis@bachelors\else
\ifx\thesis@type\thesis@masters\else
  \thesis@blocks@assignment@false
\fi\fi
%    \end{macrocode}
% Enable the inclusion of the scanned assignment inside the digital
% version of the document.
% \begin{macrocode}
\thesis@blocks@assignment@hideIfDigital@false
%    \end{macrocode}
% \begin{macro}{\thesis@blocks@bibEntry}
% The |\thesis@blocks@bibEntry| macro typesets a bibliographical
% entry. Along with the macros required by the locale file
% interface, the locale files need to define the following macros:
% \begin{itemize}
%   \item|\thesis@|\textit{locale}|@bib@title| -- The title of the
%     entire block
%   \item|\thesis@|\textit{locale}|@bib@author| -- The label of the
%     author name entry
%   \item|\thesis@|\textit{locale}|@bib@title| -- The label of the
%     title name entry
%   \item|\thesis@|\textit{locale}|@bib@programme| -- The label of
%     the programme name entry
%   \item|\thesis@|\textit{locale}|@bib@field| -- The label of the
%     field of study name entry
%   \item|\thesis@|\textit{locale}|@bib@advisor| -- The label of
%     the advisor name entry
%   \item|\thesis@|\textit{locale}|@bib@academicYear| -- The label
%     of the academic year entry
%   \item|\thesis@|\textit{locale}|@bib@pages| -- The label of the
%     number of pages entry
%   \item|\thesis@|\textit{locale}|@bib@keywords| -- The label of
%     the keywords entry
% \end{itemize}
% \changes{v0.3.45}{2017/05/26}{Bibliographical entries in
%   \texttt{style/mu/fithesis-sci.sty} now face each other when the
%   main locale is either Czech or English. [VN]}
%    \begin{macrocode}
\def\thesis@blocks@bibEntry{%
  \begin{alwayssingle}%
    % Clear only the right page, if the main locale is Czech.
    \ifthesis@czech
      \begingroup
      \let\thesis@blocks@clear\thesis@blocks@clearRight
    \fi
    \chapter*{\thesis@@{bib@title}}%
    \ifthesis@czech
      \endgroup
    \fi
    {% Calculate the width of the columns
    \let\@A\relax\newlength{\@A}\settowidth{\@A}{{%
      \bf\thesis@@{bib@author}:}}
    \let\@B\relax\newlength{\@B}\settowidth{\@B}{{%
      \bf\thesis@@{bib@thesisTitle}:}}
    \let\@C\relax\newlength{\@C}\settowidth{\@C}{{%
      \bf\thesis@@{bib@programme}:}}
    \let\@D\relax\newlength{\@D}\settowidth{\@D}{{%
      \bf\thesis@@{bib@field}:}}
    % Unless this is a rigorous thesis, we will be typesetting the
    % name of the thesis advisor.
    \let\@E\relax\newlength{\@E}
      \ifx\thesis@type\thesis@rigorous
        \setlength{\@E}{0pt}%
      \else
        \settowidth{\@E}{{\bf\thesis@@{bib@advisor}:}}
      \fi
    \let\@F\relax\newlength{\@F}\settowidth{\@F}{{%
      \bf\thesis@@{bib@academicYear}:}}
    \let\@G\relax\newlength{\@G}\settowidth{\@G}{{%
      \bf\thesis@@{bib@pages}:}}
    \let\@H\relax\newlength{\@H}\settowidth{\@H}{{%
      \bf\thesis@@{bib@keywords}:}}
    \let\@skip\relax\newlength{\@skip}\setlength{\@skip}{16pt}
    \let\@left\relax\newlength{\@left}\pgfmathsetlength{\@left}{%
      max(\@A,\@B,\@C,\@D,\@E,\@F,\@G,\@H)}
    \let\@right\relax\newlength{\@right}\setlength{\@right}{%
      \textwidth-\@left-\@skip}
    % Typeset the table
    \renewcommand{\arraystretch}{2}
    \noindent\begin{thesis@newtable@old}%
      {@{}p{\@left}@{\hskip\@skip}p{\@right}@{}}
      \textbf{\thesis@@{bib@author}:} &
        \noindent\parbox[t]{\@right}{
          \thesis@author\\
          \thesis@@{facultyName},
          \thesis@@{universityName}\\
          \thesis@department@name
        }\\
      \textbf{\thesis@@{bib@thesisTitle}:}
        & \thesis@title \\
      \textbf{\thesis@@{bib@programme}:}
        & \thesis@programme \\
      \textbf{\thesis@@{bib@field}:}
        & \thesis@field@name \\
      % Unless this is a rigorous thesis, typeset the name of the
      % thesis advisor.
      \ifx\thesis@type\thesis@rigorous\else
        \textbf{\thesis@@{bib@advisor}:}
          & \thesis@advisor \\
      \fi
      \textbf{\thesis@@{bib@academicYear}:}
        & \thesis@academicYear \\
      \textbf{\thesis@@{bib@pages}:}
        & \thesis@pages@preamble{} + \thesis@pages \\
      \textbf{\thesis@@{bib@keywords}:}
        & \thesis@TeXkeywords \\
    \end{thesis@newtable@old}}
  \end{alwayssingle}}
%    \end{macrocode}
% \end{macro}\begin{macro}{\thesis@blocks@bibEntryEn}
% The |\thesis@blocks@bibEntryEn| macro typesets a bibliographical
% entry in English unless the current locale is English.
%    \begin{macrocode}
\def\thesis@blocks@bibEntryEn{%
  \ifthesis@english\else
    {\thesis@selectLocale{english}
    \begin{alwayssingle}
      \chapter*{\thesis@english@bib@title}%
      {% Calculate the width of the columns
      \let\@A\relax\newlength{\@A}\settowidth{\@A}{{%
        \bf\thesis@english@bib@author:}}
      \let\@B\relax\newlength{\@B}\settowidth{\@B}{{%
        \bf\thesis@english@bib@thesisTitle:}}
      \let\@C\relax\newlength{\@C}\settowidth{\@C}{{%
        \bf\thesis@english@bib@programme:}}
      \let\@D\relax\newlength{\@D}\settowidth{\@D}{{%
        \bf\thesis@english@bib@field:}}
      % Unless this is a rigorous thesis, we will be typesetting
      % the name of the thesis advisor.
      \let\@E\relax\newlength{\@E}
        \ifx\thesis@type\thesis@rigorous
          \setlength{\@E}{0pt}%
        \else
          \settowidth{\@E}{{\bf\thesis@english@bib@advisor:}}
        \fi
      \let\@F\relax\newlength{\@F}\settowidth{\@F}{{%
        \bf\thesis@english@bib@academicYear:}}
      \let\@G\relax\newlength{\@G}\settowidth{\@G}{{%
        \bf\thesis@english@bib@pages:}}
      \let\@H\relax\newlength{\@H}\settowidth{\@H}{{%
        \bf\thesis@english@bib@keywords:}}
      \let\@skip\relax\newlength{\@skip}\setlength{\@skip}{16pt}
      \let\@left\relax\newlength{\@left}\pgfmathsetlength{\@left}{%
        max(\@A,\@B,\@C,\@D,\@E,\@F,\@G,\@H)}
      \let\@right\relax\newlength{\@right}\setlength{\@right}{%
        \textwidth-\@left-\@skip}
      % Typeset the table
      \renewcommand{\arraystretch}{2}
      \noindent\begin{thesis@newtable@old}%
        {@{}p{\@left}@{\hskip\@skip}p{\@right}@{}}
          \textbf{\thesis@english@bib@author:} &
            \noindent\parbox[t]{\@right}{
              \thesis@author\\
              \thesis@english@facultyName,
              \thesis@english@universityName\\
              \thesis@departmentEn@name
            }\\
          \textbf{\thesis@english@bib@thesisTitle:}
            & \thesis@titleEn \\
          \textbf{\thesis@english@bib@programme:}
            & \thesis@programmeEn \\
          \textbf{\thesis@english@bib@field:}
            & \thesis@fieldEn@name \\
          % Unless this is a rigorous thesis, typeset the name of the
          % thesis advisor.
          \ifx\thesis@type\thesis@rigorous\else
            \textbf{\thesis@english@bib@advisor:}
              & \thesis@advisor \\
          \fi
          \textbf{\thesis@english@bib@academicYear:}
            & \thesis@academicYear \\
          \textbf{\thesis@english@bib@pages:}
            & \thesis@pages@preamble{} + \thesis@pages \\
          \textbf{\thesis@english@bib@keywords:}
            & \thesis@TeXkeywordsEn \\
        \end{thesis@newtable@old}}
      \end{alwayssingle}
    }%
  \fi}
%    \end{macrocode}
% \end{macro}\begin{macro}{\thesis@blocks@abstractCs}
% The |\thesis@blocks@abstractCs| macro typesets the
% abstract in Czech. If the current locale is Czech, the
% macro produces no output. The following extra data field is
% defined for the macro: \begin{itemize}
%   \item|abstractCs| -- the Czech title of the thesis used for the
%     typesetting. This extra data field will expand to
%     |\thesis@abstract| if the current locale of the thesis
%     is Czech.
% \end{itemize}
% \changes{v0.3.45}{2017/05/28}{Defined the
%   \cs{thesis@blocks@abstractCs} macro in
%   \texttt{style/mu/fithesis-sci.sty}. The patch was submitted by
%   Juraj Pálenik. [VN]}
%    \begin{macrocode}
\thesis@def@extra[{
  \ifthesis@czech
    \thesis@abstract
  \else
    \thesis@placeholder@extra@abstractCs
  \fi
}]{abstractCs}
\def\thesis@blocks@abstractCs{%
  \ifthesis@czech\else
    {\thesis@selectLocale{czech}%
    \begin{alwayssingle}%
      \ifthesis@abstractonsinglepage@
        \thesis@blocks@clear
      \else
        % Start the new chapter without clearing the left page.
        \thesis@blocks@clearRight
      \fi
      {\let\thesis@blocks@clear\relax
      \chapter*{\thesis@czech@abstractTitle}%
      \thesis@extra@abstractCs}%
      \par\vfil\null
    \end{alwayssingle}}%
  \fi}
%    \end{macrocode}
% \end{macro}\begin{macro}{\thesis@blocks@bibEntryCs}
% The |\thesis@blocks@bibEntryCs| macro typesets a bibliographical
% entry in English unless the current locale is Czech. The
% macro uses the following extra data fields:\begin{itemize}
%   \item|programmeCs| -- the Czech name of the author's study
%     programme. This extra data field will expand to
%     |\thesis@programme| if the current locale of the thesis
%     is Czech.
%   \item|fieldCs| -- the Czech name of the author's field of
%     study. This extra data field will expand to
%     |\thesis@field@name| if the current locale of the thesis
%     is Czech.
%   \item|keywordsCs| -- the Czech keywords of the thesis.
%     This extra data field will expand to |\thesis@keywords| if
%     the current locale of the thesis is Czech.
%   \item|TeXkeywordsCs| -- the Czech \TeX{} keywords of the thesis.
%     This extra data field will expand to |\thesis@TeXkeywords| if
%     the current locale of the thesis is Czech.
% \end{itemize}
% \changes{v0.3.45}{2017/05/21}{Defined the
%   \cs{thesis@blocks@bibEntryCs} macro in
%   \texttt{style/mu/fithesis-sci.sty}. The patch was submitted by
%   Juraj Pálenik. [VN]}
%    \begin{macrocode}
\thesis@def@extra[{
  \ifthesis@czech
    \thesis@programme
  \else
    \thesis@placeholder@extra@programmeCs
  \fi
}]{programmeCs}
\thesis@def@extra[{
  \ifthesis@czech
    \thesis@field@name
  \else
    \thesis@placeholder@extra@fieldCs
  \fi
}]{fieldCs}
\thesis@def@extra[{
  \ifthesis@czech
    \thesis@title
  \else
    \thesis@placeholder@extra@titleCs
  \fi
}]{titleCs}
\thesis@def@extra[{
  \ifthesis@czech
    \thesis@keywords
  \else
    \thesis@placeholder@extra@keywordsCs
  \fi
}]{keywordsCs}
\thesis@def@extra[{
  \ifthesis@czech
    \thesis@TeXkeywords
  \else
    \thesis@placeholder@extra@keywordsCs
  \fi
}]{TeXkeywordsCs}
%    \end{macrocode}
% \changes{v0.3.45}{2017/05/26}{Bibliographical entries in
%   \texttt{style/mu/fithesis-sci.sty} now face each other when the
%   main locale is either Czech or English. [VN]}
%    \begin{macrocode}
\def\thesis@blocks@bibEntryCs{%
  \ifthesis@czech\else
    {\thesis@selectLocale{czech}
    \begin{alwayssingle}
      % Clear only the right page, if the main locale is English.
      \ifthesis@english
        \begingroup
        \let\thesis@blocks@clear\thesis@blocks@clearRight
      \fi
      \chapter*{\thesis@czech@bib@title}%
      \ifthesis@english
        \endgroup
      \fi
      {% Calculate the width of the columns
      \let\@A\relax\newlength{\@A}\settowidth{\@A}{{%
        \bf\thesis@czech@bib@author:}}
      \let\@B\relax\newlength{\@B}\settowidth{\@B}{{%
        \bf\thesis@czech@bib@thesisTitle:}}
      \let\@C\relax\newlength{\@C}\settowidth{\@C}{{%
        \bf\thesis@czech@bib@programme:}}
      \let\@D\relax\newlength{\@D}\settowidth{\@D}{{%
        \bf\thesis@czech@bib@field:}}
      % Unless this is a rigorous thesis, we will be typesetting
      % the name of the thesis advisor.
      \let\@E\relax\newlength{\@E}
        \ifx\thesis@type\thesis@rigorous
          \setlength{\@E}{0pt}%
        \else
          \settowidth{\@E}{{\bf\thesis@czech@bib@advisor:}}
        \fi
      \let\@F\relax\newlength{\@F}\settowidth{\@F}{{%
        \bf\thesis@czech@bib@academicYear:}}
      \let\@G\relax\newlength{\@G}\settowidth{\@G}{{%
        \bf\thesis@czech@bib@pages:}}
      \let\@H\relax\newlength{\@H}\settowidth{\@H}{{%
        \bf\thesis@czech@bib@keywords:}}
      \let\@skip\relax\newlength{\@skip}\setlength{\@skip}{16pt}
      \let\@left\relax\newlength{\@left}\pgfmathsetlength{\@left}{%
        max(\@A,\@B,\@C,\@D,\@E,\@F,\@G,\@H)}
      \let\@right\relax\newlength{\@right}\setlength{\@right}{%
        \textwidth-\@left-\@skip}
      % Typeset the table
      \renewcommand{\arraystretch}{2}
      \noindent\begin{thesis@newtable@old}%
        {@{}p{\@left}@{\hskip\@skip}p{\@right}@{}}
          \textbf{\thesis@czech@bib@author:} &
            \noindent\parbox[t]{\@right}{
              \thesis@author\\
              \thesis@czech@facultyName,
              \thesis@czech@universityName\\
              \thesis@extra@departmentCs
            }\\
          \textbf{\thesis@czech@bib@thesisTitle:}
            & \thesis@extra@titleCs \\
          \textbf{\thesis@czech@bib@programme:}
            & \thesis@extra@programmeCs \\
          \textbf{\thesis@czech@bib@field:}
            & \thesis@extra@fieldCs \\
          % Unless this is a rigorous thesis, typeset the name of the
          % thesis advisor.
          \ifx\thesis@type\thesis@rigorous\else
            \textbf{\thesis@czech@bib@advisor:}
              & \thesis@advisor \\
          \fi
          \textbf{\thesis@czech@bib@academicYear:}
            & \thesis@academicYear \\
          \textbf{\thesis@czech@bib@pages:}
            & \thesis@pages@preamble{} + \thesis@pages \\
          \textbf{\thesis@czech@bib@keywords:}
            & \thesis@extra@TeXkeywordsCs \\
        \end{thesis@newtable@old}}
      \end{alwayssingle}
    }%
  \fi}
%    \end{macrocode}
% \end{macro}\begin{macro}{\thesis@blocks@frontMatter}
% The |\thesis@blocks@frontMatter| macro sets up the style
% of the front matter front matter of the thesis. The front matter
% is typeset without any visible numbering, as mandated by the
% formal requirements of the faculty.
%    \begin{macrocode}
\def\thesis@blocks@frontMatter{%
  \thesis@blocks@clear
  \pagestyle{empty}
  \parindent 1.5em
  \setcounter{page}{1}
  \pagenumbering{roman}}
%    \end{macrocode}
% \end{macro}\begin{macro}{\thesis@blocks@cover}
% The |\thesis@blocks@cover| macro typesets the thesis
% cover. The following extra data field is defined for the macro:
% \begin{itemize}
%   \item|departmentCs| -- the Czech name of the department at
%     which the thesis is being written. This extra data field will
%     expand to |\thesis@department@name| if the main locale of the
%     thesis is Czech.
% \end{itemize}
% \begin{macrocode}
\thesis@def@extra[{
  \ifthesis@czech
    \thesis@department@name
  \else
    \thesis@placeholder@extra@departmentCs
  \fi
}]{departmentCs}
\def\thesis@blocks@cover{{%
  \thesis@selectLocale{czech}
  \ifthesis@cover@
    \thesis@blocks@clear
    \begin{alwayssingle}
      \begin{center}
      {\sc\thesis@titlePage@LARGE\thesis@czech@universityName\\%
          \thesis@titlePage@Large\thesis@czech@facultyName\\[0.3em]%
          \thesis@titlePage@normalsize\thesis@extra@departmentCs}
      \vfill
      {\bf\thesis@titlePage@Huge\thesis@czech@typeName}
      \vfill
      {\thesis@titlePage@large\thesis@place
       \ \thesis@year\hfill\thesis@author}
      \end{center}
    \end{alwayssingle}
  \fi}}
%    \end{macrocode}
% \end{macro}\begin{macro}{\thesis@blocks@titlePage}
% The |\thesis@blocks@titlePage| macro typesets the thesis
% title page. Depending on the value of the |\ifthesis@color@|
% conditional, the faculty logo is loaded from either
% |\thesis@logopath|, if \texttt{false}, or from
% |\thesis@logopath color/|, if \texttt{true}.
% The following extra data field is defined for the macro:
% \begin{itemize}
%   \item|TeXtitleCs| -- the Czech title of the thesis used for the
%     typesetting. This extra data field will expand to
%     |\thesis@TeXtitle| if the main locale of the thesis is Czech.
% \end{itemize}
% \begin{macrocode}
\thesis@def@extra[{
  \ifthesis@czech
    \thesis@TeXtitle
  \else
    \thesis@placeholder@extra@titleCs
  \fi
}]{TeXtitleCs}
\def\thesis@blocks@titlePage{{%
  \thesis@blocks@clear
  \thesis@selectLocale{czech}
  \begin{alwayssingle}
    % The top of the page
    \begin{adjustwidth}{-12mm}{}
      \begin{minipage}{30mm}
        \thesis@blocks@universityLogo@color[width=30mm]
      \end{minipage}\begin{minipage}{89mm}
        \begin{center}
          {\sc\thesis@titlePage@LARGE\thesis@czech@universityName\\%
              \thesis@titlePage@Large\thesis@czech@facultyName\\[0.3em]%
              \thesis@titlePage@normalsize\thesis@extra@departmentCs}
          \rule{\textwidth}{2pt}\vspace*{2mm}
        \end{center}
      \end{minipage}\begin{minipage}{30mm}
        \thesis@blocks@facultyLogo@color[width=30mm]
      \end{minipage}
    \end{adjustwidth}
    % The middle of the page
    \vfill
    \parbox\textwidth{% Prevent vfills from squashing the leading
      \bf\thesis@titlePage@Huge\thesis@extra@TeXtitleCs}
    {\thesis@titlePage@Huge\\[0.8em]}
    {\thesis@titlePage@large\thesis@czech@typeName\\[1em]}
    {\bf\thesis@titlePage@LARGE\thesis@author\\}
    \vfill\noindent
    % The bottom of the page
    {\bf\thesis@titlePage@normalsize
      % Unless this is a rigorous thesis, typeset the name of the
      % thesis advisor.
      \ifx\thesis@type\thesis@rigorous\else
          \thesis@czech@advisorTitle: \thesis@advisor\hfill
      \fi
      \thesis@place\ \thesis@year}
  \end{alwayssingle}}}
%    \end{macrocode}
% \end{macro}\begin{macro}{\thesis@blocks@thanks}
% The |\thesis@blocks@thanks| macro typesets the
% acknowledgement, if the |\thesis@thanks| macro is
% defined. Otherwise, the macro produces no output.
% As per the faculty requirements, the acknowledgement is
% positioned at the top of the page.
% \changes{v0.3.45}{2017/05/24}{Redefined the
%   \cs{thesis@blocks@thanks} and \cs{thesis@blocks@declaration}
%   macros in \texttt{style/mu/fithesis-sci.sty}. The patch was
%   submitted by Juraj Pálenik. [VN]}
%    \begin{macrocode}
\def\thesis@blocks@thanks{%
  \thesis@blocks@clear
  \ifx\thesis@thanks\undefined\else
    \begin{alwayssingle}%
      \chapter*{\thesis@@{thanksTitle}}%
      \leavevmode\thesis@thanks
    \end{alwayssingle}%
  \fi}
%    \end{macrocode}
% \end{macro}\begin{macro}{\thesis@blocks@declaration}
% The |\thesis@blocks@declaration| macro typesets the declaration
% text. Unlike the generic |\thesis@blocks@declaration| macro from
% the \texttt{style/mu/fithesis-sci.sty} file, this definition
% includes the date and a blank line for the author's signature, as
% per the requirements of the faculty.
%
% Along with the macros required by the locale file interface, the
% locale files need to define the following macros:
% \begin{itemize}
%   \item\DescribeMacro{\thesis@czech@authorSignature}
%     |\thesis@czech@authorSignature| -- The label of the author's
%       signature field
%   \item\DescribeMacro{\thesis@czech@formattedDate}
%     |\thesis@czech@formattedDate| -- A formatted date
% \end{itemize}
%    \begin{macrocode}
\def\thesis@blocks@declaration{%
  \begin{alwayssingle}%
    \leavevmode\vfill
    % Start the new chapter without clearing any page.
    {\let\thesis@blocks@clear\relax
    \chapter*{\thesis@@{declarationTitle}}}%
    \thesis@declaration
    \vskip 2cm%
    {\let\@A\relax\newlength{\@A}
      \settowidth{\@A}{\thesis@@{authorSignature}}
      \setlength{\@A}{\@A+1cm}
    \noindent\thesis@place, \thesis@czech@formattedDate\hfill
    \begin{minipage}[t]{\@A}%
      \centering\rule{\@A}{1pt}\\
      \thesis@@{authorSignature}\par
    \end{minipage}}
  \end{alwayssingle}}
%    \end{macrocode}
% \end{macro}
% Note that there is no direct support for the seminar paper and
% thesis proposal types.  If you would like to change the contents
% of the preamble and the postamble, you should modify the
% |\thesis@blocks@preamble| and |\thesis@blocks@postamble| macros.
%
% All blocks within the autolayout preamble and postamble that are
% not defined within this file are defined in the
% \texttt{style/mu/fithesis-base.sty} file. The entire front matter
% is typeset as though the locale were Czech in accordance with the
% formal requirements of the faculty.
%    \begin{macrocode}
\def\thesis@blocks@preamble{
  \thesis@blocks@coverMatter
    \thesis@blocks@cover
  \thesis@blocks@frontMatter
    \thesis@blocks@titlePage
    \thesis@blocks@clearRight
      \thesis@blocks@bibEntryCs
      \thesis@blocks@bibEntry
      \thesis@blocks@bibEntryEn
      \thesis@blocks@abstractCs
      \ifthesis@abstractonsinglepage@
        \begingroup
          \let\clearpage\relax
      \fi
          \thesis@blocks@abstract
          \thesis@blocks@abstractEn
      \ifthesis@abstractonsinglepage@
        \endgroup
      \fi
    \thesis@blocks@assignment
    {\thesis@selectLocale{czech}%
    \thesis@blocks@thanks
    \thesis@blocks@declaration
    \thesis@blocks@clear
      \pagestyle{plain}%
      \thesis@blocks@tables}}
\def\thesis@blocks@postamble{%
  \thesis@blocks@bibliography}
%    \end{macrocode}

% \subsubsection{The style files of the Faculty of Arts}
% % \file{style/mu/fithesis-phil.sty}
% This is the style file for the theses written at the Faculty of
% Arts at the Masaryk University in Brno. It has been prepared in
% accordance with the formal requirements published at the website
% of the faculty\footnote{See \url{http://is.muni.cz/auth/do/^^A
% 1421/4581421/Vzor_bakalarske_prace.pdf}}.
%    \begin{macrocode}
\NeedsTeXFormat{LaTeX2e}
\ProvidesPackage{fithesis/style/mu/fithesis-phil}[2016/04/18]
%    \end{macrocode}
% The file defines the color scheme of the respective faculty.
%    \begin{macrocode}
\thesis@color@setup{
  links={HTML}{6FCEF2},
  tableEmph={HTML}{78CEF0},
  tableOdd={HTML}{EBF6FA},
  tableEven={HTML}{D0EBF5}}
%    \end{macrocode}
% The bibliography support is enabled. The |numeric| citations are
% used and the bibliography is sorted by name, title, and year.
%    \begin{macrocode}
\thesis@bibliography@setup{
  style=iso-numeric,
  sorting=nty}
\thesis@bibliography@load
%    \end{macrocode}
% The style file configures the title page header to include the
% department and the field name.
%    \begin{macrocode}
\thesis@blocks@titlePage@department@true
\thesis@blocks@titlePage@field@true
%    \end{macrocode}
% The style file parses the value of the |\thesis@department| macro
% and recognizes the following divisions of the Faculty of Arts:
% \begin{itemize}
%   \item\texttt{kisk} -- The Division of Information and Library
%     Studies \footnote{See \url{http://kisk.phil.muni.cz/cs/pov^^A
%     innosti}.} (KISK)
%    \begin{macrocode}
\def\thesis@departments@kisk{kisk}
%    \end{macrocode}
% \end{itemize}
% Along with the macros required by the locale file interface, the
% locale files need to define the following macros:
% \begin{itemize}
%   \item\texttt{departmentName} -- The human-readable name of the
%     given recognized division.
% \end{itemize}
%    \begin{macrocode}
\ifx\thesis@department\thesis@departments@kisk
  \def\thesis@department@name{\thesis@@{departmentName}}
  \def\thesis@departmentEn@name{\thesis@english@departmentName}
\fi
%    \end{macrocode}
% \begin{macro}{\thesis@blocks@titlePage}
% In the case of a KISK thesis, the style file redefines the cover
% and title page footers to include the thesis advisor's name.
% \begin{macrocode}
\ifx\thesis@department\thesis@departments@kisk
  \def\thesis@blocks@titlePage@content{%
    {\thesis@titlePage@Huge\bf\thesis@TeXtitle\par\vfil}\vskip 0.8in
    {\thesis@titlePage@large\sc\thesis@@{typeName}\\[0.3in]}
    {\thesis@titlePage@Large\bf\thesis@author}
    % If this is a KISK thesis, typeset the name of the thesis
    % advisor.
    \ifx\thesis@department\thesis@departments@kisk
      {\thesis@titlePage@large\\[0.3in]
        {\bf\thesis@@{advisorTitle}:} \thesis@advisor}
    \fi}%
\fi
%    \end{macrocode}
% \end{macro}
% All blocks within the autolayout preamble and postamble that are
% not defined within this file are defined in the
% \texttt{style/mu/fithesis-base.sty} file.
%    \begin{macrocode}
\def\thesis@blocks@preamble{%
  \thesis@blocks@coverMatter
    \thesis@blocks@cover
    \thesis@blocks@titlePage
  \thesis@blocks@frontMatter
%    \end{macrocode}
% In KISK theses, the bibliographical entry, the abstract, and the
% keywords will be included after the cover matter.
%    \begin{macrocode}
    \ifx\thesis@department\thesis@departments@kisk
      \thesis@blocks@bibEntry
      \thesis@blocks@abstract
      \thesis@blocks@abstractEn
      \thesis@blocks@keywords
      \thesis@blocks@keywordsEn
    \fi
    \thesis@blocks@declaration
    \thesis@blocks@thanks
%    \end{macrocode}
% In KISK theses, the lists of tables and figures will be included
% behind the bibliography rather than at the beginning of the
% document.
%    \begin{macrocode}
    \ifx\thesis@department\thesis@departments@kisk
      \thesis@blocks@toc
    \else
      \thesis@blocks@tables
    \fi}
\def\thesis@blocks@postamble{%
  \ifx\thesis@department\thesis@departments@kisk
%    \end{macrocode}
% In KISK theses, the lists of tables and figures will be included
% behind the bibliography rather than at the beginning of the
% document.
%    \begin{macrocode}
    \thesis@blocks@lot
    \thesis@blocks@lof
  \fi
  \thesis@blocks@bibliography}
%    \end{macrocode}

% \subsubsection{The style files of the Faculty of Education}
% % \file{style/mu/fithesis-ped.sty}
% This is the style file for the theses written at the Faculty of
% Education at the Masaryk University in Brno. It has been prepared
% in accordance with the formal requirements published at the
% website of the faculty\footnote{See \url{http://is.muni.cz/^^A
% do/ped/VPAN/pokdek/Pokyn_dekana_c._1-2010__2_.pdf}}.
%    \begin{macrocode}
\NeedsTeXFormat{LaTeX2e}
\ProvidesPackage{fithesis/style/mu/fithesis-ped}[2015/11/29]
%    \end{macrocode}
% The file defines the color scheme of the respective faculty.
%    \begin{macrocode}
\thesis@color@setup{
  links={HTML}{FFA02F},
  tableEmph={HTML}{FFBB6B},
  tableOdd={HTML}{FFF1E0},
  tableEven={HTML}{FFDEB7}}
%    \end{macrocode}
% The style file configures the title page header to include the
% department name and the title page content to include
% advisor's name.
%    \begin{macrocode}
\thesis@blocks@titlePage@department@true
\def\thesis@blocks@titlePage@content{%
    {\thesis@titlePage@Huge\bf\thesis@TeXtitle\par\vfil}\vskip 0.8in
    {\thesis@titlePage@large\sc\thesis@@{typeName}\\[0.3in]}
    {\thesis@titlePage@Large\bf\thesis@author}
    % Typeset the name of the thesis advisor.
    {\thesis@titlePage@large\\[0.3in]
      {\bf\thesis@@{advisorTitle}:} \thesis@advisor}}
%    \end{macrocode}
% \begin{macro}{\thesis@blocks@bibEntry}
% The |\thesis@blocks@bibEntry| macro typesets a
% bibliographical entry. Along with the macros required by the
% locale file interface, the \textit{locale} files need to define
% the following macros:
% \begin{itemize}
%   \item|\thesis@|\emph{locale}|@bib@title| -- The title of the
%     entire block
%   \item|\thesis@|\emph{locale}|@bib@pages| -- The abbreviation of
%     pages used in the bibliographical entry
% \end{itemize}
%    \begin{macrocode}
\def\thesis@blocks@bibEntry{%
  \chapter*{\thesis@@{bib@title}}
  \noindent\thesis@upper{author@tail}, \thesis@author@head.
  \emph{\thesis@title}. \thesis@place: \thesis@@{universityName},
  \thesis@@{facultyName}, \thesis@department, \thesis@year.
  \thesis@pages\ \thesis@@{bib@pages}.
  \thesis@@{advisorTitle}: \thesis@advisor}
%    \end{macrocode}
% \end{macro}
% All blocks within the autolayout postamble that are not defined
% within this file are defined in the
% \texttt{style/mu/fithesis-base.sty} file.
%    \begin{macrocode}
\def\thesis@blocks@preamble{%
  \thesis@blocks@cover
  \thesis@blocks@titlePage
  \thesis@blocks@frontMatter
    \thesis@blocks@bibEntry
    \thesis@blocks@clearRight
      \thesis@blocks@abstract
      \thesis@blocks@abstractEn
      \thesis@blocks@keywords
      \thesis@blocks@keywordsEn
    \thesis@blocks@declaration
    \thesis@blocks@thanks
    \thesis@blocks@tables}
%    \end{macrocode}

% \subsubsection{The style files of the Faculty of Social Studies}
% % \file{style/mu/fithesis3-fss.sty}
% This is the style file for the theses written at the Faculty of
% Social Studies at the Masaryk University in Brno. Because of the
% inexistence of faculty-wide formal requirements and
% recommendations with each department defining their own with
% varying degrees of
% rigour\footnote{See \url{http://psych.fss.muni.cz/node/351},
% \url{http://medzur.fss.muni.cz/informace-pro-studenty/pravidla^^A
% -pro-diplomky/soubory/Pravidla pro zaverecne prace na KMSZ - v^^A
% er. 2-83.doc}, \url{http://soc.fss.muni.cz/?q=node/44}, \url{h^^A
% ttp://polit.fss.muni.cz/informace-pro-studenty/pol/}, \url{htt^^A
% p://humenv.fss.muni.cz/studium/bakalarske-studium/pravidla-pro^^A
% -vypracovani-bakalarske-prace}}, this style is a mere skeleton,
% which is unlikely to satisfy the exact requirements of
% any department and will require modification by the user.
%    \begin{macrocode}
\NeedsTeXFormat{LaTeX2e}
\ProvidesPackage{fithesis3/style/mu/fithesis3-fss}[2015/04/26]
%    \end{macrocode}
% In addition to the main locale, the file also requires the
% English locale.
%    \begin{macrocode}
\thesis@requireLocale{english}
%    \end{macrocode}
% The style file defines the autolayout preamble as the cover and
% the title page followed by the abstracts, keywords, assignment,
% declaration, acknowledgement, table of contents and
% list of tables and figures as a part of the front matter. All
% blocks are defined in the \texttt{style/mu/base.sty} file.
%    \begin{macrocode}
\def\thesis@preamble{%
  \thesis@blocks@cover%
  \thesis@blocks@titlePage%
  \thesis@blocks@frontMatter%
    \thesis@blocks@abstract%
    \thesis@blocks@abstractEn%
    \thesis@blocks@keywords%
    \thesis@blocks@keywordsEn%
    \thesis@blocks@assignment%
    \thesis@blocks@declaration%
    \thesis@blocks@thanks%
    \thesis@blocks@tables%
  \thesis@blocks@mainMatter}
%    \end{macrocode}

% \subsubsection{The style files of the Faculty of Law}
% % \file{style/mu/fithesis-law.sty}
% This is the style file for the theses written at the Faculty of
% Law at the Masaryk University in Brno. It has been prepared in
% accordance with the formal requirements published at the
% website of the faculty\footnote{See \url{http://is.muni.cz/d^^A
% o/law/ud/predp/smer/S-07-2012.pdf}}.
%    \begin{macrocode}
\NeedsTeXFormat{LaTeX2e}
\ProvidesPackage{fithesis/style/mu/fithesis-law}[2016/04/18]
%    \end{macrocode}
% The file defines the color scheme of the respective faculty.
%    \begin{macrocode}
\thesis@color@setup{
  links={HTML}{CF86EB},
  tableEmph={HTML}{D39BE8},
  tableOdd={HTML}{F2EBF5},
  tableEven={HTML}{E5CCED}}
%    \end{macrocode}
% The bibliography support is enabled. The |authoryear| citations
% are used and the bibliography is sorted by name, year, and title.
%    \begin{macrocode}
\thesis@bibliography@setup{%
  style=iso-authoryear,
  sorting=nyt}
\thesis@bibliography@load
%    \end{macrocode}
% The style file configures the cover and title page headers to
% include only the faculty name and the department name.
% Along with the macros required by the locale file interface,
% the locale files need to define the following strings:
% \begin{itemize}
%   \item\texttt{facultyLongName} -- The name of the faculty
%     combined with the name of the university.
% \end{itemize}
%    \begin{macrocode}
\def\thesis@blocks@cover@header{%
  {\sc\thesis@titlePage@Large\thesis@@{facultyLongName}\\%
    \thesis@titlePage@large\thesis@department@name\\\vskip 2em}}
\let\thesis@blocks@titlePage@header=\thesis@blocks@cover@header
%    \end{macrocode}
% \begin{macro}{\thesis@blocks@frontMatter}
% The |\thesis@blocks@frontMatter| macro sets up the style
% of the front matter of the thesis. The page numbering is arabic
% in accordance with the formal requirements of the faculty.
% \begin{macrocode}
\def\thesis@blocks@frontMatter{%
  \thesis@blocks@clear
  \pagestyle{plain}
  \parindent 1.5em
  \setcounter{page}{1}
  \pagenumbering{arabic}}
%    \end{macrocode}
% \end{macro}\begin{macro}{\thesis@blocks@mainMatter}
% The |\thesis@blocks@mainMatter| macro sets up the style
% of the main matter of the thesis. The page numbering doesn't
% reset at the beginning of the main thesis in accordance with the
% formal requirements of the faculty.
% \begin{macrocode}
\def\thesis@blocks@mainMatter{%
  \thesis@blocks@clear
  \pagestyle{thesisheadings}
  \parindent 1.5em\relax}
%    \end{macrocode}
% \end{macro}
% All blocks within the autolayout preamble that are not defined
% within this file are defined in the
% \texttt{style/mu/fithesis-base.sty} file.
%    \begin{macrocode}
\def\thesis@blocks@preamble{%
  \thesis@blocks@coverMatter
    \thesis@blocks@cover
    \thesis@blocks@titlePage
  \thesis@blocks@frontMatter
    \thesis@blocks@declaration
    \thesis@blocks@clearRight
      \thesis@blocks@abstract
      \thesis@blocks@abstractEn
      \thesis@blocks@keywords
      \thesis@blocks@keywordsEn
    \thesis@blocks@thanks
    \thesis@blocks@tables}
%    \end{macrocode}
% All blocks within the autolayout postamble that are not defined
% within this file are defined in the
% \texttt{style/mu/fithesis-base.sty} file.
%    \begin{macrocode}
\def\thesis@blocks@postamble{%
  \thesis@blocks@bibliography
  \thesis@blocks@assignment}
%    \end{macrocode}

% \subsubsection{The style files of the Faculty of Economics and
%   Administration}
% % \file{style/mu/fithesis-econ.sty}
% This is the style file for the theses written at the Faculty of
% Economics and Administration at the Masaryk University in Brno.
% It has been prepared in accordance with the formal requirements
% \changes{v0.3.46}{2017/06/02}{The documentation now points to the
%   2/2017 dean's directive for the Faculty of Economics and
%   Administration, Masaryk University, Brno. [VN]}
% published at the website of the faculty\footnote{See \url{ht^^A
% tps://is.muni.cz/auth/do/econ/predpisy/smernice/prehled/6715^^A
% 9928/SmerniceDekana2017-c.2-o_zaverecnych_pracich_2017.pdf}}.
%    \begin{macrocode}
\NeedsTeXFormat{LaTeX2e}
\ProvidesPackage{fithesis/style/mu/fithesis-econ}[2018/02/11]
%    \end{macrocode}
% The file defines the color scheme of the respective faculty. Note
% the the color definitions are in RGB, which makes the resulting
% files generally unsuitable for printing.
%    \begin{macrocode}
\thesis@color@setup{
  links={HTML}{F27995},
  tableEmph={HTML}{E8B88B},
  tableOdd={HTML}{F5ECEB},
  tableEven={HTML}{EBD8D5}}
%    \end{macrocode}
% The bibliography support is enabled. The |authoryear| citations
% are used and the bibliography is sorted by name, title, and year.
%    \begin{macrocode}
\thesis@bibliography@setup{
  style=iso-authoryear,
  sorting=nty}
\thesis@bibliography@load
%    \end{macrocode}
% The file loads the following packages:
% \begin{itemize}
%   \item\textsf{tikz} -- Used for dimension arithmetic.
%   \item\textsf{geometry} -- Allows for modifications of the type
%     area dimensions.
%   \item\textsf{array} -- Enables |<{decl.}| and |>{decl.}|
%     declarations in table preambles.
% \end{itemize}
% In addition to this, the type area width is set to
% 16\,cm in accordance with the formal requirements of the faculty.
% This leads to overfull lines and is against the good conscience
% of the author of this style.
%    \begin{macrocode}
\thesis@require{tikz}
\thesis@require{geometry}
\thesis@require{array}
\geometry{top=25mm,bottom=20mm,left=25mm,right=25mm,includeheadfoot}
%    \end{macrocode}
% \changes{v0.3.47}{2017/07/09}{Enabled the inclusion of the
%   scanned assignment inside the digital version of the document
%   in \texttt{style/mu/fithesis-econ.sty} in accordance with the
%   formal requirements of the faculty. The patch was submitted by
%   Jana Ratajská. [VN]}
% Enable the inclusion of the scanned assignment inside the digital
% version of the document.
%    \begin{macrocode}
\thesis@blocks@assignment@hideIfDigital@false
%    \end{macrocode}
% \begin{macro}{\thesis@blocks@cover}
% The |\thesis@blocks@cover| macro typesets the thesis
% cover.
%    \begin{macrocode}
\def\thesis@blocks@cover{%
  \ifthesis@cover@
    \thesis@blocks@clear
    \begin{alwayssingle}
      \thispagestyle{empty}
      \begin{center}
      {\sc\thesis@titlePage@LARGE\thesis@@{universityName}\\%
          \thesis@titlePage@Large\thesis@@{facultyName}\\}
      \vfill
      {\bf\thesis@titlePage@Huge\thesis@@{typeName}}
      \vfill
      {\thesis@titlePage@large\thesis@place
       \ \thesis@year\hfill\thesis@author}
      \end{center}
    \end{alwayssingle}
  \fi}
%    \end{macrocode}
% \end{macro}
% The style file configures the title page header to include the
% name of the field of study and redefines the title page content
% not to include the author's name and the title page footer
% to include both the author's and advisor's name, the year and
% place of the thesis defense in accordance with the formal
% requirements of the faculty.
%    \begin{macrocode}
\thesis@blocks@titlePage@field@true
\def\thesis@blocks@titlePage@content{%
  {\thesis@titlePage@Huge\bf\thesis@TeXtitle}
  \ifthesis@english\else
    {\\[0.1in]\thesis@titlePage@Large\bf\thesis@TeXtitleEn}
  \fi {\\[0.3in]\thesis@titlePage@large\sc\thesis@@{typeName}\\}}
\def\thesis@blocks@titlePage@footer{%
  {\thesis@titlePage@large
    {% Calculate the width of the thesis author and advisor boxes
     \let\@A\relax\newlength{\@A}\settowidth{\@A}{{%
       \bf\thesis@@{advisorTitle}:}}
     \let\@B\relax\newlength{\@B}\settowidth{\@B}{\thesis@advisor}
     \let\@C\relax\newlength{\@C}\settowidth{\@C}{{%
       \bf\thesis@@{authorTitle}:}}
     \let\@D\relax\newlength{\@D}\settowidth{\@D}{\thesis@author}
    \let\@left\relax\newlength{\@left}\pgfmathsetlength{\@left}{%
      max(\@A,\@B)}
    \let\@right\relax\newlength{\@right}\pgfmathsetlength{\@right}{%
      max(\@C,\@D)}
%    \end{macrocode}
% \changes{v0.3.49}{2018/02/11}{Removed an extraneous \cs{vskip} in
%   the style files for the Masaryk University in Brno. [VN]}
% \begin{macrocode}
    % Typeset the thesis author and advisor boxes
    \begin{minipage}[t]{\@left}
      {\bf\thesis@@{advisorTitle}:}\\\thesis@advisor
    \end{minipage}\hfill\begin{minipage}[t]{\@right}
      {\bf\thesis@@{authorTitle}:}\\\thesis@author
    \end{minipage}}\\[4em]\thesis@place, \thesis@year}}
%    \end{macrocode}
% \begin{macro}{\thesis@blocks@frontMatter}
% The |\thesis@blocks@frontMatter| macro sets up the style
% of the front matter of the thesis. The page numbering is arabic
% as per the formal requirements and it is hidden. In case of
% double-sided typesetting, the geometry is altered according to
% the requirements of the faculty.
% \begin{macrocode}
\def\thesis@blocks@frontMatter{%
  \thesis@blocks@clear
  % In case of double-sided typesetting, change the geometry
  \ifthesis@twoside@
    \newgeometry{top=25mm,bottom=20mm,left=35mm,
      right=15mm, includeheadfoot}
  \fi\pagestyle{empty}
  \parindent 1.5em
  \setcounter{page}{1}
  \pagenumbering{arabic}}
%    \end{macrocode}
% \end{macro}\begin{macro}{\thesis@blocks@mainMatter}
% The |\thesis@blocks@mainMatter| macro sets up the style
% of the main matter of the thesis. The page numbering doesn't
% reset at the beginning of the main thesis as per the formal
% requirements.
% \begin{macrocode}
\def\thesis@blocks@mainMatter{%
  \thesis@blocks@clear
  % In case of double-sided typesetting, change the geometry
  \ifthesis@twoside@
    \newgeometry{top=25mm,bottom=20mm,left=35mm,
      right=15mm, includeheadfoot}
  \fi\pagestyle{thesisheadings}
  \parindent 1.5em\relax}
%    \end{macrocode}
% \end{macro}\begin{macro}{\thesis@blocks@tables}
% The |\thesis@blocks@tables| macro optionally typesets the
% |\listoftables| and |\listoffigures|.
% \begin{macrocode}
\def\thesis@blocks@tables{%
  \thesis@blocks@lot
  \thesis@blocks@lof}
%    \end{macrocode}
% \end{macro}
% If the |nolot| and |nolof| options haven't been specified, the
% |\thesis@blocks@lot| and |\thesis@blocks@lof| macros are
% redefined to create an entry in the table of contents.
% \begin{macrocode}
\ifx\thesis@blocks@lot\relax\else
  \def\thesis@blocks@lot{%
    \thesis@blocks@clear
    \phantomsection
    \addcontentsline{toc}{chapter}{\listtablename}%
    \listoftables}
\fi

\ifx\thesis@blocks@lof\relax\else
  \def\thesis@blocks@lof{%
    \thesis@blocks@clear
    \phantomsection
    \addcontentsline{toc}{chapter}{\listfigurename}%
    \listoffigures}
\fi
%    \end{macrocode}
% \begin{macro}{\thesis@blocks@declaration}
% The |\thesis@blocks@declaration| macro typesets the declaration
% text. Unlike the generic |\thesis@blocks@declaration| macro from
% the \texttt{style/mu/fithesis-sci.sty} file, this definition
% includes the date and a blank line for the author's signature, as
% per the requirements of the faculty.
% \changes{v0.3.46}{2017/06/02}{Redefined
%   \cs{thesis@blocks@declaration} in
%   \texttt{style/mu/fithesis-econ.sty} in accordance with the
%   example documents. The patch was submitted by Jana Ratajská.
%   [VN]}
%    \begin{macrocode}
\def\thesis@blocks@declaration{%
  \begin{alwayssingle}%
    \thesis@blocks@clear
    \leavevmode\vfill
    % Start the new chapter without clearing any page.
    {\let\thesis@blocks@clear\relax
    \chapter*{\thesis@@{declarationTitle}}}%
    \thesis@declaration
    \vskip 2cm%
    {\let\@A\relax\newlength{\@A}
      \settowidth{\@A}{\thesis@@{authorSignature}}
      \setlength{\@A}{\@A+1cm}
    \noindent\thesis@place, \thesis@@{formattedDate}\hfill
    \begin{minipage}[t]{\@A}%
      \centering\rule{\@A}{1pt}\\
      \thesis@@{authorSignature}\par
    \end{minipage}}
  \end{alwayssingle}}
%    \end{macrocode}
% \end{macro}\begin{macro}{\thesis@blocks@abstract}
% \changes{v0.3.46}{2017/06/02}{Redefined
%   \cs{thesis@blocks@abstract}, \cs{thesis@blocks@abstractEn},
%   \cs{thesis@blocks@keywords}, and \cs{thesis@blocks@keywordsEn}
%   in \texttt{style/mu/fithesis-econ.sty} in accordance with the
%   example documents. The patch was submitted by Jana Ratajská.
%   [VN]}
% The |\thesis@blocks@abstract| macro typesets the
% abstract. This definition typesets the abstract on the same page.
% \begin{macrocode}
\def\thesis@blocks@abstract{%
  \begin{alwayssingle}%
    \vskip 40\p@
    {\let\thesis@blocks@clear\relax
    \chapter*{\thesis@@{abstractTitle}}}%
    \noindent\thesis@abstract
  \end{alwayssingle}}
%    \end{macrocode}
% \end{macro}\begin{macro}{\thesis@blocks@abstractEn}
% The |\thesis@blocks@abstractEn| macro typesets the abstract in
% English. If the current locale is English, the macro produces no
% output. This macro typesets the abstract on the same page.
% \begin{macrocode}
\def\thesis@blocks@abstractEn{%
  \ifthesis@english\else
    {\thesis@selectLocale{english}%
    \begin{alwayssingle}%
      \vskip 20\p@
      {\let\thesis@blocks@clear\relax
      \chapter*{\thesis@english@abstractTitle}}%
      \noindent\thesis@abstractEn
    \end{alwayssingle}}%
  \fi}
%    \end{macrocode}
% \end{macro}\begin{macro}{\thesis@blocks@keywords}
% The |\thesis@blocks@keywords| macro typesets the keywords. This
% definition typesets the keywords on the same page.
% \begin{macrocode}
\def\thesis@blocks@keywords{%
  \begin{alwayssingle}%
    \vskip 40\p@
    {\let\thesis@blocks@clear\relax
    \chapter*{\thesis@@{keywordsTitle}}%
    \noindent\thesis@TeXkeywords}%
  \end{alwayssingle}}
%    \end{macrocode}
% \end{macro}\begin{macro}{\thesis@blocks@keywordsEn}
% The |\thesis@blocks@keywordsEn| macro typesets the keywords in
% English. If the current locale is English, the macro produces no
% output.
% \begin{macrocode}
\def\thesis@blocks@keywordsEn{%
  \ifthesis@english\else
    {\thesis@selectLocale{english}%
    \begin{alwayssingle}%
      \vskip 20\p@
      {\let\thesis@blocks@clear\relax%
      \chapter*{\thesis@english@keywordsTitle}}%
      \noindent\thesis@TeXkeywordsEn
    \end{alwayssingle}}%
  \fi}
%    \end{macrocode}
% \end{macro}\begin{macro}{\thesis@blocks@bibEntry}
% The |\thesis@blocks@bibEntry| macro typesets a bibliographical
% entry. Along with the macros required by the locale file
% interface, the locale files need to define the following macros:
% \begin{itemize}
%   \item|\thesis@|\textit{locale}|@bib@author| -- The label of the
%     author name entry
%   \item|\thesis@|\textit{locale}|@bib@title| -- The label of the
%     title name entry
%   \item|\thesis@|\textit{locale}|@bib@titleEn| -- The label of the
%     English title name entry (\cs{thesis@english@bib@titleEn}
%     does not need to be defined)
%   \item|\thesis@|\textit{locale}|@bib@department| -- The label of
%     the department name entry
%   \item|\thesis@|\textit{locale}|@bib@advisor| -- The label of
%     the advisor name entry
%   \item|\thesis@|\textit{locale}|@bib@year| -- The label of the
%     year entry
% \end{itemize}
% \changes{v0.3.46}{2017/06/02}{Defined \cs{thesis@blocks@bibEntry}
%   in \texttt{style/mu/fithesis-econ.sty} in accordance with the
%   example documents. The patch was submitted by Jana Ratajská.
%   [VN]}
%    \begin{macrocode}
\def\thesis@blocks@bibEntry{%
  \thesis@blocks@clear
  \noindent\begin{thesis@newtable@old}{@{}>{\bfseries}ll@{}}
    \thesis@@{bib@author}:        & \thesis@author     \\
    \thesis@@{bib@thesisTitle}:   & \thesis@title      \\
  \ifthesis@english\else
    \thesis@@{bib@thesisTitleEn}: & \thesis@titleEn    \\
  \fi
    \thesis@@{bib@department}:    & \thesis@department \\
    \thesis@@{bib@advisor}:       & \thesis@advisor    \\
    \thesis@@{bib@year}:          & \thesis@year       \\
  \end{thesis@newtable@old}}
%    \end{macrocode}
% \end{macro}
% Note that there is no direct support for the seminar paper and
% thesis proposal types.  If you would like to change the contents
% of the preamble and the postamble, you should modify the
% |\thesis@blocks@preamble| and |\thesis@blocks@postamble| macros.
%
% All blocks within the autolayout preamble that are not defined
% within this file are defined in the
% \texttt{style/mu/fithesis-base.sty} file.
%    \begin{macrocode}
\def\thesis@blocks@preamble{%
  \thesis@blocks@coverMatter
    \thesis@blocks@cover
  \thesis@blocks@frontMatter
    \thesis@blocks@titlePage
    \thesis@blocks@assignment
    \thesis@blocks@bibEntry
    \thesis@blocks@abstract
    \thesis@blocks@abstractEn
    \thesis@blocks@keywords
    \thesis@blocks@keywordsEn
    \thesis@blocks@declaration
    \thesis@blocks@thanks
    \thesis@blocks@toc}
%    \end{macrocode}
% All blocks within the autolayout postamble that are not defined
% within this file are defined in the \texttt{style/mu/base.sty}
% file.
%    \begin{macrocode}
\def\thesis@blocks@postamble{%
  \thesis@blocks@bibliography
  \thesis@blocks@tables}
%    \end{macrocode}

% \subsubsection{The style files of the Faculty of Medicine}
% % \file{style/mu/fithesis-med.sty}
% This is the style file for the theses written at the Faculty of
% Medicine at the Masaryk University in Brno. It has been prepared
% in accordance with the formal requirements published at the
% website of the Department of Optometry and Orthoptics\footnote^^A
% {See \url{http://is.muni.cz/do/med/zpravyprac/Optometrie/NALE^^A
% ZITOSTI_ZAVERECNE_PRACE.doc}}.
%    \begin{macrocode}
\NeedsTeXFormat{LaTeX2e}
\ProvidesPackage{fithesis/style/mu/fithesis-med}[2015/11/29]
%    \end{macrocode}
% The file defines the color scheme of the respective faculty.
%    \begin{macrocode}
\thesis@color@setup{
  links={HTML}{F58E76},
  tableEmph={HTML}{FF9D85},
  tableOdd={HTML}{FFF5F6},
  tableEven={HTML}{FFDEDF}}
%    \end{macrocode}
% The file loads the following packages:
% \begin{itemize}
%   \item\textsf{tikz} -- Used for dimension arithmetic.
%   \item\textsf{geometry} -- Allows for modifications of the type
%     area dimensions.
%   \item\textsf{setspace} -- Allows for line height modifications.
% \end{itemize}
% In addition to this, the type area width is set to
% 16\,cm in accordance with the formal requirements of the faculty.
% This leads to overfull lines and is against the good conscience
% of the author of this style.
%    \begin{macrocode}
\thesis@require{tikz}
\thesis@require{geometry}
\thesis@require{setspace}
\geometry{top=25mm,bottom=20mm,left=25mm,right=25mm,includeheadfoot}
%    \end{macrocode}
% \begin{macro}{\thesis@blocks@cover}
% The |\thesis@blocks@cover| macro typesets the thesis
% cover.
%    \begin{macrocode}
\def\thesis@blocks@cover{%
  \ifthesis@cover@
    \thesis@blocks@clear
    \begin{alwayssingle}
      \thispagestyle{empty}
      \begin{center}
      {\sc\thesis@titlePage@LARGE\thesis@@{universityName}\\%
          \thesis@titlePage@Large\thesis@@{facultyName}\\}
      \vfill
      {\bf\thesis@titlePage@Huge\thesis@@{typeName}}
      \vfill
      {\thesis@titlePage@large\thesis@place
       \ \thesis@year\hfill\thesis@author}
      \end{center}
    \end{alwayssingle}
  \fi}
%    \end{macrocode}
% \end{macro}
% The style file redefines the title page content
% not to include the author's name and the title page footer
% to include both the author's and advisor's name, the field of
% study and the semester and place of the thesis defense in
% accordance with the requirements of the department.
%    \begin{macrocode}
\def\thesis@blocks@titlePage@content{%
  {\thesis@titlePage@Huge\bf\thesis@TeXtitle\\[0.3in]}%
  {\thesis@titlePage@large{\sc\thesis@@{typeName}}\\}}
\def\thesis@blocks@titlePage@footer{%
  {\thesis@titlePage@large
    {% Calculate the width of the thesis author and advisor boxes
     \let\@A\relax\newlength{\@A}\settowidth{\@A}{{%
       \bf\thesis@@{advisorTitle}:}}
     \let\@B\relax\newlength{\@B}\settowidth{\@B}{\thesis@advisor}
     \let\@C\relax\newlength{\@C}\settowidth{\@C}{{%
       \bf\thesis@@{authorTitle}:}}
     \let\@D\relax\newlength{\@D}\settowidth{\@D}{\thesis@author}
     \let\@E\relax\newlength{\@E}\settowidth{\@E}{{%
       \bf\thesis@@{fieldTitle}:}}
     \let\@F\relax\newlength{\@F}\settowidth{\@F}{\thesis@field}
    \let\@left\relax\newlength{\@left}\pgfmathsetlength{\@left}{%
      max(\@A,\@B)}
    \let\@right\relax\newlength{\@right}\pgfmathsetlength{\@right}{%
      max(\@C,\@D,\@E,\@F)}
    % Typeset the thesis author and advisor boxes
    \vskip 2in\begin{minipage}[t]{\@left}
      {\bf\thesis@@{advisorTitle}:}\\\thesis@advisor
    \end{minipage}\hfill\begin{minipage}[t]{\@right}
      {\bf\thesis@@{authorTitle}:}\\\thesis@author
        \\[1em]{\bf\thesis@@{fieldTitle}:}\\\thesis@field
    \end{minipage}}\\[4em]\thesis@place, \thesis@@{semester}}}
%    \end{macrocode}
% \begin{macro}{\thesis@blocks@frontMatter}
% The |\thesis@blocks@frontMatter| macro sets up the style of the
% front matter of the thesis. The page numbering is arabic in
% accordance with the formal requirements and it is hidden. In case
% of double-sided typesetting, the geometry is altered according to
% the requirements of the faculty.
% \begin{macrocode}
\def\thesis@blocks@frontMatter{%
  \thesis@blocks@clear
  % In case of double-sided typesetting, change the geometry
  \ifthesis@twoside@
    \newgeometry{top=25mm,bottom=20mm,left=35mm,
      right=15mm, includeheadfoot}
  \fi\pagestyle{empty}
  \parindent 1.5em
  \setcounter{page}{1}
  \pagenumbering{arabic}}
%    \end{macrocode}
% \end{macro}\begin{macro}{\thesis@blocks@mainMatter}
% The |\thesis@blocks@mainMatter| macro sets up the style
% of the main matter of the thesis. The leading is adjusted in
% accordance with the requirements of the faculty.
% \begin{macrocode}
\def\thesis@blocks@mainMatter{%
  \thesis@blocks@clear
  \setcounter{page}{1}
  \pagenumbering{arabic}
  \pagestyle{thesisheadings}
  \parindent 1.5em
  \onehalfspacing}
%    \end{macrocode}
% \end{macro}\begin{macro}{\thesis@blocks@bibEntry}
% The |\thesis@blocks@bibEntry| macro typesets a
% bibliographical entry. Along with the macros required by the
% locale file interface, the \textit{locale} files need to define
% the following macros:
% \begin{itemize}
%   \item|\thesis@|\emph{locale}|@bib@title| -- The title of the
%     entire block
%   \item|\thesis@|\emph{locale}|@bib@pages| -- The abbreviation of
%     pages used in the bibliographical entry
% \end{itemize}
%    \begin{macrocode}
\def\thesis@blocks@bibEntry{%
  \chapter*{\thesis@@{bib@title}}
  \noindent\thesis@upper{author@tail}, \thesis@author@head.
  \emph{\thesis@title}. \thesis@place: \thesis@@{universityName},
  \thesis@@{facultyName}, \thesis@department, \thesis@year.
  \thesis@pages\ \thesis@@{bib@pages}.
  \thesis@@{advisorTitle}: \thesis@advisor}
%    \end{macrocode}
% \end{macro}
% All blocks within the autolayout postamble that are not defined
% within this file are defined in the
% \texttt{style/mu/fithesis-base.sty} file.
%    \begin{macrocode}
\def\thesis@blocks@preamble{%
  \thesis@blocks@cover
  \thesis@blocks@frontMatter
    \thesis@blocks@titlePage
    \onehalfspacing
    \thesis@blocks@clearRight
      \thesis@blocks@abstract
      \thesis@blocks@abstractEn
      \thesis@blocks@keywords
      \thesis@blocks@keywordsEn
    \thesis@blocks@bibEntry
    \thesis@blocks@declaration
    \thesis@blocks@thanks
    \thesis@blocks@tables}
%    \end{macrocode}

% \subsubsection{The style files of the Faculty of Sports Studies}
% % \file{style/mu/fithesis-fsps.sty}
% This is the style file for the theses written at the Faculty of
% Sports Studies at the Masaryk University in Brno. It has been
% prepared in accordance with the formal requirements published at
% the website of the faculty\footnote{See \url{https://is.muni.cz/^^A
% auth/do/fsps/studijni/info-stud/SZZ/pokyny_ZP_13-5-2013.pdf}}.
%    \begin{macrocode}
\NeedsTeXFormat{LaTeX2e}
\ProvidesPackage{fithesis/style/mu/fithesis-fsps}[2017/05/15]
%    \end{macrocode}
% The file defines the color scheme of the respective faculty. Note
% the the color definitions are in RGB, which makes the resulting
% files generally unsuitable for printing.
%    \begin{macrocode}
\thesis@color@setup{
  links={HTML}{93BCF5},
  tableEmph={HTML}{A8BDE3},
  tableOdd={HTML}{EBEFF5},
  tableEven={HTML}{D1DAEB}}
%    \end{macrocode}
% The bibliography support is enabled. The |authoryear| citations
% are used and the bibliography is sorted by name, title, and year.
%    \begin{macrocode}
\thesis@bibliography@setup{
  style=iso-authoryear,
  sorting=nty}
\thesis@bibliography@load
%    \end{macrocode}
% The file loads the following packages:
% \begin{itemize}
%   \item\textsf{tikz} -- Used for dimension arithmetic.
%   \item\textsf{geometry} -- Allows for modifications of the type
%     area dimensions.
%   \item\textsf{setspace} -- Allows for line height modifications.
% \end{itemize}
% In addition to this, the type area width is set to
% 14\,cm in accordance with the formal requirements of the faculty.
%    \begin{macrocode}
\thesis@require{tikz}
\thesis@require{geometry}
\thesis@require{setspace}
\geometry{top=30mm,bottom=30mm,left=40mm,right=30mm,includeheadfoot}
%    \end{macrocode}
% The paragraph indentation is 1.25\,cm as per the requirements of the faculty.
%    \begin{macrocode}
\setlength{\parindent}{1.25cm}
%    \end{macrocode}
% The style file redefines the title page content
% not to include the author's name and the title page footer
% to include both the author's and advisor's name, the year and
% place of the thesis defense in accordance with the formal
% requirements of the faculty.
%    \begin{macrocode}
\def\thesis@blocks@titlePage@footer{%
  {\thesis@titlePage@large
    {% Calculate the width of the thesis author and advisor boxes
     \let\@A\relax\newlength{\@A}\settowidth{\@A}{{%
       \bf\thesis@@{advisorTitle}:}}
     \let\@B\relax\newlength{\@B}\settowidth{\@B}{\thesis@advisor}
     \let\@C\relax\newlength{\@C}\settowidth{\@C}{{%
       \bf\thesis@@{authorTitle}:}}
     \let\@D\relax\newlength{\@D}\settowidth{\@D}{\thesis@author}
     \let\@E\relax\newlength{\@E}\settowidth{\@E}{\thesis@field}
     \let\@F\relax\newlength{\@F}\pgfmathsetlength{\@F}{max(\@D,\@E)}
    \let\@left\relax\newlength{\@left}\pgfmathsetlength{\@left}{%
      max(\@A,\@B)}
    \let\@right\relax\newlength{\@right}\pgfmathsetlength{\@right}{%
      max(\@C,\@F)}
    % Typeset the thesis author and advisor boxes
    \vskip 2in\begin{minipage}[t]{\@left}
      {\bf\thesis@@{advisorTitle}:}\\\thesis@advisor
    \end{minipage}\hfill\begin{minipage}[t]{\@right}
      {\bf\thesis@@{authorTitle}:}\\\thesis@author\\\thesis@field
    \end{minipage}}\\[4em]\thesis@place, \thesis@year}}
%    \end{macrocode}
% \begin{macro}{\thesis@blocks@frontMatter}
% The |\thesis@blocks@frontMatter| macro sets up the style of the
% front matter of the thesis. The leading is adjusted in
% accordance with the requirements of the faculty.
% \begin{macrocode}
\def\thesis@blocks@frontMatter{%
  \thesis@blocks@clear
  \pagestyle{plain}
  \parindent 1.5em
  \setcounter{page}{1}
  \pagenumbering{roman}
  \onehalfspacing}
%    \end{macrocode}
% \end{macro}\begin{macro}{\thesis@blocks@mainMatter}
% The |\thesis@blocks@mainMatter| macro sets up the style
% of the main matter of the thesis. The leading is adjusted in
% accordance with the requirements of the faculty.
% \begin{macrocode}
\def\thesis@blocks@mainMatter{%
  \thesis@blocks@clear
  \setcounter{page}{1}
  \pagenumbering{arabic}
  \pagestyle{thesisheadings}
  \parindent 1.5em
  \onehalfspacing}
%    \end{macrocode}
% \end{macro}\begin{macro}{\thesis@blocks@bibliography}
% The |\thesis@blocks@bibliography| macro typesets the
% bibliography. The leading is adjusted in accordance
% with the requirements of the faculty.
% \begin{macrocode}
\def\thesis@blocks@bibliography{%
  \ifthesis@bibliography@loaded@
    \ifthesis@bibliography@included@\else
      \singlespacing
      \thesis@blocks@clear
      {\emergencystretch=3em%
      \printbibliography[heading=bibintoc]}%
    \fi
  \fi}
%    \end{macrocode}
% \end{macro}\begin{macro}{\thesis@blocks@declaration}
% The |\thesis@blocks@declaration| macro typesets the declaration
% text. Unlike the generic |\thesis@blocks@declaration| macro from
% the \texttt{style/mu/fithesis-sci.sty} file, this definition
% includes the date and a blank line for the author's signature, as
% per the requirements of the faculty.
%
% Along with the macros required by the locale file interface, the
% locale files need to define the following macros:
% \begin{itemize}
%   \item|\thesis@|\textit{locale}|@authorSignature| -- The
%     label of the author's signature field
%   \item|\thesis@|\textit{locale}|@formattedDate| -- A
%     formatted date
% \end{itemize}
%    \begin{macrocode}
\def\thesis@blocks@declaration{%
  \thesis@blocks@clear
  \begin{alwayssingle}%
    \chapter*{\thesis@@{declarationTitle}}%
    \thesis@declaration
    \vskip 2cm%
    {\let\@A\relax\newlength{\@A}
      \settowidth{\@A}{\thesis@@{authorSignature}}
      \setlength{\@A}{\@A+1cm}
    \noindent\thesis@place, \thesis@@{formattedDate}\hfill
    \begin{minipage}[t]{\@A}%
      \centering\rule{\@A}{1pt}\\
      \thesis@@{authorSignature}\par
    \end{minipage}}
  \end{alwayssingle}}
%    \end{macrocode}
% \end{macro}
% Note that there is no direct support for the seminar paper and
% thesis proposal types.  If you would like to change the contents
% of the preamble and the postamble, you should modify the
% |\thesis@blocks@preamble| and |\thesis@blocks@postamble| macros.
%
% All blocks within the autolayout preamble and postamble that are
% not defined within this file are defined in the
% \texttt{style/mu/fithesis-base.sty} file.
%    \begin{macrocode}
\def\thesis@blocks@preamble{%
  \thesis@blocks@coverMatter
    \thesis@blocks@cover
    \thesis@blocks@titlePage
  \thesis@blocks@frontMatter
    \thesis@blocks@declaration
    \thesis@blocks@thanks
    \thesis@blocks@tables}
\def\thesis@blocks@postamble{%
  \thesis@blocks@bibliography}
%    \end{macrocode}

