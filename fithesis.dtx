% \iffalse meta-comment
% fithesis.dtx
% Copyright 1998--2015 Daniel Marek (DM), Jan Pavlovič (JP),
% Vít Novotný (VN), Petr Sojka (PS)
% https://www.fi.muni.cz/tech/unix/tex/fithesis.xhtml
% Faculty of Informatics, Masaryk University
%
% This work may be distributed and/or modified under the
% conditions of the LaTeX Project Public License, either version 1.3
% of this license or (at your option) any later version.
% The latest version of this license is in
%   http://www.latex-project.org/lppl.txt
% and version 1.3 or later is part of all distributions of LaTeX
% version 2005/12/01 or later.
%
% This work has the LPPL maintenance status `maintained'.
% 
% The Current Maintainer of this work is Vít Novotný.
% Send bug reports, requests for additions and questions
% to the fithesis discussion forum at
% <https://is.muni.cz/auth/df/fithesis-sazba/>.
%
% This work consists of the files fithesis.dtx and fithesis.ins
% and the derived files fithesis3.cls, fithesis2.cls, fithesis.cls,
% fit10.clo, fit11.clo, fit12.clo.
%
%    \begin{macrocode}
%<*driver>

\documentclass{ltxdoc}
\usepackage[utf8]{inputenc} % this file uses UTF-8
\usepackage[english]{babel}
\usepackage{tgpagella}
\usepackage{tabularx}
\usepackage{hologo}
\usepackage[scaled=0.86]{berasans}
\usepackage[scaled=1.03]{inconsolata}
\usepackage[resetfonts]{cmap}
\usepackage[T1]{fontenc} % use 8bit fonts
\usepackage[
  hyperindex=false,
  pdfborderstyle={/S/U/W 1},
]{hyperref}
\emergencystretch 2dd

% ltxdoc class options
\CodelineIndex
\MakeShortVerb{|}
\EnableCrossrefs
\DoNotIndex{}

\begin{document}
  \RecordChanges
  \DocInput{fithesis.dtx}
  \PrintIndex
  \PrintChanges
\end{document}

%</driver>
%    \end{macrocode}
%<*class>
\NeedsTeXFormat{LaTeX2e}
% \fi
\def\thesis@version{2015/03/21 fithesis3 version 0.3.09 MU thesis class}
%
%%%%%%%%%%%%%%%%%%%%%%%%%%%%%%%%%%%%%%%%%%%%%%%%%%%%%%%%%%%%%%%%%%%%%%%%%%%%%%%
%
% \changes{v0.3.08}  {2015/03/04}{Fixed a non-terminated \cs{if} condition.
%   [VN] (backport of v0.2.18)\\Fixed mostly documentation errors reported
%   at the new fithesis discussion forum (-ti, eco$\rightarrow$econ, implicit
%   twocolumn, example extended (font setup), etc.). [PS] (backport of v0.2.17)}
% \changes{v0.3.07}  {2015/02/03}{Replaced the \cs{thesiswoman} command with
%   \cs{thesisgender}. [VN]}
% \changes{v0.3.06}  {2015/01/26}{Added the colorx package and the base colors
%   for each faculty. If the color option is specified, the tabular environment
%   gets redefined and uses the faculty colors to color alternating table rows
%   to improve readability. The hyperref links in the e-version are now likewise
%   colored according to the chosen faculty, in this case regardless of the
%   presence of the color option. Dropped the support for typesetting theses
%   outside MU. [VN]}
% \changes{v0.3.05}  {2015/01/21}{Added support for change typesetting.
%   Restructured the code to make it more amenable to literal programming.
%   Added support for \cs{CodelineIndex} typesetting. Added information about
%   the usage of \textsf{fithesis1} and \textsf{fithesis2} on the FI unix
%   machines. (backport of v0.2.16) [VN]\\Minor changes throughout the text,
%   added a link to the the fithesis forums [PS] (backport of v0.2.15@r14:15)}
% \changes{v0.3.04}  {2015/01/14}{Import the url package to allow for the use of
%   \cs{url} within the documentation. (backport of v0.2.15@r13) [VN]}
% \changes{v0.3.03}  {2015/01/14}{Small fixes (added \cs{relax} at
%   \cs{MainMatter}), generating both fithesis.cls (obsolete, loading
%   \texttt{fithesis2.cls}) and \texttt{fithesis2.cls}, minor doc edits,
%   version numbering of \texttt{.clo} fixed, switch to utf8 and ensuring that
%   \texttt{.dtx} compiles. Documentation adjusted to the status quo, added
%   link to discussion forum (backport of v0.2.14) [PS]}
% \changes{v0.3.02}  {2015/01/13}{pdf metadata stamping added for
%   \cs{thesistitle} and \cs{thesisstudent} [VN]}
% \changes{v0.3.01}  {2015/01/09}{documentation now uses babel and cmap
%   packages. the entire file was transcoded into utf8, \cs{thesiscolor} was
%   replaced by color class option, added pdf metadata stamping support [VN]}
% \changes{v0.3.00}  {2015/01/01}{fi logo is no longer special-cased (added eps
%   and pdf), \cs{thesislogopath} added to set the logo directory path,
%   \cs{thesiscolor} added to enable colorful typo elements [VN]}
% \changes{v0.2.12a}{2008--2011}{fork fithesis2 by Mr. Filipčík and Janoušek;
%   cf. \protect\url{https://github.com/liskin/fithesis}}
% \changes{v0.2.12} {2008/07/27}{Licence change to the LPPL [JP]}
% \changes{v0.2.11} {2008/01/07}{fix missing \texttt{fi-logo.mf} [JP,PS]}
% \changes{v0.2.10} {2006/05/12}{fix EN name of Acknowledgement [JP]}
% \changes{v0.2.09}  {2006/05/08}{add EN version of University name [JP]}
% \changes{v0.2.08}  {2006/01/20}{add change of University name [JP]}
% \changes{v0.2.07}  {2005/05/10}{escape all Czech letters [JP]
%   babel is used instead of stupid package czech [JP]
%   \cs{MainMatter} should be placed after \cs{tablesofcontents} [PS]}
% \changes{v0.2.06}  {2004/12/22}{fix : behind Advisor [JP]}
% \changes{v0.2.05}  {2004/05/13}{add English abstract [JP]}
% \changes{v0.2.04}  {2004/05/13}{fix SK declaration [Peter Cerensky, JP]}
% \changes{v0.2.03}  {2004/05/13}{fix title spacing [PS, JP]}
% \changes{v0.2.02}  {2004/05/12}{fix encoding bug [JP]}
% \changes{v0.2.01}  {2004/05/11}{add subsubsection to toc [JP]}
% \changes{v0.2.00}  {2004/05/03}{add sk lang [JP, Peter Cerensky]
%   set default cls class to \textsf{rapport3} [JP]}
% \changes{v0.1g}   {2004/04/01}{change of default size (12pt$\rightarrow$11pt) [JP]}
% \changes{v0.1f}   {2004/01/24}{add documentation for hyperref [JP]}
% \changes{v0.1e}   {2004/01/07}{add Brno to MU title [JP]}
% \changes{v0.1d}   {2003/03/24}{removed def schapter from fit1*.clo [JP]}
% \changes{v0.1c}   {2003/02/21}{default values of \cs{facultyname} and
%   \\\cs{@thesissubtitle} set for backward compatibility) [PS]}
% \changes{v0.1b}   {2003/02/14}{change of default size (11pt$\rightarrow$12pt) [JP]}
% \changes{v0.1a}   {2003/02/12}{minor documentation changes (CZ only,
%   sorry) [PS]}
% \changes{v0.1}    {2003/02/11}{new release, documentation editing (CZ only,
%   sorry) [PS]}
% \changes{v0.0a}   {2002}{changes by Jan Pavlovič to allow fithesis being
%   backend of docbook based system for thesis writing}
% \changes{v0.0}    {1998}{bachelor project of Daniel Marek under
%   supervision of Petr Sojka}
%
%%%%%%%%%%%%%%%%%%%%%%%%%%%%%%%%%%%%%%%%%%%%%%%%%%%%%%%%%%%%%%%%%%%%%%%%%%%%%%%
%
% \title{The \textsf{fithesis3} class for the typesetting of theses written
%   at the Masaryk Univerzity in Brno\thanks{The documentation
%   concerns the \thesis@version.}}
% \author{Daniel Marek, Jan Pavlovič, Vít Novotný, Petr Sojka}
% \date{\today}
% \maketitle
%
% \begin{abstract}
% \noindent This document describes the blah blah blah ...
% \end{abstract}
%
% \tableofcontents
%
% \section{Basic structure}
% The document class comprises the following types of files:
% \begin{enumerate}
%   \item\emph{Locale files} -- These files contain macro definitions for
%        various locales. They live in the \texttt{locale/} subtree and
%        their API is covered in section \ref{sec:locale-files}.
%   \item\emph{Style files} -- These files define the structure and the
%        look of the resulting document. They live in the \texttt{style/}
%        subtree and their API is covered in section \ref{sec:style-files}.
%   \item\emph{Logo files} -- These files contain the logos of the
%        respective faculties of the Masaryk University and they live in the
%        \texttt{logo/} subtree. New logos should be added in \textsc{Pdf}
%        and \textsc{Eps} formats in both color and black-and-white versions,
%        as described in section \ref{sec:mu-base-style-file}.
%   \item\emph{Class file} -- The class file serves as a glue, which loads
%        all of the aforementioned files and provides a public API to the
%        end user. The public API of the class is covered in section
%        \ref{sec:public-api}.
% \end{enumerate}
%
% \section{Required classes and packages}
% The class loads the \texttt{scrreprt} base class and the
% \texttt{xstring}, \texttt{keyval}, \texttt{newfile} and
% \texttt{etoolbox} packages. The \texttt{hyperref} package is
% also conditionally loaded during the expansion of the
% |\thesis@load| macro (see section \ref{sec:thesis@load}).
% Other packages may be required by the style files you are using.
%    \begin{macrocode}
\ProvidesClass{fithesis3}[\thesis@version]
\LoadClass[a4paper]{rapport3}
\RequirePackage{xstring}
\RequirePackage{keyval}
\RequirePackage{newfile}
\RequirePackage{etoolbox}
%    \end{macrocode}
% \section{Public API}
% \label{sec:public-api}
% \subsection{Options}
% Any \oarg{options} passed to the class will be handed down to the
% loaded style file. The supported options are therefore documented
% in the subsection of section \ref{sec:style-files} dedicated to
% the respective style file.
%
% \subsection{The \cs{thesissetup} macro}
% \DescribeMacro{\thesissetup}
% The only public macro is the |\thesissetup|\marg{keyvals}
% command, where \textit{keyvals} is a comma-delimited list of
% key-value pairs as defined by the \textsf{keyval} package. This
% macro needs to be included prior to the beginning of a \LaTeX\ 
% document. When used, the \textit{keyvals} are processed.
%    \begin{macrocode}
\def\thesissetup#1{%
  \setkeys{thesis}{#1}}
%    \end{macrocode}
% \subsubsection{The \texttt{basepath} key}
% The \marg{\texttt{basepath}=path} pair sets the \textit{path}
% containing the class files. The \textit{path} is prepended to
% each other path used by the class. If non-empty, the
% \textit{path} gets normalized to \textit{path/}. The normalized
% \textit{path} is stored within the private
% \DescribeMacro{\thesis@basepath}\cmd{\thesis@basepath} macro,
% whose implicit value is |fithesis3/|.
%    \begin{macrocode}
\def\thesis@basepath{fithesis3/}
\define@key{thesis}{basepath}{%
  \ifx\@empty#1\@empty%
    \def\thesis@basepath{}%
  \else%
    \def\thesis@basepath{#1/}%
  \fi}
%    \end{macrocode} 
% \subsubsection{The \texttt{logopath} key}
% The \marg{\texttt{logopath}=path} pair sets the \textit{path}
% containing the logo files, which is used by the style files
% loading the logo. If the \textit{path} doesn't begin with a
% slash (\texttt{/}), it is normalized to
% \cmd{/thesis@basepath} followed by \textit{path} via the
% private macro \DescribeMacro{\thesis@subdir}|\thesis@subdir|.
% The normalized \textit{path} is stored within the private
% \DescribeMacro{\thesis@logopath}\cmd{\thesis@logopath}
% macro, whose implicit value is |\thesis@basepath| followed by
% |logo/\thesis@university/|. By default, this expands to
% \texttt{fithesis3/logo/mu/}.
%    \begin{macrocode}
\def\thesis@logopath{\thesis@basepath logo/\thesis@university/}
\define@key{thesis}{logopath}{%
  \def\thesis@logopath{\thesis@subdir{#1}}}

\def\thesis@subdir#1{%
  \ifx\@empty#1\@empty%
    \thesis@basepath%
  \else%
    \def\@slash{/}%
    \StrLeft{#1}{1}[\@fst]%
    \ifx\@fst\@slash%
      #1/%
    \else%
      \thesis@basepath#1/%
    \fi%
  \fi}
%    \end{macrocode}
% \subsubsection{The \texttt{stylepath} key}
% The \marg{\texttt{stylepath}=path} pair sets the \textit{path}
% containing the style files. If the \textit{path} doesn't begin
% with a slash (\texttt{/}), it is normalized to
% \cmd{/thesis@basepath} followed by \textit{path}. The
% normalized \textit{path} is stored within the private
% \DescribeMacro{\thesis@stylepath}\cmd{\thesis@stylepath} macro,
% whose implicit value is |\thesis@basepath style/|. By default,
% this expands to \texttt{fithesis3/style/}.
%    \begin{macrocode}
\def\thesis@stylepath{\thesis@basepath  style/}
\define@key{thesis}{stylepath}{%
  \def\thesis@stylepath{\thesis@subdir{#1}}}
%    \end{macrocode}
% \subsubsection{The \texttt{localepath} key}
% The \marg{\texttt{localepath}=path} pair sets the \textit{path}
% containing the locale files. If the \textit{path} doesn't begin
% with a slash (\texttt{/}), it is normalized to
% \cmd{/thesis@basepath} followed by \textit{path}. The
% normalized \textit{path} is stored within the private
% \DescribeMacro{\thesis@localepath}\cmd{\thesis@localepath} macro,
% whose implicit value is |\thesis@basepath| followed by |locale/|.
% By default, this expands to \texttt{fithesis3/locale/}.
%    \begin{macrocode}
\def\thesis@localepath{\thesis@basepath locale/}
\define@key{thesis}{localepath}{%
  \def\thesis@localepath{\thesis@subdir{#1}}}
%    \end{macrocode}
% \subsubsection{The \texttt{gender} key}
% The \marg{\texttt{gender}=char} pair sets the author's gender to
% either a male, if \textit{char} is the character \texttt{m}, or
% to a female. The gender can be tested using the
% \DescribeMacro{\ifthesis@woman}|\ifthesis@woman| ... |\else| ...
% |\fi| conditional. The implicit gender is male.
%    \begin{macrocode}
\newif\ifthesis@woman\thesis@womanfalse
\define@key{thesis}{gender}{%
  \def\thesis@male{m}%
  \def\thesis@arg{#1}%
  \ifx\thesis@male\thesis@arg%
    \thesis@womanfalse%
  \else%
    \thesis@womantrue%
  \fi}
%    \end{macrocode}
% \subsubsection{The \texttt{author} key}
% The \marg{\texttt{author}=name} pair sets the author's full
% name to \textit{name}. The \textit{name} is stored within the
% private \DescribeMacro{\thesis@author}|\thesis@author|
% macro, whose implicit value is |\thesis@placeholders@author|.
%    \begin{macrocode}
\def\thesis@author{\thesis@placeholders@author}
\define@key{thesis}{author}{%
  \def\thesis@author{#1}}
%    \end{macrocode}
% \subsubsection{The \texttt{type} key}
% The \marg{\texttt{type}=type} pair sets the type of the thesis
% to \textit{type}. The following types of theses are recognized:
% \begin{center}\begin{tabular}{lc}\hline
%   The thesis type & The value of \textit{type} \\\hline
%   Bachelor's thesis & \texttt{bc} \\
%   Master's thesis & \texttt{mgr} \\
%   Doctoral thesis & \texttt{d} \\
%   Rigorous thesis & \texttt{r} \\\hline
% \end{tabular}\end{center}
% The \textit{type} is stored within the
% private \DescribeMacro{\thesis@type}|\thesis@type| macro, whose
% implicit value is |b|. For the ease of testing of the thesis
% type via |\ifx| conditions within style and locale files, the
% \DescribeMacro{\thesis@bachelors}|\thesis@bachelors|,
% \DescribeMacro{\thesis@masters}|\thesis@masters|,
% \DescribeMacro{\thesis@doctoral}|\thesis@doctoral| and
% \DescribeMacro{\thesis@rigorous}|\thesis@rigorous| macros
% containing the corresponding \textit{type} values are available
% as a part of the private API.
%    \begin{macrocode}
\def\thesis@bachelors{bc}
\def\thesis@masters{mgr}
\def\thesis@doctoral{d}
\def\thesis@rigorous{r}
\let\thesis@type\thesis@bachelors
\define@key{thesis}{type}{%
  \def\thesis@type{#1}}
%    \end{macrocode}
% \subsubsection{The \texttt{university} key}
% The \marg{\texttt{university}=id} pair sets the identifier of
% the university, at which the thesis is being written,
% to \textit{id}. The \textit{id} is stored within the private
% \DescribeMacro{\thesis@university} |\thesis@university| macro,
% whose implicit value is \texttt{mu}. The |\thesis@university|
% macro is used by the |\thesis@logopath| macro and when loading
% the style and locale files using the |\thesis@load| macro. It
% allows for the usage of the class at universities other than
% the Masaryk University in Brno without the need to alter the
% code.
%    \begin{macrocode}
\def\thesis@university{mu}
\define@key{thesis}{university}{%
  \def\thesis@university{#1}}
%    \end{macrocode}
% \subsubsection{The \texttt{faculty} key}
% The \marg{\texttt{faculty}=domain} pair sets the faculty, at
% which the thesis is being written, to \textit{domain}. The
% following \textit{domain} names are recognized:
% \begin{center}\begin{tabularx}{\textwidth}{Xc}\hline
%   The Faculty & The \textit{domain} name \\\hline
%   The Faculty of Informatics & \texttt{fi} \\
%   The Faculty of Science & \texttt{sci} \\
%   The Faculty of Law & \texttt{law} \\
%   The Faculty of Economics and Administration & \texttt{econ} \\
%   The Faculty of Social Studies & \texttt{fss} \\
%   The Faculty of Medicine & \texttt{med} \\
%   The Faculty of Education & \texttt{ped} \\
%   The Faculty of Arts & \texttt{phil} \\
%   The Faculty of Sports Studies & \texttt{fsps} \\\hline
% \end{tabularx}\end{center}
% The \textit{domain} name is stored within the
% private \DescribeMacro{\thesis@faculty}|\thesis@faculty| macro,
  % whose implicit value is \texttt{fi}.
%    \begin{macrocode}
\def\thesis@faculty{fi}
\define@key{thesis}{faculty}{%
  \def\thesis@faculty{#1}}
%    \end{macrocode}
% \subsubsection{The \texttt{department} key}
% The \marg{\texttt{department}=name} pair sets the name of the
% department, at which the thesis is being written, to
% \textit{name}. The \textit{name} is stored within the private
% \DescribeMacro{\thesis@department} |\thesis@department| macro,
% whose implicit value is |\thesis@placeholders@department|.
%    \begin{macrocode}
\def\thesis@department{\thesis@placeholders@department}
\define@key{thesis}{department}{%
  \def\thesis@department{#1}}
%    \end{macrocode}
% \subsubsection{The \texttt{programme} key}
% The \marg{\texttt{programme}=name} pair sets the name of the
% author's study programme to \textit{name}. The \textit{name}
% is stored within the private \DescribeMacro{\thesis@programme}
% |\thesis@programme| macro, whose implicit value is
% |\thesis@placeholders@programme|.
%    \begin{macrocode}
\def\thesis@programme{\thesis@placeholders@programme}
\define@key{thesis}{programme}{%
  \def\thesis@Programme{#1}}
%    \end{macrocode}
% \subsubsection{The \texttt{logo} key}
% The \marg{\texttt{logo}=filename} pair sets the filename of the
% logo file to be used to \textit{filename}. The \textit{filename}
% is stored within the private \DescribeMacro{\thesis@logo}
% |\thesis@logo| macro, whose implicit value is |\thesis@faculty|.
% The logo file is loaded from the |\thesis@logopath\thesis@logo|
% path.
%    \begin{macrocode}
\def\thesis@logo{\thesis@faculty}
\define@key{thesis}{logo}{%
  \def\thesis@logo{#1}}
%    \end{macrocode}
% \subsubsection{The \texttt{style} key}
% The \marg{\texttt{style}=filename} pair sets the filename of the
% style file to be used to \textit{filename}. The \textit{filename}
% is stored within the private \DescribeMacro{\thesis@style}
% |\thesis@style| macro, whose implicit value is
% |\thesis@university/\thesis@faculty|. The style file is loaded
% from the |\thesis@stylepath\thesis@style| path.
%    \begin{macrocode}
\def\thesis@style{\thesis@university/\thesis@faculty}
\define@key{thesis}{style}{%
  \def\thesis@style{#1}}
%    \end{macrocode}
% \subsubsection{The \texttt{styleInheritance} key}
% The \marg{\texttt{styleInheritance}=bool} pair either enables,
% if \textit{bool} is \texttt{true} or unspecified, or disables the
% inheritance for style files. The effects of the inheritance
% are documented within the subsection documenting the
% |\thesis@load| macro. The setting can be tested using the
% \DescribeMacro{\ifthesis@style@inheritance}
% |\ifthesis@style@inheritance| ...
% |\else| ... |\fi| conditional. Inheritance is enabled for style
% files by default.
%    \begin{macrocode}
\newif\ifthesis@style@inheritance\thesis@style@inheritancetrue
\define@key{thesis}{styleInheritance}[true]{%
  \def\@true{true}%
  \def\@arg{#1}%
  \ifx\@true\@arg%
    \thesis@style@inheritancetrue%
  \else%
    \thesis@style@inheritancefalse%
  \fi}
%    \end{macrocode}
% \subsubsection{The \texttt{locale} key}
% The \marg{\texttt{locale}=filename} pair sets the filename of the
% locale file(s) to be used to \textit{filename}. The
% \textit{filename} is stored within the private
% \DescribeMacro{\thesis@locale}|\thesis@style| macro, whose
% implicit value is the main language of the \textsf{babel} package
% or \texttt{czech}, when undefined. If the inheritance is
% disabled for locale files, the locale file is loaded from the
% |\thesis@localepath\thesis@locale| path.
%    \begin{macrocode}
\def\thesis@locale{%
  % Babel detection
  \ifx\languagename\undefined%
  czech\else\languagename\fi}
\define@key{thesis}{locale}{%
  \def\thesis@locale{#1}}
%    \end{macrocode}
% \subsubsection{The \texttt{localeInheritance} key}
% The \marg{\texttt{localeInheritance}=bool} pair either enables,
% if \textit{bool} is \texttt{true} or unspecified, or disables the
% inheritance. The effects of the inheritance are
% documented within the subsection documenting the |\thesis@load| 
% macro. The setting can be tested using the
% \DescribeMacro{\ifthesis@locale@inheritance}
% |\ifthesis@locale@inheritance| ...
% |\else| ... |\fi| conditional. Inheritance is enabled for locale
% files by default.
%    \begin{macrocode}
\newif\ifthesis@locale@inheritance\thesis@locale@inheritancetrue
\define@key{thesis}{localeInheritance}[true]{%
  \def\@true{true}%
  \def\@arg{#1}%
  \ifx\@true\@arg%
    \thesis@locale@inheritancetrue%
  \else%
    \thesis@locale@inheritancefalse%
  \fi}
%    \end{macrocode}
% \subsubsection{The \texttt{year} key}
% The \marg{\texttt{year}=year} pair sets the year of the thesis
% defence to \textit{year}. The \textit{year} is stored witin the
% private \DescribeMacro{\thesis@year}|\thesis@year| macro, whose
% implicit value is |\the\year|.
%    \begin{macrocode}
\def\thesis@year{\the\year}
\define@key{thesis}{year}{%
  \def\thesis@year{#1}}
%    \end{macrocode}
% \subsubsection{The \texttt{place} key}
% The \marg{\texttt{place}=place} pair sets the location of the
% faculty, at which the thesis is being prepared, to \textit{place}.
% The \textit{place} is stored within the private
% \DescribeMacro{\thesis@place}|\thesis@place| macro, whose
% implicit value is \texttt{Brno}.
%    \begin{macrocode}
\def\thesis@place{Brno}
\define@key{thesis}{place}{%
  \def\thesis@place{#1}}
%    \end{macrocode}
% \subsubsection{The \texttt{title} key}
% The \marg{\texttt{title}=title} pair sets the title of the
% thesis to \textit{title}. The \textit{title} is stored within the
% private \DescribeMacro{\thesis@title}|\thesis@title| macro, whose
% implicit value is |\thesis@placeholders@|\discretionary{}{}{}^^A
% |title|.
%    \begin{macrocode}
\def\thesis@title{\thesis@placeholders@title}
\define@key{thesis}{title}{%
  \def\thesis@title{#1}}
%    \end{macrocode}
% \subsubsection{The \texttt{titleEn} key}
% The \marg{\texttt{titleEn}=title} pair sets the English title of
% the thesis to \textit{title}. The \textit{title} is stored within
% the private \DescribeMacro{\thesis@titleEn}|\thesis@titleEn|
% macro, whose implicit value is |\undefined|.
%    \begin{macrocode}
\let\thesis@titleEn\undefined
\define@key{thesis}{titleen}{%
  \def\thesis@titleEn{#1}}
%    \end{macrocode}
% \subsubsection{The \texttt{keywords} key}
% The \marg{\texttt{keywords}=list} pair sets the keywords of the
% thesis to the comma-delimited \textit{list}. The \textit{list}
% is stored within the private
% \DescribeMacro{\thesis@keywords}|\thesis@keywords| macro, whose
% implicit value is |\thesis@placeholders@keywords|.
%    \begin{macrocode}
\def\thesis@keywords{\thesis@placeholders@keywords}
\define@key{thesis}{keywords}{%
  \def\thesis@keywords{#1}}
%    \end{macrocode}
% \subsubsection{The \texttt{keywordsEn} key}
% The \marg{\texttt{keywordsEn}=list} pair sets the English
% keywords of the thesis to the comma-delimited \textit{list}. The
% \textit{list} is stored within the private
% \DescribeMacro{\thesis@keywordsEn}|\thesis@keywordsEn| macro,
% whose implicit value is |\undefined|.
%    \begin{macrocode}
\let\thesis@keywordsEn\undefined
\define@key{thesis}{keywordsEn}{%
  \def\thesis@keywordsEn{#1}}
%    \end{macrocode}
% \subsubsection{The \texttt{abstract} key}
% The \marg{\texttt{abstract}=text} pair sets the abstract of the
% thesis to \textit{text}. The \textit{text} is stored within the
% private \DescribeMacro{\thesis@abstract}|\thesis@abstract| macro,
% whose implicit value is |\thesis@placeholders@abstract|.
%    \begin{macrocode}
\def\thesis@abstract{\thesis@placeholders@abstract}
\define@key{thesis}{abstract}{%
  \def\thesis@abstract{#1}}
%    \end{macrocode}
% \subsubsection{The \texttt{abstractEn} key}
% The \marg{\texttt{abstractEn}=text} pair sets the English
% abstract of the thesis to \textit{text}. The \textit{text}
% is stored within the private
% \DescribeMacro{\thesis@abstractEn}|\thesis@abstractEn|
% macro, whose implicit value is |\undefined|.
%    \begin{macrocode}
\let\thesis@abstractEn\undefined
\define@key{thesis}{abstractEn}{%
  \def\thesis@abstractEn{#1}}
%    \end{macrocode}
% \subsubsection{The \texttt{advisor} key}
% The \marg{\texttt{advisor}=name} pair sets the thesis advisor's
% full name to \textit{name}. The \textit{name} is stored within
% the private \DescribeMacro{\thesis@advisor}|\thesis@advisor|
% macro, whose implicit value is |\thesis@placeholders@advisor|.
%    \begin{macrocode}
\def\thesis@advisor{\thesis@placeholders@advisor}
\define@key{thesis}{advisor}{%
  \def\thesis@advisor{#1}}
%    \end{macrocode}
% \subsubsection{The \texttt{thanks} key}
% The \marg{\texttt{thanks}=text} pair sets the acknowledgement
% text to \textit{text}. The \textit{text} is stored within
% the private \DescribeMacro{\thesis@thanks}|\thesis@thanks|
% macro, whose implicit value is |\thesis@|\discretionary{}{}{}
% |placeholders@thanks|.
%    \begin{macrocode}
\def\thesis@thanks{\thesis@placeholders@thanks}
\define@key{thesis}{thanks}{%
  \long\def\thesis@thanks{#1}}
%    \end{macrocode}
% \subsubsection{The \texttt{autoLayout} key}
% The \marg{\texttt{autoLayout}=bool} pair either enables,
% if \textit{bool} is \texttt{true} or unspecified, or disables
% autolayout. Autolayout injects the
% |\thesis@preamble| and |\thesis@postamble| private macros
% at the beginning and the end of the document, respectively. The
% setting can be tested using the \DescribeMacro{\ifthesis@auto}
% |\ifthesis@auto| ... |\else| ... |\fi| conditional. The
% autolayout is enabled by default.
%    \begin{macrocode}
\newif\ifthesis@auto\thesis@autotrue
\define@key{thesis}{autoLayout}[true]{%
  \def\@true{true}%
  \def\@arg{#1}%
  \ifx\@true\@arg%
    \thesis@autotrue%
  \else%
    \thesis@autofalse%
  \fi}
%    \end{macrocode}
% The \DescribeMacro{\thesis@preamble}|\thesis@postamble|
% and \DescribeMacro{\thesis@postamble}|\thesis@preamble|
% private macros are defined as empty strings by default and are
% subject to redefinition by the style files.
%    \begin{macrocode}
\def\thesis@preamble{}
\def\thesis@postamble{}
%    \end{macrocode}
% \section{Private API}
% \subsection{Main routine}\label{sec:thesis@load}
% The |\thesis@load| macro is responsible for preparing the
% environment for, and consequently loading, the necessary locale
% and style files. By default, the \DescribeMacro{\thesis@load}
% |\thesis@load| macro gets expanded at the end of the preamble,
% but it can be inserted manually prior to that, if necessary to
% prevent package clashes. The \DescribeMacro{\ifthesis@loaded}
% |\ifthesis@loaded| semaphore ensures that the expansion is only
% performed once.
%    \begin{macrocode}
\newif\ifthesis@loaded\thesis@loadedfalse
\AtEndPreamble{\thesis@load}
\def\thesis@load{%
  \ifthesis@loaded\else%
    \thesis@loadedtrue
    \makeatletter%
%    \end{macrocode}
% First, the locale files are included. If inheritance is
% enabled for locale files, then each of the following files is
% input in sequence, provided they exist:\begin{enumerate}
%   \item|\thesis@localepath base.tex|
%   \item|\thesis@localepath\thesis@locale.tex|
%   \item|\thesis@localepath\thesis@university/\thesis@locale.tex|
%   \item|\thesis@localepath\thesis@university/\thesis@faculty/|^^A
%     |\thesis@locale.tex|
% \end{enumerate}If inheritance is disabled for locale files,
% then only the |\thesis@localepath|\discretionary{}{}{}^^A
% |\thesis@locale.tex| file is loaded.
%    \begin{macrocode}
      \ifthesis@locale@inheritance
        \input{\thesis@localepath base}
      \fi
      \thesis@input{\thesis@localepath\thesis@locale}%
      \ifthesis@locale@inheritance
        \thesis@input{\thesis@localepath\thesis@university/%
          \thesis@locale}% 
        \thesis@input{\thesis@localepath\thesis@university/%
          \thesis@faculty/\thesis@locale}%
      \fi
%    \end{macrocode}
% With the placeholder strings loaded from the locale files, we
% can now inject metadata into the resulting PDF file. To this
% end, the \textsf{hyperref} package is conditionally included.
% Consequently, the following values are assigned to the PDF
% headers:\begin{itemize}
%   \item\texttt{Title} is set to either |\thesis@titleEn|, if
%     defined, or to |\thesis@title|.
%   \item\texttt{Author} is set to |\thesis@author|.
%   \item\texttt{Keywords} is set to either |\thesis@keywordsEn|,
%     if defined, or to |\thesis@keywords|.
%   \item\texttt{Creator} is set to \texttt{\thesis@version}.
%   \item\texttt{Subject} is set to either |\thesis@abstractEn|, if
%     defined, or to |\thesis@abstract|.
% \end{itemize}
%    \begin{macrocode}
      \@ifpackageloaded{hyperref}{}{\RequirePackage{hyperref}}%
      \hypersetup{%
        pdftitle={\ifx\thesis@titleEn\undefined%
          \thesis@title%
        \else%
          \thesis@titleEn%
        \fi}, pdfauthor={\thesis@author},%
        pdfkeywords={\ifx\thesis@keywordsEn\undefined%
          \thesis@keywords%
        \else%
          \thesis@keywordsEn%
        \fi}, pdfcreator={\thesis@version},%
        pdfsubject={\ifx\thesis@abstractEn\undefined%
          \thesis@abstract%
        \else%
          \thesis@abstractEn%
        \fi}
      }%
%    \end{macrocode}
% Consequently, the style files are loaded with the class
% options passed onto them. If inheritance is enabled for
% style files, then each of the following files is
% loaded in sequence, provided they exist:\begin{enumerate}
%   \item|\thesis@stylepath base.sty|
%   \item|\thesis@stylepath\thesis@university/base.sty|
%   \item|\thesis@stylepath\thesis@style.sty|
% \end{enumerate}If inheritance is disabled for style files,
% then only the |\thesis@stylepath\thesis@|^^A
% \discretionary{}{}{}|style.sty| file is loaded.
%    \begin{macrocode}
      \ifthesis@style@inheritance
        \thesis@exists{\thesis@stylepath base.sty}{%
          \RequirePackageWithOptions{\thesis@stylepath base}}%
        \thesis@exists{\thesis@stylepath\thesis@university/%
          base.sty}{\RequirePackageWithOptions{\thesis@stylepath%
          \thesis@university/base}}%
      \fi
      \thesis@exists{\thesis@stylepath\thesis@style.sty}{%
        \RequirePackageWithOptions{\thesis@stylepath\thesis@style}}%
%    \end{macrocode}
% If autolayout is enabled, the |\thesis@preamble| and
% |\thesis@postamble| macros are scheduled for expansion at the
% beginning and at the end of the document, respectively.
%    \begin{macrocode}
      \ifthesis@auto%
        \AtBeginDocument{\thesis@preamble}%
        \AtEndDocument{\thesis@postamble}%
      \fi%
%    \end{macrocode}
% Lastly, a \BibTeX\ file named |\jobname.bib| containing the
% bibliographical entry for the thesis is scheduled to be
% generated at the end of the document in the working directory
% using the |\thesis@bibgen| macro. The document needs to be
% typeset at least twice for the style files to be able to use the
% file.
%    \begin{macrocode}
      \AtEndDocument{%
        % Generate the BibTeX file
        \newoutputstream{bib}%
        \openoutputfile{\jobname.bib}{bib}%
        \thesis@bibgen{bib}
        \closeoutputstream{bib}}%
    \makeatother%
  \fi}
%    \end{macrocode}
% The \DescribeMacro{\thesis@exists}|\thesis@exists| and
% \DescribeMacro{\thesis@input}|\thesis@input| macros are used to
% include locale files and test the existance of files in general.
%    \begin{macrocode}
\def\thesis@exists#1#2{%
  \IfFileExists{#1}{#2}{%
  \typeout{File #1 doesn't exist.}}}

\def\thesis@input#1{%
  \thesis@exists{#1}{\input{#1}}}
%    \end{macrocode}
% The \DescribeMacro{\thesis@bibgen}|\thesis@bibgen| macro
% generates the contents of a \BibTeX\ file containing a
% bibliographical entry for the thesis.
%    \begin{macrocode}
% Temporarily swap the catcodes of {} and <>
{\catcode`\<=1
\catcode`\>=2
\catcode`\{=12
\catcode`\}=12
\catcode`\_=13
\gdef\thesis@bibgen#1<<%
  % Helper macros
  \def\add<\addtostream<#1>>%
  \let\ea\expandafter%
  %% Find the last space-separated word
  \def\tail##1<\xtail##1 \relax>%
  \def\xtail##1 ##2<%
    \ifx\relax##2%
      ##1%
      \ea\@gobbletwo%
    \fi%
    \xtail##2>%
  %% Pre-cooked parts of the output
  \edef\thesis@author@toks<\thesis@author>
  \def\surname<\ea\tail\ea<\thesis@author@toks>>%
  \edef\entryType<@\ifx\thesis@type\thesis@masters%
      MastersThesis%
    \else\ifx\thesis@type\thesis@doctoral%
      PhdThesis%
    \else%
      misc%
    \fi\fi>%
  % Generate the file
  <%% Temporarily turn _s into spaces
    \catcode`\_=13 \let_=\space
    % Temporarily disable the UTF-8 encoding
    \def\UTFviii@two@octets##1##2<%
      \string##1\string##2>%
    \def\UTFviii@three@octets##1##2##3<%
      \string##1\string##2\string##3>%
    \def\UTFviii@four@octets##1##2##3##4<%
      \string##1\string##2\string##3\string##4>%
    % Fill the output stream
    \add<\entryType{\surname\thesis@year thesis,>%
    \add<__AUTHOR_____=_"\thesis@author",>%
    \add<__TITLE______=_"\thesis@title",>%
    \add<__YEAR_______=_"\thesis@year",>%
    \add<__TYPE_______=_"\thesis@typeName",>%
    \add<__SCHOOL_____=_"\thesis@universityName,
      \thesis@facultyName",>%
    \add<__SUPERVISOR_=_"\thesis@advisor",>%
    \add<__PAGES______=_"\thepage">%
    \add<}>
  >>>%
>
%    \end{macrocode}
% \subsection{General utility macros}
% The \DescribeMacro{\thesis@lower}|\thesis@lower|
% and \DescribeMacro{\thesis@upper}|\thesis@upper|
% private macros are to be used for upper- and lowercasing within
% locale files. To cast the |\thesis@|\textit{name} macro
% to the lower- or uppercase, |\thesis@lower{|\textit{name}|}| or
% |\thesis@upper{|\textit{name}|}| would be used, respectively.
%    \begin{macrocode}
\def\thesis@lower#1{%
  \edef\thesis@expanded{\csname thesis@#1\endcsname}%
  \expandafter\lowercase\expandafter{\thesis@expanded}}
\def\thesis@upper#1{%
  \edef\thesis@expanded{\csname thesis@#1\endcsname}%
  \expandafter\uppercase\expandafter{\thesis@expanded}}
%    \end{macrocode}
% \iffalse
%</class>
%
%<*oldclass1>

\NeedsTeXFormat{LaTeX2e}
\ProvidesClass{oldfithesis1}[2015/03/04 old fithesis will load fithesis3 MU thesis class]

\errmessage{%
  You are using the fithesis class, which has been deprecated.
  The fithesis3 class will be used instead.
  For more information, see <https://www.fi.muni.cz/tech/unix/tex/fithesis.xhtml>%
}\LoadClass{fithesis3}

%</oldclass1>
%
%<*oldclass2>

\NeedsTeXFormat{LaTeX2e}
\ProvidesClass{oldfithesis2}[2015/03/04 old fithesis2 will load fithesis3 MU thesis class]

\errmessage{%
  You are using the fithesis2 class, which has been deprecated.
  The fithesis3 class will be used instead.
  For more information, see <https://www.fi.muni.cz/tech/unix/tex/fithesis.xhtml>%
}\LoadClass{fithesis3}

%</oldclass2>
% \fi
%
% \subsection{Locale files}
% \label{sec:locale-files}
% Locale files contain macro definitions for various locales. They
% live in the \texttt{locale/} subtree and they are loaded during
% the main routine (see section \ref{sec:thesis@load}). A locale
% file needs to define the following private macros:\begin{itemize}
%   \item\DescribeMacro{\thesis@placeholders@title}
%                      |\thesis@placeholders@title| -- The
%                      placeholder title of the thesis
%   \item\DescribeMacro{\thesis@placeholders@keywords}
%                      |\thesis@placeholders@keywords| -- The
%                      placeholder keywords of the thesis
%   \item\DescribeMacro{\thesis@placeholders@abstract}
%                      |\thesis@placeholders@abstract| -- The
%                      placeholder abstract of the thesis 
%   \item\DescribeMacro{\thesis@placeholders@author}
%                      |\thesis@placeholders@author| -- The
%                      placeholder name of the thesis author
%   \item\DescribeMacro{\thesis@universityName}
%                      |\thesis@universityName| -- The name of the
%                      university
%   \item\DescribeMacro{\thesis@facultyName}
%                      |\thesis@facultyName| -- The name of the
%                      faculty
%   \item\DescribeMacro{\thesis@placeholders@advisor}
%                      |\thesis@placeholders@advisor| -- The
%                      placeholder name of the advisor
%   \item\DescribeMacro{\thesis@placeholders@department}
%                      |\thesis@placeholders@department| -- The
%                      placeholder department name
%   \item\DescribeMacro{\thesis@placeholders@programme}
%                      |\thesis@placeholders@programme| -- The
%                      placeholder programme name
%   \item\DescribeMacro{\thesis@placeholders@thanks}
%                      |\thesis@placeholders@thanks| -- The
%                      placeholder acknowledgement text
%   \item\DescribeMacro{\thesis@declaration}
%                      |\thesis@declaration| -- The declaration
%                      text
%   \item\DescribeMacro{\thesis@advisorTitle}
%                      |\thesis@advisorTitle| -- The title of the
%                      advisor
%   \item\DescribeMacro{\thesis@abstractTitle}
%                      |\thesis@abstractTitle| -- The title of the
%                      abstract section
%   \item\DescribeMacro{\thesis@keywordsTitle}
%                      |\thesis@keywordsTitle| -- The title of the
%                      keywords section
%   \item\DescribeMacro{\thesis@thanksTitle}
%                      |\thesis@thanksTitle| -- The title of the
%                      acknowledgement section
%   \item\DescribeMacro{\thesis@declarationTitle}
%                      |\thesis@declarationTitle| -- The title of
%                      the declaration section
%   \item\DescribeMacro{\thesis@typeName}
%                      |\thesis@typeName| -- The name of the
%                      thesis type
% \end{itemize}
% ^^A % \file{locale/base.def}
% If inheritance is enabled for locale files, then this file is
% always the first locale file to be loaded, regardless of the
% value of the |\thesis@locale| macro. The file loads the English
% locale strings as a fallback for any potentially missing strings
% in the subsequently loaded locale files and defines the following
% two additional private macros to be used in the style files:
% \begin{itemize}
%   \item\DescribeMacro{\thesis@abstractEnTitle}
%                      |\thesis@abstractEnTitle| -- The title of the
%                      abstract section in English
%   \item\DescribeMacro{\thesis@keywordsEnTitle}
%                      |\thesis@keywordsEnTitle| -- The title of the
%                      keywords section in English
% \end{itemize}
%    \begin{macrocode}
\ProvidesFile{base.def}[2015/04/08]
% A symbolic link to the English locale
\input{\thesis@localepath english.def}
\let\thesis@abstractEnTitle=\thesis@abstractTitle
\let\thesis@keywordsEnTitle=\thesis@keywordsTitle
%    \end{macrocode}

% % \file{locale/fithesis-english.def}
% This is the base file of the English locale.\iffalse
%<*base>
% \fi\begin{macrocode}
\ProvidesFile{fithesis/locale/fithesis-english.def}[2017/09/08]
%    \end{macrocode}
% The locale file defines all the private macros mandated by the
% locale file interface.
% \begin{macrocode}

% Placeholders
\gdef\thesis@english@universityName{University name}
\gdef\thesis@english@facultyName{Faculty name}
%    \end{macrocode}
% \changes{v0.3.47}{2017/07/09}{Moved the \cs{ifthesis@digital}
%   tests from \texttt{locale/*.def} to \texttt{locale/mu/*.def},
%   since \cs{ifthesis@digital} is undefined in
%   \texttt{fithesis3.cls}. [VN]}
%    \begin{macrocode}
\gdef\thesis@english@assignment{%
  This is where a copy of the official signed thesis assignment
  is located in the printed version of the document.}
\gdef\thesis@english@declaration{Declaration text ...}

% Csquotes style
\gdef\thesis@english@csquotesStyle{english}

% Time strings
\gdef\thesis@english@spring{Spring}
\gdef\thesis@english@fall{Fall}
\gdef\thesis@english@semester{%
  \thesis@{english@\thesis@season} \thesis@seasonYear}
\gdef\thesis@english@formattedDate{{%
  \thesis@day.
  \newcount\@month\expandafter\@month\thesis@month\relax
  \ifnum\@month=1%
    January
  \else\ifnum\@month=2%
    February
  \else\ifnum\@month=3%
    March
  \else\ifnum\@month=4%
    April
  \else\ifnum\@month=5%
    May
  \else\ifnum\@month=6%
    June
  \else\ifnum\@month=7%
    July
  \else\ifnum\@month=8%
    August
  \else\ifnum\@month=9%
    September
  \else\ifnum\@month=10%
    October
  \else\ifnum\@month=11%
    November
  \else\ifnum\@month=12%
    December
  \else
    <<unknown month (\the\@month)>>
  \fi\fi\fi\fi\fi\fi
  \fi\fi\fi\fi\fi\fi
  \thesis@year}}

% Miscellaneous
\gdef\thesis@english@authorSignature{Author's signature}
\gdef\thesis@english@fieldTitle{Field of study}
\gdef\thesis@english@advisorTitle{Advisor}
\gdef\thesis@english@authorTitle{Author}
\gdef\thesis@english@abstractTitle{Abstract}
\gdef\thesis@english@keywordsTitle{Keywords}
%    \end{macrocode}
% \changes{v0.3.48}{2017/09/08}{Changed
%   \cs{thesis@english@thanksTitle} to plural. [VN]}
%    \begin{macrocode}
\gdef\thesis@english@thanksTitle{Acknowledgements}
\gdef\thesis@english@declarationTitle{Declaration}
\gdef\thesis@english@idTitle{ID}
\gdef\thesis@english@typeName@sempaper{Seminar Paper}
\gdef\thesis@english@typeName@bachelors{Bachelor's Thesis}
\gdef\thesis@english@typeName@masters{Master's Thesis}
\gdef\thesis@english@typeName@proposal{Thesis Proposal}
\gdef\thesis@english@typeName@doctoral{Doctoral Thesis}
\gdef\thesis@english@typeName@rigorous{Rigorous Thesis}
\gdef\thesis@english@typeName{%
  \ifx\thesis@type\thesis@sempaper
    \thesis@english@typeName@sempaper
  \else\ifx\thesis@type\thesis@bachelors
    \thesis@english@typeName@bachelors
  \else\ifx\thesis@type\thesis@masters
    \thesis@english@typeName@masters
  \else\ifx\thesis@type\thesis@proposal
    \thesis@english@typeName@proposal
  \else\ifx\thesis@type\thesis@doctoral
    \thesis@english@typeName@doctoral
  \else\ifx\thesis@type\thesis@rigorous
    \thesis@english@typeName@rigorous
  \else
    <<Unknown thesis type (\thesis@type)>>%
  \fi\fi\fi\fi\fi\fi}
%    \end{macrocode}\iffalse
%</base>
% \fi\file{locale/mu/fithesis-english.def}
% This is the English locale file specific to the Masaryk
% University in Brno. It replaces the \texttt{universityName}
% placeholder with the correct value and defines the
% \texttt{declaration} and \texttt{idTitle} strings.
% \iffalse
%<*mu>
% \fi\begin{macrocode}
\ProvidesFile{fithesis/locale/mu/fithesis-english.def}[2017/07/09]
\gdef\thesis@english@universityName{Masaryk University}
\gdef\thesis@english@declaration{%
  Hereby I declare that this paper is my original authorial work,
  which I have worked out on my own. All sources, references, and
  literature used or excerpted during elaboration of this work are
  properly cited and listed in complete reference to the due source.}

% Placeholders
%    \end{macrocode}
% \changes{v0.3.47}{2017/07/09}{Moved the \cs{ifthesis@digital}
%   tests from \texttt{locale/*.def} to \texttt{locale/mu/*.def},
%   since \cs{ifthesis@digital} is undefined in
%   \texttt{fithesis3.cls}. [VN]}
%    \begin{macrocode}
\gdef\thesis@english@assignment{%
  \ifthesis@digital@
  \else
  \fi}
%    \end{macrocode}
% \changes{v0.3.47}{2017/07/09}{Moved the \cs{ifthesis@digital}
%   tests from \texttt{locale/*.def} to \texttt{locale/mu/*.def},
%   since \cs{ifthesis@digital} is undefined in
%   \texttt{fithesis3.cls}. [VN]}
% \changes{v0.3.47}{2017/07/09}{Added an
% \cs{ifthesis@blocks@assignment@hideIfDigital@} test to the
% definition of the \texttt{assignment} string for the Masaryk
% University in Brno. [VN]}
%    \begin{macrocode}
\gdef\thesis@english@assignment{%
  \ifthesis@blocks@assignment@hideIfDigital@
    \ifthesis@digital@
      This is where a copy of the official signed thesis assignment
      is located in the printed version of the document.
    \else
      Replace this page with a copy of the official signed thesis
      assignment.
    \fi
  \else
    Set the PDF document containing the official signed thesis
    assignment using the <<assignment>> key.
  \fi}

% Bibliographic entry
\gdef\thesis@english@bib@title{Bibliographic record}
\gdef\thesis@english@bib@pages{p}
%    \end{macrocode}
% \changes{v0.3.46}{2017/06/02}{Lifted the \texttt{bib@author},
%   \texttt{bib@thesisTitle}, and \texttt{bib@advisor} strings from
%   \texttt{locale/mu/sci/*.def} to \texttt{locale/mu/*.def},
%   so that they can be shared with \texttt{locale/mu/econ/*.def}.
%   [VN]}
%    \begin{macrocode}
\global\let\thesis@english@bib@author\thesis@english@authorTitle
\gdef\thesis@english@bib@thesisTitle{Title of Thesis}
\gdef\thesis@english@bib@advisor{Supervisor}

% Miscellaneous
\gdef\thesis@english@idTitle{UČO}
%    \end{macrocode}\iffalse
%</mu>
% \fi\file{locale/mu/law/fithesis-english.def}
% This is the English locale file specific to the Faculty of Law at
% the Masaryk University in Brno. It replaces the
% \texttt{facultyName} placeholder with the correct value and
% defines the \texttt{facultyLongName} required by the
% |\thesis@blocks@cover| and the |\thesis@blocks@titlePage| blocks.
% \iffalse
%<*mu/law>
% \fi\begin{macrocode}
\ProvidesFile{fithesis/locale/mu/law/fithesis-english.def}[2015/06/26]
\gdef\thesis@english@facultyName{Faculty of Law}
\gdef\thesis@english@facultyLongName{The Faculty of Law of the
  Masaryk University}
%    \end{macrocode}\iffalse
%</mu/law>
% \fi\file{locale/mu/fsps/fithesis-english.def}
% This is the English locale file specific to the Faculty of Sports
% Studies at the Masaryk University in Brno. It replaces the
% \texttt{facultyName} placeholder with the correct value and
% redefines the \texttt{fieldTitle} string in accordance with the
% common usage at the faculty.
% \iffalse
%<*mu/fsps>
% \fi\begin{macrocode}
\ProvidesFile{fithesis/locale/mu/fsps/fithesis-english.def}[2017/06/02]

% Placeholders
\gdef\thesis@english@facultyName{Faculty of Sports Studies}

% Miscellaneous
\gdef\thesis@english@fieldTitle{Specialization}
%    \end{macrocode}\iffalse
%</mu/fsps>
% \fi\file{locale/mu/fss/fithesis-english.def}
% This is the English locale file specific to the Faculty of Social
% Studies at the Masaryk University in Brno. It replaces the
% \texttt{facultyName} and \texttt{assignment} strings with the
% correct values.
% \iffalse
%<*mu/fss>
% \fi\begin{macrocode}
\ProvidesFile{fithesis/locale/mu/fss/fithesis-english.def}[2016/05/25]

% Placeholders
\gdef\thesis@english@facultyName{Faculty of Social Studies}
\gdef\thesis@english@assignment{%
  \ifthesis@digital@
    This is where a copy of the official signed thesis assignment
    or a copy of the Statement of an Author or both are located
    in the printed version of the document.
  \else
    Replace this page with a copy of the official signed thesis
    assignment or a copy of the Statement of an Author or both,
    depending on the requirements of the respective department.
  \fi}
%    \end{macrocode}\iffalse
%</mu/fss>
% \fi\file{locale/mu/econ/fithesis-english.def}
% This is the English locale file specific to the Faculty of
% Economics and Administration at the Masaryk University in Brno.
% It replaces the \texttt{facultyName} and \texttt{abstractTitle}
% placeholders with the correct value. The locale file also defines
% the private macros required by the
% |\thesis@blocks@|\discretionary{}{}{}|bibEntry| block defined
% within the \texttt{style/mu/fithesis-econ.sty} style file.
% \iffalse
%<*mu/econ>
% \fi\begin{macrocode}
\ProvidesFile{fithesis/locale/mu/econ/fithesis-english.def}[2017/06/02]

% Placeholders
\gdef\thesis@english@facultyName{Faculty of Economics
  and Administration}

% Bibliographic entry
%    \end{macrocode}
% \changes{v0.3.46}{2017/06/02}{Defined strings required by
%   \cs{thesis@blocks@bibEntry} from
%   \texttt{style/mu/fithesis-econ.sty} in
%   \texttt{locale/mu/econ/*.def}. [VN]}
%    \begin{macrocode}
\gdef\thesis@english@bib@department{Department}
\gdef\thesis@english@bib@year{Year of Defense}

% Miscellaneous
%    \end{macrocode}
% \changes{v0.3.46}{2017/06/02}{Updated the
%   \cs{abstractTitle} string in \texttt{locale/mu/econ/*.def} in
%   accordance with the 2/2017 dean's directive. The patch was
%   submitted by Jana Ratajská. [VN]}
%    \begin{macrocode}
\gdef\thesis@english@abstractTitle{Annotation}
%    \end{macrocode}\iffalse
%</mu/econ>
% \fi\file{locale/mu/med/fithesis-english.def}
% This is the English locale file specific to the Faculty of
% Medicine at the Masaryk University in Brno.
% It replaces the \texttt{facultyName} placeholder with the
% correct value and redefines the \texttt{abstractTitle} string
% with the common usage at the faculty. The file also defines
% the \texttt{bib@title} and \texttt{bib@pages} strings required
% by the |\thesis@blocks@bibEntry| block defined within the
% \texttt{style/mu/\discretionary{}{}{}fithesis-med.sty}
% style file.
% \iffalse
%<*mu/med>
% \fi\begin{macrocode}
\ProvidesFile{fithesis/locale/mu/med/fithesis-english.def}[2016/03/23]

% Placeholders
\gdef\thesis@english@facultyName{Faculty of Medicine}

% Miscellaneous
\gdef\thesis@english@abstractTitle{Annotation}
%    \end{macrocode}\iffalse
%</mu/med>
% \fi\file{locale/mu/fi/fithesis-english.def}
% This is the English locale file specific to the Faculty of
% Informatics at the Masaryk University in Brno.  It replaces the
% \texttt{facultyName} placeholder with the correct value and
% redefines the string in accordance with the requirements of the
% faculty.  The file also defines the \texttt{advisorSignature}
% string required by the |\thesis@blocks@titlePage| block defined
% within the
% \texttt{style/mu/\discretionary{}{}{}fithesis-fi.sty}
% style file.
% \iffalse
%<*mu/fi>
% \fi\begin{macrocode}
\ProvidesFile{fithesis/locale/mu/fi/fithesis-english.def}[2016/05/25]

% Placeholders
\gdef\thesis@english@facultyName{Faculty of Informatics}
\gdef\thesis@english@assignment{Replace this page with a copy
  of the official signed thesis assignment and a copy of the
  Statement of an Author.}
\gdef\thesis@english@assignment{%
  \ifthesis@digital@
    This is where a copy of the official signed thesis assignment
    and a copy of the Statement of an Author is located in the
    printed version of the document.
  \else
    Replace this page with a copy of the official signed thesis
    assignment and a copy of the Statement of an Author.
  \fi}

% Others
\gdef\thesis@english@advisorSignature{Signature of Thesis
  \thesis@english@advisorTitle}
\gdef\thesis@english@typeName@proposal{Ph.D. Thesis Proposal}
%    \end{macrocode}\iffalse
%</mu/fi>
% \fi\file{locale/mu/phil/fithesis-english.def}
% This is the English locale file specific to the Faculty of
% Arts at the Masaryk University in Brno.
% It replaces the \texttt{facultyName} placeholder with the
% correct value. It also defines the \texttt{departmentName}
% string, which is used by the \texttt{style/mu/fithesis-phil^^A
% .sty} style file, when typesetting the names of known
% departments.
% \iffalse
%<*mu/phil>
% \fi\begin{macrocode}
\ProvidesFile{fithesis/locale/mu/phil/fithesis-english.def}[2016/03/22]
\gdef\thesis@english@facultyName{Faculty of Arts}
\gdef\thesis@english@departmentName{%
  \ifx\thesis@department\thesis@departments@kisk
    Division of Information and Library Studies%
  \else
    <<Unknown department (\thesis@department)>>%
  \fi}
%    \end{macrocode}\iffalse
%</mu/phil>
% \fi\file{locale/mu/ped/fithesis-english.def}
% This is the Slovak locale file specific to the Faculty of
% Education at the Masaryk University in Brno.  It replaces the
% \texttt{facultyName} placeholder with the correct value. The file
% also defines the \texttt{bib@title} and \texttt{bib@pages}
% strings required by the |\thesis@blocks@bibEntry| block defined
% within the
% \texttt{style/mu/\discretionary{}{}{}fithesis-ped.sty}
% style file.
% \iffalse
%<*mu/ped>
% \fi\begin{macrocode}
\ProvidesFile{fithesis/locale/mu/ped/fithesis-english.def}[2016/03/22]

% Placeholders
\gdef\thesis@english@facultyName{Faculty of Education}
%    \end{macrocode}\iffalse
%</mu/ped>
% \fi\file{locale/mu/sci/fithesis-english.def}
% This is the English locale file specific to the Faculty of
% Science at the Masaryk University in Brno.  It defines the
% private macros required by the |\thesis@blocks@bibEntryEn| block
% defined within the
% \texttt{style/mu/\discretionary{}{}{}fithesis-sci.sty}
% style file. It also replaces the \texttt{facultyName}
% placeholder with the correct value and redefines the
% \texttt{advisorTitle} string in accordance with the formal
% requirements of the faculty.
% \iffalse
%<*mu/sci>
% \fi\begin{macrocode}
\ProvidesFile{fithesis/locale/mu/sci/fithesis-english.def}[2017/06/02]

% Placeholders
\gdef\thesis@english@facultyName{Faculty of Science}

% Miscellaneous
\global\let\thesis@english@advisorTitleEn=\thesis@english@bib@advisor

% Bibliographic entry
\gdef\thesis@english@bib@programme{Degree Programme}
\global\let\thesis@english@bib@field\thesis@english@fieldTitle
\gdef\thesis@english@bib@academicYear{Academic Year}
\gdef\thesis@english@bib@pages{Number of Pages}
\global\let\thesis@english@bib@keywords\thesis@english@keywordsTitle
%    \end{macrocode}\iffalse
%</mu/sci>
% \fi

% % \file{locale/fithesis-czech.def}
% This is the base file of the Czech locale.\iffalse
%<*base>
% \fi\begin{macrocode}
\ProvidesFile{fithesis/locale/fithesis-czech.def}[2017/07/09]
%    \end{macrocode}
% The locale file defines all the private macros mandated by the
% locale file interface.
% \begin{macro}{\thesis@czech@gender@koncovka}
% The locale file also defines the |\thesis@czech@gender@koncovka|
% macro, which expands to the correct verb ending based on the
% value of the |\thesis@ifwoman| macro and the
% \end{macro}\begin{macro}{\thesis@czech@typeName@akuzativ}
% |\thesis@czech@typeName@akuzativ| containing the accusative case
% of the thesis type name.
% \end{macro}\begin{macrocode}

% Pomocná makra
\gdef\thesis@czech@gender@koncovka{%
  \ifthesis@woman a\fi}

% Csquotes styl
\gdef\thesis@czech@csquotesStyle{german}

% Zástupné texty
\gdef\thesis@czech@universityName{Název univerzity}
\gdef\thesis@czech@facultyName{Název fakulty}
%    \end{macrocode}
% \changes{v0.3.47}{2017/07/09}{Moved the \cs{ifthesis@digital}
%   tests from \texttt{locale/*.def} to \texttt{locale/mu/*.def},
%   since \cs{ifthesis@digital} is undefined in
%   \texttt{fithesis3.cls}. [VN]}
%    \begin{macrocode}
\gdef\thesis@czech@assignment{%
  Na tomto místě se v~tištěné práci nachází oficiální podepsané
  zadání práce.}
\gdef\thesis@czech@declaration{Text prohlášení ...}

% Časové údaje
\gdef\thesis@czech@spring{jaro}
\gdef\thesis@czech@fall{podzim}
\gdef\thesis@czech@semester{%
  \thesis@{czech@\thesis@season} \thesis@seasonYear}
\gdef\thesis@czech@formattedDate{{%
  \thesis@day.
  \newcount\@month\expandafter\@month\thesis@month\relax
  \ifnum\@month=1%
    ledna
  \else\ifnum\@month=2%
    února
  \else\ifnum\@month=3%
    března
  \else\ifnum\@month=4%
    dubna
  \else\ifnum\@month=5%
    května
  \else\ifnum\@month=6%
    června
  \else\ifnum\@month=7%
    července
  \else\ifnum\@month=8%
    srpna
  \else\ifnum\@month=9%
    září
  \else\ifnum\@month=10%
    října
  \else\ifnum\@month=11%
    listopadu
  \else\ifnum\@month=12%
    prosince
  \else
    <<neznámý měsíc (\the\@month)>>
  \fi\fi\fi\fi\fi\fi
  \fi\fi\fi\fi\fi\fi
  \thesis@year}}

% Různé
\gdef\thesis@czech@authorSignature{Podpis autora}
\gdef\thesis@czech@fieldTitle{Obor}
\gdef\thesis@czech@advisorTitle{Vedoucí práce}
\gdef\thesis@czech@authorTitle{Autor}
\gdef\thesis@czech@abstractTitle{Shrnutí}
\gdef\thesis@czech@keywordsTitle{Klíčová slova}
\gdef\thesis@czech@thanksTitle{Poděkování}
\gdef\thesis@czech@declarationTitle{Prohlášení}
\gdef\thesis@czech@idTitle{ID}
\gdef\thesis@czech@typeName@sempaper{Seminární práce}
\gdef\thesis@czech@typeName@bachelors{Bakalářská práce}
\gdef\thesis@czech@typeName@masters{Diplomová práce}
\gdef\thesis@czech@typeName@proposal{Teze závěrečné práce}
\gdef\thesis@czech@typeName@doctoral{Disertační práce}
\gdef\thesis@czech@typeName@rigorous{Rigorózní práce}
\gdef\thesis@czech@typeName{%
  \ifx\thesis@type\thesis@sempaper
    \thesis@czech@typeName@sempaper
  \else\ifx\thesis@type\thesis@bachelors
    \thesis@czech@typeName@bachelors
  \else\ifx\thesis@type\thesis@masters
    \thesis@czech@typeName@masters
  \else\ifx\thesis@type\thesis@proposal
    \thesis@czech@typeName@proposal
  \else\ifx\thesis@type\thesis@doctoral
    \thesis@czech@typeName@doctoral
  \else\ifx\thesis@type\thesis@rigorous
    \thesis@czech@typeName@rigorous
  \else
    <<Neznámý typ práce (\thesis@type)>>%
  \fi\fi\fi\fi\fi\fi}
\gdef\thesis@czech@typeName@akuzativ@sempaper{Seminární práci}
\gdef\thesis@czech@typeName@akuzativ@bachelors{Bakalářskou práci}
\gdef\thesis@czech@typeName@akuzativ@masters{Diplomovou práci}
\gdef\thesis@czech@typeName@akuzativ@proposal{Tezi závěrečné práce}
\gdef\thesis@czech@typeName@akuzativ@doctoral{Disertační práci}
\gdef\thesis@czech@typeName@akuzativ@rigorous{Rigorózní práci}
\gdef\thesis@czech@typeName@akuzativ{%
  \ifx\thesis@type\thesis@sempaper
    \thesis@czech@typeName@akuzativ@sempaper
  \else\ifx\thesis@type\thesis@bachelors
    \thesis@czech@typeName@akuzativ@bachelors
  \else\ifx\thesis@type\thesis@masters
    \thesis@czech@typeName@akuzativ@masters
  \else\ifx\thesis@type\thesis@proposal
    \thesis@czech@typeName@akuzativ@proposal
  \else\ifx\thesis@type\thesis@doctoral
    \thesis@czech@typeName@akuzativ@doctoral
  \else\ifx\thesis@type\thesis@rigorous
    \thesis@czech@typeName@akuzativ@rigorous
  \else
    <<Neznámý typ práce (\thesis@type)>>%
  \fi\fi\fi\fi\fi\fi}
%    \end{macrocode}\iffalse
%</base>
% \fi\file{locale/mu/fithesis-czech.def}
% This is the Czech locale file specific to the Masaryk
% University in Brno. It replaces the \texttt{universityName}
% placeholder with the correct value and defines the
% \texttt{declaration} and \texttt{idTitle} strings.
% \iffalse
%<*mu>
% \fi\begin{macrocode}
\ProvidesFile{fithesis/locale/mu/fithesis-czech.def}[2017/07/09]

% Zástupné texty
\gdef\thesis@czech@universityName{Masarykova univerzita}
\gdef\thesis@czech@declaration{Prohlašuji, že jsem
  \thesis@lower{czech@typeName@akuzativ} zpracoval%
  \thesis@czech@gender@koncovka\ samostatně a~%
  použil\thesis@czech@gender@koncovka\ jen prameny
  uvedené v~seznamu literatury.}
%    \end{macrocode}
% \changes{v0.3.47}{2017/07/09}{Moved the \cs{ifthesis@digital}
%   tests from \texttt{locale/*.def} to \texttt{locale/mu/*.def},
%   since \cs{ifthesis@digital} is undefined in
%   \texttt{fithesis3.cls}. [VN]}
% \changes{v0.3.47}{2017/07/09}{Added an
% \cs{ifthesis@blocks@assignment@hideIfDigital@} test to the
% definition of the \texttt{assignment} string for the Masaryk
% University in Brno. [VN]}
%    \begin{macrocode}
\gdef\thesis@czech@assignment{%
  \ifthesis@blocks@assignment@hideIfDigital@
    \ifthesis@digital@
      Na tomto místě se v~tištěné práci nachází oficiální podepsané
      zadání práce.
    \else
      Místo tohoto listu vložte kopii oficiálního podepsaného zadání
      práce.
    \fi
  \else
    Nastavte pomocí klíče <<assignment>> název PDF souboru
    s~oficiálním podepsaným zadáním práce.
  \fi}

% Bibliografický záznam
\gdef\thesis@czech@bib@title{Bibliografický záznam}
\gdef\thesis@czech@bib@pages{str}
%    \end{macrocode}
% \changes{v0.3.46}{2017/06/02}{Lifted the \texttt{bib@author},
%   \texttt{bib@thesisTitle}, and \texttt{bib@advisor} strings from
%   \texttt{locale/mu/sci/*.def} to \texttt{locale/mu/*.def},
%   so that they can be shared with \texttt{locale/mu/econ/*.def}.
%   [VN]}
%    \begin{macrocode}
\global\let\thesis@czech@bib@author\thesis@czech@authorTitle
\gdef\thesis@czech@bib@thesisTitle{Název práce}
\global\let\thesis@czech@bib@advisor\thesis@czech@advisorTitle

% Různé
\gdef\thesis@czech@idTitle{UČO}
%    \end{macrocode}\iffalse
%</mu>
% \fi\file{locale/mu/law/fithesis-czech.def}
% This is the Czech locale file specific to the Faculty of Law at
% the Masaryk University in Brno. It replaces the
% \texttt{facultyName} placeholder with the correct value, defines
% the \texttt{facultyLongName} required by the
% |\thesis@blocks@cover| and the |\thesis@blocks@titlePage| blocks
% and replaces the \texttt{abstractTitle} string in accordance
% with the requirements of the faculty.
% \iffalse
%<*mu/law>
% \fi\begin{macrocode}
\ProvidesFile{fithesis/locale/mu/law/fithesis-czech.def}[2015/06/26]

% Různé
\gdef\thesis@czech@abstractTitle{Abstrakt}

% Zástupné texty
\gdef\thesis@czech@facultyName{Právnická fakulta}
\gdef\thesis@czech@facultyLongName{Právnická fakulta Masarykovy
  univerzity}
%    \end{macrocode}\iffalse
%</mu/law>
% \fi\file{locale/mu/fsps/fithesis-czech.def}
% This is the Czech locale file specific to the Faculty of Sports
% Studies at the Masaryk University in Brno. It replaces the
% \texttt{facultyName} placeholder with the correct value and
% redefines the \texttt{fieldTitle} string in accordance with the
% common usage at the faculty. The locale file also redefines the
% \texttt{declaration} string in accordance with the requirements
% of the faculty.
% \iffalse
%<*mu/fsps>
% \fi\begin{macrocode}
\ProvidesFile{fithesis/locale/mu/fsps/fithesis-czech.def}[2017/05/15]

% Zástupné texty
\gdef\thesis@czech@facultyName{Fakulta sportovních studií}
\gdef\thesis@czech@declaration{Prohlašuji, že jsem
  \thesis@lower{czech@typeName@akuzativ} vypracoval%
  \thesis@czech@gender@koncovka\ samostatně a~na základě
  literatury a~pramenů uvedených v~použitých zdrojích.}

% Různé
\gdef\thesis@czech@fieldTitle{Specializace}
%    \end{macrocode}\iffalse
%</mu/fsps>
% \fi\file{locale/mu/fss/fithesis-czech.def}
% This is the Czech locale file specific to the Faculty of Social
% Studies at the Masaryk University in Brno. It replaces the
% \texttt{facultyName} and \texttt{assignment} placeholders with
% the correct values.
% \iffalse
%<*mu/fss>
% \fi\begin{macrocode}
\ProvidesFile{fithesis/locale/mu/fss/fithesis-czech.def}[2016/05/25]

% Zástupné texty
\gdef\thesis@czech@facultyName{Fakulta sociálních studií}
\gdef\thesis@czech@assignment{%
  \ifthesis@digital@
    Na tomto místě se v~tištěné práci nachází oficiální podepsané
    zadání práce, prohlášení autora školního díla nebo obojí.
  \else
    Místo tohoto listu vložte kopie oficiálního podepsaného zadání
    práce nebo prohlášení autora školního díla nebo obojí
    v~závislosti na požadavcích příslušné katedry.
  \fi}

%    \end{macrocode}\iffalse
%</mu/fss>
% \fi\file{locale/mu/econ/fithesis-czech.def}
% This is the Czech locale file specific to the Faculty of
% Economics and Administration at the Masaryk University in Brno.
% It replaces the \texttt{facultyName} and \texttt{abstractTitle}
% placeholders with the correct values. The locale file also
% redefines the \texttt{declaration} string in accordance with
% the requirements of the faculty and defines the private macros
% required by the |\thesis@blocks@|\discretionary{}{}{}|bibEntry|
% block defined within the \texttt{style/mu/fithesis-econ.sty}
% style file.
% \iffalse
%<*mu/econ>
% \fi\begin{macrocode}
\ProvidesFile{fithesis/locale/mu/econ/fithesis-czech.def}[2017/07/09]

% Zástupné texty
\gdef\thesis@czech@facultyName{Ekonomicko-správní fakulta}

% Bibliografický záznam
%    \end{macrocode}
% \changes{v0.3.46}{2017/06/02}{Defined strings required by
%   \cs{thesis@blocks@bibEntry} from
%   \texttt{style/mu/fithesis-econ.sty} in
%   \texttt{locale/mu/econ/*.def}. [VN]}
%    \begin{macrocode}
\gdef\thesis@czech@bib@thesisTitleEn{Název práce v angličtině}
\gdef\thesis@czech@bib@department{Katedra}
\gdef\thesis@czech@bib@year{Rok obhajoby}

% Různé
%    \end{macrocode}
% \changes{v0.3.46}{2017/06/02}{Updated the
%   \cs{abstractTitle} string in \texttt{locale/mu/econ/*.def} in
%   accordance with the 2/2017 dean's directive. The patch was
%   submitted by Jana Ratajská. [VN]}
%    \begin{macrocode}
\gdef\thesis@czech@abstractTitle{Anotace}
%    \end{macrocode}
% \changes{v0.3.46}{2017/06/02}{Updated the \texttt{declaration} string
%   in \texttt{locale/mu/econ/*.def} in accordance with the 2/2017
%   dean's directive. [VN]}
% \changes{v0.3.47}{2017/07/09}{Updated the \texttt{declaration} string
%   in \texttt{locale/mu/econ/*.def} in accordance with the 2/2017
%   dean's directive. [VN]}
% The following extra data field is defined for
% \texttt{declaration} string: \begin{itemize}
%   \item|advisorCsGenitiv| -- the advisor's name in
%     genitive following Czech morphology.
% \end{itemize}
%    \begin{macrocode}
\thesis@def@extra{advisorCsGenitiv}
\gdef\thesis@czech@declaration{Prohlašuji, že jsem
  \thesis@lower{czech@typeName@akuzativ} \thesis@title{} zpracoval%
  \thesis@czech@gender@koncovka\ samostatně pod vedením
  \thesis@extra@advisorCsGenitiv\
  a~uvedl\thesis@czech@gender@koncovka\ v~ní všechny
  odborné zdroje v~souladu s~právními předpisy, vnitřními
  předpisy Masarykovy univerzity a~vnitřními akty řízení
  Masarykovy univerzity a~Ekonomicko-správní fakulty MU.}
%    \end{macrocode}\iffalse
%</mu/econ>
% \fi\file{locale/mu/med/fithesis-czech.def}
% This is the Czech locale file specific to the Faculty of
% Medicine at the Masaryk University in Brno.
% It replaces the \texttt{facultyName} placeholder with the
% correct value and redefines the \texttt{abstractTitle} string in
% accordance with the common usage at the faculty. The file also
% defines the \texttt{bib@title} and \texttt{bib@pages} strings
% required by the |\thesis@blocks@bibEntry| block defined within
% the \texttt{style/mu/fithesis-med.sty} style file.
% \iffalse
%<*mu/med>
% \fi\begin{macrocode}
\ProvidesFile{fithesis/locale/mu/med/fithesis-czech.def}[2016/03/23]

% Zástupné texty
\gdef\thesis@czech@facultyName{Lékařská fakulta}

% Různé
\gdef\thesis@czech@abstractTitle{Anotace}
%    \end{macrocode}\iffalse
%</mu/med>
% \fi\file{locale/mu/fi/fithesis-czech.def}
% This is the Czech locale file specific to the Faculty of
% Informatics at the Masaryk University in Brno.
% It replaces the \texttt{facultyName} placeholder with the
% correct value and redefines the \texttt{declaration} string in
% accordance with the requirements of the faculty. The file also
% defines the \texttt{advisorSignature} string required by the
% |\thesis@blocks@titlePage| block defined within the
% \texttt{style/mu/\discretionary{}{}{}fithesis-fi.sty}
% style file.
% \iffalse
%<*mu/fi>
% \fi\begin{macrocode}
\ProvidesFile{fithesis/locale/mu/fi/fithesis-czech.def}[2016/05/25]

% Zástupné texty
\gdef\thesis@czech@facultyName{Fakulta informatiky}
\gdef\thesis@czech@assignment{%
  \ifthesis@digital@
    Na tomto místě se v~tištěné práci nachází oficiální podepsané
    zadání práce a prohlášení autora školního díla.
  \else
    Místo tohoto listu vložte kopie oficiálního podepsaného zadání
    práce a prohlášení autora školního díla.
  \fi}
\gdef\thesis@czech@declaration{%
  Prohlašuji, že tato \thesis@lower{czech@typeName} je mým
  původním autorským dílem, které jsem vypracoval%
  \thesis@czech@gender@koncovka\ samostatně. Všechny zdroje,
  prameny a~literaturu, které jsem při vypracování
  používal\thesis@czech@gender@koncovka\ nebo z~nich
  čerpal\thesis@czech@gender@koncovka, v~práci řádně cituji
  s~uvedením úplného odkazu na příslušný zdroj.}

% Ostatní
\gdef\thesis@czech@advisorSignature{Podpis vedoucího}
\gdef\thesis@czech@typeName@proposal{Teze disertační práce}
\gdef\thesis@czech@typeName@akuzativ@proposal{Tezi disertační práce}
%    \end{macrocode}\iffalse
%</mu/fi>
% \fi\file{locale/mu/phil/fithesis-czech.def}
% This is the Czech locale file specific to the Faculty of
% Arts at the Masaryk University in Brno.
% It replaces the \texttt{facultyName} placeholder with the
% correct value. It also redefines the \texttt{declaration},
% \texttt{typeName} and \texttt{typeName@akuzativ} strings in
% accordance with the requirements of the faculty.
%
% The locale file also defines the \texttt{departmentName}
% string, which is used by the \texttt{style/mu/fithesis-phil^^A
% .sty} style file, when typesetting the names of known
% departments.
% \iffalse
%<*mu/phil>
% \fi\begin{macrocode}
\ProvidesFile{fithesis/locale/mu/phil/fithesis-czech.def}[2016/03/22]

% Zástupné texty
\gdef\thesis@czech@facultyName{Filozofická fakulta}
\gdef\thesis@czech@departmentName{%
  \ifx\thesis@department\thesis@departments@kisk
    Kabinet informačních studií a knihovnictví%
  \else
    <<Neznámé oddělení (\thesis@department)>>%
  \fi}
\gdef\thesis@czech@declaration{%
  \ifx\thesis@department\thesis@departments@kisk
    Prohlašuji, že jsem předkládanou práci zpracoval%
    \thesis@czech@gender@koncovka\ samostatně a~použil%
    \thesis@czech@gender@koncovka\ jen uvedené prameny a~%
    literaturu. Současně dávám svolení k~tomu, aby elektronická
    verze této práce byla zpřístupněna přes informační systém
    Masarykovy univerzity.%
  \else
    Prohlašuji, že jsem \thesis@lower{czech@typeName@akuzativ}
    vypracoval\thesis@czech@gender@koncovka\ samostatně s~využitím
    uvedené literatury.%
  \fi}

% Ostatní
\global\let\thesis@czech@typeName@super
  \thesis@czech@typeName
\gdef\thesis@czech@typeName{%
  \ifx\thesis@type\thesis@bachelors
    Bakalářská diplomová práce%
  \else\ifx\thesis@type\thesis@masters
    Magisterská diplomová práce%
  \else
    \thesis@czech@typeName@super
  \fi\fi}

\global\let\thesis@czech@typeName@akuzativ@super
  \thesis@czech@typeName@akuzativ
\gdef\thesis@czech@typeName@akuzativ{%
  \ifx\thesis@type\thesis@bachelors
    Diplomovou práci%
  \else\ifx\thesis@type\thesis@masters
    Diplomovou práci%
  \else
    \thesis@czech@typeName@akuzativ@super
  \fi\fi}
%    \end{macrocode}\iffalse
%</mu/phil>
% \fi\file{locale/mu/ped/fithesis-czech.def}
% This is the Czech locale file specific to the Faculty of
% Education at the Masaryk University in Brno.
% It replaces the \texttt{facultyName} placeholder with the
% correct value. The file also defines the
% \texttt{bib@title} and \texttt{bib@pages} strings required by the
% |\thesis@blocks@bibEntry| block defined within the
% \texttt{style/mu/\discretionary{}{}{}fithesis-ped.sty}
% style file.
% \iffalse
%<*mu/ped>
% \fi\begin{macrocode}
\ProvidesFile{fithesis/locale/mu/ped/fithesis-czech.def}[2016/03/22]

% Zástupné texty
\gdef\thesis@czech@facultyName{Pedagogická fakulta}
%    \end{macrocode}\iffalse
%</mu/ped>
% \fi\file{locale/mu/sci/fithesis-czech.def}
% This is the Czech locale file specific to the Faculty of Science
% at the Masaryk University in Brno.  It defines the private macros
% required by the |\thesis@blocks@|\discretionary{}{}{}|bibEntryCs|
% block defined within the
% \texttt{style/mu/fithesis-sci.sty} style file.  It also
% replaces the \texttt{facultyName} placeholder with the correct
% value and redefines the \texttt{abstractTitle} and
% \texttt{declaration} strings in accordance with the formal
% requirements of the faculty.
% \iffalse
%<*mu/sci>
% \fi\begin{macrocode}
\ProvidesFile{fithesis/locale/mu/sci/fithesis-czech.def}[2017/10/28]

% Zástupné texty
\gdef\thesis@czech@facultyName{Přírodovědecká fakulta}

% Ostatní
\gdef\thesis@czech@abstractTitle{Abstrakt}
\gdef\thesis@czech@declaration{%
  Prohlašuji, že jsem \thesis@lower{czech@typeName@%
  akuzativ} vypracoval\thesis@czech@gender@koncovka\ samostatně,
  s~využitím pouze citovaných pramenů, dalších informací a zdrojů
  v~souladu s~Disciplinárním řádem pro studenty Pedagogické fakulty
  Masarykovy univerzity a se zákonem č.\ 121/2000 Sb., o~právu
  autorském, o~právech souvisejících s~právem autorským a o~změně
  některých zákonů (autorský zákon), ve znění pozdějších předpisů.}

% Bibliografický záznam
\gdef\thesis@czech@bib@programme{Studijní program}
\global\let\thesis@czech@bib@field\thesis@czech@fieldTitle
\gdef\thesis@czech@bib@academicYear{Akademický rok}
\gdef\thesis@czech@bib@pages{Počet stran}
\global\let\thesis@czech@bib@keywords\thesis@czech@keywordsTitle
%    \end{macrocode}\iffalse
%</mu/sci>
% \fi

% % \file{locale/slovak.def}
% This is the base file of the Slovak locale. It defines all the
% private macros mandated by the locale file interface. 
% \begin{macro}{\thesis@slovak@gender@koncovka}
% The locale file defines the |\thesis@slovak@gender@koncovka|
% macro, which expands to the correct verb ending based on the
% value of the |\thesis@ifwoman| macro.
% \end{macro}\iffalse
%<*base>
% \fi\begin{macrocode}
\ProvidesFile{fithesis3/locale/slovak.def}[2015/04/18]

% Pomocná makrá
\def\thesis@slovak@gender@koncovka{%
  \ifthesis@woman a\fi}

% Zástupné texty
\def\thesis@slovak@placeholders@title{Názov práce}
\def\thesis@slovak@placeholders@keywords{kľúčové slovo 1,
  kľúčové slovo 2, ...}
\def\thesis@slovak@placeholders@abstract{Text zhrnutie}
\def\thesis@slovak@placeholders@author{Meno autora}
\def\thesis@slovak@placeholders@author@firstName{Meno}
\def\thesis@slovak@placeholders@author@surname{Priezvisko}
\def\thesis@slovak@universityName{Názov univerzity}
\def\thesis@slovak@facultyName{Názov fakulty}
\def\thesis@slovak@placeholders@advisor{Meno vedúceho}
\def\thesis@slovak@placeholders@department{Názov katedry}
\def\thesis@slovak@placeholders@programme{Názov študijného
  programu}
\def\thesis@slovak@placeholders@field{Název študijného odboru}
\def\thesis@slovak@assignment{Namiesto tejto stránky vložte kópiu
  oficiálneho podpísaného zadania práce.}
\def\thesis@slovak@declaration{Text prehlásenie ...}

% Rôzne
\def\thesis@slovak@fieldTitle{Odbor}
\def\thesis@slovak@advisorTitle{Vedúci práce}
\def\thesis@slovak@authorTitle{Autor}
\def\thesis@slovak@abstractTitle{Zhrnutie}
\def\thesis@slovak@keywordsTitle{Kľúčové slová}
\def\thesis@slovak@thanksTitle{Poďakovanie}
\def\thesis@slovak@declarationTitle{Prehlásenie}
\def\thesis@slovak@winter{Jar}
\def\thesis@slovak@summer{Jeseň}
\def\thesis@slovak@semester{%
  \thesis@{slovak@\thesis@season}\ \thesis@year}
\def\thesis@slovak@typeName{%
  \ifx\thesis@type\thesis@bachelors%
    Bakalárska práca%
  \else\ifx\thesis@type\thesis@masters%
    Diplomová práca%
  \else\ifx\thesis@type\thesis@doctoral%
    Dizertačná práca%
  \else\ifx\thesis@type\thesis@rigorous%
    Rigorózna práca%
  \else%
    <<Neznámy typ práce (\thesis@type)>>%
  \fi\fi\fi\fi}
\def\thesis@slovak@typeName@akuzativ{%
  \ifx\thesis@type\thesis@bachelors%
    Bakalársku prácu%
  \else\ifx\thesis@type\thesis@masters%
    Diplomovú prácu%
  \else\ifx\thesis@type\thesis@doctoral%
    Dizertačnú prácu%
  \else\ifx\thesis@type\thesis@rigorous%
    Rigoróznu prácu%
  \else%
    <<Neznámy typ práce (\thesis@type)>>%
  \fi\fi\fi\fi}
%    \end{macrocode}\iffalse
%</base>
% \fi\file{locale/mu/slovak.def}
% This is the Slovak locale file specific to the Masaryk
% University in Brno. It replaces the \texttt{universityName}
% placeholder with the correct value and defines a placeholder
% declaration text.
% \iffalse
%<*mu>
% \fi\begin{macrocode}
\ProvidesFile{fithesis3/locale/mu/slovak.def}[2015/04/26]

% Zástupné texty
\def\thesis@slovak@universityName{Masarykova Univerzita}
\def\thesis@slovak@declaration{%
  Prehlašujem, že som predloženú \thesis@lower{%
  slovak@typeName@akuzativ} vypracoval%
  \thesis@slovak@gender@koncovka\ samostatne len s~použitím
  uvedenej literatúry a prameňov.}
%    \end{macrocode}\iffalse
%</mu>
% \fi\file{locale/mu/law/slovak.def}
% This is the Slovak locale file specific to the Faculty of Law at
% the Masaryk University in Brno. It replaces the
% \texttt{facultyName} placeholder with the correct value and
% replaces the \texttt{abstractTitle} and
% \texttt{placeholders@abstract} strings in accordance with the
% requirements of the faculty.
% \iffalse
%<*mu/law>
% \fi\begin{macrocode}
\ProvidesFile{fithesis3/locale/mu/law/slovak.def}[2015/04/26]

% Rôzne
\def\thesis@slovak@abstractTitle{Abstrakt}

% Zástupné texty
\def\thesis@slovak@placeholders@abstract{Text abstraktu}
\def\thesis@slovak@facultyName{Právnická fakulta Masarykovej
  univerzity}
%    \end{macrocode}\iffalse
%</mu/law>
% \fi\file{locale/mu/fsps/slovak.def}
% This is the Slovak locale file specific to the Faculty of Sports
% Studies at the Masaryk University in Brno. It replaces the
% \texttt{facultyName} placeholder with the correct value and the
% \texttt{placeholders@field} and \texttt{fieldTitle} strings in
% accordance with the common usage at the faculty.
% \iffalse
%<*mu/fsps>
% \fi\begin{macrocode}
\ProvidesFile{fithesis3/locale/mu/fsps/slovak.def}[2015/04/18]

% Zástupné texty
\def\thesis@slovak@placeholders@field{Názov špecializácie}
\def\thesis@slovak@facultyName{Fakulta športových štúdií}

% Rôzne
\def\thesis@slovak@fieldTitle{Špecializácie}
%    \end{macrocode}\iffalse
%</mu/fsps>
% \fi\file{locale/mu/fss/slovak.def}
% This is the Slovak locale file specific to the Faculty of Social
% Studies at the Masaryk University in Brno. It replaces the
% \texttt{facultyName} and \texttt{placeholders@assignment}
% strings with the correct value.
% \iffalse
%<*mu/fss>
% \fi\begin{macrocode}
\ProvidesFile{fithesis3/locale/mu/fss/slovak.def}[2015/04/26]

% Zástupné texty
\def\thesis@slovak@facultyName{Fakulta sociálnych štúdií}
\def\thesis@slovak@assignment{Namiesto tejto stránky vložte kópiu
  oficiálneho podpísaného zadania práce alebo prehlásenie autora
  školského diela alebo obidve~v závislosti na požiadavkách
  príslušnej katedry.}
%    \end{macrocode}\iffalse
%</mu/fss>
% \fi\file{locale/mu/econ/slovak.def}
% This is the Slovak locale file specific to the Faculty of
% Economics and Administration at the Masaryk University in Brno.
% It replaces the \texttt{facultyName} placeholder with the
% correct value.
% \iffalse
%<*mu/econ>
% \fi\begin{macrocode}
\ProvidesFile{fithesis3/locale/mu/econ/slovak.def}[2015/04/18]
\def\thesis@slovak@facultyName{Ekonomicko-správna fakulta}
%    \end{macrocode}\iffalse
%</mu/econ>
% \fi\file{locale/mu/med/slovak.def}
% This is the Slovak locale file specific to the Faculty of
% Medicine at the Masaryk University in Brno.
% It replaces the \texttt{facultyName} placeholder with the
% correct value and redefines the \texttt{abstractTitle},
% and \texttt{placeholders@abstract} strings in accordance
% with the common usage at the faculty. The file also defines the
% \texttt{bib@title} and \texttt{bib@pages} strings required by the
% |\thesis@blocks@bibEntry| block defined within the
% \texttt{style/mu/fithesis3-med.sty} style file.

% \iffalse
%<*mu/med>
% \fi\begin{macrocode}
\ProvidesFile{fithesis3/locale/mu/med/slovak.def}[2015/04/26]

% Rôzne
\def\thesis@slovak@abstractTitle{Anotácie}

% Zástupné texty
\def\thesis@slovak@placeholders@abstract{Text abstraktu}
\def\thesis@slovak@facultyName{Lekárska fakulta}

% Bibliografický zoznam
\def\thesis@slovak@bib@title{Bibliografický záznam}
\def\thesis@slovak@bib@pages{str}
%    \end{macrocode}\iffalse
%</mu/med>
% \fi\file{locale/mu/fi/slovak.def}
% This is the Slovak locale file specific to the Faculty of
% Informatics at the Masaryk University in Brno.
% It replaces the \texttt{facultyName} placeholder with the
% correct value and updates the \texttt{placeholders@assignment}
% and \texttt{declaration} strings in accordance with the
% requirements of the faculty. The file also defines the
% \texttt{advisorSignature} string required by the
% |\thesis@blocks@titlePage| block defined within the
% \texttt{style/mu/fithesis3-fi.sty} style file.
% \iffalse
%<*mu/fi>
% \fi\begin{macrocode}
\ProvidesFile{fithesis3/locale/mu/fi/slovak.def}[2015/04/18]

% Zástupné texty
\def\thesis@slovak@facultyName{Fakulta informatiky}
\def\thesis@slovak@assignment{Namiesto tejto stránky vložte kópiu
  oficiálneho podpísaného zadania práce a prehlásenie autora
  školského diela.}
\def\thesis@slovak@declaration{%
  Prehlasujem, že táto \thesis@lower{slovak@typeName} je mojím
  pôvodným autorským dielom, ktoré som vypracoval%
  \thesis@slovak@gender@koncovka samostatne. Všetky zdroje,
  pramene a literatúru, ktoré som pri vypracovaní
  používal\thesis@slovak@gender@koncovka\ alebo z~nich
  čerpal\thesis@slovak@gender@koncovka, v~práci riadne citujem
  s~uvedením úplného odkazu na príslušný zdroj.}

% Rôzne
\def\thesis@slovak@advisorSignature{Podpis vedúceho}
%    \end{macrocode}\iffalse
%</mu/fi>
% \fi\file{locale/mu/phil/slovak.def}
% This is the Slovak locale file specific to the Faculty of
% Arts at the Masaryk University in Brno.
% It replaces the \texttt{facultyName} placeholder with the
% correct value. It also defines the \texttt{declaration} string
% and redefines the \texttt{typeName} and
% \texttt{typeName@akuzativ} strings in accordance with the
% requirements of the faculty.
% \iffalse
%<*mu/phil>
% \fi\begin{macrocode}
\ProvidesFile{fithesis3/locale/mu/phil/slovak.def}[2015/04/26]

% Zástupné texty
\def\thesis@slovak@facultyName{Filozofická fakulta}
\def\thesis@slovak@declaration{%
  Prehlašujem, že som predloženú \thesis@lower{%
  slovak@typeName@akuzativ} vypracoval%
  \thesis@slovak@gender@koncovka\ samostatne na
  základe vlastných zistení a len s~použitím
  uvedenej literatúry a prameňov.}

% Rôzne
\def\thesis@slovak@typeName{%
  \ifx\thesis@type\thesis@bachelors%
    Bakalárska diplomová práca%
  \else\ifx\thesis@type\thesis@masters%
    Magisterská diplomová práca%
  \else\ifx\thesis@type\thesis@doctoral%
    Dizertačná práca%
  \else%
    <<Neznámy typ práce (\thesis@type)>>%
  \fi\fi\fi}
\def\thesis@slovak@typeName@akuzativ{%
  \ifx\thesis@type\thesis@bachelors%
    Diplomovú prácu%
  \else\ifx\thesis@type\thesis@masters%
    Diplomovú prácu%
  \else\ifx\thesis@type\thesis@doctoral%
    Dizertačnú prácu%
  \else%
    <<Neznámý typ práce (\thesis@type)>>%
  \fi\fi\fi}
%    \end{macrocode}\iffalse
%</mu/phil>
% \fi\file{locale/mu/ped/slovak.def}
% This is the Slovak locale file specific to the Faculty of
% Education at the Masaryk University in Brno.
% It replaces the \texttt{facultyName} placeholder with the
% correct value. The file also defines the
% \texttt{bib@title} and \texttt{bib@pages} strings required by the
% |\thesis@blocks@bibEntry| block defined within the
% \texttt{style/mu/fithesis3-ped.sty} style file.
% \iffalse
%<*mu/ped>
% \fi\begin{macrocode}
\ProvidesFile{fithesis3/locale/mu/ped/slovak.def}[2015/04/18]

% Zástupné texty
\def\thesis@slovak@facultyName{Pedagogická fakulta}

% Bibliografický zoznam
\def\thesis@slovak@bib@title{Bibliografický záznam}
\def\thesis@slovak@bib@pages{str}
%    \end{macrocode}\iffalse
%</mu/ped>
% \fi\file{locale/mu/sci/slovak.def}
% This is the Slovak locale file specific to the Faculty of
% Science at the Masaryk University in Brno.
% It replaces the \texttt{facultyName} placeholder with the
% correct value.
% \iffalse
%<*mu/sci>
% \fi\begin{macrocode}
\ProvidesFile{fithesis3/locale/mu/sci/slovak.def}[2015/04/18]

% Zástupné texty
\def\thesis@slovak@facultyName{Prírodovedecká fakulta}
%    \end{macrocode}\iffalse
%</mu/sci>
% \fi

%
% \subsection{Style files}
% \label{sec:style-files}
%
% \subsubsection{The \texttt{style/mu/base.sty} style file}
% \label{sec:mu-base-style-file}
